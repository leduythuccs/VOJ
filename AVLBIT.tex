

Một dãy cấp số cộng là một dãy số mà 2 cặp phần tử liên tiếp bất kỳ có hiệu bằng nhau và khác 0. Trường hợp dãy số chỉ gồm 2 số khác nhau vẫn tính là một dãy cấp số cộng

Ví dụ:
\begin{itemize}
	\item 2, 5 là dãy cấp số cộng.
	\item 8, 3 là dãy cấp số cộng.
	\item 1, 2, 3, 4, 5 là dãy cấp số cộng.
	\item 11, 8, 5, 2 là dãy cấp số cộng.
	\item 1, 2, 4, 5, 7 không phải là dãy cấp số cộng.
\end{itemize}

Cho một dãy số A gồm N số nguyên dương. Cho Q truy vấn dạng (x, y). Mỗi truy vấn yêu cầu kiểm tra xem đoạn từ x tới y có phải là hoán vị của một dãy cấp số cộng không.

\subsubsection{\emph{Dữ liệu vào }}

Dòng đầu chứa 2 số nguyên N, Q.

Số thứ i trong N số ở dòng thứ 2 là A $_ i. $

Dòng thứ i trong Q dòng tiếp theo chứa 2 số nguyên x, y mô tả truy vấn thứ i.

 

\subsubsection{\emph{Dữ liệu ra }}

Gồm Q dòng.

Dòng thứ i trong Q dòng sẽ trả lời cho truy vấn thứ i.

In ra YES nếu đoạn từ x tới y là hoán vị của một dãy cấp số cộng. Ngược lại thì ghi ra NO.

\emph{\textbf{Ràng buộc }}

11 test có N, Q  $\le$  1000 .

10 test có N  $\le$  1000, Q  $\le$  10^6 .

10 test có N  $\le$  10^5, Q  $\le$  10^5.

A $_ i $  $\le$  10^9

\emph{\textbf{Ví dụ }}
\begin{verbatim}
Input
 5 2
 1 3 2 5 4
 1 5
 2 4

Output
 YES
 NO


\end{verbatim}

 
