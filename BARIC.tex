



   Học theo những gì đọc được từ một bài báo viết về việc tăng sản lượng sữa, Bessie đã biến mình trở thành một con bò Ba-ri bằng cách tìm hiểu về áp suất không khí để lấy lòng Farmer John.  

   Cô ta đã làm N phép đo (1  $\le$  N  $\le$  100) trong ngày. Để thuận tiện, chúng lần lượt được gọi là M\_1, M\_2, ..., M\_N (M\_i  $\le$  1 000 000). Các phép đo được đánh số theo thứ tự Bessie thực hiện chúng.  

   Để thể hiện được những phân tích về áp suất khí quyển trong ngày, Bessie đang để tâm đến việc tìm một tập hợp con của các phép đo, biểu thị bởi K (1  $\le$  K  $\le$  N) chỉ số s\_j (trong đó 1  $\le$  s\_1  $\le$  s\_2  $\le$  ...  $\le$  s\_K  $\le$  N), thể hiện chính xác được toàn bộ các phép đo, tức là, giới hạn sai số trong khoảng cho phép.  

   Trong bất kì tập hợp các phép đo nào, một lỗi xuất hiện ở mỗi phép đo có vị trí:  
\begin{enumerate}
	\item     trước phép đo đầu tiên trong tập hợp;   
	\item     giữa hai phép đo liên tiếp trong tập hợp;   
	\item     sau phép đo cuối cùng trong tập hợp.   
\end{enumerate}

   Tổng sai số của tập hợp đó được tính bằng tổng các lỗi trong các phép đo.  

   Cụ thể, với mỗi phép đó có chỉ số i mà không nằm trong tập hợp các phép đo đang xét:  
\begin{itemize}
	\item     Nếu i $<$ s\_1 , sai số được tính như sau: 2 * | M\_i - M\_(s\_1) |   
	\item     Nếu s\_j $<$ i $<$ s\_(j+1), sai số được tính như sau: | 2 * M\_i - Sum(s\_j, s\_(j+1)) |      trong đó Sum(x,y) = M\_x + M\_y   
	\item     Nếu s\_K $>$ i, sai số được tính như sau: 2 * | M\_i - M\_(s\_K) |   
\end{itemize}

   Cho biết sai số tối đa là E (E  $\le$  1 000 000), xác định số phần tử của tập hợp nhỏ nhất tạo ra sai số không vượt quá E.  

\subsubsection{   Dữ liệu  }

   Dòng 1: Chứa hai số nguyên cách nhau bởi khoảng trắng: N và E  

   Dòng 2..N+1: Dòng i+1 chứa số nguyên M\_i  
\begin{verbatim}
Dữ liệu mẫu
4 20
10
3
20
40

Giải thích


Bessie tiến hành 4 phép đo; sai số tối đa cho phép là 20. 
Giá trị các phép đo lần lượt là: 10,3,20,40
\end{verbatim}

\subsubsection{   Kết quả  }

   Dòng 1: Hai số nguyên cách nhau bởi khoảng trắng: số phần tử của tập hợp các phép đo nhỏ nhất tạo ra sai số không vượt quá E và sai số tập hợp đó tạo ra.  
\begin{verbatim}
Kết quả mẫu
2 17

Giải thích
Chọn phép đo thứ hai và thứ tự là giải pháp tốt nhất, cho sai số là 17. 
Sai số trong phép đo thứ nhất là 2*|10-3|=14; 
sai số trong phép đo thứ hai là |2*20-(3+40)|=3.
\end{verbatim}
