

Người ta định nghĩa đệ qui dãy ngoặc và cấp của dãy như sau:
\begin{itemize}
	\item Xâu rỗng được gọi là dãy ngoặc cấp 0.
	\item Nếu S là xâu ngoặc cấp k thì (S) là xâu ngoặc cấp k+1.
	\item Nếu A, B là các dãy ngoặc thì S = AB là một dãy ngoặc với cấp bằng số lớn hơn trong cấp của A và B.
\end{itemize}

Định nghĩa này chỉ áp dụng cho những xâu sinh ra theo qui tắc đệ qui trên.

Cho 2 số nguyên dương N và k, gọi S là tập các dãy ngoặc cấp k độ dài N.
\begin{enumerate}
	\item Cho biết S có bao nhiêu phần tử.
	\item Cho một dãy ngoặc thuộc, hãy cho biết thứ tự từ điển của dãy này trong tập S.
\end{enumerate}

\subsubsection{Input}
\begin{itemize}
	\item Dòng đầu ghi 2 số N, k (N chẵn, N  $\le$  60, k  $\le$  n/2).
	\item Dòng hai ghi 1 xâu ngoặc cấp k độ dài N.
\end{itemize}

\subsubsection{Output}

Gồm hai dòng, mỗi dòng trả lời 1 yêu cầu theo thứ tự trên.

\subsubsection{Example}
\begin{verbatim}
Input:
6 2
(())()

Output:
3
2
\end{verbatim}
