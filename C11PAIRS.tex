



   N người đang đứng xếp hàng chờ mua vé vào buổi hòa nhạc. Mọi người đều phát chán khi phải chờ đợi, vì vậy họ nhìn quanh xem có ai quen hay không.  

   Hai người A và B đứng trong hàng có thể nhìn thấy nhau nếu:  
\begin{itemize}
	\item     Người A và người B đang đứng cạnh nhau.   
	\item     Giữa người A và người B,    \textbf{     không có ai cao hơn hẳn    }    một trong hai người.   
\end{itemize}

   Hãy đếm xem có bao nhiêu cặp có thể nhìn thấy nhau trong hàng.  

\subsubsection{   Dữ liệu  }
\begin{itemize}
	\item     Dòng đầu tiên chứa số nguyên dương N, là số người đang đứng trong hàng.   
	\item     Mỗi dòng trong N dòng tiếp theo chứa một số nguyên là chiều cao của một người tính bằng nanomet. (Tất cả mọi người đều thấp hơn 2    $^     31    $    nanomet).   
\end{itemize}

\subsubsection{   Kết quả  }
\begin{itemize}
	\item     Một số nguyên duy nhất là kết quả cần tìm.   
\end{itemize}

\subsubsection{   Ví dụ  }
\begin{verbatim}
\textbf{Input:}
\\7
\\2
\\4
\\1
\\2
\\2
\\5
\\1 \end{verbatim}
\begin{verbatim}
\textbf{Output:}
\\10\end{verbatim}

\subsubsection{   Giải thích  }

   Các cặp có thể nhìn thấy nhau là (1, 2), (2, 3), (2, 4), (2, 5), (2, 6), (3, 4), (4, 5), (4, 6), (5, 6), (6, 7).  

\subsubsection{   Giới hạn  }
\begin{itemize}
	\item     1 ≤ N ≤ 5.10    $^     5    $
	\item     Trong 1/3 số test 1 ≤ N ≤ 5000   
\end{itemize}