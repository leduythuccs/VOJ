



   Cho 2 số nguyên N và K (1 $<$= N $<$= 2   $^    64   $   - 1, 3 $<$= K $<$= 10). Tìm số nguyên lớn nhất không vượt quá N và là tích của K số nguyên tố liên tiếp.  

\subsubsection{   Input  }
\begin{itemize}
	\item     Dòng đầu là số nguyên T tương ứng với số bộ test (1 $<$= T $<$= 15)   
	\item     T dòng tiếp theo mỗi dòng là 1 cặp số (N, K) cách nhau 1 dấu cách   
\end{itemize}

\subsubsection{   Output  }
\begin{itemize}
	\item     Gồm T dòng là kết quả của T bộ test tương ứng, nếu không tìm được số thỏa mãn in ra -1   
\end{itemize}

\subsubsection{   Example  }
\begin{verbatim}
\textbf{Input:}
2
\\100 4
\\110 3
\\
\\\textbf{Output:}
-1
\\105 \end{verbatim}