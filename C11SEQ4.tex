



   Sau khi được đố câu hỏi Violympic: “Có bao nhiêu số lẻ có 6 chữ số, các chữ số đứng cạnh nhau thì khác nhau”, Songuku95 nghĩ ra một bài toán mở rộng hơn:   
\\   Cho 2 số tự nhiên N, M. Tập hợp A=\{0,1,2,…,n\}. Câu hỏi đặt ra là có bao nhiêu dãy X gồm k phần tử: \{x1,x2,x3,…,xk\} thỏa mãn:  

   -          Xi thuộc tập hợp A ( 1 $<$= i $<$= k )  

   -          Xk là số lẻ  

   -          X1 khác 0  

   -          Xi và X(i+1) là 2 số khác nhau ( với 1 $<$=  i  $<$ k )   

    -      1 $<$= k $<$= N  

   Do kết quả có thể rất lớn nên các bạn chỉ cần in ra Kết quả mod M.  

   Lưu ý: 2 dãy X,Y khác nhau nếu tồn tại 1 vị trí p sao cho Xp khác Yp  

\textbf{    Giới hạn:   }

   -          20\% số test có n,m $<$= 1000  

   -          20\% số test tiếp theo có n,m $<$= 10\textasciicircum6  

   -          20\% số test tiếp theo có n $<$= 10\textasciicircum6, m $<$= 10\textasciicircum15  

   -          20\% số test tiếp theo có n $<$= 10\textasciicircum15, m $<$= 1000  

   -          20\% số test còn lại có n,m $<$= 10\textasciicircum15  

\subsubsection{   Input  }

   Gồm 1 dòng duy nhất chứa hai số N,M ( 3 $<$= N,M $<$= 10\textasciicircum15 )  

\textbf{     Output    }

   Gồm 1 dòng duy nhất là kết quả bài toán  



\subsubsection{   Example  }
\begin{verbatim}
\textbf{Input:}
3 123456
\textbf{Output:}
20\end{verbatim}
\begin{verbatim}
\textbf{Giải thích: }Có 20 số thỏa mãn là: 1, 3, 13, 21, 23, 31, 101, 103, 121, 123, 131, 201, 203, 213, 231, 301, 303, 313, 321, 323\end{verbatim}