



   Có một sân cỏ được chia làm m x n ô vuông, mỗi ô vuông có thể chứa cỏ hoặc là chứa các hòn đá.   \emph{    yenthanh132   }   đã mua một máy cắt cỏ có kích thước k x k ô vuông, máy cắt cỏ có thể cắt toàn bộ cỏ trong k x k ô vuông mà máy cắt cỏ đi qua. Lưu ý máy cắt cỏ không thể đi ra ngoài sân cỏ và cũng không được đi vào vùng chứa các hòn đá, tức là không có ô chứa các hòn đá nào nằm trong vùng k x k ô vuông đặt máy cắt cỏ. Ban đầu   \emph{    yenthanh132   }   sẽ đặt máy cắt cỏ trong một vị trí nào đó trên sân. Sau đó   \emph{    yenthanh132   }   sẽ di chuyển máy cắt cỏ theo 1 trong 4 hướng đông, tây, nam, bắc một 1 ô vuông, nói cách khác nếu gọi x,y là tọa độ góc trái trên của máy cắt cỏ thì ta có thể di chuyển máy cắt cỏ đến 1 trong 4 vị trí có góc trái trên là (x-1,y), (x+1,y), (x,y-1), (x,y+1). Và tất nhiên vị trí mới này cũng không thể chứa các ô vuông có đá.  

   Hãy giúp   \emph{    yenthanh132   }   tìm vị trí đặt máy cắt cỏ sao cho diện tích cỏ cắt được là nhiều nhất, diện tích có cắt được bằng số ô vuông chứa cỏ mà máy cắt cỏ đã đi qua (mỗi ô chứa cỏ chỉ tính 1 lần).  

\subsubsection{   Input  }
\begin{itemize}
	\item     Dòng đầu tiên chứa 3 số nguyên dương m, n, k ( k  $\le$  min(m,n) ).   
	\item     m dòng tiếp theo, mỗi dòng chứa n kí tự, kí tự '.' đại diện cho ô chứa cỏ, kí tự '\#' đại diện cho ô chứa các hòn đá.   
\end{itemize}

\subsubsection{   Output  }
\begin{itemize}
	\item     Xuất ra một số nguyên duy nhất là diện tích cỏ cắt được lớn nhất. Nếu không có vị trí nào để đặt được máy cắt cỏ, tức    \emph{     yenthanh132    }    không thể cắt được bất cứ ô vuông chứa cỏ nào, ta xuất ra -1.   
\end{itemize}

\subsubsection{   Giới hạn:  }
\begin{itemize}
	\item     Trong 20\% số test có 1  $\le$  m,n  $\le$  100   
	\item     Trong tất cả các test có 1  $\le$  m,n  $\le$  2000   
\end{itemize}

\subsubsection{   Ví dụ:  }
\begin{verbatim}
\textbf{Input 1:}
4 5 2
\\.....
\\...#.
\\.....
\\..#..

\textbf{Output 1:}
11
\\
\\Ta có thể đặt máy cắt cỏ vào ô (2,2), thì có 11 ô có có thể cắt được bằng máy cắt cỏ
\\(kí tự 'x' đại diện cho vùng cỏ cắt được, kí tự '@' đại diện cho vị trí đặt máy cắt cỏ ban đầu
\\ và tất nhiên cũng là vùng cỏ cắt được)
\\xxx..
\\x@@#.
\\x@@..
\\xx#..\end{verbatim}
\begin{verbatim}
\textbf{Input 2:}
4 4 3
\\....
\\....
\\....
\\....

\textbf{Output 2:}
16
\\
\\Ta có thể đặt máy cắt cỏ ở bất kì vị trí nào trong sân để cắt được toàn bộ cỏ trong sân.\end{verbatim}
\begin{verbatim}
\textbf{Input 3:}
4 5 3
\\#....
\\..#..
\\.....
\\....#

\textbf{Output 3:}
-1
\\
\\Không có vị trí nào hợp lệ để đặt được máy cắt cỏ vào sân.\end{verbatim}
