



   Một bàn cờ thế là một bảng gồm 4 dòng, 4 cột. Mỗi thế cờ là một cách sắp xếp 8 quân cờ, hai quân khác nhau ở hai ô khác nhau.            Bài toán đặt ra là cho hai thế cờ 1 và 2, hãy tìm một số ít nhất bước di chuyển quân để chuyển từ thế 1 sang thế 2; một bước di chuyển quân là một lần chuyển quân cờ sang ô trống kề cạnh với ô quân cờ đang đứng.  

\subsubsection{   Dữ liệu vào  }

   Từ file văn bản gồm 8 dòng, mỗi dòng là một xâu nhị phân độ dài 4 mà số 1/0 tương ứng với vị trí có hoặc không có quân cờ. Bốn dòng đầu là thế cờ 1, bốn dòng sau là thế cờ 2.  

\subsubsection{   Dữ liệu ra  }

   Gồm 1 dòng duy nhất là số bước chuyển quân ít nhất  

\subsubsection{   Ví dụ  }
\begin{verbatim}
Dữ liệu vào:
1111
0000
1110
0010
1010
0101
1010
0101

Dữ liệu ra :
4
\end{verbatim}