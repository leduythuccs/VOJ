

 

Cho 1 bàn cờ kích thước 3 dòng, N cột (N $\le$ 10\textasciicircum9). Các ô thuộc dòng trên cùng có một quân tốt đen ở mỗi ô, các ô ở dòng dưới cùng có một quân tốt trắng ở mỗi ô.
\\Hai người chơi 1 trò chơi, quân trắng đi trước. Hai người thực hiện nước đi của mình luân phiên, ai đến lượt mình mà không thể đi được nữa ( do không còn nước đi hợp lệ ) sẽ thua. Tại mỗi nước đi, một người sẽ đi các quân tốt theo luật cờ vua, 2 luật quan trọng nhất là :
\begin{itemize}
	\item Quân tốt chỉ có thể đi 1 ô dọc mỗi nước, và không được lùi.
	\item Quân tốt có thể ăn quân tốt khác màu theo 1 ô chéo.
\end{itemize}

Tuy nhiên có 2 luật sau khác với luật cờ vua :
\begin{itemize}
	\item Quân tốt không thể phong cấp ( thành hậu, mã, ...).
	\item Tại một nước đi, việc ăn tốt khác là bắt buộc ( nếu tồn tại nước ăn thì người chơi bắt buộc phải ăn tốt đối phương, nếu có nhiều lựa chọn ăn thì sự lựa chọn thuộc về người chơi).
\end{itemize}

Biết cả 2 người chơi đều cố gắng chơi tốt nhất có thể. Hãy xác định ai là người chiến thắng.

\subsubsection{Input}

Gồm nhiều dòng, mỗi dòng ghi một số N.

\subsubsection{Output}

Với mỗi số N tương ứng ở input, in ra "White" nếu người chơi quân trắng thắng, in ra "Black" nếu người chơi quân đen thắng. ( Không in ra dấu nháy kép " )

\subsubsection{Example}
\begin{verbatim}
Input:
3
4
5

Output:
White
Black
White
\end{verbatim}
