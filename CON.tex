



   Các phần tử của dãy số a[1], a[2], ..., a[n] được xếp lần lượt lên 1 vòng tròn theo chiều kim đồng hồ.  

   Người ta xây dựng dãy số b[] bằng cách tính lần lượt: b[i] = a[i-1] + a[i] + a[i+1], với i = 1, 2, ..., n và quy ước a[0] = a[n], a[n+1] = a[1].  

   Cho dãy số b[], tìm dãy số a[].  

\subsubsection{   Input  }

   Dòng đầu ghi n là số phần tử của 2 dãy số (3  $\le$  n  $\le$  100000)  

   Dòng thứ 2 ghi n số nguyên b[1], b[2], ..., b[n] (1  $\le$  b[i]  $\le$  100000)  

\subsubsection{   Output  }

   Một dòng duy nhất chứa n số nguyên a[1], a[2], ..., a[n]. Nếu có nhiều đáp án đúng, chỉ cần ghi ra 1 trong số đó.  

\subsubsection{   Example  }
\begin{verbatim}
Input:
4
3 4 4 4

Output:
1 1 2 1
\end{verbatim}
