

 

Xét một tập N đối tượng có thể so sánh được (2$<$=n$<$=10). Giữa 2 đối tượng a và b có thể tồn tại 1 trong 3 quan hệ phân loại:

a = b; a $<$ b; a $>$ b;

Như vậy, với 3 đối tượng (a, b, c) có thể tồn tại 13 quan hệ phân loại như sau:

a = b = c; a = b $<$ c; c $<$ a = b; a $<$ b = c
\\b = c $<$ a; a = c $<$ b; b $<$ a = c; a $<$ b $<$ c
\\a $<$ c $<$ b; b $<$ a $<$ c; b $<$ c $<$ a; c $<$ a $<$ b
\\c $<$ b $<$ a;

Cho số n, hãy xác định số lượng quan hệ phân loại khác nhau.

\subsubsection{Input}

Gồm nhiều số n. Mỗi số trên 1 dòng. Kết thúc file là -1.

\subsubsection{Output}

Với mỗi n, đưa ra số lượng quan hệ phân loại tìm được, mỗi số trên 1 dòng (không có dòng trống).

\subsubsection{Example}
\begin{verbatim}
\textbf{Input}
2
3
-1
\textbf{Output}
3
13
\end{verbatim}