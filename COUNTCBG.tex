



   Với 1 số tự nhiên N(1$<$= N $<$= 10\textasciicircum9) ta có thể phân tích nó thành tổng của một số số tự nhiên liên tiếp( tất nhiên những số này phải nhỏ hơn N). Ví dụ với N = 5 ta có duy nhất 1 cách phân tích là 5 = 2+3.      Bài toán đặt ra là cho số tự nhiên N, hãy cho biết có bao nhiêu cách phân tích số tự nhiên N thành tổng của các số tự nhiên liên tiếp.  

\subsubsection{   Input  }

   Gồm nhiều dòng, mỗi dòng chứa một số nguyên N. (Giới hạn : số dòng $<$= 100)  

\subsubsection{   Output  }

   Mỗi dòng ghi một số nguyên là số cách phân tích số N đọc được ở dòng tương ứng trong input.  

\subsubsection{   Ví dụ  }
\begin{verbatim}
Input:
12
5
4
13
45
100
234
3
175

Output:
1
1
0
1
5
2
5
1
5
\end{verbatim}