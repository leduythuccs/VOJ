



   Cho một tập hợp S các số nguyên, bạn hãy lập trình thực hiện các thao tác sau:  
\begin{itemize}
	\item     ADD x: thêm số x vào tập S   
	\item     DELETE x: xóa số x khỏi tập S   
	\item     MININUM: tìm số nhỏ nhất trong tập S   
	\item     MAXIMUM: tìm số lớn nhất trong tập S   
	\item     SUCC x: tìm số nhỏ nhất lớn hơn x trong tập S   
	\item     SUCC\_2 x: tìm số nhỏ nhất và không nhỏ hơn x trong tập S   
	\item     PRED x: tìm số lớn nhất nhỏ hơn x trong tập S   
	\item     PRED\_2 x: tìm số lớn nhất không vượt quá x trong tập S   
\end{itemize}

   Ghi chú: Đối với thao tác DELETE, giữ nguyên tập S nếu x không có trong tập S. Đối với các thao tác MINIMUM, MAXIMUM, SUCC, SUCC\_2, PRED và PRED\_2, in ra 'empty' nếu tập S rỗng. Đối với các thao tác SUCC, SUCC\_2, PRED và PRED\_2, in ra 'no' nếu không tìm được số thỏa mãn.  

   Các thao tác ADD, DELETE, MINIMUM, MAXIMUM, SUCC, SUCC\_2, PRED, PRED\_2 lần lượt được mã hóa bởi các chỉ số 1 2 3 4 5 6 7 8.  

\subsubsection{   Dữ liệu  }

   Gồm nhiều dòng, mỗi dòng bắt đầu bằng một số từ 0 đến 8 cho biết chỉ số thao tác cần thực hiện. Số 0 báo hiệu kết thúc dữ liệu nhập. Đối với các thao tác 1, 2, 5, 6, 7, 8, số tiếp theo trên dòng là số nguyên x (|x| ≤ 10   $^    9   $   ) cho biết tham số của thao tác. Biết số thao tác cần thực hiện không vượt quá 300000.  

\subsubsection{   Kết quả  }

   Đối với mỗi thao tác loại 3, 4, 5, 6, 7, 8 in ra một dòng là kết quả của thao tác.  

\subsubsection{   Ví dụ  }
\begin{verbatim}
Dữ liệu
4
1 10
1 5
3
1 7
4
2 5
3
5 10
6 10
7 10
8 10
2 10
7 100
0

Kết quả
empty
5
10
7
no
10
7
10
7
\end{verbatim}

\subsubsection{   Gợi ý  }

   Bạn hãy dùng cấu trúc set trong thư viện STL của C++. Bài này bạn chỉ nộp được bằng C++ mà thôi.  