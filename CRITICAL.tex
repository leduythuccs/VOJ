

Đất nước Hạnh Phúc có N thành phố được nối với nhau bởi M đường nối hai chiều. Giữa hai thành phố bất kỳ chỉ có nhiều nhất một con đường.

Chính quyền nước này đưa ra một tiêu chí để đánh giá độ quan trọng của mỗi thành phố, theo đó độ quan trọng của một thành phố X được tính bằng số cặp thành phố A và B mà để di chuyển từ A đến B (và ngược lại) bắt buộc phải đi qua thành phố X.

Bạn hãy lập trình tính độ quan trọng trung bình của tất cả các thành phố.

\subsubsection{Dữ liệu}

Dòng đầu tiên chứa hai số nguyên N, M (1 $<$= N $<$= 20000, 0 $<$= M $<$= 200000).

M dòng tiếp theo mỗi dòng chứa 2 số nguyên u, v (1$<$=u,v $<$=N) mô tả một đường nối.

\subsubsection{Kết quả}

Gồm một số thực duy nhất là độ quan trọng trung bình của các thành phố, làm tròn đến 2 chữ số thập phân.

\subsubsection{Ví dụ}
\begin{verbatim}
\textbf{Dữ liệu}
5 5
1 2
2 3
3 4
4 5
5 3

\textbf{Kết quả}
1.40

\end{verbatim}


\includegraphics{http://vn.spoj.com/VO10/content/critical.png}

Giải thích: Độ quan trọng của các thành phố 1, 2, 3, 4, 5 lần lượt là 0, 3, 4, 0, 0. Độ quan trọng của thành phố 3 là 4 vì có 4 cặp thành phố mà khi di chuyển đến nhau cần đi qua thành phố 3: (1, 4), (1, 5), (2, 4), (2,5).

\emph{Tác giả: Ngô Minh Đức }