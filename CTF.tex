



   Mọi số hữu tỉ đều có thể biểu diễn dưới dạng phân số hoặc dạng thập phân vô hạn tuần hoàn.  
\begin{itemize}
	\item \textbf{     (1)    }    Biểu diễn dạng phân số của một số hữu tỉ là a/b, với a, b là các số nguyên, b khác 0, và ước chung lớn nhất của a và b bằng 1 (để đảm bảo phân số tối giản).   
	\item \textbf{     (2)    }    Biểu diễn dạng thập phân vô hạn tuần hoàn của một số hữu tỉ là x.y(z), với x, y, z là các số nguyên (có thể có số 0 ở đầu). Trong biểu diễn này, (z) thể hiện cho phần tuần hoàn.   
\end{itemize}

   Biểu diễn phải đảm bảo trước tiên là y có ít chữ số nhất có thể, sau đó là z có ít chữ số nhất có thể. Ví dụ, biểu diễn 0.12(1212) là không hợp lệ; biểu diễn đúng là 0.(12).  

   Lưu ý rằng biểu diễn có thể không có phần y, và nếu số hữu tỉ có phần thập phân hữu hạn thì (z) = (0). Ví dụ, 1/3 = 0.(3), 20/13 = 1.(538461), 557/495 = 1.1(25), 5/1 = 5.(0), 3/2 = 1.5(0) …  

   Cho biểu diễn dạng thập phân vô hạn tuần hoàn của một số số hữu tỉ, hãy tìm biểu diễn dạng phân số của chúng.  

\subsubsection{   Dữ liệu  }

   Dòng đầu ghi T là số lượng số hữu tỉ cần tính toán. T dòng tiếp theo, mỗi dòng ghi một số hữu tỉ được biểu diễn dưới dạng thập phân vô hạn tuần hoàn, theo định dạng như mô tả ở (2).  

\subsubsection{   Kết quả  }

   Gồm T dòng, mỗi dòng ghi biểu diễn dạng phân số của số hữu tỉ tương ứng, theo định dạng như mô tả ở (1).  

\subsubsection{   Giới hạn  }

   Trong 50\% số điểm:  
\begin{itemize}
	\item     0 ≤ x ≤ 10^3   
	\item     0 ≤ y ≤ 10^3   
	\item     0 ≤ z ≤ 10^3   
\end{itemize}

   Trong 50\% số điểm còn lại:  
\begin{itemize}
	\item     0 ≤ x ≤ 10^10   
	\item     0 ≤ y ≤ 10^10   
	\item     0 ≤ z ≤ 10^10   
\end{itemize}

\subsubsection{   Ví dụ  }

\textbf{    Dữ liệu   }
\begin{verbatim}
5
0.(3)
1.(538461)
1.1(25)
5.(0)
1.5(0)
\end{verbatim}

\textbf{    Kết quả   }
\begin{verbatim}
1/3
20/13
557/495
5/1
3/2
\end{verbatim}
