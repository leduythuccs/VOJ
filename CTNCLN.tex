

Cho N số nguyên dương A1, A2, ..., An. Mỗi số có giá trị không vượt quá N.

Yêu cầu:

Hãy chia N số này thành một số nhóm sao cho:
\begin{itemize}
	\item Mỗi nhóm là một dãy các số liên tiếp.
	\item Trọng số của mỗi nhóm được tính theo công thức: Bình phương của \textbf{\emph{ số giá trị khác nhau }} trong nhóm đó.
\end{itemize}

Ví dụ:
\begin{itemize}
	\item 1 2 3 -$>$ số giá trị khác nhau bằng 3, trọng số = 9.
	\item 1 2 1  -$>$ số giá trị khác nhau bằng 2, trọng số = 4.
	\item 1 1 1  -$>$ số giá trị khác nhau bằng 1, trọng số = 1.
\end{itemize}

Trọng số của dãy số bằng Tổng trọng số của tất cả các nhóm. Hãy tìm cách chia sao cho Tổng trọng số đạt giá trị nhỏ nhất.

\subsubsection{Input}

Dòng 1: Gồm 2 số N và M - Số lượng số và số giá trị khác nhau trong dãy.

Dòng 2..N+1: Mỗi dòng chứa một số nguyên.

\subsubsection{Output}

Trọng số nhỏ nhất của dãy số.

\subsubsection{Example}
\begin{verbatim}
\textbf{Input:}
13 4
1
2
1
3
2
2
3
4
3
4
3
1
4

\textbf{Output:}
11\end{verbatim}


\\
\\\textbf{\emph{Giải thích}}: Chúng ta sẽ chia dãy số thành 8 nhóm
\begin{itemize}
	\item 4 nhóm đầu tiên mỗi nhóm chứa một số duy nhất.
	\item nhóm thứ 5 chứa hai số nguyên đều có giá trị là 2.
	\item nhóm thứ 6 gồm các số từ 7-$>$12 : (3, 4, 3, 4, 3)
	\item 2 nhóm cuối cùng mỗi nhóm gồm 1 số duy nhất.
\end{itemize}


\\Kết quả là 1\textasciicircum2 + 1\textasciicircum2 + 1\textasciicircum2 + 1\textasciicircum2 + 1\textasciicircum2 + 2\textasciicircum2 + 1\textasciicircum2 + 1\textasciicircum2= 11.
\\
\\\textbf{Giới hạn}: N $<$= 40 000.