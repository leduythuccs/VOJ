

Chàng Ngốc rất thích trò chơi ô ăn quan.nhưng có vẻ trò chơi quá "dễ" nên chàng nghĩ ra một trò chơi ô ăn quan của riêng mình:cho N ô vuông đánh số từ 1 đến N,chúng đc xếp thành một vòng tròn(ô thứ N+1 trùng với ô thứ 1).Trên mỗi ô vuông có sẵn một số sỏi.Với một lượt đi bắt đầu tại ô thứ i,bạn bốc hết sỏi ở ô đó,đi lần lượt qua các ô theo chiều tăng dần chỉ sô các ô vuông,qua mỗi ô bạn thả một viên sỏi vào ô đó.Cho đên khi bạn hết sỏi trên tay.Giả sử bạn thả viên sỏi cuối cùng tại ô thứ i,nếu ô i+1 có sỏi thì bạn bốc hết số sỏi ở ô đó và tiếp tục đi.Còn nếu ô thứ i+1 đã hết sỏi,bạn được "ăn" tất cả số sỏi ở ô thứ i+2,sau đó nếu ô thứ i+3 hết sỏi,bạn được ăn số sỏi ở ô thứ i+4...! Yêu cầu:cho ô bắt đầu,tính xem bạn dc ăn bao nhiêu viên sỏi!

\subsubsection{Input}

Dòng 1:số N là số ô vuông (3 $\le$ N $\le$ 10000) Dòng 2:N số nguyên dương,số thứ i là số sỏi ở ô thứ i.Số sỏi trong mỗi ô vuông  $\le$ 10000; Các dòng tiếp theo,mỗi dòng một số nguyên S,nghĩa là bắt đầu tại ô thứ S (1 $\le$ S $\le$ N) Dòng cuối,là số 0,báo hiệu kết thúc yêu cầu.

\subsubsection{Output}

Với mỗi yêu cầu ghi ra một số nguyên,là số sỏi bạn được ăn,mỗi số trên một dòng.

\subsubsection{Example}
\begin{verbatim}
INPUT1:
3
2 1 1
1
0

OUTPUT1:
2

INPUT2:
4
1 1 1 1
1 0

OUTPUT2:
4\end{verbatim}
