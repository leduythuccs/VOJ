



   Một xâu S có độ dài n gồm các kí tự x1x2…xn được gọi là xâu đối xứng nếu nó thỏa mãn xi=xn-i+1 với mọi i=1…n. Xâu s được gọi là CVPdrome nếu ta đảo lại trật tự các kí tự thì thu được một xâu đối xứng. Cho xâu s gồm n kí tự.   
\\\textbf{    Yêu cầu   }   :Tính số xâu con (gồm các kí tự liên tiếp của s) là CVPdrome.  


\\\textbf{    Input:   }
\\   Dòng 1: Số nguyên dương N (1$<$=N$<$=3*10\textasciicircum5) là độ dài xâu s.   
\\   Dòng 2: Xâu s, các kí tự trong s là các chữ cái latin thường hoặc hoa (‘A’-$>$’Z’ hoặc ‘a’-$>$’z’).  


\\\textbf{    Ouput:   }

   Một dòng duy nhất ghi số nguyên kết quả.  

   Ví dụ:  


\\\textbf{    Input:   }
\\   3   
\\   aaa   
\\\textbf{    Output:   }
\\   6  


\\\textbf{    Input:   }
\\   3   
\\   aAA   
\\\textbf{    Output:   }
\\   5  


\\   Giải thích các xâu (1,1),(1,3),(2,2),(2,3),(3,3) là các CVPdrome. Xâu aAA không là xâu đối xứng nhưng ta có thể viết lại  nó thành AaA là một xâu đối xứng.  