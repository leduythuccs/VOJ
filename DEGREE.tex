



   Một số nguyên dương A gọi là có bậc K đối với cơ số B nếu như :   
\\   • A = B^x1 + B^x2 + … + B^xk   
\\   ( trong đó x1 , x2 , … , xk là các số nguyên không âm thoả mãn x1 $<$$>$ x2 $<$$>$ x3 … $<$$>$ xk )   
\\   Ví dụ :   
\\   • 17 có bậc 2 đối với cơ số 2 vì 17 = 2^4 + 2^0 .   
\\   • 151 có bậc 3 đối với cơ số 5 vì 151 = 5^3 + 5^2 + 5^0.   
\\   Yêu cầu : Cho trước 1 đoạn [X,Y] . Hãy xác định xem trong đoạn này có bao nhiêu số có bậc K đối với cơ số B.   
\\   Giới hạn :   
\\   • 1  $\le$  X  $\le$  Y  $\le$  10^9   
\\   • 1  $\le$  K  $\le$  25, 2  $\le$  B  $\le$  9   
\\   • Chạy được với bộ nhớ thông báo $<$ 800 K bạn mới thực sự là thành công   
\\

\subsubsection{   Input  }

   1 dòng gồm 4 số nguyên dương X , Y , K , B  

\subsubsection{   Output  }

   Gồm 1 dòng duy nhất ghi ra số lượng số tìm được .  

\subsubsection{   Example  }
\begin{verbatim}
Input:
15 20 2 2

Output:
3 
\end{verbatim}
