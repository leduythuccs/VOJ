

Xét trò chơi xếp hình chữ nhật với các que diêm như sau: Có n que diêm, que thứ i có độ dài di. Người chơi cần chọn ra 4 que diêm để có thể xếp thành một hình chữ nhật, giả sử 4 que diêm mà người chơi chọn có độ dài lần lượt là a, b, c, d (a ≤ b ≤ c ≤ d), khi đó có thể xếp được thành một hình chữ nhật nếu a = b và c = d. Người chơi xếp được hình chữ nhật có diện tích càng lớn sẽ càng được điểm cao.

\textbf{Yêu cầu}: Cho d1, d2,..,dn là độ dài của n que diêm. Hãy tìm cách chọn 4 que diêm để xếp được thành một hình chữ nhật có diện tích lớn nhất.

\textbf{Dữ liệu}: Vào từ thiết bị vào chuẩn: Dòng đầu tiên ghi số nguyên dương K là số lượng bộ dữ liệu. Tiếp đến là K nhóm dòng, mỗi nhóm tương ứng với một bộ dữ liệu có cấu trúc như sau:
\begin{itemize}
	\item Dòng thứ nhất ghi một số nguyên dương n;
	\item Dòng tiếp theo chứa n số nguyên dương d1, d2,..,dn (di ≤ 10^9).
	\item Các số trên cùng một dòng được ghi cách nhau ít nhất một dấu cách.
\end{itemize}

\textbf{Kết quả}: Ghi ra thiết bị ra chuẩn gồm K dòng, mỗi dòng ghi một số nguyên là diện tích của hình chữ nhật xếp được tương ứng với bộ dữ liệu trong file dữ liệu vào (ghi -1 nếu không tồn tại cách chọn nào xếp được hình chữ nhật).

\textbf{Giới hạn}:
\begin{itemize}
	\item Subtask 1 (33\%): Giả thiết là n ≤ 30.
	\item Subtask 2 (33\%): Giả thiết là n ≤ 3000.
	\item Subtask 3 (33\%): Giả thiết là n ≤ 300000.
\end{itemize}

 

Ví dụ:
\begin{verbatim}
\textbf{Dữ liệu}
2
5
5 3 1 5 1
4
1 2 3 4

\textbf{Kết quả}
5
-1\end{verbatim}

 
