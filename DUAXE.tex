

 

Trong 1 cuộc đua xe phân khối lớn, đường đua gồm N đoạn đường. Điểm chuyển tiếp giữa 2 đoạn đường liên tiếp gọi là các nút, có tất cả N-1 nút như thế. Rõ ràng ai cũng muốn đi với tốc độ cao nhất để về đíhc nhanh nhất, tuy nhiên có các ràng buộc sau:
\begin{itemize}
	\item Vận tốc tại vị trí xuất phát và về đích phải bằng 0 (tức là xe phải dừng ở đúng đích)
	\item Xe chỉ có thể tăng tốc ở mức gia tốc tối đa A (m/s\textasciicircum2)
	\item Xe chỉ có thể giảm tốc (phanh) ở mức gia tốc tối đa B (m/s\textasciicircum2) Điều này có nghĩa là tại mỗi thời điểm, gia tốc a của xe phải thoả mãn -B$<$=a$<$=A
	\item Xe không bị giới hạn vận tốc trên đường nhưng khi qua các nút chuyển tiếp, để đảm bảo an toàn, lái xe không được phép vượt quá một vận tốc V (m/s) nào đó, tuỳ thuộc vào từng nút.
\end{itemize}

Giả thiết vận tốc của xe không bị giới hạn nào khác ngoài các ràng buộc trên. Hãy tính thời gian ngắn nhất để xe có thể về đích.

\subsubsection{Input}

Dữ liệu nhập vào có khuôn dạng như sau:
\begin{itemize}
	\item Dòng 1 ghi 3 số nguyên dương N, A, B (N $<$= 1000)
	\item N dòng tiếp, mỗi dòng mô tả 1 chặng đường, gồm 2 số nguyên dương là độ dài của chặng và vận tốc giới hạn tại điểm nút cuối chặng. Chú ý rằng điểm xuất phát là điểm đầu của chặng 1 và điểm đích là điểm cuối chặng N
\end{itemize}

\subsubsection{Output}

Thời gian ngắn nhất (đơn vị giây) để xe có thể đến đích. Kết quả ghi dưới dạng số thực chính xác đến 3 chữ số thập phân. Bài của bạn sẽ được coi là Accept nếu kết quả sai lệch không quá 0.01 so với kết quả của file đáp án.

\subsubsection{Example}
\begin{verbatim}
\textbf{Input:}
1 1 1
100 0

\textbf{Output:}
20.000
\end{verbatim}
\begin{verbatim}
\textbf{Input:}
2 1 1
10 10
90 0

\textbf{Output:}
20.000\end{verbatim}