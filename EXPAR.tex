



   Cho biểu thức số học chỉ chứa hai biến x, y, các hằng số và các phép toán +,−,×. Hãy xác định tính chẵn – lẻ của kết quả biểu thức dựa trên tính chẵn – lẻ của hai biến x, y.  

\subsubsection{   Input  }

   Dòng 1: xâu S có độ dài không quá 10   $^    6   $   chỉ gồm các kí tự chữ số, chữ cái ′x′,  ′y′ và các dấu  phép toán +,−,∗. Xâu S đảm bảo là biểu thức toán học hợp lệ, các hằng số đều là số nguyên không âm và không có chữ số vô nghĩa.  

   Dòng 2,3: mỗi dòng ghi xâu ′Odd′ hay ′Even' chỉ tính lẻ hay chẵn của tương ứng biến x và y.  

\subsubsection{   Output  }

   Ghi xâu ′Odd′ hay ′Even' là kết quả của bài toán.  

\subsubsection{   Example  }
\begin{verbatim}
\textbf{Input:}

x-y*2

Odd

Odd \textbf{Output:}Odd\end{verbatim}