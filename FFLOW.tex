



   Cho một đồ thị với N (2 ≤ N ≤ 5,000) đỉnh được đánh số từ 1 đến N và M (1 ≤ M ≤ 30,000) cạnh vô hướng, có trọng số, hãy tính giá trị của   \href{http://en.wikipedia.org/wiki/Maximum_flow_problem}{    luồng cực đại / lát cắt cực tiểu   }   từ đỉnh 1 đến đỉnh N  

\subsubsection{   Input  }

   Dòng đầu chứa hai số nguyên N và M. M dòng tiếp theo, mỗi dòng chứa ba số nguyên A, B, và C, thể hiện việc có một cạnh với khả năng thông qua C (1 ≤ C ≤ 10   $^    9   $   ) giữa nút A và nút B (1 ≤ A, B ≤ N). Lưu ý rằng có thể có nhiều cạnh giữa hai nút, cũng như có thể có một cạnh từ một nút đến chính nó.  

\subsubsection{   Output  }

   Viết ra một số nguyên duy nhất (có thể vượt quá kiểu số nguyên 32 bit) thể hiện giá trị của luồng cực đại / lát cắt cực tiểu giữa 1 và N.  

\subsubsection{   Example  }
\begin{verbatim}
Input:
4 6
1 2 3
2 3 4
3 1 2
2 2 5
3 4 3
4 3 3

Output:
5
\end{verbatim}

   Nhìn bài toán dưới dạng luồng cực đại, ta có thể cho 3 đơn vị luồng chảy qua đường 1 - 2 - 3 - 4 và 2 đơn vị luồng qua đường 1 - 3 - 4. Nhìn dưới góc độ lát cắt cực tiểu, ta có thể cắt cạnh thứ nhất và thứ 3. Cả 2 cách đều có tổng giá trị là 5.  

   Bài gốc:   \href{https://www.spoj.pl/problems/FASTFLOW/}{    https://www.spoj.pl/problems/FASTFLOW/   }   .  