

Vùng đất huyền thoại Gondor có một hệ thống truyền tin gồm các tháp canh để báo hiệu trong trường hợp khẩn cấp.

Khi mỗi tháp canh được báo hiệu, nhân viên truyền tin ở tháp đó sẽ ngay lập tức truyền thông tin đến tất cả các tháp chưa được nhận thông tin, và theo 1 thứ tự cho trước. Việc truyền tin xảy ra đồng thời (xem ví dụ 2: ngay khi tháp 1 nhận được tin, thông tin được truyền đến cả tháp 2 và 4 -$>$ thời gian tháp 2 nhận được tin là 2). Tuy nhiên nhân viên truyền tin ở tháp i chỉ có thể truyền thông tin tới không quá s[i] tháp khác. Thời gian để truyền tin giữa 2 tháp bằng khoảng cách giữa 2 tháp đó. Tại thời điểm 0, thông tin bắt đầu được truyền từ tháp 1. Tính thời gian để mỗi tháp nhận được thông tin.

\subsubsection{Input}

Dòng đầu tiên: N (1 $<$= N $<$= 100) là số lượng tháp. Các tháp đánh số từ 1 đến N.

N dòng tiếp theo, dòng thứ i gồm:
\begin{itemize}
	\item Số nguyên X và Y (1$<$=X,Y$<$=1000) là tọa độ của tháp trong hệ toạ độ
	\item Số nguyên S (1$<$=S$<$=100) là số tháp mà tháp đó có thể truyền tin đi
	\item N-1 số nguyên phân biệt trong khoảng 1 đến N, là danh sách các tháp mà nhân viên ở tháp i được yêu cầu truyền tin. Nhân viên truyền tin ở tháp này sẽ phải lần lượt truyền thông tin đi theo thứ tự trong danh sách. Không có 2 số nào trùng nhau trong danh sách, và trong danh sách thứ i không chứa số i
\end{itemize}

Dữ liệu đảm bảo không có 2 tháp nào nhận được tin tại cùng 1 thời điểm.

\subsubsection{Output}

Gồm N dòng, mỗi dòng chứa một số thực. Dòng thứ i chứa thời gian mà tháp thứ i nhận được thông tin. Kết quả của bạn sẽ được tính là chính xác nếu kết quả sai khác không quá 0.01 so với đáp án.

\subsubsection{Example}
\begin{verbatim}
\textbf{Input:}
4
1 1 1 2 3 4
1 2 1 4 1 3
2 1 1 2 1 4
2 2 1 3 2 1


\textbf{Output:}
0.000000
1.000000
3.000000
2.000000

\end{verbatim}
\begin{verbatim}
\textbf{Input:}
5
4 3 2 5 2 4 3
4 5 1 4 1 5 3
4 4 1 1 4 5 2
2 4 1 5 2 3 1
3 4 2 2 4 3 1


\textbf{Output:}
0.000000
2.000000
4.414214
2.414214
1.414214

\end{verbatim}

\subsubsection{ }

 