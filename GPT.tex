



   Tam giác Pascal là một cách sắp xếp hình học của các hệ số nhị thức vào một tam giác. Hàng thứ n (n ≥ 0) của tam giác bao gồm các hệ số trong khai triển của đa thức f(x,y) = (x + y)   $^    n   $   . Hay nói cách khác, phần tử tại cột thứ k, hàng thứ n của Tam giác Pascal là C(n, k), tức tổ hợp chập k của tập n phần tử (với 0 ≤ k ≤ n).  

   Dưới đây là hình vẽ thể hiện các hàng từ 0 đến 16 của Tam giác Pascal:  


\includegraphics{http://vn.spoj.pl/content/pascal.jpg}



   Cho số tự nhiên n. Hãy tính GPT(n) là ước chung lớn nhất của các số nằm giữa hai số 1 trên hàng thứ n của Tam giác Pascal.  

\subsubsection{   Input  }

   Dòng đầu ghi T là số lượng Test. T dòng tiếp theo, mỗi dòng ghi một số nguyên n.  

\textbf{    Output   }

   Gồm T dòng, mỗi dòng ghi GPT(n) tương ứng.  

\subsubsection{\textbf{    Giới hạn   }}

   -       1 ≤ T ≤ 20.  

   -       2 ≤ n ≤ 10   $^    9   $   .  

\subsubsection{   Example  }
\begin{verbatim}
\textbf{Input:}

5

2

3

4

56

\textbf{Output:}

2

3

2

51\end{verbatim}