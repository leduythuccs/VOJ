

 

Một cách chia hình chữ nhật đúng được định nghĩa như sau:
\begin{enumerate}
	\item Không chia gì cả
	\item Chia hình chữ nhật thành 2 nữa trên và dưới. 2 nửa này lại được chia tiếp
	\item Chia hình chữ nhật thành 2 nửa trái và phải. 2 nửa này lại được chia tiếp
\end{enumerate}

Cho 2 cách chia hình chữ nhật A và B, 2 hình chữ nhật này có cùng kích thước.
\begin{itemize}
	\item Tìm cách chia đúng C sử dụng ít lần chia nhất, sao cho mỗi đường chia của A và B đều có trong cách chia C
	\item Tìm cách chia đúng D sử dụng nhiều lần chia nhất, sao cho mỗi đường chia của D đều có trong cả A và B
\end{itemize}

\subsubsection{Input}

Gồm nhiều bộ test, mỗi bộ: Dòng 1: M, N (1$<$=M,N$<$=20) lần lượt là số cột và số hàng của hình chữ nhật ban đầu N+1 dòng tiếp, mỗi dòng 4*M + 3 kí tự mô tả cách chia (các cách chia đều chính xác và có dạng hoàn toàn giống phía dưới) . Riêng dòng đầu tiên của mỗi mô tả chỉ có 4*M + 2 kí tự.

\subsubsection{Output}

Với mỗi test in ra : Dòng 1: "Case X", trong đó X là số thứ tự của bộ test, đánh số từ 1. N+1 dòng tiếp: 2 cách chia giống dạng như input

\subsubsection{Example}
\begin{verbatim}
Input:
4 3
 _ _ _ _   _ _ _ _ 
|_ _ _ _| |_|_ _ _|
|   |   | |       |
|_ _|_ _| |_ _ _ _|
3 4
 _ _ _   _ _ _ 
| |   | | |   |
| |   | |_|_ _|
|_|_ _| |   | |
|_ _|_| |_ _|_|
0 0

Output:
Case 1:
 _ _ _ _   _ _ _ _ 
|_|_ _ _| |_ _ _ _|
|   |   | |       |
|_ _|_ _| |_ _ _ _|
Case 2:
 _ _ _   _ _ _ 
| |   | |     |
|_|_ _| |     |
|_|_|_| |     |
|_ _|_| |_ _ _|

\end{verbatim}

Note: unofficial test data