



   Với mục đích cung cấp điện cho một khu dân cư mới, có N cột điện đã được dựng lên. Tất cả N cột điện được dựng trên một đường thẳng (thẳng hàng với trạm điện). Vị trí của mỗi cột điện được thể hiện bởi một số nguyên chính bằng khoảng cách từ trạm điện đến cột đó.  

   Để phù hợp với mĩ quan đô thị, công ti điện lực quyết định di chuyển một số cột điện sao cho đảm bảo điều kiện khoảng cách giữa hai cột điện liên tiếp nhỏ nhất là K (mét) và lớn nhất là P (mét). Việc di chuyển một cột điện là công việc nặng nhọc và tốn kém. Vì vậy, công ti điện lực muốn số lượng cột điện cần phải di chuyển là ít nhất.  

\section{   Yêu cầu  }

   Xác định số lượng nhỏ nhất các cột điện cần di chuyển.  

\section{   Input  }
\begin{itemize}
	\item     Dòng 1: Ghi 3 số nguyên dương N, K, P.   
	\item     Dòng 2: Ghi N số nguyên x1, x2, ..., xn trong đó xi là vị trí của cột điện i.   
\end{itemize}

\section{   Output  }
\begin{itemize}
	\item     Ghi một số nguyên là số lượng ít nhất các cột điện cần di chuyển.   
\end{itemize}

\section{   Giới hạn  }
\begin{itemize}
	\item     1 $<$= N $<$= 2500   
	\item     0 $<$= x1 $<$ x2 $<$ ... $<$ xn $<$= 2500; xi nguyên   
	\item     1 $<$= K $<$= 10   
	\item     K $<$= P $<$= 50   
\end{itemize}

\section{   Ví dụ  }
\begin{verbatim}
Input
3 3 5
0 1 2

Output
2
\end{verbatim}