

Cho dãy số nguyên a 1 , a 2 , ..., a N và số nguyên M. Xét thuật toán đưa M số nhỏ nhất về đầu dãy như sau

Cho dãy số nguyên a 1 , a 2 , ..., a N và số nguyên M. Xét thuật toán đưa M số nhỏ nhất về đầu dãy như sau:
\begin{verbatim}
for i$<$-1 to M do
  for j$<$-i+1 to N do
    if a[i]$>$a[j] then
      swap(a[i],a[j])\end{verbatim}

Trong đó swap là hàm đổi giá trị của 2 biến cho nhau.

Cho số nguyên M và dãy số nguyên a 1 , a 2 , ..., a N . Hãy đếm số lần thực hiện hàm swap sau khi

thuật toán kết thúc.

 

\subsubsection{Input}

Gồm nhiều bộ dữ liệu. Với mỗi bộ dữ liệu:

Dòng 1: Hai số nguyên dương N và M với 0 $<$ N, M  $\le$  10\textasciicircum5 .

Dòng 2: N số nguyên thể hiện dãy a, mỗi số cách nhau ít nhất 1 dấu cách. -10\textasciicircum9 ≤ a i ≤ 10\textasciicircum9.

\subsubsection{Output}

Ứng với mỗi test, in ra 1 dòng duy nhất chứa số lần thực hiện hàm swap theo yêu cầu bài toán

\subsubsection{Example}
\begin{verbatim}
\textbf{Input:}
3 3
2 1 3
4 1
3 2 -1 -10

\textbf{Output:}
1
3\end{verbatim}
