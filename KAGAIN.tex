



   Lại nói về Lục Vân Tiên , sau khi vượt qua vòng loại để trở thành Tráng Sỹ , anh đã gặp được Đôrêmon và được chú mèo máy cho đi quá giang về thế kỷ 19 . Trở lại quê hương sau nhiều năm xa cách , với tấm bằng Tráng Sỹ hạng 1 do Liên Đoàn Type Thuật cấp , anh đã được Đức Vua cử làm đại tướng thống lãnh 3 quân chống lại giặc Ô Qua xâm lăng . Đoàn quân của anh sẽ gồm N đại đội , đại đội i có A[i] ( $>$ 0 ) người . Quân sỹ trong 1 đại đội sẽ đứng thành 1 cột từ người 1 -$>$ người A[i] , như vậy binh sỹ sẽ đứng thành N cột . Vì Vân Tiên quyết 1 trận sẽ đánh bại quân Ô Qua nên đã cử ra 1 quân đoàn hùng mạnh nhất . Trong sử cũ chép rằng , quân đoàn của Vân Tiên cử ra lúc đó là một nhóm các đại đội có chỉ số liên tiếp nhau ( tức là đại đội i , i + 1 , … j ) . Vì sử sách thì mối mọt hết cả nên chỉ biết được mỗi thế . Ngoài ra theo giang hồ đồn đại thì sức mạnh của 1 quân đoàn = số người của đại đội ít người nhất * số đại đội được chọn . Nhiệm vụ của bạn là dựa trên các thông số của các nhà khảo cổ có được , hãy cho biết quân đoàn mà Vân Tiên đã chọn ra là từ đại đội nào đến đại đội nào . Chú ý nếu có nhiều phương án thì ghi ra phương án mà chỉ số của đại đội đầu tiên được chọn là nhỏ nhất .  

\subsubsection{   Bài này O(N) mới thực sự coi là accept . Còn lại O(NlogN) , O(N^2) thì đó là do bạn may mắn accept thôi.  }

\subsubsection{   Input  }

   Dòng 1 : Số T là số bộ test .   
\\   T nhóm dòng tiếp theo , mỗi nhóm dòng mô tả 1 bộ test . Nhóm dòng thứ i :   
\\   Dòng 1: N (  $\le$  30000 )   
\\   Dòng 2: N số nguyên mô tả N số A[1] , A[2] , … A[N] ( các số nguyên dương  $\le$  30000 ).   
\\   .  

\subsubsection{   Output  }

   Kết quả mỗi test ghi ra trên 1 dòng , gồm 3 số : sức mạnh quân đoàn mạnh nhất , chỉ số của đại đội đầu tiên và chỉ số của đại đội cuối cùng được chọn .  

\subsubsection{   Example  }
\begin{verbatim}
Input:
2
4
3 4 3 1
4
1 2 1 3

Output:
9 1 3
4 1 4
\end{verbatim}
