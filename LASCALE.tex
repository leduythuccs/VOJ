



   Cửa hàng của duyhung123abc có một cái cân và  các quả cân có khối lượng có dạng 3   $^    k   $   (tức lũy thừa của 3). VD: 1, 3, 9, 27, 81, ...  

   Khối lượng của các quả cân khác nhau từng đôi một. Duyhung123abc có một vật nặng M kg, vật nặng được đặt vào đĩa bên trái của cái cân. Hãy giúp anh ta đặt các quả cân vào 2 đĩa sao cho cân thãng bằng  

\subsubsection{   Input  }

   - Chứa 1 số nguyên M duy nhất (0  $\le$  M  $\le$  100 000 000)  

\subsubsection{   Output  }

   - Kết quả gồm 2 dòng  

   - Dòng 1: số A là số quả cân đặt vào đĩa bên trái, theo sau gồm A số là khối lượng của các quả cân theo thứ tự tăng dần  

   - Dòng 2: số B là số quả cân đặt vào đĩa bên phải, theo sau gồm B số là khối lượng của các quả cân theo thứ tự tăng dần  

\subsubsection{   Example  }
\begin{verbatim}
\textbf{Input:}
\\42
\\
\\\textbf{Output:}
\\3 3 9 27
\\1 81
\\\end{verbatim}
