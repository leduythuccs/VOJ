



   Hồi còn bé sherry thường chơi với bố 1 trò chơi có tên gọi là White Black :D  

   Bố có 1 mảnh giấy HCN thật dài cỡ 1 x N chia thành N ô vuông bé dàn thành hàng ngang. Ban đầu ô vuông nào cũng có màu trắng. Quy tắc chơi sẽ là mỗi bước bố có thể tô màu 1 đoạn các ô vuông từ ô L đến ô R cùng 1 màu ( có thể là màu đen hoạc màu trắng ) 1 lúc sau tờ giấy sẽ có rất nhiều ô đen trắng đan xen nhau và câu hỏi của bố dành cho sherry là có bao nhiêu ô vuông màu trắng liên tiếp ( sao cho số lượng các ô này là nhiều nhất )  

   sherry cũng thông minh lắm nên hôm nào cũng thắng ( tuy nhiên sherry chơi hơi chậm 1 chút \textasciicircum\textasciicircum ) Sao bạn không thử tham gia trò chơi này nhỉ :D  

\subsubsection{   Input  }

   Dòng 1: N (1 $<$= N $<$= 10000)  

   Dòng 2: M (1 $<$= M $<$= 100000) ( tổng số lần tô màu và số lần bố đố sherry )  

   M dòng tiếp theo: Mỗi dòng có dạng:  

   1  L  R  (1 $<$= L $<$= R $<$= N) tô các ô vuông từ L -$>$ R màu trắng  

   2  L  R  (1 $<$= L $<$= R $<$= N) tô các ô vuông từ L -$>$ R màu đen  

   3 đếm số lượng ô màu trắng liên tiếp dài nhất  

\subsubsection{   Output  }

   Gồm 1 số dòng tương ứng với các câu trả lời của sherry cho câu hỏi của bố  

\subsubsection{   Example  }
\begin{verbatim}
Input:
6
7
2 1 2
2 4 5
3
1 3 4
3
1 1 1
3

Output:
1
2
2
\end{verbatim}