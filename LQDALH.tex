

Bài toán Alhazen về phản xạ của tia sáng từ mặt cầu là một bài toán kinh điển của quang học.
\\Bài toán được tóm tắt như sau:
\\Cho đường tròn bán kính R tâm ở gốc tọa độ .Đường tròn này phản xạ tia sáng có nguồn ở điểm (x1,y1) .Tia phản xạ đi qua điểm có tọa độ (x2,y2).Các điểm này đều nằm ngoài vòng tròn .
\\\textbf{Yêu cầu } : Hãy xác định điểm tới của tia sáng trên đường tròn với độ chính xác 4 chữ số sau dấu chấm thập phân .
\\
\includegraphics{http://upload.wikimedia.org/wikipedia/commons/thumb/d/d8/Problème_dAlhazen.svg/424px-Problème_dAlhazen.svg.png}\textbf{}

\textbf{Input }

Dòng đầu tiên chứa số nguyên t - số bộ test (1 $\le$ t $\le$ 300)
\\Mỗi bộ dữ liệu cho trên 3 dòng :
\begin{itemize}
	\item Dòng thứ nhất chứa số nguyên R (1 $\le$ R $\le$ 1000)
	\item Dòng thứ 2 chứa 2 số nguyên x1,y1
	\item Dòng thứ 3 chứa 2 số nguyên x2,y2
\end{itemize}

Tọa độ các điểm có giá trị tuyệt đối không vượt quá 10000
\\
\\\textbf{\textbf{Output }}

Kết quả mỗi test đưa trên một dòng dưới dạng 2 số thực
\\
\\
\\ 

\subsubsection{Example}
\begin{verbatim}
\textbf{Input:
}1
2
3 2
0 3

\textbf{Output:
}0.8411 1.8146


\end{verbatim}
