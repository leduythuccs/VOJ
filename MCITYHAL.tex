

Ma trận kích thước M hàng, N cột biểu diễn tường, 1-tường tốt, 0-tường hỏng. 
\begin{verbatim}

\texttt{1110000111
1100001111
1000000011
1111101111
1110000111 
}\end{verbatim}

Hình minh hoạ

Sửa = cách đặt các khối thẳng đứng vào các vùng hỏng. Các khối có thể sử dụng có độ rộng 1 và chiều cao có thể là \{1,2, ..., M\}. Cần xác định số khối từng loại sao cho số lượng khối là ít nhất.

\subsubsection{Input}Dòng đầu là hai số M và N (1 $<$= M, N $<$= 200). M dòng sau đó gồm N kí tự  1 hoặc 0. 

\subsubsection{Output}Xác định số khối cần sử dụng đối với từng chiều cao k Ck với k ∈ \{1,2, ..., M\} là chiều cao của khối và Ck là số khối cần sử dụng. Không in ra các dòng có Ck = 0 và in ra theo thứ tự tăng dần của k.
\begin{verbatim}
\textbf{Sample Input}
5 10 
1110000111
1100001111
1000000011
1111101111
1110000111\end{verbatim}
\begin{verbatim}
\textbf{Sample output}
1 7 
2 1 
3 2 
5 1 
\end{verbatim}