

Xét hai số n chữ số A và B không có số 0 ở đầu.

Cần tìm hai số có n chữ số gần A nhất, một số $>$= A và một số $<$ A mà gồm mọi chữ số của B theo một thứ tự nào đó.

Ví dụ:
\begin{itemize}
	\item Nếu A=3022 và B=1232
\begin{itemize}
	\item Các số thu được từ B là: 1223, 1232, 1322, 2123, 2132, 2213, 2231, 2312, 2321, 3122, 3212 và 3221
	\item Số nhỏ nhất $>$= A là 3122, và số lớn nhất $<$ A là 2321.
\end{itemize}
	\item Nếu A=1232 và B=3022
\begin{itemize}
	\item Các số thu được từ B là 2023, 2032, 2203, 2230, 2302, 2320, 3022, 3202 và 3220.
	\item Số nhỏ nhất $>$=A là 2023, và không có số nào $<$ A.
\end{itemize}
\end{itemize}

Cho A, B, tìm 2 số gần nhất A như trên.

\subsubsection{INPUT}

Gồm hai dòng là hai số n chữ số A, B tương ứng (1 ≤ n ≤ 60).

\subsubsection{OUTPUT}
\begin{itemize}
	\item Dòng 1:  Số nhỏ nhất $>$= A theo định nghĩa trên, không có số 0 ở đầu. Nếu không tồn tại, in ra 0.
	\item Dòng 2: số lớn nhất $<$ A theo định nghĩa trên, không có số 0 ở đầu. Nếu không tồn tại, in ra 0.
\end{itemize}
\begin{verbatim}
SAMPLE INPUT
Ví dụ 1          Ví dụ 2
3075             3000203 
6604             4562454\end{verbatim}
\begin{verbatim}
SAMPLE OUTPUT
Ví dụ 1        Ví dụ 2
4066           4244556 
0              2655444 
\end{verbatim}

Problem for kid - Please, think like kid.