







   Asen và Boyan cùng chơi trò chơi với các đồng xu sau. Chúng chọn 2 số nguyên dương K và L khác nhau và chơi trò chơi với 1 tháp gồm N xu. Asen luôn chơi trước, tiếp theo là Boyan, sau đó lại là Asen ...  

   Mỗi lần chơi, chúng có thể lấy đi 1, K hoặc L đồng xu. Người thực hiện lượt đi cuối cùng là người chiến thắng. Asen  nhận thấy có những trường hợp nó luôn có thể thắng, và các trường hợp mà Boyan có thể thắng bất kể nó chơi   
\\   như nào. Do đó, trước khi bắt đầu chơi, Asen muốn biết kết quả có thể của lần chơi. Hãy viết 1 chương trình dự đoán kết quả với K, L và N cho trước.  

\subsubsection{   INPUT  }

   Dữ liệu vào mô tả m lần chơi. Dòng đầu tiên gồm các số  K, L và m, 1 $<$ K $<$ L $<$ 10, 3 $<$ m $<$ 50. Dòng thứ hai gồm m số nguyên N1, N2, …, Nm, 1 ≤ Ni ≤ 1000 000,  i = 1, 2, …., m, là số đồng xu trong từng lần chơi.  
\begin{verbatim}
SAMPLE INPUT
\\2 3 5 
\\3 12 113 25714 88888\end{verbatim}

\subsubsection{   OUTPUT  }

   Hiển thị một xâu m kí tự 'A' và 'B', 'A' nếu Asen thắng và 'B' nếu ngược lại trong lần chơi thứ i.  
\begin{verbatim}

\\SAMPLE OUTPUT
\\ABAAB
\\\end{verbatim}

\textbf{    Problem for kid - Please, think like kid.   }
\\\textbf{    Cơ bản về thuật toán trò chơi : Bạn ở vị trí thắng nếu có 1 cách đi từ vị trí này đến vị trí thua.   }
\\\textbf{    Lập bảng phương án các khả năng có thể - sau đó QHĐ hay duyệt   }

