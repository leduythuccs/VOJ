

                     Chú ý : C,S,B được sử dụng thay cho H,F,I, trong các test     

   Hal và Dave chơi game trên bảng RxC. Quy tắc như sau:  

   • Thay nhau di chuyển 1 quân cờ.  

   • Quân cờ chỉ di chuyển theo quy tắc : xuông dưới hoặc sang phải  hoặc xuống dưới và sang phải.  

   • Một số ô cấm, không thể di chuyển vào ô đó.  

   • Một ô có thể có 1 trong 3 loại đồ vật H,F,I. Đi vào ô chứa  H thu được 1 điểm, F-3 điểm và I-5 điểm.  

   • Game kết thúc khi không di chuyển được quân cờ nữa (ra ngoài  bảng hoặc gặp toàn ô cấm).  

   • Nếu 2 người bằng điểm thì người không đi được nữa thua.  

   • Nếu điểm khác nhau thì ai nhiều điểm hơn thắng.  

   • Điểm ban đầu của 2 người là 0. Hal đi trước. Vị trí ô xuất phát  là bất kỳ và không chứa H, F, I.  

   Cho một bảng và một số vị trí xuất phát, xác định người thắng.  

\subsubsection{   Input  }

   Dòng đầu gồm hai số R, C, (2 ≤ R ≤ 100), (2 ≤ C ≤ 100) , số hàng, cột. Sau đó là R dòng C kí tự.Ô cấm là '\#', ô có đồ vật là H hoặc F hoặc I.  Ô rỗng là '.'  

   Tiếp theo là N, 1 ≤ N ≤ 100, số vị trí xuất phát có thể. N dòng tiếp theo mỗi dòng hai số A (1 ≤ A ≤ R) và B (1 ≤ B ≤ C),  hàng, cột của vị trí xuất phát thứ i.  Hàng, cột đánh số từ 1.  

\subsubsection{   Output  }

   In ra N dòng, dòng i là tên người chiến thắng ở lần chơi thứ i.  

\subsubsection{   Sample  }
\begin{verbatim}
GAME.IN

3 4
.H#.
I...
##H.
3
1 1
1 4
2 3

GAME.OUT

HAL
DAVE
HAL

GAME.IN

4 5
.#...
#.#.F
.#..F
.#...
3
3 1
3 3
1 5

GAME.OUT

HAL
HAL
HAL

GAME.IN

5 6
##..#.
..#FH#
..#..#
###...
.....I
4
2 1
5 1
1 4
1 6

GAME.OUT

HAL
HAL
DAVE
DAVE

\end{verbatim}