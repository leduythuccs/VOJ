



   Xét số nguyên dương X và gọi S là tổng tất cả các ước dương của 2004\textasciicircumX .  

   Cần tính phần dư của S cho 29. Ví dụ, với X=1, các ước dương của 2004\textasciicircum1 là 1, 2, 3, 4, 6, 12, 167, 334, 501, 668, 1002 và 2004. Do đó S = 4704  và số dư của S chia cho 29 là 6.  

       Input     

   Gồm nhiều bộ test, mỗi bộ là một số nguyên X (1 $<$= X $<$= 10000000).  

   Bộ test với X = 0 để kết thúc chương trình và không cần xử lý.  
\begin{verbatim}
Sample Input
1 
10000 
0
\end{verbatim}     Output    



   Với mỗi bộ test, in ra một kết quả của số dư S chia cho 29 trên 1 dòng.  
\begin{verbatim}
Sample Input
6 
10 
\end{verbatim}