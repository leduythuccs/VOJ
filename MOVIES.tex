



\subsubsection{   Đề bài  }

   Bờm và Cuội đi xem phim trong một rạp có n hàng ghế, mỗi hàng có m ghế. Hàng đánh số từ 1..n từ đầu đến cuối rạp, và ghế đánh số 1..m từ trái sang phải. Một số ghế đã có người mua vé.  

   Tính số cách để Bờm và Cuội có thể mua 2 ghế cạnh nhau trên cùng một hàng.  

\subsubsection{   Dữ liệu  }
\begin{itemize}
	\item     Mỗi test bắt đầu bằng thẻ "[CASE]", các test cách nhau bởi một dòng trắng. Thẻ "[END]" báo hiệu kết thúc file input.   
	\item     Với mỗi test, dòng đầu tiên chứa số n, dòng thứ hai chứa số m.   
	\item     Tiếp theo là dòng "$<$$<$".   
	\item     Các dòng tiếp theo chứa số hiệu hàng của các ghế đã có người ngồi   
	\item     Kết thúc bằng dòng "$>$$>$'.   
	\item     Tiếp theo là dòng "$<$$<$".   
	\item     Các dòng tiếp theo chứa số hiệu ghế của các ghế đã có người ngồi   
	\item     Kết thúc bằng dòng "$>$$>$'.   
\end{itemize}

\subsubsection{   Kết quả  }
\begin{itemize}
	\item     Với mỗi test in ra số lượng cách để Bờm và Cuội mua được hai vé ngồi cạnh nhau.   
\end{itemize}

\subsubsection{   Giới hạn  }
\begin{itemize}
	\item     1  $\le$  n  $\le$  m  $\le$  10^9   
	\item     Số ghế đã có người ngồi nằm trong phạm vi từ 1..47.   
\end{itemize}

\subsubsection{   Ví dụ  }
\begin{verbatim}
Dữ liệu
[CASE]
2
3
$<$$<$
1
2
$>$$>$
$<$$<$
2
3
$>$$>$

[CASE]
2
3
$<$$<$
1
1
1
2
2
2
$>$$>$
$<$$<$
1
2
3
1
2
3
$>$$>$

[END]
Kết quả
1
0
\end{verbatim}

   Ví dụ 1: Ghế 1 và 2 ở hàng 2 là hai ghế cạnh nhau duy nhất còn trống.  

   Ví dụ 2: Tất cả các ghế đều đã có người ngồi.  
