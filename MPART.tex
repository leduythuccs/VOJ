



   Phân người vào các nhóm sao cho nhóm có nhiều người nhất có số lượng người ít nhất có thể.  

\subsubsection{   Input  }

   Có không quá 20 test. Dòng đầu mỗi test là hai số N,M : N số người, M số nhóm.  

   N dòng tiếp theo, đầu tiên là tên người, sau đó là danh sách các nhóm mà người đó có thể được phân vào (1$<$=N$<$=1000,M$<$=500).  

   Tên người chỉ gồm kí tự chữ cái và có độ dài  $<$=15 , không có 2 người trùng tên. Mã nhóm đánh số từ 0 cho đến M-1.  

   Kết thúc test là hai số 0 0.       Sample Input    
\begin{verbatim}
3 2 
John 0 1 
Rose 1 
Mary 1 
5 4 
ACM 1 2 3 
ICPC 0 1  
Asian 0 2 3 
Regional 1 2 
ShangHai 0 2 
0 0 
\end{verbatim}

\subsubsection{   Output  }

   Hiện ra số người của nhóm mà nhiều người nhất mà thỏa mãn điều kiện trên.       Sample output    
\begin{verbatim}
2
2
\end{verbatim}