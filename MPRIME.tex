



   Xét dãy A các số nguyên tố  

   2, 3, 5, 7, 11, 13, 17, 19,...  

   và dãy B gồm các số thu được từ dãy A bằng cách ghép hai số liên tiếp trong A:  

   23, 57, 1113, 1719, ...  

   Trong dãy B có những phần tử là số nguyên tố. Chẳng hạn 23, 3137, 8389, 157163...  

   Các số nguyên tố trong dãy B gọi là   \textit{    số nguyên tố ghép.   }

   Yêu cầu: Cho trước số nguyên dương K ≤ 500, hãy tìm số nguyên tố ghép thứ K.  

\subsubsection{   Dữ liệu  }

   Gồm 1 số nguyên dương K duy nhất.  

\subsubsection{   Kết qủa  }

   In ra 1 số nguyên dương duy nhất là số nguyên tố ghép thứ K.  

\subsubsection{   Ví dụ  }
\begin{verbatim}
Dữ liệu:
2

Kết qủa
3137
\end{verbatim}