

Dãy Recaman được định nghĩa như sau : a0 = 0 ; với m $>$ 0, a(m) = a(m−1) − m nếu a(m) là dương và chưa xuất hiện trong dãy, ngược lại a(m) = a(m−1) + m. Một số phần tử đầu tiên của dãy là 0, 1, 3, 6, 2, 7, 13, 20, 12, 21, 11, 22,10, 23, 9 · · · .

Cho k, tính ak.

\subsubsection{Input}

Gồm vài test case, mỗi dòng chứa một số nguyên k. (0$<$=k$<$=500000). Kết thúc là số -1.

\subsubsection{Output}

In ra ak trên 1 dòng.
\begin{verbatim}
\textbf{Sample Input}
7
10000
-1
\textbf{Sample output}
20
18658
\end{verbatim}