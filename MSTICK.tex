

Có n đoạn gỗ. Để xử lý chúng cần thời gian để chuẩn bị :

(a) Thời gian chuẩn bị cho đoạn gỗ đầu tiên là 1 phút.

(b) Sau khi xử lý xong đoạn gỗ có chiều dài l và trọng lượng w, không mất thời gian xử lý nếu đoạn gỗ tiếp theo có độ dài l' và trọng lượng w' thỏa l ≤ l' and w ≤ w'. Ngược lại mất 1 phút để chuẩn bị.

 

Tìm thời gian chuẩn bị ít nhất cho n đoạn gỗ.

Ví dụ có 5 đoạn ( 9 , 4 ) , ( 2 , 5 ) , ( 1 , 2 ) , ( 5 , 3 ) , và ( 4 , 1 ), thì thời gian ít nhất là 2 vì có thể xử lý theo thứ tự như sau ( 4 , 1 ) , ( 5 , 3 ) , ( 9 , 4 ) , ( 1 , 2 ) , ( 2 , 5 ) .

\subsubsection{Input}

Dòng đầu là số lượng test, T. Mỗi test gồm 2 dòng:
\begin{itemize}
	\item Dòng đầu là số n , 1 $<$= n $<$= 5000 ,
	\item Dòng thứ hai gồm 2n số nguyên dương l1 , w1 , l2 , w2 ,..., ln , wn $<$= 10000, li và wi là độ dài và trọng lượng của đoạn gỗ thứ i. 
\end{itemize}

 

\subsubsection{Output}
\begin{verbatim}
Ghi ra thời gian ít nhất trên từng dòng.
​
SAMPLE INPUT
3 
5 
4 9 5 2 2 1 3 5 1 4 
3 
2 2 1 1 2 2 
3 
1 3 2 2 3 1

SAMPLE OUTPUT
2
1
3\end{verbatim}