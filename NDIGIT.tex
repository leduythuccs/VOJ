

Số A(K) là dãy thu được bằng cách ghép liên tiếp các số 1^K, 2^K, 3^K, ... Số nhỏ hơn ở phía sau (bên phải).

Với K = 1, A(K) = ...181716151413121110987654321.

Với K = 2, A(K) = ...169144121100816449362516941.

Xét tổng S = A(1) + A(2). Đoạn cuối của S là: ...350860272513937560350171262

Cho N, K1, K2, hãy tìm chữ số thứ N tính từ phải sang của tổng S = A(K1) + A(K2) (số ngòai cùng bên phải của tổng S được tính là chữ số thứ 1)

\subsubsection{Dữ liệu}

Mỗi input gồm 3 test, mỗi test được ghi trên 1 dòng gồm 3 số N K1 K2. 1 ≤ K1, K2 ≤ 5. 1 ≤ N ≤ 1,000,000,000

\subsubsection{Kết quả}

In ra 3 đáp số tương ứng.

\subsubsection{Chấm điểm}

Nếu đúng 3/3 bạn được 5 điểm.

Nếu đúng 2/3 bạn được 3 điểm.

Nếu đúng 1/3 bạn được 1 điểm.

Ngoài ra, bạn không được điểm.

\subsubsection{Ví dụ}
\begin{verbatim}
Dữ liệu
1 1 2
3 1 2
5 1 2

Kết quả
2
2
7
\end{verbatim}
