







   Mirko và Slavko chơi trò các con thú đồ chơi. Đầu tiên, Mirko và Slavko chọn một trong 3 bàn cờ như hình dưới đây. Mỗi bàn cờ bao gồm các ô (dưới dạng hình tròn trong hình vẽ) sắp xếp trên một lưới 1, 2 hoặc 3 chiều.  


\includegraphics{http://www.spoj.com/content/paulmcvn:pairs.jpg}

   Sau đó Mirko sẽ đặt N con thú đồ chơi lên các ô.  

   Khoảng cách giữa 2 ô là số bước đi nhỏ nhất để một con thú đi từ ô này đến ô kia. Trong mỗi bước đi. con thú có thể bước đến 1 trong 4 ô kề với nó (nối với nhau bằng đoạn thẳng trong hình vẽ).  

   Hai con thú có thể nghe thấy nhau nếu khoảng cách giữa 2 ô chúng đứng không vượt quá D. Nhiệm vụ của Slavko là tính số cặp con thú có thể nghe thấy nhau.  

\subsubsection{   Dữ liệu  }

   Dòng đầu tiên chứa 4 số nguyên dương theo thứ tự:  
\begin{itemize}
	\item     Loại bàn cờ B (1 ≤ B ≤ 3).   
	\item     Số con thú N (1 ≤ N ≤ 100000).   
	\item     Khoảng cách lớn nhất D mà hai con thú có thể nghe thấy nhau (1 ≤ D ≤ 100000000).   
	\item     Kích thước bàn cờ M (tọa độ lớn nhất xuất hiện trong dữ liệu).    
\begin{itemize}
	\item       Khi B=1, M không vượt quá 75000000.     
	\item       Khi B=2, M không vượt quá 75000.     
	\item       Khi B=3, M không vượt quá 75.     
\end{itemize}
\end{itemize}

   Mỗi dòng trong số N dòng sau chứa B số nguyên cách nhau bởi khoảng trắng, cho biết các tọa độ của một con thú đồ chơi. Mỗi tọa độ sẽ thuộc phạm vi [1, M]. Có thể có nhiều con thú nằm trên cùng 1 ô.  

\subsubsection{   Kết qủa  }

   Gồm 1 số nguyên duy nhất là số lượng con thú có thể nghe thấy nhau.  

   Lưu ý: sử dụng số nguyên 64-bit để tính kết quả (long long trong C/C++, int64 trong Pascal).  

\subsubsection{   Hạn chế  }



\subsubsection{   Ví dụ  }
\begin{verbatim}
\textbf{Dữ liệu:}
1 6 5 100
25
50
50
10
20
23

\textbf{Kết qủa}
4

\textbf{Dữ liệu:}
2 5 4 10
5 2
7 2
8 4
6 5
4 4

\textbf{Kết qủa}
8

\textbf{Dữ liệu:}
3 8 10 20
10 10 10
10 10 20
10 20 10
10 20 20
20 10 10
20 10 20
20 20 10
20 20 20

\textbf{Kết qủa}
12
\end{verbatim}

