



   Các trưởng đoàn đội tuyển tin học vùng Balkan muốn chọn ra những thí sinh mạnh nhất trong khu vực từ N thí sinh (3 ≤ N ≤ 100000). Các trưởng   đoàn tổ chức 3 kỳ thi, mỗi thí sinh sẽ tham dự cả 3. Biết rằng không có 2 thí sinh nào có cùng điểm số trong mỗi kỳ thi. Ta nói thí sinh A       giỏi   hơn      thí sinh B nếu A được xếp hạng trước B trong cả 3 kỳ thi. Một thí sinh A được gọi là       xuất sắc      nếu như không có thí sinh nào giỏi hơn   A.  

   Yêu cầu: Hãy giúp các trưởng đoàn đếm số thí sinh xuất sắc.  

\subsubsection{   Dữ liệu vào  }
\begin{itemize}
	\item     Dòng thứ nhất chứa 1 số nguyên dương N.   
	\item     3 dòng sau, mỗi dòng chứa N số nguyên dương cách nhau bởi khoảng trắng, là chỉ số của các thí sinh theo thứ tự xếp hạng từ cao đến thấp của kỳ   thi tương ứng.   
\end{itemize}

\subsubsection{   Kết qủa  }

   Gồm 1 số nguyên duy nhất cho biết số thí sinh xuất sắc.  

\subsubsection{   Ví dụ  }
\begin{verbatim}
Dữ liệu mẫu
3 
2 3 1
3 1 2
1 2 3

Kết qủa
3
\end{verbatim}

   Không có thí sinh nào giỏi hơn thí sinh khác nên cả 3 thí sinh đều xuất sắc.  
\begin{verbatim}
Dữ liệu mẫu
10 
2 5 3 8 10 7 1 6 9 4
1 2 3 4 5 6 7 8 9 10
3 8 7 10 5 4 1 2 6 9

Kết qủa
4
\end{verbatim}

   Thí sinh 1, 2, 3, 5 là những thí sinh xuất sắc.  