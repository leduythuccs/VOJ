



    Kal El là 1 kỹ sư tài năng. Anh ta chế tạo ra chiếc máy in, nhưng vì vừa làm việc vừa ngủ nên chiếc máy in chế tạo ra chỉ có thể điều khiển bằng 3 lệnh:   

    1. Set(x): Đưa ký tự x vào bộ nhớ chính.   

    2. Next(x): Đưa ký tự x vào bộ nhớ phụ.   

    3. Write: In ra một ký tự. Nếu ngay trước đó là lệnh next thì sẽ in ra ký tự có trong bộ nhớ phụ, nếu không, in ký tự trong bộ nhớ chính.   

    Ví dụ: VNOI có thể dùng 8 lệnh:   

    Set(V),Write,Set(N),Write,Next(O),Write,Set(I),Write.   

    Cho trước xâu ký tự S chỉ gồm các chữ cái in hoa tiếng anh, cần biết số lệnh ít nhất dùng để in xâu S với điều kiện lệnh đầu tiên luôn phải là Set.   

\subsubsection{    Input   }

    1 dòng duy nhất chứa xâu S có độ dài không quá 5*10^6   

\subsubsection{    Output   }

    1 dòng duy nhất là số lệnh ít nhất cần dùng để máy in ra xâu S   

\subsubsection{    Example   }
\begin{verbatim}
\textbf{Input:}
  VNOI

\textbf{Output:}   8\end{verbatim}
