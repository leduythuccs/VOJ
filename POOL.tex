



   Trong hệ trục tọa độ Oxy như hình vẽ, xét bàn bi-a hình chữ nhật có kích thước a x b (a, b chẵn). Bàn bi-a có 6 lỗ (mỗi lỗ có thể xem như một điểm), trong đó 4 lỗ đặt ở đúng 4 đỉnh của bàn, 2 lỗ còn lại được đặt ở trung điểm cạnh nằm ngang. Coi quả bóng bi-a như một chất điểm. Khi đánh bóng vào băng bóng bật lại theo định luật phản xạ ánh sáng (góc phản xạ  b bằng góc tới  a như được mô tả trên hình vẽ). Bóng được xem như vào lỗ khi và chỉ khi tọa độ bóng trùng với tọa độ lỗ. Là một tay chơi nhà nghề, Thu có thể ngắm bóng vào bất kì một điểm có tọa độ nguyên nào trên bàn. Để trình diễn trước bạn bè, Thu muốn đánh bóng vào lỗ sau khi đập băng nhiều lần nhất có thể. Bạn hãy tính tọa độ của điểm mà Thu nên ngắm vào.  


\includegraphics{http://vn.spoj.com/content/POOL}

\subsubsection{   Dữ liệu  }
\begin{itemize}
	\item     Dòng thứ nhất ghi 2 số nguyên a và b, a và b chẵn.   
	\item     Dòng thứ hai ghi 2 số nguyên x và y là tọa độ ban đầu của bóng ( 0 ≤ x ≤ a ; 0 ≤ y ≤ b ).   
\end{itemize}

\subsubsection{   Kết quả  }
\begin{itemize}
	\item     Dòng đầu ghi số lần đập băng nhiều nhất có thể.   
	\item     Dòng thứ hai ghi 2 số z và t là tọa độ của điểm mà Thu nên ngắm vào. Nếu có nhiều nghiệm thì ghi ra tọa độ có hoành độ z nhỏ nhất, nếu vẫn có nhiều nghiệm thì ghi ra nghiệm có tung độ t nhỏ nhất.   
\end{itemize}

\subsubsection{   Giới hạn  }
\begin{itemize}   Kích thước: 0 $<$ a, b ≤ 500   
	\item     Thời gian: 3 s/test   
	\item     Bộ nhớ: 1 MB   
\end{itemize}

\subsubsection{   Ví dụ  }
\begin{verbatim}
\textbf{Dữ liệu}
6 4
4 3	

\textbf{Kết quả}
5
0 1
\end{verbatim}