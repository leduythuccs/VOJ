



   Văn bản là một dãy gồm N từđánh số từ 1 đến N. Từ thứ i có độ dài là wi (i=1, 2,... N). Phân trang là một cách xếp lần lượt các từ của văn bản vào dãy các dòng, mỗi dòng cóđộ dài L, sao cho tổng độ dài của các từ trên cùng một dòng không vượt quá L.Ta gọi hệ số phạt của mỗi dòng trong cách phân trang là hiệu số (L-S), trong đóS là tổng độ dài của các từ xếp trên dòng đó. Hệ số phạt của cách phân trang là giá trị lớn nhất trong số các hệ số phạt của các dòng.Tìm cách phân trang với hệ số phạt nhỏ nhất.  

\subsubsection{   Input  }
\begin{itemize}
	\item     Dòng 1 chứa 2 số nguyên dương N, L (N$<$=6000,L$<$=1000)   
	\item     Dòng thứ i trong số N dòng tiếp theo chứa số nguyên dương wi (wi$<$=L), i=1, 2,.., N   
\end{itemize}

\subsubsection{   Output  }
\begin{itemize}
	\item     In ra hệ số phạt nhỏ nhất   
\end{itemize}

\subsubsection{   Ví dụ  }
\begin{verbatim}
Input:
4 5
3
2
2
4
Output:
2 
\end{verbatim}