

Nhân dịp tổ chức PreVOI ở Quảng Ninh, thầy Minh mời N bạn thí sinh, đánh số từ 1 đến N tham gia trò chơi "Dịch chuyển thức thời" trong khu du lịch của mình.

Mỗi thí sinh sẽ có một tấm ván toán học mượn được của giáo sư Ngô Bảo Châu. Trong mỗi bước dịch chuyển, tấm ván sẽ đưa thí sinh đang ở tọa độ x đến tọa độ x*p hoặc x/p (nếu x chia hết cho p). Ở đây p là số nguyên tố bất kỳ.

Ban đầu, N bạn thí sinh đứng ở các tọa độ nguyên dương x\_1, x\_2, ..., x\_N.

Thầy Minh muốn đặt một bàn ăn tại một tọa độ sao cho tổng số bước dịch chuyển của tất cả các bạn thí sinh từ vị trí ban đầu đến bàn ăn là nhỏ nhất.

\subsubsection{Input}
\begin{itemize}
	\item Dòng 1 chứa số nguyên dương N ≤ 100,000
	\item Dòng 2 chứa N số nguyên dương, số thứ i là x\_i ≤ 10,000.
\end{itemize}

Các số trên một dòng cách nhau bởi dấu cách.

\subsubsection{Output}

In ra 2 số nguyên cách nhau 1 dấu cách: số thứ nhất là tổng số bước di chuyển tối thiểu. Số thứ 2 là tọa độ thầy Minh đặt bàn ăn. Nếu có nhiều tọa độ thỏa mãn thì in ra tọa độ bế nhất.

\subsubsection{Example}
\begin{verbatim}
\textbf{Input:}
2
2 3

\textbf{Output:}
2 1\end{verbatim}