

Để chuẩn bị cho lễ bế mạc PreVOI 2014 hoành tráng, thầy Minh giao cho thầy Hải tổ chức một bàn tiệc to để chiêu đãi tất cả các thí sinh. L(m, n) là bàn ăn có một hình dạng đặc biệt được đặc trưng bởi 2 số nguyên 0 $<$= n $<$ m : đó là một lưới kích thước m*m với một góc phía trên bên phải bị mất một phần có kích thước n*n. Ví dụ, L(5, 3):
\begin{verbatim}

\texttt{x x
x x
x x
x x x x x
x x x x x}\end{verbatim}

Thầy Hải chuẩn bị K = m*m - n*n món ăn có độ hấp dẫn khác nhau và có giá trị từ 1 đến K.

Thầy Minh yêu cầu thầy Hải xếp mỗi món ăn vào một ô trên bàn ăn L(m, n) sao cho độ hấp dẫn của món ăn ở một ô nhỏ hơn độ hấp dẫn của món ăn ở ô bên dưới và ô bên trái ô đó.

Ví dụ, sau đây là 2 cách xếp món ăn trên bàn L(5, 3).
\begin{verbatim}

\texttt{ 6  4
 7  5
14  9
15 11  8  3  1
16 13 12 10  2}\end{verbatim}

Gọi LC(m, n) là số lượng cách xếp món ăn trên bàn L(m, n).

\textbf{Yêu cầu } : tính LC(m, n).

\subsubsection{Input}

Dòng 1 chứa 3 số nguyên dương m, n $<$= 1000 và p, là số nguyên tố có không quá 11 chữ số.

\subsubsection{Output}

In ra LC(m, n) modulo p.

\subsubsection{Example}
\begin{verbatim}
\textbf{Input:}
2 0 7

\textbf{Output:}
2\end{verbatim}
\begin{verbatim}
\textbf{Input:}
2 1 7

\textbf{Output:}
2\end{verbatim}