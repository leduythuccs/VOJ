



   Ngày nay khi nghiên cứu quan hệ giữa các phần tử các nhà khoa học không đơn giản chỉ nghiên cứu các quan hệ bình thường mà để thêm phần phức tạp là thêm vào đó 1 vài bộ điều kiện. Một trong những điều kiện đó là số quan hệ '='  

   Như ta đã biết giữa 2 phần tử a, b sẽ có 3 quan hệ:  

   a = b, a $>$ b, a $<$ b.  

   Các nhà khoa học đưa ra 1 bộ gồm n phần tử. Sau khi tìm ra số lượng các quan hệ của n phần tử này họ muốn biết nếu như số quan hệ '=' trong tập n phần tử này đúng bằng k thì sẽ có bao nhiêu quan hệ như thế?  

\subsubsection{   Input  }

   Gồm nhiều bộ số n, k. Mỗi bộ số trên 1 dòng. Kết thúc file là -1.  ( 1 $<$ n $<$ 11 )  

\subsubsection{   Output  }

   Với mỗi bộ số (n, k) đưa ra số quan hệ có điều kiện tìm được  

\subsubsection{   Example  }
\begin{verbatim}
Input:
3 0
3 1
3 2
3 3
-1



Output:
6
6
0
1

Giải thích:
 Với bộ 3 phần tử (a, b, c). 
n=3, k=0:
 a $<$ b $<$ c;   a $<$ c $<$ b;   b $<$ a $<$ c; 
 b $<$ c $<$ a;   c $<$ a $<$ b;   c $<$ b $<$ a;
n=3, k=1:
 a = b $<$ c;   c $<$ a = b;   a $<$ b = c
 b = c $<$ a;   a = c $<$ b;   b $<$ a = c;
n=3, k=3:
 a = b = c; 

\end{verbatim}