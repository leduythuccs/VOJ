



   Cho một bảng A kích thước m x n (1  $\le$  m, n  $\le$  100), trên đó ghi các số nguyên a   $_    ij   $   (|a   $_    ij   $   |  $\le$  100). Một người xuất phát tại ô nào đó của cột 1, cần sang cột n (tại ô nào cũng được).  

   Quy tắc đi: Từ ô (i, j) chỉ được quyền sang một trong 3 ô (i, j + 1); (i - 1, j + 1); (i + 1, j + 1)  

\subsubsection{   Input  }

   Dòng 1: Ghi hai số m, n là số hàng và số cột của bảng.  

   M dòng tiếp theo, dòng thứ i ghi đủ n số trên hàng i của bảng theo đúng thứ tự từ trái qua phải  

\subsubsection{   Output  }

   Gồm 1 dòng duy nhất ghi tổng lớn nhất tìm được  

\subsubsection{   Example  }
\begin{verbatim}
Input:
5 7
9 -2 6 2 1 3 4
0 -1 6 7 1 3 3
8 -2 8 2 5 3 2
1 -1 6 2 1 6 1
7 -2 6 2 1 3 7

Output:
41
\end{verbatim}
