



   Cho một đồ thị cây có N đỉnh N-1 cạnh, gốc của cây là đỉnh 1, mỗi đỉnh có một trọng số không âm là A   $_    i   $   . Dễ dàng nhận thấy, ngoại trừ đỉnh 1, các đỉnh còn lại đều có một đỉnh cha và nhận nó làm đỉnh con. Từ mảng A người ta tiến hành xây dựng mảng P như sau:  

   P   $_    u   $   =A   $_    u   $   nếu như đỉnh u đó không có đỉnh con, ngược lại P   $_    u   $   =A   $_    u   $   *max(P   $_    v1   $   , P   $_    v2   $   ...P   $_    vm   $   ) với v1,v2...vm lần lượt là các đỉnh con trực tiếp có cạnh nối với u. Nhiệm vụ của bạn là tính P   $_    1   $   .  

\subsubsection{   Input  }

   Dòng 1: Gồm một số nguyên N và M(1≤N≤10   $^    5   $   , 1≤M≤10   $^    18   $   ).  

   Dòng 2: Gồm N số nguyên A   $_    1   $   ... A   $_    N   $   với A   $_    i   $   là trọng số của đỉnh thứ i. (A   $_    i   $   ≤10   $^    18   $   ).  

   N-1 dòng tiếp theo, mỗi dòng gồm hai số nguyên u và v, thể hiện cạnh nối giữa u và v (1≤u,v≤N).  

\subsubsection{   Output  }

   Một số nguyên duy nhất là P   $_    1   $   , do kết quả có thể rất lớn nên chỉ cần in ra phần dư cho một số nguyên M  

\subsubsection{   Example  }
\begin{verbatim}
\textbf{Input:}

3 1000

1 2 3

1 2

1 3\textbf{Output:}
3\end{verbatim}