



   Một số nguyên dương được gọi là may mắn nếu tổng một số chữ số bằng tổng của các chữ số còn lại, ví dụ: 561743 sẽ là số may mắn vì 5 + 1 + 4 + 3 = 6 + 7. Tuy nhiên số may mắn không nhiều, nên người ta muốn đếm xem có bao nhiêu số không may mắn.  

\textbf{\emph{     Yêu cầu    }}   : Tính số lượng số không may mắn có   \textbf{\emph{     n    }}   chữ số và chỉ chứa các chữ số trong phạm vi từ 0 đến   \textbf{\emph{     k    }}   . Các số có thể bắt đầu bằng các số 0.  

\subsubsection{   Input  }

   Gồm nhiều dòng, mỗi dòng chứa 2 số   \textbf{\emph{     n    }}   và   \textbf{\emph{     k    }}   (1 ≤   \textbf{\emph{     n    }}   ≤ 20, 1 ≤   \textbf{\emph{     k    }}   ≤ 9, có không quá 5 dòng).  

\subsubsection{   Output  }

   Gồm nhiều dòng, mỗi dòng là kết quả tương ứng với dữ liệu vào.  

\subsubsection{   Example  }
\begin{verbatim}
\textbf{Input:}
1 5
\\4 3

\textbf{Output:}
5
\\164 \end{verbatim}