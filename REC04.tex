



    Cho N điểm trên mặt phẳng. Tìm 1 hình chữ nhật nhận 4 trong các điểm đó là 4 đỉnh và có diện tích lớn nhất.   

    Input:   
\begin{itemize}
	\item 

      Dòng  đầu tiên gồm số nguyên N: số lượng điểm trên mặt  phẳng     
	\item 

      N  dòng sau: mỗi dòng gồm 2 số nguyên x[i],y[i] là tọa độ  của điểm thứ i     
\end{itemize}

    Output:   
\begin{itemize}
	\item 

      Gồm  1 dòng duy nhất ghi 1 số nguyên dương là diện tích hình  chữ nhật thu được     
\end{itemize}



    Giới hạn:   
\begin{itemize}
	\item 

      4   $\le$  N  $\le$  1500     
	\item 

      -10^8   $\le$  x[i],y[i]  $\le$  10^8     
	\item 

      Không  có 2 điểm nào trùng nhau     
	\item 

      Dữ  liệu đảm bảo luôn tồn tại ít nhất 1 hình chữ nhật     
\end{itemize}



    Ví dụ:   

        Input       

        8       

        -2 3       

        -2 -1       

        0 3       

        0 -1       

        1 -1       

        2 1       

        -3 1       

        -2 1       

        Output       

        10       
