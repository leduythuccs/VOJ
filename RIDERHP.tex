



   Con đường Vạn Hoa dài                                                    km mà giáo sư X thường đi ngắm cảnh trong kỳ nghỉ đang vào mùa lễ hội, ngày nào cũng có     lễ hội trên đường đánh số từ 1 tới m. Lễ hội thứ i diễn ra tại điểm cách đầu đường i km và tiến hành từ đầu ngày (thời điểm 0) cho tới hết thời điểm t\_i trong ngày, trong thời gian lễ hội tổ chức không xe nào được đi qua điểm diễn ra lễ hội mà phải đợi tới khi lễ hội kết thúc mới được đi qua.  

   Giáo sư X không quan tâm lắm tới các lễ hội mà ông chỉ đam mê tốc độ trong khung cảnh thiên nhiên hoang dã, trong mỗi ngày đi dạo (bằng mô-tô) từ đầu tới cuối con đường Vạn Hoa, ông muốn tính toán xem mình có thể đi với tốc độ không đổi lớn nhấtbằng bao nhiêu mà không phải dừng lại chờ bất cứ lễ hội  nào.  

\textbf{    Yêu cầu:   }   Cho biết tốc độ tối đa có thể của giáo sư X trong mỗi ngày, biết rằng kỳ nghỉ của giáo sư diễn ra trong n ngày và vào ngày thứ j giáo sư bắt đầu đi vào thời điểm s\_j.  

\subsubsection{   Input  }
\begin{itemize}
	\item     Dòng 1 chứa số nguyên dương m  $\le$  10\textasciicircum5   
	\item     Dòng 2 chứa m số nguyên dương t\_1, t\_2, ...t\_m  $\le$  10\textasciicircum9 cách nhau ít nhất một dấu cách   
	\item     Dòng 3 chứa số nguyên dương n  $\le$  10\textasciicircum5   
	\item     Dòng 4 chứa     số nguyên không âm s\_1, s\_2, ..., s\_n cách nhau ít nhất một dấu cách, với mọi i, s\_i  $\le$  max(t\_j).   
\end{itemize}

\emph{    Ít nhất 50\% số điểm ứng với các test có M, N  $\le$  1000   }\emph{}

\subsubsection{   Output  }

   Ghi ra file văn bản n dòng, dòng thứ     ghi tốc độ tối đa (số km/1 đơn vị thời gian) của giáo sư trong ngày thứ     dưới dạng một số thực làm tròn lấy đúng 6 chữ số sau dấu chấm thập phân  

\subsubsection{   Example  }
\begin{verbatim}
\textbf{Input:}

4

3 5 6 1

30 3 5

\textbf{Output:}

0.333333

1.0000003.000000\end{verbatim}
