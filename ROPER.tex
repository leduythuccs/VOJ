

Cho 1 dãy số có N phần tử. Có 2 thao tác được phép sử dụng để biến đổi dãy số:
\begin{enumerate}
	\item 

Đổi chỗ 2 phần tử đầu tiên của dãy
	\item 

Cho 1 hoán vị P, di chuyển phần tử thứ nhất đến vị trí P[1], phần tử thứ 2 đến vị trí P[2],..
\end{enumerate}

Các thao tác được phép thực hiện với số lần không hạn chế và theo bất kì thứ tự nào.

Cho M truy vấn “u v”, bạn cần trả lời xem có thể có thể biến đổi hoán vị ban đầu sao cho phần tử thứ u có thể di chuyển đến vị tri v hay không?

\textbf{Input:}
\begin{itemize}
	\item 

Dòng 1: 2 số N và M (0 $<$ N,M  $\le$  10\textasciicircum5)
	\item 

Dòng 2: hoán vị P (1 hoán vị của \{1,2,…,N\})
	\item 

Dòng 3..M + 2: mỗi dòng chứa 2 số nguyên u và v (0 $<$ u,v  $\le$  N)
\end{itemize}

\textbf{Output:}
\begin{itemize}
	\item 

Với mỗi truy vấn, đưa ra “Yes” hoặc “No” tương ứng
\end{itemize}

\textbf{Example}
\begin{verbatim}
\textbf{Input}
4 2
1 4 3 2
4 1
3 4

\textbf{Output}
Yes
No\end{verbatim}

 

 
