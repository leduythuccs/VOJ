



   Sau khi đội tuyển CQB đã nắm khá chắc phần điểm và làm tốt các bài toán về điểm trong hình học. Thầy Thạch chuyển sang giải quyết các bài toán về hình tròn, dạng này thường rất có trong những cuộc thi lớn. Thầy ra cho các bạn 1 bài như sau:  

   Trên 1 tờ giấy thầy vẽ n đường tròn, sau đó thầy tô màu n hình tròn này.  

   Yêu cầu: Hãy tính diện tích của phần được tô màu.  

\subsubsection{   Input  }

   Dòng thứ nhất ghi số nguyên dương N là số hình tròn.  

   Dòng thứ i trong N dòng tiếp theo ghi 3 số nguyên X   $_    i   $   , Y   $_    i   $   và R   $_    i   $   là tọa độ và bán kính của đường tròn thứ i trong N đường tròn.  

\subsubsection{   Output  }

   Ghi ra duy nhất 1 số thực là diện tích phần giấy phải tô màu. Chính xác tới 5 chữ số sau dấu phẩy ( không làm tròn, cắt tại chữ số thứ 5)  

\subsubsection{   Example  }
\begin{verbatim}
Input:
2
5 6 3
5 5 5

Output:
78.53981

Giới hạn:
1 ≤ N ≤ 50
Kích thước tờ giấy là 10000 * 10000
Tất cả các hình tròn nằm trong tờ giấy.
\end{verbatim}