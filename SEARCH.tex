

Cho ba dãy số nguyên dương A=(A\_1,..,A\_M) B=(B\_1,...,B\_N) và C=(C\_1,...,C\_P)

Hãy tìm một dãy con dài nhất gồm các phần tử \textbf{ liên tiếp } của dãy C thỏa mãn hai điều kiện:
\begin{itemize}
	\item Mọi phần tử của dãy A đều xuất hiện trong dãy con được chọn
	\item Không phần tử nào của dãy B xuất hiện trong dãy con được chọn
\end{itemize}

 

\subsubsection{Input}

Dòng 1 chứa ba số nguyên dương M, N, P

Dòng 2 chứa M số nguyên dương A\_1,...,A\_M

Dòng 3 chứa N số nguyên dương B\_1,...,B\_N

Dòng 4 chứa P số nguyên dương C\_1,...,C\_P

\emph{Các số trong file dữ liệu đều là số nguyên dương không lớn hơn 10\textasciicircum5 }\emph{ , các số trên cùng một dòng được ghi cách nhau bởi dấu cách. Dữ liệu vào đảm bảo tìm được dãy con khác rỗng gồm các phần tử liên tiếp của }\emph{ thỏa mãn yêu cầu đề bài. Trong 50\% số test, m,n,p  $\le$  1000 }

\subsubsection{Output}

Một số nguyên duy nhất là độ dài của dãy con tìm được

\subsubsection{Example}
\begin{verbatim}
\textbf{Input:}
3 2 11
1 2 3
5 9
1 2 9 2 2 1 4 5 3 1 2

\textbf{Output:}
3\end{verbatim}
