



   Cho tập hợp E gồm n phần tử. Một chỉnh hợp chập k  của n phần tử đó là một bộ sắp thứ tự k phần tử của A, các phần tử đôi một khác nhau.  

   Bài toán đặt ra là:  

   - Cho số hiệu của một chỉnh hợp chập k của n số nguyên dương đầu tiên, hãy tìm chỉnh hợp chập đó.  

   - Cho chỉnh hợp chập k của n số nguyên dương đầu tiên, hãy tính số hiệu của chỉnh hợp chập đó.  

\subsubsection{   Input  }

   Gồm 2 dòng có dạng như sau:  

   Dòng 1: Ghi 2 số nguyên N, K ( 3  $\le$  N  $\le$  100 )  

   Dòng 2: Ghi số nguyên S  

   Dòng 3: Gồm K số nguyên B   $_    1   $   , B   $_    2   $   , ... B   $_    K   $   .  

\subsubsection{   Output  }

   Dòng 1: Ghi ra dãy số A   $_    1   $   , A   $_    2   $   , ... A   $_    K   $   là chỉnh hợp chập k của n số nguyên dương đầu tiên có số hiệu S.  

   Dòng 2: Ghi số hiệu của chỉnh hợp chập k: B   $_    1   $   , B   $_    2   $   , ... B   $_    K   $   .  

\subsubsection{   Example  }
\begin{verbatim}
Input:
3 2 
4
3 1

Output:
2 3
5

Giải thích:
       Số hiệu               Chỉnh hợp
          1                     1 2
          2                     1 3
          3                     2 1
          4                     2 3
          5                     3 1
          6                     3 2

\end{verbatim}
