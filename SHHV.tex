



   Xét tất cả các hoán vị của dãy số tự nhiên (1, 2,..., n) (1  $\le$  n  $\le$  12)  

   Giả sử rằng các hoán vị được sắp xếp theo thứ tự từ điển.  

   Yêu cầu:  

   1: Cho trước 1 hoán vị. Tìm số hiệu của hoán vị đó trong dãy đã sắp xếp  

   2: Cho trước số hiệu của 1 hoán vị trong dãy hoán vị đã sắp xếp. Tìm hoán vị đó  

\subsubsection{   Input  }

   Dòng 1: Chứa n số a1, a2, …, an ( dãy hoán vị n phần tử )  

   Dòng 2: Chứa số p ( số hiệu của hoán vị trong dãy hoán vị n phần tử )  

\subsubsection{   Output  }

   Dòng 1: Ghi số q ( số hiệu của dãy hoán vị a   $_    i   $   )  

   Dòng 2: Ghi n số b   $_    1   $   , b   $_    2   $   , …, b   $_    n   $   ( dãy hoán vị có số hiệu p )  

\subsubsection{   Example  }
\begin{verbatim}
Input:
2 1 3
4

Output:
3
2 3 1 

\end{verbatim}
