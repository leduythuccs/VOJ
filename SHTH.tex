

 

Cho tập hợp A gồm N phần tử. Mỗi tập con gồm K (1  $\le$  K  $\le$  N) phần tử của A được gọi là một tổ hợp chập K của N phần tử đã cho

Bài toán đặt ra là:
\begin{itemize}
	\item Cho số hiệu của một tổ hợp chập K của N số nguyên dương đầu tiên, hãy tìm tổ hợp chập đó.
	\item Cho tổ hợp chập K của N số nguyên dương đầu tiên, hãy tính số hiệu của tổ hợp chập đó.
\end{itemize}

\subsubsection{Input}

Gồm 2 dòng có dạng như sau:

Dòng 1: Ghi 2 số nguyên N, K ( 3  $\le$  N  $\le$  300 )

Dòng 2: Ghi số nguyên S

Dòng 3: Gồm K số nguyên B $_ 1 $ , B $_ 2 $ , ... B $_ K $ ( B $_ 1 $ $<$ B $_ 2 $ $<$ ... $<$ B $_ K $ )

\subsubsection{Output}

Dòng 1: Ghi ra dãy số A $_ 1 $ , A $_ 2 $ , ... A $_ K $ là tổ hợp chập K của N số nguyên dương đầu tiên có số hiệu S. Các số viết theo thứ tự tăng dần.

Dòng 2: Ghi số hiệu của tổ hợp chập K: B $_ 1 $ , B $_ 2 $ , ... B $_ K $ .

\subsubsection{Example}
\begin{verbatim}
Input:
3 2 
2
2 3

Output:
1 3
3

\end{verbatim}
