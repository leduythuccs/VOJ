



   Fibonacci sequence is defined as follow: F   $_    1   $   = 1, F   $_    2   $   = 2, F   $_    i   $   = F   $_    i-1   $   + F   $_    i-2   $   (i $>$ 2).  

   Each natural number X can be expressed by the maximum numbers that are less than or equal to X in Fibonacci sequence: X = a   $_    1   $   xF   $_    1   $   + a   $_    2   $   xF   $_    2   $   + … Therefore, in Fibonacci system, X is known as: a   $_    n   $   a   $_    n-1   $   …a   $_    1   $   . For example, 1 = 1   $_    F   $   , 2 = 10   $_    F   $   , etc. If we write all natural numbers successively in Fibonacci system, we will obtain a sequence like this: 1\_1\_0… This is called “Fibonacci bit sequence of natural numbers”.  

   Your task is counting the numbers of times that bit 1 appears in the first N bits of this sequence.  

\subsubsection{   Input  }

   Line 1: An integer N (1 $<$= N $<$= 10   $^    15   $   )  

\subsubsection{   Output  }

   Line 1: An integer K is the result  

\subsubsection{   Example  }
\begin{verbatim}
\textbf{Input:}
\\2
\\
\\\textbf{Output:}
\\2
\\\end{verbatim}