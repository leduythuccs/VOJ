

Bài này bạn cần quản lý thông tin kết nối giữa các hành tinh.

Hành tinh A, B được kết nối nếu chúng có đường đi trực tiếp hoặc có 1 dãy các hành tinh

P1, P2,..Pn mà P1=A, Pn=B và có kết nối giữa Pk và Pk-1, k ∈ \{2,..n\}.

Kết nối là 2 chiều và có thể có nhiều kết nối giữa 2 thành phố.

\subsubsection{Input}

Gồm nhiều test, mỗi test vài dòng, mỗi dòng bắt đầu bằng ‘D’, ‘C’ hoặc ‘Q’ (chữ hoa hoặc

thường) sau đó là 1 -$>$ 5 số nguyên với ý nghĩa như sau:
\begin{itemize}
	\item ‘D’ N -$>$ xác định số hành tinh là N, N $\le$ 6000000, hành tinh đánh số từ 1..N.
	\item ‘C’ -$>$ tạo kết nối giữa các cặp hành tinh.
	\item ‘Q’ -$>$ kiểm tra các cặp hành tinh có kết nối.
\end{itemize}

Lệnh C và Q có cùng định dạng như sau (kí hiệu là chữ X )
\begin{itemize}
	\item X src dst – Tạo/truy vấn kết nối giữa 2 hành tinh src và dst
	\item X src dst nnn – Tạo/truy vấn kết nối giữa hành tinh src và nnn hành tinh liên tiếp từ dst. VD:
\begin{itemize}
	\item C 1 100 1 tạo 3 kết nối (1,100), (1,101), (1,102).
	\item C 1 100 3 tạo 3 kết nối sau (1,100), (1,105), (1,110).
	\item C 1 100 5 tạo 3 kết nối sau (1,100), (1,105), (1,110).
\end{itemize}
	\item X src dst nnn dststep srcstep – Tạo/truy vấn kết nối giữa nnn cặp thành phố từ src với bước nhảy là srcstep tại src và tới dst với bước nhảy là dststep. VD:
\begin{itemize}
	\item C 1 100 3 5 15 tạo 3 kết nối (1,100), (16,105), (31,110).
\end{itemize}
\end{itemize}

\subsubsection{Output}

Với mỗi truy vấn ‘Q’ in ra 2 số: số đầu tiên là số cặp thành phố kết nối, số thứ hai là số cặp thành phố không kết nối trong truy vấn tương ứng.

\subsubsection{Sample}
\begin{verbatim}
Sample input: 
d 5 
C 1 3 
D 20 
q 1 3 
c 1 10 10 
Q 1 2 18 1 1 

Sample output: 
0 1 
9 9  

Sample input 2:
d 5
d 1
q 1 1
d 10
q 1 6 5 1 1
c 1 2 9
q 1 6 5 1 1
Sample output 2:
1  0
0  5
5  0
\end{verbatim}

 

 

 
