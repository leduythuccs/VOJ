



   Ngày sinh nhật của cô bò Bessie đang đến, cô muốn mừng sinh nhật trong D (1 $<$= D $<$= 100,000; 70\% dữ liệu có 1 $<$= D $<$= 500) ngày sắp tới.  

   Đàn bò ít chú ý nên Bessie muốn có các đồ chơi để góp vui cho buổi tiệc. Cô đã tính toán rằng cần phải có T\_i (1 $<$= T\_i $<$= 50) đồ chơi trong ngày i.  

   Trường mẫu giáo của Bessie có rất nhiều dịch vụ cho các lập trình viên bò, trong đó có một cửa hàng đồ chơi bán đồ chơi với giá Tc (1 $<$= Tc $<$= 60) dollars. Bessie muốn tiết kiệm tiền bằng cách dùng lại đồ chơi, nhưng bác John lo về nguy cơ bệnh truyền nhiễm nên yêu cầu các đồ chơi phải được khử trùng trước khi sử dụng (cửa hàng sẽ khử trùng đồ chơi khi bán chúng).  

   Có hai dịch vụ khử trùng gần trang trại. Dịch vụ thứ nhất đòi C1 dollars và cần N1 đêm để hoàn thành. Dịch vụ thứ hai đòi C2 dollars và cần N2 đêm để hoàn thành (1 $<$= N1 $<$= D; 1 $<$= N2 $<$= D; 1 $<$= C1 $<$= 60; 1 $<$= C2 $<$= 60). Bessie đem đồ chơi đến các dịch vụ này sau buổi tiệc và có thể trả tiền đồng thời lấy đồ chơi về sáng hôm sau nếu dịch vụ cần một đêm để làm việc, hoặc trong các buổi sáng sau, nếu dịch vụ cần nhiều đêm hơn.  

\subsubsection{    Dữ liệu   }

    * Dòng 1: 6 số nguyên cách nhau bởi khoảng trắng: D, N1, N2, C1, C2, Tc   

    * Dòng 2..D+1: Dòng i+1 chứa một số nguyên duy nhất: T\_i   

\subsubsection{    Kết quả   }

    * Dòng 1: Chi phí ít nhất để cung cấp cách đồ chơi an toàn cho các buổi tiệc sinh nhật của Bessie   

\subsubsection{    Ví dụ   }
\begin{verbatim}
Dữ liệu:
4 1 2 2 1 3
8
2
1
6

Kết quả:
35
\end{verbatim}