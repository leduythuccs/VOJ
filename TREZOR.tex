



   Mirko quyết định mở một dịch vụ kinh doanh mới – các hầm cho ngân hàng. Một chi nhánh của ngân hàng có thể được hình dung trong một mặt phẳng, các hầm là các điểm trên mặt phẳng. Chi nhánh của Mirko chứa chính xác L·(A+1+B) hầm, sao cho mỗi điểm với tọa độ nguyên nằm trong hình chữ nhật với các góc là (1, −A) và (L, B) có đúng một hầm.  

   Các hầm được quan sát bởi 2 lính canh – một người ở (0, −A),  và người kia ở (0, B). Một lính canh có thể nhìn thấy một hầm nếu không có bất cứ hầm nào khác trên đường thẳng nối giữa lính canh và hầm đó.  

   Một hầm là không an toàn nếu như không có lính canh nào nhìn thấy nó, an toàn nếu chỉ có 1 lính canh nhìn thấy nó, và cực kỳ an toàn nếu cả 2 lính canh đều nhìn thấy nó.  

   Cho A, B và L, ghi ra số lượng các hầm không an toàn, an toàn, và cực kỳ an toàn.  

\subsubsection{   Input  }

   Dòng đầu tiên chứa các số nguyên A và B (1 ≤ A ≤ 2000, 1 ≤ B ≤ 2000).  

   Dòng thứ 2 chứa số nguyên L (1 ≤ L ≤ 1 000 000 000).  

\subsubsection{   Output  }

   Ghi ra trên 3 dòng phân biệt lần lượt các số là số lượng các hầm không an toàn, an toàn, và cực kỳ an toàn.  

\subsubsection{   Example  }
\begin{verbatim}
Input:
1 1
3

Output:
2
2
5
\end{verbatim}
\begin{verbatim}
Input:
2 3
4

Output:
0
16
8
\end{verbatim}
\begin{verbatim}
Input:
7 11
1000000

Output:
6723409
2301730
9974861
\end{verbatim}