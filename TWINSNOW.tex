



   Người ta nói rằng không có hai bông tuyết nào giống hệt nhau! Bạn hãy thử viết một chương trình nho nhỏ để kiểm tra xem điều này có đúng không nhé :D Mỗi bông tuyết được nhắc tới ở đây đều có 6 cánh, và hai bông tuyết được gọi là giống nhau nếu chúng có độ dài các cánh tương ứng bằng nhau ;)  
\includegraphics{http://www.spoj.pl/content/anhdq:twinsnow_snowflake.jpg}

\subsubsection{   Dữ liệu  }

   - Dòng đầu tiên chứa số nguyên n là số bông tuyết cần kiểm tra.   
\\   - n dòng tiếp theo, mỗi dòng gồm 6 số nguyên (trong khoảng 0..10   $^    7   $   ) là độ dài các cánh của bông tuyết tương ứng, được liệt kê theo thứ tự vòng quanh bông tuyết.  

\subsubsection{   Kết quả  }

   Nếu không có bông tuyết nào giống nhau, hãy ghi ra:  
\begin{verbatim}
No two snowflakes are alike.\end{verbatim}

   Nếu tìm được ít nhất một cặp bông tuyết giống nhau, hãy ghi ra:  
\begin{verbatim}
Twin snowflakes found.\end{verbatim}

\subsubsection{   Ví dụ  }
\begin{verbatim}
Dữ liệu:
2
1 2 3 4 5 6
3 4 5 6 1 2

Kết quả:
Twin snowflakes found.
\end{verbatim}

\subsubsection{   Giới hạn  }

   - 0 $<$ n ≤ 10   $^    5   $   .  

\paragraph{\textit{    Bài không khó, để hẳn TimeLimit 2s, nhưng AC thì là cả một vấn đề tinh tế ;))   }}