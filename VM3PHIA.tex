



   Một đồ thị vô hướng được gọi là đồ thị 3 phía nếu tồn tại một cách chia tập đỉnh V thành 3 tập V1, V2, V3 khác rỗng sao cho mọi cặp đỉnh u v có cạnh nối thì u, v thuộc 2 tập con khác nhau. Cho đồ thị G=$<$V,E$>$ là một đồ thị 3 phía, tìm cách chia tập V thành 3 tập V1, V2, V3 khác rỗng thỏa mãn.  

\subsubsection{   Input  }
\begin{itemize}
	\item     Dòng đầu tiên ghi hai số N và M  trong đó N là số đỉnh của đồ thị, M là số cạnh của đồ thị.   
	\item     M dòng tiếp theo, mỗi dòng ghi 2 số u v thể hiện cạnh nối giữa u và v. (u $<$$>$ v)   
\end{itemize}

\subsubsection{   Output  }
\begin{itemize}
	\item     In ra xâu N kí tự. Kí tự thứ i là 1/2/3 tương ứng với đỉnh i thuộc tập V1/V2/V3.   
	\item     Dữ liệu đảm bảo luôn có đáp án. Bạn chỉ cần đưa ra một đáp án bất kỳ.   
\end{itemize}

\subsubsection{   Example  }
\begin{verbatim}
\textbf{Input:}
5 6
1 2
1 5
1 4
2 3
3 5
5 4
\end{verbatim}
\begin{verbatim}
\textbf{Output:}
12132\end{verbatim}
\begin{verbatim}
\textbf{Giới hạn}: 3 $\le$ N $\le$ 200, trong 20% số test, N  $\le$  15.\end{verbatim}
