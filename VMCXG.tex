



   Cho tập   \textbf{    S   }   gồm   \textbf{    N   }   điểm trên mặt phẳng thỏa mãn không có 3 điểm nào thẳng hàng.   \emph{    Cối xay gió   }   là một quá trình bắt đầu với một đường thẳng   \textbf{    d   }   đi qua   \textbf{    chỉ một   }   điểm   \textbf{    P   }   thuộc   \textbf{    S   }   . Đường thẳng này quay theo ngược chiều kim đồng hồ xung quanh tâm   \textbf{    P   }   cho đến khi lần đầu tiên gặp một điểm khác nào đó của   \textbf{    S   }   . Điểm này, ký hiệu   \textbf{    Q   }   , lại được lấy làm tâm quay mới, và bây giờ đường thẳng   \textbf{    d   }   tiếp tục quay theo ngược chiều kim đồng hồ quanh tâm   \textbf{    Q   }   cho đến khi gặp điểm tiếp theo của   \textbf{    S   }   . Quá trình trên cứ tiếp tục như vậy.  

   Biết mỗi giây, đường thẳng   \textbf{    d   }   quay một góc   \textbf{    1 radian   }   , hỏi sau khi quá trình được thực hiện   \textbf{    T   }   giây, đường thẳng   \textbf{    d   }   nhận mỗi điểm làm tâm quay trong bao nhiêu thời gian.  

\subsubsection{   Input  }
\begin{itemize}
	\item     Dòng đầu tiên ghi số nguyên    \textbf{     N    }    .   
	\item     Dòng thứ    \textbf{     i    }    trong    \textbf{     N    }    dòng tiếp theo, mỗi dòng ghi 2 số nguyên    \textbf{     x y    }    là tọa độ của điểm thứ    \textbf{     i    }    thuộc tập    \textbf{     S    }    .   
	\item     Dòng tiếp theo ghi số nguyên    \textbf{     T    }    .   
	\item     Dòng cuối cùng ghi ba số nguyên    \textbf{     a    }    ,    \textbf{     b    }    ,    \textbf{     c    }    mô tả đường thẳng    \textbf{     d    }    có phương trình    \emph{     ax + by + c = 0    }    .   
	\item     Dữ liệu trong input đảm bảo thỏa mãn các điều kiện của đề bài.   
\end{itemize}

\subsubsection{   Output  }
\begin{itemize}
	\item     Ghi ra    \textbf{     N    }    dòng. Dòng thứ    \textbf{     i    }    ghi một số thực    \textbf{     w     $_      i     $}    thể hiện thời gian mà điểm    \textbf{     i    }    làm tâm quay.   
	\item     Các số cần ghi ra với độ sai số không quá 10    $^     -2    $
\end{itemize}

\subsubsection{   Giới hạn  }
\begin{itemize}
	\item     Trong 40\% test đầu tiên, N ≤ 50, T ≤ 100;   
	\item     Trong 30\% test tiếp theo, N ≤ 500, T ≤ 10    $^     6    $    ;   
	\item     Trong tất cả các test, N ≤ 5000, T ≤ 10    $^     9    $    , a, b, c và các tọa độ có trị tuyệt đối không vượt quá 10    $^     6    $    .   
\end{itemize}

\subsubsection{   Chấm điểm  }

   Bài của bạn sẽ được chấm trên thang điểm 100. Điểm mà bạn nhận được sẽ tương ứng với \% test mà bạn giải đúng.  

   Trong quá trình thi, bài của bạn sẽ chỉ được chấm với 1 test ví dụ có trong đề bài.  

   Khi vòng thi kết thúc, bài của bạn sẽ được chấm với bộ test đầy đủ.  
\begin{itemize}
\end{itemize}

\subsubsection{   Ví dụ  }
\begin{verbatim}
\textbf{Input:}
4
2 0
-2 0
0 2
0 -2
3
0 1 -2

\textbf{Output:}
0.000000
1.570796
0.785398
0.643806
\end{verbatim}