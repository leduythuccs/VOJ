



   ConanKudo và RR chơi 1 trò chơi vô cùng thú vị dành cho 2 người như sau:  
\begin{itemize}
	\item     Ban đầu 2 người được cho một bảng vuông kích thước N*N, trên ô ở hàng i cột j của bảng có ghi số A(i,j).   
	\item     2 người chơi lần lượt chơi theo lượt xen kẽ nhau. ConanKudo chơi trước. Ở mỗi lượt, một người chơi sẽ chọn 1 số nguyên dương trong khoảng [1,N] mà chưa được chọn trước đó bởi bất kỳ người chơi nào.   
	\item     Sau khi cả 2 người đã chọn hết tất cả các số nguyên dương từ 1 đến N, điểm của 2 người chơi được tính như sau:    
\begin{itemize}
	\item       Với mỗi bộ 2 số nguyên dương i, j (i có thể bằng j) mà một người chơi chọn được, điểm của người chơi đó sẽ được cộng thêm A(i,j).     
\end{itemize}
\end{itemize}

   Đặt f = (điểm mà ConanKudo nhận được) - (điểm mà RR nhận được).  

   Biết rằng cả 2 người chơi đều thực hiện các nước đi một cách tối ưu, ConanKudo luôn tìm cách chơi để f đạt giá trị lớn nhất, RR luôn tìm cách chơi để f đạt giá trị nhỏ nhất. Hãy tìm các nước đi của ConanKudo.  

\subsubsection{   Mô phỏng trò chơi  }

   Bài toán sẽ được chấm theo kiểu interactive: Bạn vào vai ConanKudo, và máy tính vào vai RR.  

   Khi bắt đầu, trình chấm sẽ viết ra file input của bạn N+1 dòng:  
\begin{itemize}
	\item     Dòng đầu chứa số nguyên dương N (N $<$ 101).   
	\item     N dòng tiếp theo, mỗi dòng chứa N số nguyên dương trong khoảng [1,10    $^     6    $    ] được phân cách nhau bởi ít nhất một dấu cách.   
	\item     Sau khi đọc input từ trình chấm, bạn sẽ viết ra output cách di chuyển của mình ở lượt đầu tiên, gồm một số nguyên dương duy nhất thể hiện nước di chuyển của bạn   
	\item     Tiếp đó, trình chấm sẽ viết vào input của bạn nước di chuyển mà trình chấm thực hiện (luôn là nước đi tối ưu)   
	\item     Sau đó, bạn lại viết ra output cách di chuyển của mình ở lượt tiếp theo   
	\item     Quá trình trên lặp lại cho đến khi bạn và trình chấm đã thực hiện đủ N lượt chơi.   
\end{itemize}

   Nếu tất cả các nước di chuyển bạn đưa ra đều tối ưu, bạn được trọn vẹn điểm cho test đó, nếu không, bạn không nhận được điểm nào.  

\subsubsection{   Chú ý  }
\begin{itemize}
	\item     Sau khi output ra một số nguyên, bạn cần output thêm 1 dấu cách hoặc 1 dấu xuống dòng.   
	\item     Để trình chấm nhận được output của bạn, sau mỗi lần in ra một nước di chuyển, bạn cần dùng thêm lệnh fflush(stdout) với C++ và flush(output) với pascal. Với pascal, bạn không nên dùng cách đọc từ file có tên trống để tránh lệnh flush(output) hoạt động không như mong muốn.   
\end{itemize}

\subsubsection{   Chấm điểm  }

   Trong thời gian thi, bài của bạn chỉ được chấm với   \textbf{    50\%   }   số test của bài. Trong quá trình thi, điểm của bạn thể hiện phần trăm test giải đúng trong các test đó (trên thang điểm   \textbf{    100   }   ).  

\subsubsection{   Example  }
\begin{verbatim}
\textbf{Input \& Output:}
InputOutput3
\\2 8 1
\\3 5 5
\\3 2 7  23  1\end{verbatim}

\subsubsection{   Giải thích  }

   Điểm của ConanKudo: A(2,2) + A(1,1) + A(1,2) + A(2,1) = 18  

   Điểm của RR: A(3,3) = 7  

   f = 11  

   Nếu ở lượt chơi đầu tiên, ConanKudo chọn 1 hoặc 3, thì ở lượt 2, RR sẽ chọn 2. Khi đó, điểm của ConanKudo là 13, điểm của RR là 5, f = 8.  