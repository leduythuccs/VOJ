

Cân nặng đã ở mức đáng báo động, admin K quyết định tập chạy để giảm cân. Khu vực admin K sống có một con đường dài, điểm đầu của con đường là điểm 0, điểm cuối của con đường là điểm N-1. Các điểm cách đều nhau một khoảng 1 đơn vị độ dài; điểm thứ i có độ cao H[i]. Admin K muốn chọn ra một đoạn đường để tập chạy sao cho:
\begin{itemize}
	\item Đoạn đường có chiều dài ít nhất là L.
	\item Chênh lệch độ cao giữa điểm cao nhất và điểm thấp nhất trên đoạn đường không vượt quá D.
\end{itemize}

Bạn hãy giúp xác định xem có bao nhiêu đoạn đường thỏa mãn.

\subsubsection{Input}
\begin{itemize}
	\item Dòng đầu tiên ghi ba số N L D.
	\item Dòng thứ hai ghi N số H[i] là độ cao của điểm thứ i.
\end{itemize}

\subsubsection{Output}

In ra số đoạn đường thỏa mãn.

\subsubsection{Giới hạn}
\begin{itemize}
	\item 1 $<$= L $<$ N $<$= 10\textasciicircum6
	\item 0 $<$= D $<$= 10000
	\item 1 $<$= H[i] $<$= 10000
	\item Trong 15\% số test, N $<$= 500
	\item Trong 20\% số test tiếp theo, N $<$= 10\textasciicircum4
	\item Trong 25\% số test tiếp theo, N $<$= 10\textasciicircum5
\end{itemize}

 

\subsubsection{Example}
\begin{verbatim}
\textbf{Input:}
10 3 4
5 6 9 7 4 3 5 6 8 8

\textbf{Output:}
5\end{verbatim}