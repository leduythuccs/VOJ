





   Bé mới bắt đầu học môn hình học tại trường mẫu giáo. Bé được học rất nhiều điều thú vị như hình vuông có 4 cạnh bằng nhau, cách lấy điểm đối xứng qua 1 điểm.  

   Trước khi được nghỉ hè, cô giáo giao cho bé một bài toán và sau hè bé phải nộp lại kết quả cho cô giáo.  

   “Cho một hình vuông, 4 đỉnh theo thứ tự là A, B, C, D có tọa độ nguyên. Chọn một điểm X   $_    1   $   nằm ngoài hình vuông đó, và điểm X   $_    2   $   đối xứng với X   $_    1   $   qua A, X   $_    3   $   đối xững X   $_    2   $   qua B, X   $_    4   $   đối xứng X   $_    3   $   qua C, X   $_    5   $   đối xứng X   $_    4   $   qua D.  

   Nhiệm vụ của bé là đếm số cách chọn điểm X   $_    1   $   có tọa độ nguyên thỏa mãn:  
\begin{itemize}
	\item     Qua các bước lấy đối xứng trên ta có X    $_     1    $    trùng với X    $_     5    $
	\item     Tứ giác X    $_     1    $    X    $_     2    $    X    $_     3    $    X    $_     4    $    (theo đúng thứ tự X    $_     1    $    , X    $_     2    $    , X    $_     3    $    , X    $_     4    $    ) là tứ giác lồi   
\end{itemize}

   Nhưng mà bé đã đi du lịch khắp đất nước Việt Nam xinh đẹp cùng bố mẹ trong mùa hè, do đó bé quên bẵng mất bài tập hè cô giáo giao cho. Đến khi nhớ đến thì đã quá muộn để hoàn thành (do bé không biết code). Các bạn hãy giúp đỡ bé để bé có được phiếu bé ngoan nhé.  

\subsubsection{   Input  }

   Gồm 8 số nguyên lần lượt lượt là tọa độ các điểm A, B, C, D  

\subsubsection{   Output  }

   Một số nguyên duy nhất là số cách chọn điểm X   $_    1   $

\subsubsection{   Giới hạn  }

   Subtask 1 (30\%): Giá trị tuyệt đối tọa độ các đỉnh ≤ 1000  

   Subtask 2 (30\%): Giá trị tuyệt đối tọa độ các đỉnh ≤ 100,000  

   Subtask 3 (40\%): Giá trị tuyệt đối tọa độ các đỉnh ≤ 1000,000,000  

\subsubsection{   Example  }

\textbf{    Input:   }
\begin{verbatim}
0 0 0 2 2 2 2 0
\end{verbatim}

\textbf{    Output:   }
\begin{verbatim}
1\end{verbatim}