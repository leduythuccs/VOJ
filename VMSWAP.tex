

 

Cho dãy A là một hoán vị gồm \textbf{ N } số. Một nghịch thế là một cặp số u, v sao cho u $<$ v và A $_ u $ $>$ A $_ v $ . Bạn được thực hiện thao tác đổi chỗ hai số bất kì trong dãy A không quá một lần. Hãy tính số nghịch thế nhỏ nhất có thể đạt được trong dãy A.

\subsubsection{Input}
\begin{itemize}
	\item Dòng 1 chứa số nguyên dương \textbf{ N } .
	\item Dòng 2 chứa N số nguyên dương \textbf{ A $_ 1 $} , \textbf{ A $_ 2 $} , ..., \textbf{ A $_ N $} .
\end{itemize}

\subsubsection{Output}

Ghi ra số nghịch thế nhỏ nhất.

\subsubsection{Giới hạn}
\begin{itemize}
	\item N ≤ 1000
	\item A $_ i $ ≤ N
\end{itemize}

Trong 70\% số test, N ≤ 100.

\subsubsection{Example}

\textbf{Input }
\begin{verbatim}
5
4 5 2 3 1\end{verbatim}

\textbf{Output }
\begin{verbatim}
3\end{verbatim}