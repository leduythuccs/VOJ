



   Một sàn nhà kích thước   \textbf{    M*N   }   , được chia thành các hình vuông nhỏ kích thước 1*1. Trên đó có một số ô cấm.  

   Người ta cần lát kín sàn bằng các viên gạch 1*2 và 2*1, sao cho:  
\begin{itemize}
	\item     Không có 2 viên gạch nào chồng lên nhau   
	\item     Không có viên gạch nào lát đè lên ô cấm   
	\item     Ngoài các ô cấm, tất cả các ô còn lại đều được lát bởi đúng 1 viên gạch   
\end{itemize}

   Nhiệm vụ: Bạn được download input là thông tin về 10 sàn nhà. hãy tính số cách lát sàn thỏa mãn các điều kiện trên, và submit 1 file text gồm 10 dòng, mỗi dòng chứa một số nguyên dương duy nhất, là số cách lát sàn nhà thỏa mãn, lấy modulo 10   $^    9   $   + 7.  

\subsubsection{   Input  }

   Dòng đầu chứa số nguyên dương   \textbf{    T   }   (1 ≤ T ≤ 10).  

   Tiếp theo là T test, mỗi test gồm:  
\begin{itemize}
	\item     Dòng đầu chứa 2 số nguyên dương M, N (1 ≤ M, N ≤ 1000).   
	\item     Tiếp theo là M dòng, mỗi dòng gồm đúng N ký tự. Ký tự ở hàng i, cột j là '\#' nếu ô tương ứng là ô cấm, và '.' trong trường hợp ngược lại.   
\end{itemize}

   Bộ test có thể download ở:   \href{../../../content/voj:VMTILE}{    link   }

\subsubsection{   Output  }

   Gồm T dòng, mỗi dòng chứa một số nguyên dương duy nhất là số cách lát sàn M*N mod 10   $^    9   $   + 7.  

\subsubsection{   Chấm điểm  }

   Trong quá trình thi, chương trình của bạn sẽ được chấm với   \textbf{    5 test đầu tiên   }   . Trong quá trình thi, điểm mà bạn nhận được thể hiện phần trăm test mà bạn giải đúng trong số 5 test này (trên thang điểm   \textbf{    100   }   ).  

\subsubsection{   Example  }
\begin{verbatim}
\textbf{Input:}
4
5 4
....
.#..
....
....
....

5 4
....
....
....
....
....

5 4
...#
....
....
....
#...

5 5
#####
.....
.....
#####
#####

\textbf{Output:}
0
95
23
8
\end{verbatim}