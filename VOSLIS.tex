

Cho 2 dãy a[1..N] và b[1..M]. Gọi c[1..k] là 1 dãy con chung (không cần liên tiếp) bất kì của 2 dãy này. Đặt f(c) = abs(c[2] - c[1]) + abs(c[3] - c[2]) + .. + abs(c[k] - c[k - 1]). Nếu số phần tử của c $<$ 2 thì f(c) = 0.

Xác định dãy con chung có giá trị f lớn nhất và in ra giá trị đó.

\subsubsection{Input}
\begin{itemize}
	\item Dòng 1: 2 số nguyên N và M
	\item Dòng 2: N số nguyên biểu diễn dãy A
	\item Dòng 3: M số nguyên biểu diễn dãy B
\end{itemize}

\subsubsection{Output}

Một dòng duy nhất ghi số nguyên kết quả.

\subsubsection{Giới hạn:}
\begin{itemize}
	\item 20\% số test có 1 $<$= N, M $<$= 20
	\item 40\% số test có 1 $<$= N, M $<$= 200
	\item 60\% số test có 1 $<$= N, M $<$= 2000
\end{itemize}

Trong tất cả các test 1$<$= N, M $<$= 5000

Trong tất cả các test -10\textasciicircum9 $<$= a[i], b[i] $<$= 10\textasciicircum9

\subsubsection{Example}
\begin{verbatim}
\textbf{Input:} 
4 4
1 15 8 7
15 1 7 8 
\textbf{Output:
}8\end{verbatim}

\textbf{Giải thích:}

Dãy 15 7