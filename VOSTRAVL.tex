

Đất nước Monterey có rất nhiều danh lam thánh cảnh đẹp. Brogan đã lên kế hoạch cho chuyến du lịch sắp tới của mình ở Monterey. Theo kế hoạch, Brogan sẽ đi tham quan \textbf{K} danh lam thánh cảnh đẹp nhất ở Monterey. Nhưng Brogan vẫn chưa biết chọn lộ trình sao cho hợp lý. Brogan muốn lộ trình sẽ không đi qua một con đường theo cùng một chiều quá 1 lần và khi kết thúc lộ trình Brogan phải quay về thành phố lúc ban đầu. Ban đầu, Brogan ở thành phố \textbf{S}.

Hãy kiểm tra xem Brogan có thể tìm được lộ trình thỏa các điều kiện trên hay không. Nếu không tồn tại lộ trình như trên thì xuất ra \textbf{“NIE”} còn nếu tồn tại thì xuất ra  \textbf{“TAK”} và một lộ trình bất kỳ thỏa mản yêu cầu trên.

\subsubsection{DỮ LIỆU VÀO}
\begin{itemize}
	\item Dòng đầu tiên chứa 3 số nguyên N, M, S, K lần lượt là số lượng thành phố ở Monterey,  số lượng đường đi ở Monterey, thành phố hiện giờ Brogan ở và số lượng danh lam thánh cảnh mà Brogan muốn tham quan.
	\item M dòng tiếp theo mỗi dòng chứa hai số u, v  có nghĩa là có đương đi hai chiều từ thành phố u tới thành phố v.
	\item Dòng tiếp theo chứa K số là gồm thử tự của những thành phố mà Brogan muốn tham quan.
	\item Lưu ý: đồ thị nhập vào đảm bảo là đồ thị đơn.
\end{itemize}

\subsubsection{DỮ LIỆU RA}
\begin{itemize}
	\item Dòng đầu tiên chứa “NIE” hoặc “TAK”.
	\item Nếu là “TAK”, dòng tiếp theo se chứa số nguyên D là số lượng thành phố nằm trên lộ trình kể cả thành phố xuất phát. Theo sau số D sẽ là dãy gồm D số miêu tả lộ trình tìm được.
\end{itemize}

\subsubsection{RÀNG BUỘC}
\begin{itemize}
	\item N, M $<$= 106.
	\item K $<$= N.
	\item 10\% số test M $<$= 10.
	\item 20\% số test đồ thị không có chu trình.
\end{itemize}

\subsubsection{VÍ DỤ}
\begin{verbatim}
\textbf{Input}
3 2 1 1
1 2
2 3
3

\textbf{Output}
TAK
5 1 2 3 2 1\end{verbatim}

 

\textbf{Giải thích:} ta không thể chọn lộ trình 1 -$>$ 2 -$>$ 1 -$>$ 2 -$>$ 3 -$>$ 2 -$>$ 1 vì đoạn đường 1-$>$ 2 đi qua 2 lần theo cùng một chiều từ 1 sang 2.