

Cho N quân domino, quân thứ i (1 ≤ i ≤ N) ghi hai số nguyên A $_ ­i $ , B $_ i $ trên tương ứng hai nửa trái, phải. Cần xếp các quân domino thành một dãy thẳng theo quy tắc:
\begin{itemize}
	\item Không xoay hay lật các quân domino
	\item Hai quân domino xếp kề nhau phải có số ở nửa phải quân bên trái trùng với số ở nửa trái quân bên phải.
\end{itemize}

Hãy xác định dãy domino dài nhất xếp được có bao nhiêu quân.

\subsubsection{Input}

Dòng 1: số nguyên N (1 ≤ N ≤ 10 $^ 5 $ )

Dòng 2…N+1: dòng i + 1 ghi hai số nguyên A $_ i $ , B $_ i $ (0 ≤ A $_ i $ ≤B $_ i $ ≤ 10 $^ 9 $ )

\subsubsection{Output}

Số nguyên là số quân domino nhiều nhất xếp được thành một dãy.

\subsubsection{Example}
\begin{verbatim}
\textbf{Input:}
7
2 6
5 6
2 5
2 2
6 8
2 2
0 2 
\textbf{Output:
}6\end{verbatim}