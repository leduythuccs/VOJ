



   Các con bò của nông dân John có sở thích là hay đi khám phá những  vùng xung quanh nông trang. Ban đầu, tất cả N (1  $\le$  N  $\le$  1,000,000,000)  con bò tập trung thành 1 nhóm và cùng bắt đầu chuyến đi trên  1 con đường. Cho tới khi gặp một ngã ba đường thì chúng đôi  khi chọn cách chia làm 2 nhóm nhỏ hơn ( mỗi nhóm ít nhất 1 bò )  và mỗi nhóm lại tiếp tục hành trình trên con đường của nhóm  chúng. Khi một trong những nhóm này gặp 1 ngã ba khác thì nhóm  này lại có thể tách ra tiếp, và cứ như vậy.  

   Các con bò đã hình thành nên 1 quy tắc về việc chia nhóm như sau: nếu  chúng có thể chia thành 2 nhóm mà chênh lệch số bò của 2 nhóm là đúng  bằng K (1  $\le$  K  $\le$  1000) thì tại ngã ba đó chúng sẽ chia làm 2; nếu  không thì chúng sẽ dừng cuộc hành trình và đứng ở đó nhấm nháp cỏ non.  

   Giả sử rằng luôn có những ngã ba mới trên các con đường, hãy  tính xem cuối cùng có bao nhiêu nhóm bò tất cả.  

\subsubsection{   Dữ liệu  }
\begin{itemize}
	\item     Dòng 1: 2 số nguyên cách nhau bởi dấu cách: N và K   
\end{itemize}

\subsubsection{   Kết quả  }
\begin{itemize}
	\item     Dòng 1: Một số nguyên cho biết số lượng nhóm bò sau cùng.   
\end{itemize}

\subsubsection{   Ví dụ  }
\begin{verbatim}
Dữ liệu
6 2

Giải thích:
Có 6 con bò và độ chênh lệch khi xét chia nhóm là 2.

Kết quả
3

Giải thích:
Cuối cùng có 3 nhóm bò (1 nhóm có 2 bò, 1 nhóm có 1 và 1 nhóm có 3 ).

   6
  / \
 2   4
    / \
   1   3
\end{verbatim}
