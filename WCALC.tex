



    Khi còn bé, các bạn học sinh học được cách trừ phân số bằng cách quy đồng mẫu số, rồi mới thực hiện phép trừ      .   
\includegraphics{http://i37.photobucket.com/albums/e81/beo_chay_so/1-1_zpsbbfe63f5.png}

    Nhưng một lần, An tính thử hiệu hai phân số bằng cách lấy hiệu hai tử số và hiệu hai mẫu số và thấy thật ngạc nhiên là kết quả vẫn đúng.    
\includegraphics{http://i37.photobucket.com/albums/e81/beo_chay_so/2_zps662f8985.png}

     An thấy tính chất này thật kỳ diệu và An muốn biết, với phân số         cho trước, có bao nhiêu cặp giá trị         a$>$=0 và m$>$=0         sao cho     
\includegraphics{http://i37.photobucket.com/albums/e81/beo_chay_so/3_zps611a6584.png}

\subsubsection{   Input  }

    Một dòng chứa hai số nguyên dương b       và n       cách nhau ít nhất một dấu cách (1 $<$= b, n $<$= 10\textasciicircum6; trong 50\% số test b, n $<$= 1000       )      .  

\subsubsection{   Output  }

    một số nguyên duy nhất là số lượng cặp (a,m)       tính được      .  

\subsubsection{   Example  }
\begin{verbatim}
\textbf{Input:}
9 12

\textbf{Output:}
5
\end{verbatim}