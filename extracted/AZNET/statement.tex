Ngân hàng AZ có n chi nhánh, mỗi chi nhánh có một máy chủ là đầu mối đảm bảo truyền thông với các chi nhánh còn lại. Các máy chủ ở các chi nhánh được đánh số từ 1 đến n. Để đảm bảo truyền thông giữa các chi nhánh, ngân hàng thuê m kênh truyền tin của hai công ty A và B để kết nối n máy chủ của các chi nhánh thành một mạng máy tính. Các kênh truyền tin được đánh số từ 1 đến m, không có hai kênh truyền tin nào kết nối cùng một cặp máy chủ. Kênh truyền tin i (thuê của công ty A hoặc B) đảm bảo việc truyền tin hai chiều giữa máy chủ của chi nhánh ui và vi (i = 1, 2, ..., m). Mạng máy tính có tính chất thông suốt nghĩa là đảm bảo từ máy chủ của một chi nhánh bất kỳ có thể truyền tin đến tất cả các máy chủ của các chi nhánh còn lại theo kênh truyền tin trực tiếp giữa chúng hoặc thông qua đường truyền đi qua một số máy chủ của các chi nhánh nào đó. Trong thời gian tới do tình hình tài chính gặp khó khăn, ngân hàng muốn cắt giảm tối đa việc thuê các kênh truyền tin nhưng vẫn đảm bảo mạng thông suốt. Do chi phí thuê bao phụ thuộc vào số lượng kênh truyền tin phải thuê, nêu sau khi hỏi ý kiến các chuyên gia, ngân hàng được biết là để đảm bảo tính thông suốt của mạng, tối thiểu phải thuê n - 1 kênh truyền tin. Từ bảng đơn giá thuê bao kênh truyền tin với hai công ty ta biết ak và bk tương ứng là giá thuê bao k kênh truyền tin của công ty A và B (k = 1, 2, ..., n - 1). Ngân hàng muốn tìm phương án giữ lại đúng n - 1 kênh truyền tin trong số m kênh truyền tin đã thuê của hai công ty sao cho tổng chi phí thuê bao phải trả là nhỏ nhất mà vẫn đảm bảo tính thông suốt của mạng.
Yêu cầu
Cho biết danh sách các kênh truyền tin và các chi phí ak, bk (k = 1, 2, ..., n - 1). Hãy tìm phương án giữ lại đúng n - 1 kênh truyền tin trong số m kênh truyền tin đã thuê của hai công ty sao cho tổng chi phí thuê bao phải trả là nhỏ nhất mà vẫn đảm bảo tính thông suốt của mạng.