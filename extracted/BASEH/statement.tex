Cho một dãy số a[1],a[2],a[3],...,a[n] và hai số K,H được xác định như sau:  
\begin{itemize}
	\item     a[1]=1;   
	\item     Nếu K chẵn thì a[K]=H*a[K/2].   
	\item     Nếu K lẻ thì a[K]=H*a[(K-1)/2]+1.   
\end{itemize}

   Các bạn hãy lập trình tính số thứ K của dãy viết trong hệ cơ số H.