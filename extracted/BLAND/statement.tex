\subsection{   Giá trị chiếc quạt mo  }

   Phú ông có một mảnh đất hình chữ nhật được chia thành lưới ô vuông gồm M hàng và N cột. Các hàng của lưới được đánh số từ trên xuống dưới bắt đầu từ 1, còn các cột – đánh số từ trái sang phải, bắt đầu từ 1. Ô nằm giao của hàng i và cột j là ô đất i,j (i=1..M,j=1..N) có độ cao là $h_{ij}$   . Phú ông đã đưa ra đề nghị đổi chiếc quạt mo lấy đất như sau:  
\begin{itemize}
	\item     Bờm được quyền chọn 2 mảnh đất con (một mảnh để làm nhà, một mảnh để trồng rau), hai mảnh đều có dạng hình chữ nhật và chứa nguyên các ô.   
	\item     Mỗi mảnh đất có độ chênh lệch không quá K, nghĩa là hiệu độ cao của ô có độ cao nhất với ô có độ cao thấp nhất không vượt quá K.   
	\item     Hai mảnh đất được chọn không không giao nhau nhưng có thể tiếp xúc nhau.   
	\item     Hãy giúp Bờm chọn được hai mảnh đất thỏa mãn điều kiện của phú ông và có tổng diện tích là lớn nhất.   
\end{itemize}