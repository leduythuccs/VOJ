Cho một hình lăng trụ với mỗi đáy là 1 đa giác n đỉnh. Một chất điểm xuất phát từ đỉnh 1 và muốn đi đến đỉnh x của hình lăng trụ và phải đi qua đúng p bước. Ta đánh số các đỉnh của đa giác như sau:  

   \_ Mặt đáy trên của đa giác được đánh số từ 1 đến n ngược chiều kim đồng hồ.  

   \_ Đỉnh nằm ở mắt đáy dưới và chung cạnh với đỉnh 1 sẽ là đỉnh n+1 và đánh dấu lần lượt ngược chiều kim đồng hồ cho đến đỉnh thứ 2*n.  

   Ví dụ với n = 5:  


\includegraphics{https://dl.dropboxusercontent.com/u/44735005/C11%20Contest/C11DK2.png}

\textbf{    Yêu cầu:   }

   \_ Đếm số cách đi từ đỉnh 1 đến đỉnh x   \textbf{    qua đúng p bước   }   sao cho   \textbf{    tại mỗi bước chất điểm sẽ không đi lại đỉnh mà chất điểm đã thăm ở bước ngay trước đó.   }\textbf{}

   \_ Hành trình phải đi qua trên các cạnh.  

   \_ Mỗi cạnh được phép đi nhiều lần trên hành trình.  

\textbf{    \_   }   Kết quả theo module 2012.  

\