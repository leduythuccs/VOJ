Búp bê Nga hay Búp bê babushka (Búp bê lồng nhau, Búp bê làm tổ, ...) là một loại búp bê đặc trưng của Nga. Thật ra đó là một bộ gồm những búp bê rỗng ruột có kích thước từ lớn đến nhỏ. Con búp bê nhỏ nhất sẽ được chứa đựng trong lòng con búp bê lớn hơn nó một chút, đến lượt mình con búp bê lớn được chứa trong một con búp bê khác lớn hơn, và cứ thế cho đến con lớn nhất sẽ chứa tất cả những con búp bê còn lại trong bộ.


\includegraphics{http://nuocnga.net/uploaded/2/2163_matrioska_1.jpg}

   Hôm nay   yenthanh132   mời bạn đến tham quan bố sưu tập búp bê nga của anh ta. Nhưng việc tham quan chỉ là phụ thôi, việc chính là   yenthanh132   muốn đố bạn một bài toán liên quan đến những con búp bê nga này.  

   Đầu tiên   yenthanh132   sắp xếp N con búp bê của anh ta từ trái sang phải theo một thứ tự bất kì, con thứ i có kích thước $a_{i}$   (1 ≤ $a_{i}$   ≤ M). Sau đó anh ta yêu cầu bạn tạo khoảng trống ở đầu mút trái, để làm được điều đó, bạn cần xếp những con búp bê có kích thước nhỏ hơn ở bên trái đặt vào những con búp bê có kích thước lớn hơn bên phải, tuy nhiên mỗi con búp bê lớn hơn chỉ chứa đúng 1 con búp bê nhỏ hơn nó. Bạn chỉ được quyền dùng 3 thao tác: Lấy một con búp bê ở bên trái lên, di chuyển nó sang phải và       đặt vào bên trong      một con búp bê lớn hơn. Bạn cần tìm cách sắp xếp sao cho khoảng trống ở đầu mút trái là lớn nhất.  

   Nói cách khác,   yenthanh132   yêu cầu bạn tìm một số nguyên K lớn nhất (1 ≤ K ≤ N/2) sao cho K con búp bê trái nhất có để đặt vào bên trong K con búp bê ngay sau đó (từ k+1..k*2), mỗi con búp bê chỉ chứa đúng 1 con búp bê, theo một thứ tự nào đó. Lưu ý con búp bê i có thể đặt vào bên trong con búp bê j nếu ($a_{i}$   $<$ $a_{j}$   ).