Cho dãy các phép tính số học chỉ gồm các phép cộng trừ các số nguyên không âm. Ví dụ :  
\begin{itemize}
	\item     1 – 2 + 3 – 4 – 5   
\end{itemize}

   Bạn  được phép đặt các dấu ngoặc ‘(‘, ‘)’ vào dãy phép tính mà ko được thay  đổi các dấu cộng trừ. Với mỗi cách đặt bạn sẽ được các kết quả khác  nhau. Ví dụ :  
\begin{itemize}
	\item     1 - 2 + 3 - 4 - 5 = -7   
	\item     1 - (2 + 3 - 4 - 5) = 5   
	\item     1 - (2 + 3) - 4 - 5 = -13   
	\item     1 - 2 + 3 - (4 - 5) = 3   
	\item     1 - (2 + 3 - 4) - 5 = -5   
	\item     1 - (2 + 3) - (4 - 5) = -3   
\end{itemize}

   Câu hỏi đặt ra cho bạn là có bao nhiêu giá trị khác nhau có thể nhận  được bằng cách đặt các dấu ngoặc vào dãy phép tính như trên?