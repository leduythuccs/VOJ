Ổ khóa nhà của \emph{ yenthanh132 } rất đặc biệt. Nó gồm 5 vòng số. Mỗi vòng có đúng n số, mỗi số có một giá trị nhất định. Cụ thể, vòng thứ i sẽ chứa n số: a(i,1), a(i,2), a(i,3),..., a(i, n).

Do vốn tính hay quên nên \emph{ yenthanh132 } không để một mật mã nhất định mà thay vào đó \emph{ yenthanh132 } đã yêu cầu người ta thiết kế ra một ổ khóa đặt biệt như sau: \emph{ yenthanh132 } sẽ chọn ra một số nguyên k. Để mở được ổ khóa, ta cần phải xoay các vòng số, sao cho tổng 5 số hiện trên 5 vòng số này bằng k.

\textbf{Yêu cầu: } Cho các giá trị trên 5 vòng số. Hãy giúp yenthanh132 đếm xem có bao nhiêu cách để mở ổ khóa của anh ta. Giả sử có 2 cách xoay để chọn các số trên vòng số là [a(1,i1), a(2,i2), a(3,i3), a(4,i4), a(5,i5)] và [a(1,j1), a(2,j2), a(3,j3), a(4,j4), a(5,j5)]. Hai cách đó được xem là khác nhau nếu: hoặc i1 ≠ j1, hoặc i2 ≠ j2, hoặc i3 ≠ j3, hoặc i4 ≠ j4, hoặc i5 ≠ j5. (xem ví dụ để hiểu rõ hơn).