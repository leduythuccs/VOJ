Ví dụ
\begin{verbatim}
\textbf{Input 1:}
4 4 3
\\1 2 3 5
\\7 8 1 6
\\9 8 7 3
\\2 4 1 4
\\
\\\textbf{Output 1:}
1
\\1 2\end{verbatim}

   (Giải thích: Hình vuông 3x3 có tích 'giá trị ưa chuộng' lớn nhất có góc trái trên ở ô (1,2), có giá trị 2x3x5x8x1x6x8x7x3=241920. Sau khi đoàn đầu đặt ghế xong thì không còn đủ ghế trống để tiếp tục đặt ghế thỏa mãn hình vuông 3x3 nữa).  
\begin{verbatim}

\\\textbf{Input 2:
\\}4 4 2
\\1 1 1 1
\\2 2 1 1
\\2 2 1 1
\\2 2 2 2
\\
\\\textbf{Output 2:
\\}3
\\2 1
\\3 3
\\1 3\end{verbatim}


\\   (Giải thích: Khi đoàn đầu tiên đến đặt chổ, có 2 vị trí hình vuông có cùng giá trị lớn nhất, đó là hình vuông góc trái trên là (2,1) và (3,1). Do đó trưởng đoàn sẽ ưu tiên đặt ghế trong hình vuông có góc trái trên ở vị trí (2,1). 2 đoàn tiếp theo sẽ lần lượt đặt ghế theo giá trị lớn nhất, có vị trí góc trái trên là (3,3) rồi (1,3)).