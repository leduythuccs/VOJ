\begin{itemize}
	\item 1  $\le$  d  $\le$  10.
	\item Chiều cao của cây thông là 1 số nguyên trong đoạn [1, 10 $^ 9 $ ]
	\item Trong 20\% test đầu có N  $\le$  100.
	\item Trong 20\% test tiếp theo có N  $\le$  1000.
	\item Trong tất cả các test có N  $\le$  10000.
\end{itemize}
\begin{verbatim}
\textbf{Input:}
3 2
1 5 3

\textbf{Output:}
3\end{verbatim}
\begin{verbatim}
\textbf{Giải thích: }Có cách để sau 3 ngày 3 cây thông sẽ có chiều cao như nhau:
- Ngày 1: Phun thuốc vào cây thông thứ 2, chiều cao của 3 cây sau ngày 1 là: 3 5 5
- Ngày 2: Phun thuốc vào cây thông thứ 3, chiều cao của 3 cây sau ngày 2 là: 5 7 5
- Ngày 3: Phun thuốc vào cây thông thứ 2, chiều cao của 3 cây sau ngày 3 là: 7 7 7.
Vậy sau 3 ngày chiều cao của 3 cây thông đã bằng nhau\end{verbatim}