Ví dụ:
\textbf{    Input:   }

   6 4  

   7 9 5 3 2 4  

   8  

   7 9 3 4  



\textbf{    Output:   }



   7 9 3 4  

   8  



    Cho dãy số nguyên A gồm N phần tử đôi một khác nhau. Từ dãy A chọn ra K phần tử và giữ nguyên thứ tự như trong A tạo thành một dãy con.Sắp xếp tất cả các dãy con K phần tử theo thứ tự từ điển. Yêu cầu:   

    1) Hãy tìm dãy con có thứ tự từ điển thứ M   

    2) Cho dãy con K phần tử của dãy A. Hãy cho biết thứ tự từ điển của dãy con đó.   

\textbf{Input: C11A.INP}

    \_ Dòng đầu gồm 2 số nguyên N và K. (1  $\le$  K  $\le$  N  $\le$  60)   

    \_ Dòng thứ 2 ghi N số nguyên a[1], a[2],…, a[n] (-10^6  $\le$  a[i]  $\le$  10^6)   

    \_ Dòng thứ 3 ghi số nguyên M (theo yêu cầu 1).   

    \_ Dòng thứ 4 ghi K số nguyên là 1 dãy con của dãy A (theo yêu cầu 2).   

\textbf{Output: C11A.OUT}

    \_Dòng 1: trả lời yêu cầu 1, ghi ra dãy con K phần tử tìm được, giữa 2 số có 1 khoảng trắng.   

    \_Dòng 2: trả lời yêu cầu 2, thứ tự từ điển của dãy con đó.