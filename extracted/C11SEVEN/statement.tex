Như các bạn biết (hoặc có thể sắp biết) thì   \emph{    yenthanh132   }   rất thích số 7.   \emph{    T7   }   cũng là một nickname khác của   \emph{    yenthanh132   }   . Nhưng ta sẽ nói về chuyện đó sau... Hôm nay   \emph{    yenthanh132   }   có một bài toán cho các bạn và có liên quan tới số 7.  

\emph{    yenthanh132   }   sắp n hòn đá thành một hàng thẳng, mỗi hòn đá có một   \textbf{    mức năng lượng   }   là a   $_    i   $   , biết rằng mức năng lượng của một hòn đá là một số nguyên dương. Các hòn đá được đánh số từ trái sang phải lần lượt các số từ 0 đến n - 1 (n ≤ 10   $^    5   $   ). Sau đó   \emph{    yenthanh132   }   yêu cầu các bạn trả lời các truy vấn sau:  
\begin{itemize}
	\item     1 i : Xóa hòn đá ở có số thứ tự i (0 ≤ i ≤ n-1). Sau đó các hòn đá bên phải hòn đá bị xóa sẽ dịch sang trái 1 đơn vị, và tất nhiên số thứ tự của mỗi hòn đá cũng như giá trị n (tổng số hòn đá) sẽ giảm đi 1.   
	\item     2 i v : Thay đổi mức năng lượng của hòn đá có số thứ tự i thành v. (1 ≤ v ≤ 10    $^     9    $    ; 0 ≤ i ≤ n-1)   
	\item     3 k : Tính    \textbf{     giá trị năng lượng kết hợp    }    của các hòn đá có số thứ tự là 7*t + k (với 0 ≤ t ≤ (n-k-1) div 7). Nói cách khác đó là các hòn đá có số thứ tự i sao cho i mod 7 = k (0 ≤ i ≤ n-1).   
\end{itemize}

   Biết rằng   \textbf{    giá trị năng lượng kết hợp   }   của các hòn đá bằng tích   \textbf{    mức năng lượng   }   của các hòn đá đó. Do giá trị này có thể rất lớn nên   \emph{    yenthanh132   }   chỉ yêu cầu các bạn tìm phần dư của giá trị này sau khi chia cho 1.000.000.007 (tức 10   $^    9   $   + 7).  

   Trong mọi thời điểm, số lượng hòn đá còn lại luôn lớn hơn hoặc bằng 7.