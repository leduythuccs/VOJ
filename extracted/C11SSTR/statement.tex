Hôm nay chúng ta sẽ sẽ cùng ôn về   \textbf{     Suffix Array}   !!!  

   Cho 1 xâu S gồm n kí tự, ta sinh ra n xâu con của S là: T[1] = S[1..n], T[2] = S[2..n], T[3] = S[3..n],.. T[n] = S[n..n];  

   Sau đó ta sẽ sắp xếp n xâu T lại theo thứ tự   \textbf{    từ điển tăng dần   }   , 1 xâu A được xem là có thứ tự từ điển nhỏ hơn B nếu tồn tại 1 vị trí   \textbf{    k   }   sao cho   \textbf{    A[1..k-1] = B[1..k-1]   }   và   \textbf{    A[k] $<$ B[k]   }   ;  

   Cuối cùng ta sẽ xây dựng mảng a[1], a[2], .., a[n] với a[i] = x nếu xâu T[x] nằm ở vị trí i sau khi đã được sắp xếp tăng dần. Mảng   \textbf{    a   }   được gọi là   \textbf{     Suffix Array}   của xâu S.  

   Ví dụ:  
\begin{itemize}
	\item     Cho xâu S = 'bcab'.   
	\item     Ta có 4 xâu con:    
\begin{itemize}
	\item       T[1] = 'bcab'     
	\item       T[2] = 'cab'     
	\item       T[3] = 'ab'     
	\item       T[4] = 'b'     
\end{itemize}
	\item     Mảng T sau khi sắp xếp lại theo thứ tự tăng dần là: T[3], T[4], T[1], T[2].   
	\item     Vậy ta có Suffix Array    \textbf{     a    }    là: 3 4 1 2   
\end{itemize}
\begin{enumerate}
\end{enumerate}

\textbf{    Yêu cầu:   }
\begin{itemize}
	\item     Cho trước Suffix Array    \textbf{     a,    }    và số    \textbf{     K.    }    Trong những xâu có    \textbf{Suffix Array     a,    }    hãy tìm xâu S có thứ tự từ điển nhỏ thứ    \textbf{     K.    }
	\item     Để đơn giản các xâu chỉ bao gồm các kí tự từ 'a' đến 'z'   
\end{itemize}

   Chú thích:  
\begin{itemize}
	\item     Với S[i..j] (i$<$j) là xâu con bao gồm các kí tự liên tiếp của S trong đoạn từ kí tự thứ i đến kí tự thứ j (kí tự đầu tiên trong xâu là kí tự thứ 1). Ví dụ với xâu S = 'abac' thì S[3..4] = 'ac';   
	\item     S[k] là kí tự thứ k của xâu S.   
	\item     |S| là độ dài của xâu S hay số lượng kí tự trong xâu S.   
	\item     Nếu k $>$ |S| thì S[k] là một kí tự rỗng, kí tự rỗng được xem là nhỏ hơn bất kì kí tự nào khác   
\end{itemize}