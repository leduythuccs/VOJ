alex   có một bộ sưu tập đồng xu rất đẹp.   yenthanh132   cũng thích những đồng xu nhưng do lười nên không có sưu tầm xu và kết quả là tới giờ vẫn không có đồng xu nào cả. Thấy   yenthanh132   có vẻ thích bộ sưu tập đồng xu của mình nên một hôm   alex   quyết định sẽ tặng cho   yenthanh132   một số đồng xu trong bộ sưu tập. Nhưng vốn biết tính tham lam của   yenthanh132   , nếu cho anh ta tùy ý lựa chọn thì thế nào anh ta cũng sẽ lấy hết tất cả đồng xu trong bộ sưu tập của   alex.   Thế là   alex   đã nghĩ ra một cách để   yenthanh132   không thể lấy hết được các đồng xu của mình…  

   Trong bộ sưu tập đồng xu của   alex   có tất cả   \textbf{    M   }   loại khác nhau (được đánh số từ 1 đến M). Đầu tiên   alex   chọn ra   \textbf{    N   }   đồng xu trong bộ sưu tập đồng xu của mình (không phải chọn ra tất cả). Biết rằng   \textbf{    N   }   là một số nguyên dương và luôn chia hết cho 4. Các đồng xu được đánh số từ 1 đến   \textbf{    N   }   , sau đó anh ta lấy ra   \textbf{    N/2   }   bao thư, đánh số từ 1 đến   \textbf{    N/2   }   , mỗi bao thư bỏ 2 đồng xu liên tiếp vào. Nói cách khác bao thư thứ nhất chứa đồng xu 1, 2; bao thư thứ 2 chứa đồng xu 3,4;… bao thư thứ   \textbf{    N/2   }   chứa đồng xu (n-1) và n. Cuối cùng   alex   để n bao thư này lên bàn và bảo với   yenthanh132   được phép chọn một số bao thư với điều kiện sau:  
\begin{itemize}
	\item     Với mỗi cặp bao thư được đánh số k*2-1 và k*2 (k từ 1 đến    \textbf{     N/4    }    ) thì    yenthanh132    hoặc không chọn cả 2 hoặc chỉ có thể chọn một trong 2 bao thư.   
	\item     Với mỗi bao thư mà    yenthanh132    được chọn, thì    yenthanh132    sẽ được phải lấy cả 2 đồng xu trong bao thơ đó.   
	\item     Sau khi    yenthanh132    đã lấy xong các đồng xu thì nếu như: trong số các bao thư mà    yenthanh132    lấy được,    alex    có thể chọn ra 1 tập các bao thư (khác rỗng) sao cho với mỗi loại trong    \textbf{     M    }    loại đồng xu khác nhau trong bộ sưu tập của anh ta thì tổng số lượng đồng xu loại đó trong tập các bao thư    alex    chọn sẽ: hoặc bằng    \textbf{     0    }    hoặc là một số    \textbf{     chẵn    }    . Thì    yenthanh132    phải trả lại toàn bộ những đồng xu đó cho    alex    (tất nhiên trong quá trình chọn các bao thư chứa đồng xu    yenthanh132    sẽ phải chọn như thế nào đó để tránh gặp kết quả như vậy)    .
\end{itemize}

   Vốn tính tham lam nên mặt dù   alex   cho điều kiện ngặc nghèo như vậy nhưng   yenthanh132   vẫn cố gắng để làm sao có thể lấy được càng nhiều đồng xu càng tốt từ   alex.

\textbf{    Yêu cầu:   }   Hãy giúp   yenthanh132   tính xem anh ta có thể lấy được nhiều nhất bao nhiêu đồng xu từ bộ sưu tập các đồng xu của   alex   . Nếu bạn giúp được, có thể anh ta sẽ chia cho bạn vài đồng đấy :)