Cho bộ bài gồm n lá bài được xếp thành dãy thứ tự từ 1 tới n, đầu tiên người ta ghi vao mỗi lá bài một số nguyên là số thứ tự ban đầu của lá bài đó. Xét phép tráo S(i,m,j) : Lấy ra khỏi bộ bài m lá liên tiếp bắt đầu từ lá bài thứ i, sau đó chèn m lá bài này vào trước lá bài thứ j trong số n-m lá bài còn lại 1 $\le$ i,j $\le$ n-m+1. Quy ước nếu j = n-m+1 thì m lá bài lấy ra sẽ được đưa vào cuối dãy  

   Ví dụ với n = 9  

   Bộ bài ban đầu : (1,2,3,4,5,6,7,8,9)  

   Thực hiện S(1,5,2) : (1,2,3,4,5,6,7,8,9) -$>$ (6,1,2,3,4,5,7,8,9)  

   Thực hiện tiếp S(5,4,6) : (6,1,2,3,4,5,7,8,9) -$>$ (6,1,2,3,9,4,5,7,8)  

   Thực hiện tiếp S(8,2,1) : (6,1,2,3,9,4,5,7,8) -$>$ (7,8,6,1,2,3,9,4,5)  

   Yêu cầu : Hãy cho biết số ghi trên các lá bài sau khi thực hiện x phép tráo bài cho trước  

\