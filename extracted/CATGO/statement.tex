Ở một xưởng gỗ có rất nhiều đoạn gỗ thừa. Để đạt năng suất cao, người ta muốn tận dụng những thanh gỗ này. Tất nhiên giá trị của mỗi đoạn gỗ sẽ phụ thuộc vào độ dài của chúng. Tuy nhiên sự phụ thuộc này không đơn giản chỉ là sự phụ thuộc tuyến tính: các thanh gỗ càng dài càng có giá trị cao. Do đó, nếu cần thiết, người ta sẽ cắt các thanh gỗ này ra làm nhiều đoạn nhỏ hơn.  

   Người ta có một máy cắt, mỗi lần có thể cắt một thanh gỗ ra làm hai thanh có độ dài ngắn hơn. Do lưỡi cưa sẽ mòn dần trong quá trình cắt, chi phí của mỗi lần cắt sẽ được tính như sau: lần đầu sẽ mất 1VNĐ, lần thứ 2 sẽ là 2VNĐ, lần thứ 3 sẽ là 3VNĐ,...  

   Nhiệm vụ của bạn sẽ là tính lợi nhuận lớn nhất có thể thu được từ các đoạn gỗ thừa này.