Cho một xâu DNA S gồm các ký tự \{A, C, G, T\}. Bạn sẽ làm việc trên một xâu T, ban đầu có giá trị rỗng. Tìm số thao tác sao chép nhỏ nhất để biến T thành một xâu cho trước. Biết rằng mỗi thao tác sao chép có một trong hai dạng:  
\begin{itemize}
	\item     sao chép S i j k: sao chép đoạn S[i..j] vào xâu T bắt đầu từ vị trí k   
	\item     sao chép T i j k: sao chép đoạn T[i..j] vào xâu T bắt đầu từ vị trí k   
\end{itemize}

   Lưu ý nếu i $>$ j có nghĩa là ta sao chép đoạn xâu theo thứ tự ngược. Mỗi ký tự trong T chỉ được tạo ra đúng một lần, nghĩa là không được sao chép đè lên ký tự đó.  

   Ví dụ: Với S = “ACTG” hãy tạo T = “GTACTATTATA”  
\begin{enumerate}
	\item     Tạo GT......... bằng cách sao chép và đảo xâu “TG” từ S.   
	\item     Tạo GTAC....... bằng cách sao chép “AC” từ S.   
	\item     Tạo GTAC...TA.. bằng cách sao chép “TA” từ T.   
	\item     Tạo GTAC...TAAT bằng cách sao chép và đảo xâu “TA” từ T.   
	\item     Tạo GTACAATTAAT bằng cách sao chép “AAT” từ T.   
\end{enumerate}