Cho một tập hợp S các số nguyên, bạn hãy lập trình thực hiện các thao tác sau:  
\begin{itemize}
	\item     ADD x: thêm số x vào tập S   
	\item     DELETE x: xóa số x khỏi tập S   
	\item     MININUM: tìm số nhỏ nhất trong tập S   
	\item     MAXIMUM: tìm số lớn nhất trong tập S   
	\item     SUCC x: tìm số nhỏ nhất lớn hơn x trong tập S   
	\item     SUCC\_2 x: tìm số nhỏ nhất và không nhỏ hơn x trong tập S   
	\item     PRED x: tìm số lớn nhất nhỏ hơn x trong tập S   
	\item     PRED\_2 x: tìm số lớn nhất không vượt quá x trong tập S   
\end{itemize}

   Ghi chú: Đối với thao tác DELETE, giữ nguyên tập S nếu x không có trong tập S. Đối với các thao tác MINIMUM, MAXIMUM, SUCC, SUCC\_2, PRED và PRED\_2, in ra 'empty' nếu tập S rỗng. Đối với các thao tác SUCC, SUCC\_2, PRED và PRED\_2, in ra 'no' nếu không tìm được số thỏa mãn.  

   Các thao tác ADD, DELETE, MINIMUM, MAXIMUM, SUCC, SUCC\_2, PRED, PRED\_2 lần lượt được mã hóa bởi các chỉ số 1 2 3 4 5 6 7 8.