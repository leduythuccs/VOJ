Trong số học, định lý Số Nguyên Tố cho biết sự phân bố tiệm cận của các số nguyên tố. Gọi π(x) là số số nguyên tố không vượt quá x. Định lý Số Nguyên Tố khẳng định:  
\includegraphics{http://upload.wikimedia.org/math/c/9/3/c93061b930d29877a2364a62e5ecc1a5.png}

   Bạn hãy viết chương trình xác định xem định lý Số Nguyên Tố có thể dùng để tính xấp xỉ π(x) tốt đến đâu. Cụ thể hơn, với mỗi giá trị x, bạn cần tính sai số phần trăm |π(x) - x/lnx| / π(x) \%.