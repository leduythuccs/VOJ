Những hình nhân nhảy múa là một loại mật mã bí ẩn đã xuất hiện trong một câu chuyện về thám tử Sherlock Homes.  Ngày nay, người ta vẫn còn dùng loại mật mã này, nhưng các hình nhân được truyền đi bằng hình ảnh qua Internet và khó giải mã hơn. Chúng sẽ tự động biến đổi đi theo thời gian để trở nên khó nhận dạng so với mật mã ban đầu. Các hình nhân có chiều cao khác nhau, mỗi hình nhân có thể quay lên phía trên hoặc quay xuống phía dưới.  
\includegraphics{http://vn.spoj.pl/VM08/content/DANCING1.gif}

   Để tiện lợi, ta quy ước chiều cao của mỗi hình nhân là số dương nếu quay đầu lên trên và số âm nếu quay đầu xuống dưới.  Sau mỗi giây, các hình nhân sẽ biến đổi như sau. Ba hình nhân liên tiếp bất kỳ sẽ được chọn. Chiều cao của hai hình nhân ở bên trái và phải sẽ được cộng thêm một lượng bằng       chiều cao kể cả dấu      của hình nhân ở giữa. Sau đó, hình nhân ở giữa sẽ quay ngược đầu lại. Hình dưới đây minh họa sự biến đổi của ba hình nhân liên tiếp:  
\includegraphics{http://vn.spoj.pl/VM08/content/DANCING2.gif}

   Biết rằng ban đầu, các hình nhân       đều quay lên phía trên.     

   Biết dãy các hình nhân tại một thời điểm nào đó, bạn hãy xác dịnh dãy hình nhân ban đầu, hoặc thông báo dãy hình nhân không hợp lệ, nếu không tìm được dãy hình nhân ban đầu hoặc dãy hình nhân ban đầu không phải là duy nhất.