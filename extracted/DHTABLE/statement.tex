Các bảng thông tin điện tử được lắp trên các đường phố thành phố Hạ Long nhằm cung cấp ngắn gọn các thông tin quan trọng, các sự kiện, khẩu hiệu trong các dịp lễ hội. Công ty điện tử LĐK được lựa chọn là đơn vị cung cấp các bảng thông tin điện tử. Công ty vừa cho xuất xưởng một bảng thông tin điện tử có dạng một hàng gồm n vị trí, mỗi vị trí hiển thị một ký tự. Các vị trí được đánh số từ 1 đến n từ trái qua phải. Các ký tự chạy từ phải qua trái. Cứ mỗi giây ký tự ở vị trí i chuyển sang vị trí i− 1 ( i = 2, 3, …, n ) và ký tự mới từ xâu dữ liệu vào được lên bảng ở vị trí n . Ban đầu, tất cả các vị trí đều chứa dấu cách.

Trong thời gian thử nghiệm, để kiểm tra chất lượng bảng Công ty LĐK cho phát lên bảng xâu S được tạo thành từ cách viết liên tiếp các số tự nhiên 1, 2, 3, 4, ..., $10^{15}$ . Như vậy, phần đầu của xâu, khi viết đến số 14 sẽ là

1234567891011121314

Nếu n = 5 thì ở giây thứ 19 kể từ lúc bắt đầu phát thử nghiệm trên bảng thông tin sẽ có nội dung

\textbf{2   1   3   1   4 }

\textbf{Yêu cầu: } Cho xâu T độ dài n , chỉ chứa các ký tự số trong phạm vi từ 0 đến 9. Hãy xác định thời điểm lần đầu tiên xuất hiện xâu T , giả thiết là thời điểm bắt đầu phát thử nghiệm là 0.