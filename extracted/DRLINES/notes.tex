Giới hạn
\begin{itemize}
	\item     N nằm trong phạm vi từ 1..10000.   
	\item     Số đoạn thẳng Bờm đã vẽ từ 1..50.   
\end{itemize}
Ví dụ
\begin{verbatim}
Dữ liệu
[CASE]
3
$<$$<$
2
$>$$>$
$<$$<$
3
$>$$>$

[CASE]
5
$<$$<$
1
4
$>$$>$
$<$$<$
3
1
$>$$>$

[END]
Kết quả
1.5
5.5
\end{verbatim}

   Ví dụ 1:  
\includegraphics{http://www.topcoder.com/contest/problem/DrawingLines/drawinglinesnew.png}

   Có 4 cách để Bờm chọn các điểm  
\begin{verbatim}
[Trên 1]-[Dưới 1] [Trên 3]-[Dưới 2]
[Trên 1]-[Dưới 2] [Trên 3]-[Dưới 1]
[Trên 3]-[Dưới 1] [Trên 1]-[Dưới 2]
[Trên 3]-[Dưới 2] [Trên 1]-[Dưới 1]
\end{verbatim}

   Cách 1 và 4 tương ứng với hình bên trái. Cách 2 và 3 tương ứng với hình bên phải. Đoạn thẳng xanh là đoạn thẳng vẽ ban đầu bởi Bờm. Trong hình trái, có 1 cặp đoạn thẳng cắt nhau. Trong hình phải, có 2 cặp đoạn thẳng cắt nhau. Do đó giá trị kỳ vọng của số cặp đoạn thẳng cắt nhau là 1.5.