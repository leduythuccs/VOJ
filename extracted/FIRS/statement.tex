Biên giới giữa hai nước Ngược và Xuôi có dạng một đoạn thẳng với độ dài N-1 mét.  

   Vua nước Ngược vì muốn che giấu bí mật quốc gia đã trồng trên đường biên giới N cây tán lá xum xuê (các cây cách đều nhau với khoảng cách một mét). Vua nghĩ rằng nhờ hàng cây này mà các điệp viên của vua nước Xuôi không thể do thám nước mình được. Để chăm sóc các cây này, vua sai một người làm vườn mỗi buổi sáng chọn cây thưa lá nhất (có số lá ít nhất) và tưới một loại phân bón đặc biệt (nếu có nhiều cây thưa lá nhất thì người làm vườn sẽ chọn cây đầu tiên). Phân bón có bán kính tác dụng là 1 mét (nghĩa là sẽ tác dụng lên từ 1 đến 3 cây).  

   Tuy nhiên, vua nước Xuôi quyết định chống lại chiến lược này và thuê một người làm vườn khác. Mỗi buổi chiều, người làm vườn này tưới phân bón lên cùng cái cây đã được người làm vườn nước Ngược tưới vào buổi sáng, nhưng bằng một loại phân bón khác. Loại phân bón này làm chết tất cả các cây trong bán kính 1 mét!  

   Bạn được bộ trưởng tài chính của vua nước Xuôi thuê để giúp tính xem sau bao nhiêu ngày thì tất cả các cây đều bị chết. Hãy lập trình tính giá trị này.