Người cổ đại có một trò chơi vẫn còn lưu truyền tới ngày nay ở   Việt Nam.   
\\   Diễn đạt 1 cách ngắn gọn, trò chơi như sau :   
\\   Người chơi được cho 1 ngọn tháp được xây dựng bởi N cái   hộp với rất nhiều màu sắc, đỏ, tím, vàng, xanh, ...   
\\   Họ được yêu cầu phải dỡ ngọn tháp này ra và biến nó thành   hình 1 bông hoa, bông hoa là 1 hình đa giác đều với các cánh   hoa chính là các hộp (xem hình vẽ dưới).   
\\
\includegraphics{http://vn.spoj.pl/content/flower.gif}
\\

   Các thao tác mà người chơi có thể thực hiện là như sau :   
\\
\begin{itemize}
	\item     Dỡ chiếc hộp ở trên cùng của ngọn tháp ra và đặt vào 1 cái   giỏ.   
	\item     Trong các hộp có trong các giỏ, chọn 1 chiếc, rút nó ra   khỏi giỏ và xếp nó vào bên trái/phải các cánh hoa đã xếp trước   đó.   
	\item     Dỡ chiếc hộp ở trên cùng của ngọn tháp ra nhưng không   đặt chiếc hộp vào giỏ mà xếp nó luôn vào bên trái/phải các   cánh hoa đã xếp trước đó.   
\end{itemize}

   Điểm của người chơi tính bằng số lượng những chiếc giỏ cần   dùng.   
\\   Hãy tính xem điểm nhỏ nhất mà người chơi có thể đạt được là   bao nhiêu ?   
\\
\\       Chú ý      : Khi rút hộp ra khỏi giỏ thì giỏ lại có thể tiếp   tục dùng để đựng các hộp khác và một giỏ chỉ chứa được   không quá 1 hộp.   
\\