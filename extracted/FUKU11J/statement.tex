Jim đang chuẩn bị đi thăm một trong số những người bạn tốt nhất của cậu ấy ở trong thành phố trên núi cao. Đầu tiên, cậu ấy rời khỏi thành phố của mình và đi đến thành phố đích, gọi là lượt đi. Sau đó cậu ấy lại quay trở về thành phố của mình, gọi là lượt về. Bạn được yêu cầu viết một chương trình tính chi phí nhỏ nhất cho chuyến hành trình này, được tính bằng tổng chi phí của lượt đi và lượt về.     

      Có một mạng lưới đường đi giữa những thành phố này. Mọi con đường đều là một chiều. Mỗi đường cần một chi phí nhất định để di chuyển qua nó.     

      Ngoài chi phí phải trả cho các con đường, còn phải trả một khoản phí cho mỗi thành phố trên đường đi. Tuy nhiên, vì đây là phí visa cho thành phố, nên ta chỉ phải trả 1 lần duy nhất khi lần đầu tiên tới 1 thành phố nào đó.     

      Độ cao của mỗi thành phố được cho trước. Trong lượt đi, không được phép sử dụng các con đường đi xuống (tức là nếu đi từ a đến b thì độ cao của a không được lớn hơn độ cao của b). Trong lượt về thì không được sử dụng các đường đi lên. Nếu độ cao của a và b bằng nhau thì nó có thể được sử dụng ở cả lượt đi lẫn lượt về.     

\