Hàm có tính nhân là các hàm thỏa mãn tính chất f(m*n) = f(m) * f(n). Bây giờ, ta đặt thêm một ràng buộc cho hàm nhân, đó là nếu m và n là 2 số nguyên tố cùng nhau thì f(m) và f(n) cũng nguyên tố cùng nhau. Thêm vào đó, nó phải thỏa mãn f(1)=1. f(x) được định nghĩa cho các số nguyên dương, và trả về kết quả cũng là các số nguyên dương.  

   Bây giờ bạn được cung cấp một số số x và f(x) tương ứng. Nhiệm vụ của bạn là phải kiểm tra xem, có thể có duy nhất giá trị của f(y) với một số y cho trước hay không; và nếu có thì hãy tính giá trị đó.  

Dataset 1: Số lượng test nhỏ hơn 20. N ≤=50. x và f(x) ≤ 10^50 . x và f(x) không có ước nguyên tố nào lớn hơn 100005.  

   Số lượng câu hỏi không quá 50. Mỗi số trong các câu hỏi đều nhỏ hơn 10^50. Bạn có thể chắc chắn rằng nếu câu trả lời là duy nhất thì nó có ít hơn 400 chữ số. Time limit: 12s