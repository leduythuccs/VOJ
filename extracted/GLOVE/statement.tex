Trong tầng hầm tối tăm tại căn nhà của giáo sư hóa học Acidrain, có 2 ngăn kéo đựng toàn găng tay – một chứa găng tay trái và cái còn lại chứa găng tay phải. Trong mỗi ngăn kéo đều có những chiếc găng tay với n màu khác nhau. Giáo sư biết có bao nhiêu chiếc găng tay của mỗi màu trong mỗi ngăn kéo (số lượng găng tay cùng màu có thể khác nhau trong cả 2 ngăn). Ông cũng biết chắc chắn rằng có thể tìm được một cặp găng tay cùng màu.  

   Thí nghiệm của giáo sự chỉ có thể thành công nếu như ông sự dụng đúng đôi găng tay cùng màu (không quan trọng là màu gì), cho nên trước mỗi cuộc thí nghiệm, giáo sự đi xuống tầng hầm và chọn găng tay từ 2 ngăn kéo, hi vọng rằng sẽ có ít nhất một cặp găng cùng màu. Tầng hầm quá tối đến nỗi không thể nhận ra được màu của bất kỳ chiếc găng tay nào mà không phải đi ra khỏi đó. Giáo sư rất ghét việc phải đi vào tầng hầm hơn một lần (trong trường hợp không có cặp găng nào cùng màu), cũng như việc mang theo một lượng quá lớn những chiếc găng không cần thiết đến phòng thí nghiệm.