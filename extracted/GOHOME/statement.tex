Kì thi lập trình sinh viên ACM-ICPC kết thúc cũng là lúc   \textit{    ktuan   }   và   \textit{    microsoft   }   trở về Việt Nam ăn Noel. Sau khi gói gém xong đồ đạc, cả hai đồng ý với nhau sẽ đi MRT (Mass Rapid Transit, hệ thống xe điện ở Singapore) từ NTU ra sân bay Changi. Hai bạn quyết định mang đi một số đồng xu để trả tiền MRT, các đồng xu gồm nhiều mệnh giá khác nhau, mỗi mệnh giá bao gồm 1 số lượng đồng xu nhất định và thỏa mãn điều kiện:  
\begin{itemize}
	\item     Mỗi đồng xu có mệnh giá là một số nguyên dương   
	\item     Tổng giá trị của tất cả các đồng xu là S   
	\item     Mọi số tiền trong đoạn 1 -$>$ S đều chỉ có duy nhất một cách dùng các đồng xu mang theo để trả. Ví dụ với S = 5, những cách mang tiền đi là (1, 2, 2), (1, 1, 1, 1, 1) và (1, 1, 3). Những cách mang tiền không đúng ý của hai bạn là : (1, 1, 1, 2) vì số tiền 2 có thể trả bằng (1, 1) hoặc (2) ; số tiền 3 cũng có thể trả bằng (1, 1, 1) và (1, 2).   
\end{itemize}

   Trên đường đi hai bạn phải đi qua một số trạm (NTU và sân bay Changi cũng là trạm MRT), và để đi từ trạm này đến trạm khác hai bạn sẽ phải trả phí. Tuy nhiên cách thức thu phí ở Singapore cũng khá lạ đời. Nếu có chuyến MRT từ trạm X đến trạm Y thì phí của chuyến này là $F_{XY}$   , tuy nhiên bạn không cần phải trả số tiền $F_{XY}$   mà       phải chọn một loại mệnh giá xu         bạn đang có        trên tay sao cho còn lại đúng $F_{XY}$    đồng xu mệnh giá này và trả hết số xu mệnh giá đó.     

       Yêu cầu:      Không cho biết các mệnh giá xu và số lượng xu mang đi, hãy giúp hai bạn chọn các đồng xu       thỏa mãn điều kiện ở trên      và hai bạn có thể dùng các đồng xu này để       có thể ra sân bay      mà       đi qua ít trạm MRT nhất      . Hai bạn không muốn mang xu lên máy bay nên họ muốn       sử dụng hết số xu mang đi      .  

Hạn chế
\begin{itemize}
	\item     Có 50\% số test thỏa mãn N ≤ 100.   
\end{itemize}