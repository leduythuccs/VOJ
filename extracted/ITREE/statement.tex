Cho đồ thị cây có trọng số gồm N đỉnh , các đỉnh được đánh số từ 1 -> N . Gốc của cây là đỉnh 1 . Cha của đỉnh u là 1 đỉnh có số hiệu nhỏ hơn u . Mỗi đỉnh có một nhãn là 1 số thực A[i] . Trong đó nhãn của đỉnh 1 bằng 1 và nhãn của đỉnh lá bằng 0 . Biết rằng A[v] ≤ A[u] nếu v là con của u .   


   Giá trị của 1 cây = Tổng (  ( A[u] – A[v] ) * Trọng số cạnh (u,v)  , với u là cha của v )   


   Bây giờ người ta cho biết các cạnh của đồ thị và trọng số của các cạnh này nhưng không cho biết các A[i]. Hãy tính xem giá trị của cây thấp nhất là bao nhiêu.