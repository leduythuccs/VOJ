Pirate đi shopping ở Nhật Bản và sưu tập được rất nhiều đồng xu Nhật. Khi về nhà, anh đem các đồng xu ra chơi với em gái.  

   Trò chơi diễn ra như sau:  
\begin{itemize}
	\item     Đầu tiên, Pirate chọn ra N đồng xu để chơi và chọn thêm một số nguyên K từ 0 đến N.   
	\item     Trong mỗi lượt chơi, Pirate hoặc em gái của mình sẽ đưa ra một cách chọn K đồng xu sao cho cách chọn đó không trùng với bất cứ cách chọn nào đã được sử dụng trước đó.   
	\item     Đến lượt của mình, ai không đưa ra được cách chọn nào nữa thì thua.   
\end{itemize}

   Pirate là một người anh mẫu mực nên anh luôn nhường em gái đi trước và luôn giành phần thua cho mình. Các bạn hãy tính xem có bao nhiêu cách chọn K để Pirate thua nhé.  

   Lưu ý: Một cách chọn K đồng xu là một bộ K số ($a_{1}$   , $a_{2}$   , ..., $a_{K}$   ) sao cho 1 ≤ $a_{1}$   $<$ $a_{2}$   $<$ ... $<$ $a_{K}$   ≤ N. Hai cách chọn ($a_{1}$   , $a_{2}$   , ..., $a_{K}$   ) và ($b_{1}$   , $b_{2}$   , ..., $b_{K}$   ) khác nhau nếu tồn tại $a_{i}$   $<$$>$ $b_{i}$   (1 ≤ i ≤ K).