You are given a sequence A[1], A[2], ..., A[N] ( 0 ≤ A[i] ≤ 10^8 , 2 ≤ N ≤ 10^5 ). There are two types of operations and they are defined as follows:
\begin{itemize}
	\item \textbf{Update: }
\begin{itemize}
	\item \textbf{​}This will be indicated in the input by a 'U' followed by space and then two integers i and x.
	\item U i x , 1 ≤ i ≤ N, and x, 0 ≤ x ≤ 10^8.
	\item This operation sets the value of A[i] to x.
\end{itemize}
	\item \textbf{Query:}
\begin{itemize}
	\item This will be indicated in the input by a 'Q' followed by a single space and then two integers i and j.
	\item Q x y , 1 ≤ x $<$ y ≤ N.
	\item For Query, you must find i and j such that x ≤ i, j ≤ y and i != j, such that the sum A[i]+A[j] is maximized. Print the sum A[i]+A[j].
\end{itemize}
\end{itemize}