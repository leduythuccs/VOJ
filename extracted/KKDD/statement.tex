Một dãy số được gọi là   \textit{    K – không đơn độc   }   nếu mỗi phần tử của dãy đều thuộc một đoạn gồm ít nhất       K      phần tử liên tiếp có giá trị giống nhau. Ví dụ dãy 1 1 2 2 2 1 1 là   \textit{    2 – không đơn độc   }   , nhưng không phải là   \textit{    3 – không đơn độc   }   vì phần tử đầu tiên chỉ thuộc một đoạn gồm 2 số 1. Nếu một dãy số chưa phải là   \textit{    K – không đơn độc   }   , bạn có quyền thực hiện các thao tác biến đổi, mỗi thao tác sẽ cộng ( hoặc trừ ) một đơn vị vào một phần tử của dãy.
Yêu cầu
Hãy đếm số thao tác ít nhất cần thực hiện để biến một dãy số thành dãy   \textit{    K – không đơn độc   }   .
Hạn chế
\begin{itemize}
	\item     Có 30\% số test thỏa mãn N ≤ 200.   
	\item     Có 50\% số test thỏa mãn N ≤ 2000.   
\end{itemize}