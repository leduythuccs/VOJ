 

Dãy C = c1, c2, .. ck được gọi là dãy con của dãy A = a1, a2, .., an nếu C có thể nhận được bằng cách xóa bớt một số phần tử của dãy A và giữ nguyên thứ tự của các phần tử còn lại, nghĩa là tìm được dãy các chỉ số 1 ≤ l1 $<$ l2 $<$ … $<$ lk ≤ n sao cho c1 = a\_l1, c2 = a\_l2, …, ck = a\_lk. Ta gọi độ dài của dãy là số phần tử của dãy.

Cho hai dãy A = a1, a2, …, am và B = b1, b2, …, bn Dãy C = c1, c2, …, ck được gọi là dãy con chung bội hai của dãy A và B nếu C vừa là dãy con của dãy A, vừa là dãy con của dãy B và thỏa mãn điều kiện 2 × ci ≤ c(i+1) (i = 1, 2, …, k – 1).

Yêu cầu
Cho hai dãy A và B. Hãy tìm độ dài dãy con chung bội hai có độ dài lớn nhất của hai dãy A và B.