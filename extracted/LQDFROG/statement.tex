Có một chú ếch con trong hồ . Hồ được miêu tả như lưới tọa độ Oxy . Có n tảng đá trên hồ . Các tảng đá được đánh số từ 1..N . Ban đầu chú ếch đang ở tảng số 1 .  

   Từ tảng đá có tọa độ (x,y) ,chú ếch có thể nhảy đến tảng đá có tọa độ  

   1, (x+P,y+P) với P là số nguyên dương bất kỳ.Ta gọi hướng này là hướng A  

   2, (x+P,y-P) với P là số nguyên dương bất kỳ.Ta gọi hướng này là hướng B  

   3, (x-P,y+P) với P là số nguyên dương bất kỳ.Ta gọi hướng này là hướng C  

   4, (x-P,y-P) với P là số nguyên dương bất kỳ.Ta gọi hướng này là hướng D  

   Chú ếch sẽ nhảy đến ô có tọa độ gần nhất theo hướng đã chọn . Nếu không có ô nào  

   trên hướng nhảy thì chú ếch sẽ đứng yên và chuẩn bị nhảy theo hướng tiếp theo.Sau khi nhảy thì tảng đá (x,y) sẽ biến mất.  

   Tính tọa độ cuối cùng của chú ếch với K bước nhảy cho trước