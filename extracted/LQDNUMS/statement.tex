\\





    He choose a number N ( 1 ≤ N ≤ 10^18 ),   

    then wrote all the numbers from 1 to N to form a continuous string of digits. Next he replaced substrings of indentical digits with a single digit. For example string fragment "14445556677666" would be changed to "145676". He named this shortened string S. Then he specified a problem for his fellow professors: given a length of string S determine the number N witch results in that kind of string S. The task has proven to be too much for Mathematicans. can you solve it?   



    During a meeting with professors in the Asian Confederation of Mathematics, a Russian professor came up with a problem:   



    He choose a number N ( 1 ≤ N ≤ 10^18 ), then wrote all the numbers from 1 to N to form a continuous string of digits. Next he replaced substrings of indentical digits with a single digit. For example string fragment "14445556677666" would be changed to "145676". Then he specified a problem for his fellow professors: given a length of string S determine the number N witch results in that kind of string S. The task has proven to be too much for Mathematicans. can you solve it?   



\textbf{     Your task:    }



    Write a program to help your country's mathematicians.