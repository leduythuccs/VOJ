Trong một số nước châu Á, 8 và 6 được coi là những chữ số may mắn. Bất cứ số nguyên nào chỉ chứa chữ số 8 và 6 được coi là số may mắn, ví dụ 6, 8, 66, 668, 88, 886 …. Nguyên là một học sinh rất thích toán. Nguyên thích các số may mắn nhưng chỉ thích các số có dạng  

   S = 8…86…6  

   trong đó S có ít nhất một chữ số và chữ số 6 và 8 không nhất thiết phải đồng thời xuất hiện. Ví dụ, 8, 88, 6, 66, 86, 886, 8866 … là các số có dạng S.  

   Cho trước một số nguyên dương X (1 $<$ X $<$ 10 000), Nguyên muốn tìm số may mắn nhỏ nhất dạng S, có không quá 200 chữ số và chia hết cho X.  

   Nhiệm vụ của bạn là viết một chương trình tìm số đó cho Nguyên.  

\