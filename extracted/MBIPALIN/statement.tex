Pero và Slavko là 2 sinh viên thích toán. Họ đã biết về khái niệm palindrome : xâu giống nhau nếu đọc từ trái sang phải và ngược lại, (ví dụ,  "ANA",  "1991"  và "RADAR"). Sau đó, Pero đưa ra một khái niệm mới –  bipalindrome (viết tắt là bipalin).  

   Một bipalin  là một số tạo thành từ việc ghép 2 số palindrome có cùng độ dài, chỉ gồm các chữ số và số palindrome đầu tiên không bắt đầu bằng số 0. Ví dụ  393020 là một bipalin (tạo bởi 393 và 020), trong khi đó 222 và 010202 không phải là bipalin.  

   Bây giờ Slavko muốn biết có bao nhiêu bipalin độ dài N mà chia hết cho M,