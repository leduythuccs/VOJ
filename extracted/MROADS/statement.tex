Có N thành phố và N-1 cặp đường nối chúng, có duy nhất 1 đường nối 2 thành phố khác nhau.

Đường đã bị xuống cấp và với mỗi đường ta biết 2 số A, B : A(s) thời gian để đi qua đường này và B(s) là thời gian ít nhất để đi qua đường này nếu nâng cấp hết cả đường.

Có 1 lượng tiền đầu tư để sửa đường, với mỗi đoạn đường, kết quả sẽ tỉ lệ với lượng tiền đầu tư. Đầu tư 1 euro cho 1 đoạn đường sẽ giảm thời gian trên đoạn đường đó đi 1s. Tuy nhiên nó không thể giảm quá thời gian tối thiểu B của đoạn đường này.

Cần phân bố lượng tiền trên cho các đoạn đường khác nhau để thời gian cần thiết đi từ thành phố 1 tới thành phố xa nhất (đi mất nhiều thời gian nhất sau khi thực hiện mọi sửa chữa) là nhỏ nhất có thể.

Xác định thời gian nhỏ nhất này.

\