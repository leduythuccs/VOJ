TV muốn chiếu một trận bóng đá. Hệ thống mạng của họ gồm một số bộ  truyền dẫn, khuyếch đại và người dùng ; hệ thống này có thể mô tả bằng  một cây.    Gốc của cây là bộ phát tín hiệu về trận bóng, nút lá là người xem và các  nút khác là bộ truyền dẫn/ Biết chi phí của việc truyền tín hiện từ bộ truyền dẫn tới người dùng,  hoặc tới bộ truyền dẫn khác, thì chi phí của việc phát sóng là tổng chi phí  của các truyền dẫn được sử dụng. Mỗi người dùng trả một số tiền để xem bóng đá và nhà đài quyết định xem có  cung cấp cho họ tín hiệu không (vì trả bèo quá). Tính số lượng người xem bóng đá tối đa mà nhà đài không mất tiền do  việc truyền trận bóng.