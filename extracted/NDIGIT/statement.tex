Số A(K) là dãy thu được bằng cách ghép liên tiếp các số 1^K, 2^K, 3^K, ... Số nhỏ hơn ở phía sau (bên phải).

Với K = 1, A(K) = ...181716151413121110987654321.

Với K = 2, A(K) = ...169144121100816449362516941.

Xét tổng S = A(1) + A(2). Đoạn cuối của S là: ...350860272513937560350171262

Cho N, K1, K2, hãy tìm chữ số thứ N tính từ phải sang của tổng S = A(K1) + A(K2) (số ngòai cùng bên phải của tổng S được tính là chữ số thứ 1)