Lưu ý  
\begin{itemize}
	\item     Các giá trị của H ở công thức nhảy cao và nhảy xa thể hiện độ cao ở thời điểm nhảy, không phải độ cao ban đầu.   
	\item     Ở lần nhảy về, hướng nhảy thay đổi và bạn không thể áp dụng y nguyên công thức của lần nhảy đi (ví dụ, bạn có thể đánh lại chỉ số các cọc từ phải qua trái trước khi áp dụng công thức nhảy cao và nhảy xa).   
	\item     Trong thời gian thi, bài nộp của bạn sẽ được chấm với cả 4 ví dụ bên dưới.   
\end{itemize}
   Giới hạn  
\begin{itemize}
	\item     2 ≤    \textbf{     N    }    ,    \textbf{     $H_{i}$}    ≤ 100; 1 ≤    \textbf{     x    }    ,    \textbf{     y    }    ,    \textbf{     u    }    ,    \textbf{     v    }    ≤ 100.   
	\item     50\% số test có N ≤ 20.   
	\item     70\% số test có N ≤ 40.   
\end{itemize}
   Ví dụ  
\begin{verbatim}
\textbf{Dữ liệu 1:}
5
2 1 5 50
9 2 6 2 4\end{verbatim}
\begin{verbatim}
\textbf{Kết quả 1:}
59
Cách nhảy: 1 → 3 → 5, 5 → 4 → 3 → 2 → 1\end{verbatim}
\begin{verbatim}
\textbf{Dữ liệu 2:}
2
5 1 3 1
11 11\end{verbatim}
\begin{verbatim}
\textbf{Kết quả 2:}
8
Cách nhảy: 1 → 2, 2 → 1\end{verbatim}
\begin{verbatim}
\textbf{Dữ liệu 3:}
7
1 1 100 1
7 2 2 2 2 2 4

\textbf{Kết quả 3:}
612
Cách nhảy: 1 → 2 → 3 → 4 → 5 → 6 → 7, 7 → 2 → 1\textbf{Dữ liệu 4:}
6
5 1 20 1
9 5 2 7 2 5

\textbf{Kết quả 4:}
207
Cách nhảy: 1 → 6, 6 → 5 → 4 → 2 → 1\end{verbatim}
\begin{verbatim}
Dữ liệu 2:11 11Cách nhảy: 1 -$>$ 6, 6 -$>$ 5 -$>$ 4 -$>$ 2 -$>$ \end{verbatim}