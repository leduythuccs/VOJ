Bạn ninja   Rantaro   đang chăm chỉ luyện khinh công. Một ngày,   Rantaro   cắm các cọc tre từ bờ bên trái sang bờ bên phải của một con sông và luyện tập như sau:   Rantaro   sẽ nhảy lên cọc đầu tiên bên trái, nhảy qua một số cọc tre về phía bên phải trước khi nhảy lên cọc cuối cùng và sang bờ bên kia.   Rantaro   luôn nhảy hướng về đích chứ không bao giờ nhảy lùi, đồng thời, một số cọc có thể bị nhảy qua, nhưng   Rantaro   luôn nhảy lên cọc đầu tiên và cuối cùng.  

   Giả sử độ cao hiện tại của các cọc tre lần lượt là   \textbf{    $H_{1}$    , $H_{2}$    , …, $H_{n}$}   từ trái qua phải. Ở mỗi bước,   Rantaro   có thể lựa chọn:  
\begin{itemize}
	\item     Nhảy cao đến cọc kế tiếp và mất    \textbf{     x    }    năng lượng cho mỗi đơn vị độ cao. Nói cách khác,    Rantaro    có thể nhảy từ cọc    \textbf{     i    }    sang cọc    \textbf{     i+1    }    và mất max(    \textbf{     $H_{i+1}$}    -    \textbf{     $H_{i}$}    , 0) *    \textbf{     x    }    năng lượng.   
	\item     Nhảy xa đến một cọc khác nếu cọc đó và tất cả các cọc ở giữa đều thấp hơn cọc đang đứng. Mỗi đơn vị độ dài sẽ làm cho    Rantaro    mất    \textbf{     y    }    năng lượng. Nói cách khác,    Rantaro    có thể nhảy từ cọc    \textbf{     i    }    sang cọc    \textbf{     j    }    nếu    \textbf{     H     \textbf{$_       i+1      $}     , $H_{i+2}$     , …, $H_{j}$     $<$ $H_{i}$}    và mất (    \textbf{     j    }    -    \textbf{     i    }    ) *    \textbf{     y    }    năng lượng.   
\end{itemize}

   Mỗi khi   Rantaro   nhảy lên một cọc tre, cọc tre sẽ bị lún và độ cao của cọc đó sẽ giảm đi 1 trước khi   Rantaro   thực hiện bước nhảy tiếp theo. Ví dụ, nếu có 2 cọc tre với độ cao là (3, 5).   Rantaro   sẽ nhảy lên cọc đầu tiên và làm độ cao của cọc đó giảm xuống 2. Khi   Rantaro   nhảy lên cọc cuối sẽ mất (5 - 2) *   \textbf{    x   }   năng lượng. Sau khi sang bờ bên kia, độ cao của 2 cọc sẽ là (2, 4). Trong bài này, chúng ta bỏ qua năng lượng để nhảy từ bờ lên cọc đầu tiên và từ cọc cuối cùng xuống bờ bên kia.  

   Sau khi sang bờ bên phải,   Rantaro   lại nhảy lại về bờ bên trái theo cách tương tự. Tuy nhiên,   Rantaro   sẽ mất   \textbf{    u   }   năng lượng cho mỗi đơn vị độ cao và   \textbf{    v   }   năng lượng cho mỗi đơn vị độ dài. Bạn cần giúp Rantaro tính tổng số năng lượng ít nhất cần dùng cho cả hai lượt nhảy.  

   Ví dụ, nếu   \textbf{    x   }   = 2,   \textbf{    y   }   = 1,   \textbf{    u   }   = 5,   \textbf{    v   }   = 50, độ cao các cọc là (9, 2, 6, 2, 4). Ở lần nhảy từ trái qua phải,   Rantaro   có thể chỉ mất 4 năng lượng nếu nhảy từ cọc 1 đến cọc 5. Sau khi sang bờ bên phải, độ cao các cọc lần lượt là (8, 2, 6, 2, 3). Ở lần nhảy về,   Rantaro   sẽ nhảy 5 → 4 → 3 → 2 → 1 và mất ((6-1) + (8-1)) * 5 = 60 năng lượng. Tổng cộng   Rantaro   sẽ mất 64 năng lượng. Nếu ở lần nhảy từ trái qua phải,   Rantaro   nhảy 1 → 3 → 5 sẽ mất 4 năng lượng và độ cao các cọc sau khi nhảy là (8, 2, 5, 2, 3). Ở lần nhảy về,   Rantaro   sẽ chỉ mất 55 năng lượng và tổng cộng sẽ là 59 năng lượng.