Xét các số nguyên từ 1 đế N. Các số này được sắp xếp theo thứ tự từ điển. Ví dụ với N=11, ta có dãy số sau khi sắp xếp là 1, 10, 11, 2, 3, 4, 5, 6, 7, 8, 9.

Ký hiệu Q$_N,K $ là vị trí của số K trong dãy được sắp xếp theo cách nói trên. Ví dụ Q$_11,2 $ =4 Cho các số nguyên K và M. Hãy tìm số nguyên N nhỏ nhất thỏa mãn Q$_N,K $ =M