Xét D là dãy vô hạn các chữ số trong hệ đếm cơ số 16 (Hexa) bằng cách viết liên tiếp các số tăng dần từ 1 trở đi: 1, 2, 3, 4, . . ., N, ... Phần đầu của dãy D là

123456789ABCDEF10111\textbf{21}31415161718191A1B1C1D1E1F20\textbf{21}22...

Có thể coi dãy D là một xâu vô hạn các ký tự số hệ 16. Gọi S là xâu bất kỳ chỉ bao gồm các ký tự số của hệ 16. Số lần xâu S xuất hiện trong D như một xâu con là vô hạn. Khoảng cách giữa hai lần xuất hiện liên tiếp không giao nhau của S là số ký tự của D nằm giữa hai lần xuất hiện này. Ví dụ, nếu S = ’21’ thì khoảng cách giữa lần xuất hiện thứ nhất và thứ hai là 27 (như minh họa trên).

Yêu cầu: Cho xâu S độ dài không quá 30 ký tự, hãy xác định khoảng cách giữa hai lần xuất hiện thứ nhất và thứ hai của S trong D.