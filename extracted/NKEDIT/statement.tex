Cho trước một số nguyên K và hai văn bản dưới dạng hai xâu S và P (có độ dài không quá 50 ký tự), chỉ gồm các chữ cái in thường ('a'..'z').  

   Người ta hiệu chỉnh cả hai văn bản theo quy tắc sau: tìm xâu con (nghĩa là một đoạn gồm các ký tự liên tiếp) chung dài nhất của hai xâu S và P. Sau đó nếu xâu con chung này có độ dài $>$=K thì xóa xâu con chung này khỏi S và P.  

   Trong trường hợp có nhiều xâu con chung dài nhất, người ta chọn xâu để xóa theo quy tắc sau:  
\begin{itemize}
	\item     Chọn xâu con chung dài nhất có vị trí trái nhất thuộc xâu S   
	\item     Nếu xâu này vẫn xuất hiện nhiều lần ở xâu P, chọn xâu có vị trí trái nhất thuộc xâu P   
\end{itemize}

   Quá trình này được lặp lại cho đến khi S và P không còn xâu con chung nào có độ dài $>$= K.  

   Ví dụ, với K=2, S=aabhh, P=haahaa  

   Bước 1:  S=       aa      bhh  P=h       aa      haa  

   Bước 2:  S=b       hh      P=       hh      aa  

   Kết thúc: S=b   P=aa  

   Đến đây S và P không còn xâu con chung nào có độ dài $>$= 2. Ta kết thúc quá trình hiệu chỉnh văn bản.  

   Bạn hãy lập trình thực hiện quá trình hiệu chỉnh văn bản trên và in ra số bước hiệu chỉnh, xâu S và P cuối cùng.