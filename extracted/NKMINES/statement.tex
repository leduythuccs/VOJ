Một bãi mìn hình chữ nhật có cạnh M × N nguyên dương. Bãi mìn được chia thành M × N ô vuông đơn vị bằng các đường song song với các cạnh,   các dòng ô vuông đánh số từ 1 đến M từ trên xuống dưới, các cột ô vuông đánh số từ 1 đến N từ trái sang phải, hai ô vuông khác nhau được gọi là kề   nhau nếu chúng có ít nhất một đỉnh chung. Mỗi ô vuông có không quá một quả mìn. Để ghi nhận tình trạng mìn tại các ô đồng thời có thể giữ bí mật phần   nào, người ta lập một mảng hai chiều M dòng N cột mà A[U, V] bằng số ô mìn có điểm chung với ô [U, V] của bãi mìn (có nhiều nhất 8 ô có điểm   chung với một ô cho trước).  

   Cho mảng A, hãy tìm cách xác định các ô có mìn.