Một hệ thống n máy tính (các máy tính được đánh số từ 1 đến n) được nối lại thành một mạng bởi m kênh nối, mỗi kênh nối hai máy nào đó và cho phép ta truyền tin một chiều từ máy này đến máy kia. Giả sử s và t là 2 máy tính trong mạng. Ta gọi đường truyền từ máy s đến máy t là một dãy các máy tính và các kênh nối chúng có dạng:  

   s = $u_{1}$   , $e_{1}$   , $u_{2}$   , ..., $u_{i}$   , $e_{i}$   , $u_{i+1}$   , ..., $u_{k-1}$   , $e_{k-1}$   , $u_{k}$   = t  

   trong đó $u_{1}$   , $u_{2}$   , ..., $u_{k}$   là các máy tính trong mạng, $e_{i}$   - kênh truyền tin từ máy $u_{i}$   đến máy $u_{i+1}$   . (i = 1, 2,... , k-1).  

   Mạng máy tính được gọi là thông suốt nếu như đối với hai máy u, v bất kỳ ta luôn có đường truyền tin từ u đến v và đường truyền tin từ v đến u. Mạng máy tính được gọi là hầu như thông suốt nếu đối với hai máy u, v bất kỳ, hoặc là có đường truyền từ u đến v, hoặc là có đường truyền từ v đến u.  

   Biết rằng mạng máy tính đã cho là hầu như thông suốt nhưng không thông suốt.  

   Yêu cầu: hãy xác định xem có thể bổ sung đúng một kênh truyền tin để biến mạng đã cho trở thành thông suốt được không?
Hạn chế
Trong tất cả các test, n ≤ 2000, m ≤ 30000.