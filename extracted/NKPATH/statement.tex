Cho một lưới ô vuông gồm m dòng và n cột. Các dòng được đánh số từ 1 đến m từ trên xuống dưới, các cột được đánh số từ 1 đến n từ trái qua phải. Ô nằm ở vị trí dòng i và cột j của lưới được gọi là ô   (i, j) và khi đó, i được gọi là tọa độ dòng còn j được gọi là tọa độ cột của ô này. Trên ô (i, j) của lưới ghi số nguyên dương $a_{ij}$   , i = 1, 2, …, m; j = 1, 2, …, n. Trên lưới đã cho, từ ô (i, j) ta có thể   di chuyển đến ô (p, q) nếu các điều kiện sau đây được thỏa mãn:  
\begin{itemize}
	\item     j $<$ n; i ≤ p; j ≤ q và i + j $<$ p + q;   
	\item     $a_{ij}$    và $a_{pq}$    có ước số chung lớn hơn 1.   
\end{itemize}

   Ta gọi một cách di chuyển từ mép trái sang mép phải của lưới là cách di chuyển bắt đầu từ một ô có tọa độ cột bằng 1 qua các ô của lưới theo qui tắc di chuyển đã nêu và kết thúc ở một ô có tọa độ cột   bằng n.  

   Yêu cầu: Tính số cách di chuyển từ mép trái lưới sang mép phải lưới.
Hạn chế
Trong tất cả các test: 1 $<$ m, n ≤ 100; $a_{ij}$   ≤ 30000, i=1,2,…,m;j=1,2,…,n. Có 50\% số lượng test với m, n ≤ 50.