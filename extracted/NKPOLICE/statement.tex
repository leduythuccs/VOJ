Để truy bắt tội phạm, cảnh sát xây dựng một hệ thống máy tính mới. Bản đồ khu vực bao gồm N thành phố và E đường nối 2 chiều. Các thành phố được đánh số từ 1 đến N.  

   Cảnh sát muốn bắt các tội phạm di chuyển từ thành phố này đến thành phố khác. Các điều tra viên, theo dõi bản đồ, phải xác định vị trí thiết lập trạm gác. Hệ thống máy tính mới phải trả lời được 2 loại   truy vấn sau:  
\begin{itemize}
	\item     1. Đối với hai thành phố A, B và một đường nối giữa hai thành phố $G_{1}$    , $G_{2}$    , hỏi tội phạm có thể di chuyển từ A đến B nếu đường nối này bị chặn (nghĩa là tên tội phạm không   thể sử dụng con đường này) không?   
	\item     2. Đối với 3 thành phố A, B, C, hỏi tội phạm có thể di chuyển từ A đến B nếu như toàn bộ thành phố C bị kiểm soát (nghĩa là tên tội phạm không thể đi vào thành phố này) không?   
\end{itemize}