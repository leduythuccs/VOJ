\begin{verbatim}
\textbf{Input:}
4 4
1 2 0
1 3 0
3 4 0
2 4 1
3
0
00
01\end{verbatim}

\textbf{Output:}
\\+1 2
\\+2 1
\\-1 2

\textbf{Giải thích: } Trong 2 lần thí nghiệm cuối robot đã chọn phòng 4 làm đích đến. Có 2 đường đi ngắn nhất giữa 1 và 4 :

1 --(0)--$>$ 3 --(0)--$>$ 4

1 --(0)--$>$ 2 --(1)--$>$ 4

Trong đó 1, 3, 4 là đường có thứ tự từ điển nhỏ hơn.

00 --$>$ chạy đúng, robot đi được độ dài K=2 và dừng tại S=1 phòng (phòng số 4).

01 --$>$ mặc dù vẫn đi được đến phòng số 4, nhưng không phải là đường đẹp nhất. Nên robot sẽ chạy đúng đến sau bước thứ K=1 , sau bước đó số phòng có thể có chứa Robot là S=2 phòng (phòng số 2 và 3).

\textbf{Giới hạn: }

- 25\% số test đồ thị có dạng cây khung.

- 40\% số test N $\le$ 100 và M $\le$ 200.