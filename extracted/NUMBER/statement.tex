Cho M máy biến đổi số được đánh số từ 1 đến M và 1 số nguyên dương N. Hoạt động của máy i được xác định bởi cặp số nguyên dương (ai,bi) (1 $\le$ ai,bi $\le$ N). Máy nhận đầu vào là số nguyên dương ai và trả lại ở đầu ra số nguyên dương bi.

Ta nói một số nguyên dương X có thể biến đổi thành số nguyên dương Y nếu hoặc X=Y hoặc tồn tại dãy hữu hạn các số nguyên dương X= P1,P2,...,Pk =Y sao cho đối với 2 phần tử liên tiếp Pi và Pi+1 bất kỳ trong dãy, luôn tìm được 1 trong số các máy đã cho để biến đổi Pi thành Pi+1

Cho trước 1 số nguyên dương T (T  $\le$  N). Hãy bổ sung thêm 1 số ít nhất các máy biến đổi số để bất kì số nguyên dương nào từ 1 đến N đều có thể biến đổi thành T