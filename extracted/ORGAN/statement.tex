Xưởng sản xuất đồ chơi XYZ đã mua các lô hàng ống đàn để làm nguyên liệu sản xuất đàn ống. Mỗi lô gồm n (n $>$ 2) ống đàn với độ cao đôi một khác nhau lần lượt là $h_{1}$ , $h_{2}$ , ..., $h_{n}$ để khi nhạc công gõ vào các ống đàn với độ cao khác nhau, ống sẽ phát ra các âm thanh khác nhau. Ống đàn thứ i có trọng lượng là $h_{i}$ x m (1 ≤ i ≤ n). Quy trình sản xuất đàn của hãng thực hiện theo dây chuyền tự đông hoàn toàn như sau: Bắt đầu, robot A sẽ tự động mở một lô và xếp lần lượt n ống có độ cao $h_{1}$ , $h_{2}$ , ..., $h_{n}$ lên dây chuyền. Tiếp theo, các ống sẽ được robot B phân thành s (1 $<$ s ≤ n) lô con. Lô con thứ nhất gồm các ống từ 1 đến $k_{1}$ , lô con thứ hai gồm các ống từ $k_{1}$ + 1 đến $k_{2}$ , ..., lô con thứ s gồm các ống từ $k_{s-1}$ + 1 đến n (1 ≤ $k_{1}$ $<$ $k_{2}$ $<$ ... $<$ $k_{s-1}$ $<$ n). Mỗi một lô con sẽ được chuyển cho robot C để lắp thành một chiếc đàn. Robot C sẽ tiến hành sắp xếp các ống thành một dãy đảm bảo điều kiện có không quá w vị trí ống đứng trước cao hơn ống đứng liền kề sau nó (nếu có). Có thể có nhiều phương án sắp xếp các ống đàn trong một lô con thỏa mãn điều kiện này. Mỗi một phương án như vậy sẽ được gọi là một loại đàn. Sau khi khảo sát thị hiếu người tiêu dùng, Ban giám đốc nhận thấy: trọng lượng hợp lý của một chiếc đàn (được tính bởi tổng trọng lượng của các ống đàn) là một số không nhỏ hơn $b_{min}$ và không lớn hơn $b_{max}$ ; ngoài ra, không có hai khách hàng nào lại muốn dùng đàn giống nhau. Dễ thấy, số lượng loại đàn khác nhau có thể tạo ra phụ thuộc vào việc n ống thành s lô con. Do đó, Ban giám đốc muốn lựa chọn cách phân n ống thành s lô con sao cho tổng trọng lượng các ống trong mỗi lô con đều nằm trong đoạn từ $b_{min}$ đến $b_{max}$ và số lượng các loại đàn ống khác nhau có thể sản xuất được là nhiều nhất.
Hãy tìm cách phân n ống thành s lô con thỏa mãn các điều kiện đặt ra và sao cho số lượng loại đàn ống khác nhau có thể sản xuất được là nhiều nhất.