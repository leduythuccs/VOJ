Xét bàn cờ vuông kích thước n×n. Các dòng được đánh số từ 1 đến n, từ dưới lên trên. Các cột được đánh số từ 1 đến n từ trái qua phải.

Ô nằm trên giao của dòng i và cột j được gọi là ô (i,j). Trên bàn cờ có m (0 ≤ m ≤ n) quân cờ. Với m $>$ 0, quân cờ thứ i ở ô ($r_{i}$ , $c_{i}$ ), i = 1,2,..., m. Không có hai quân cờ nào ở trên cùng một ô. Trong số các ô còn lại của bàn cờ, tại ô (p, q) có một quân tượng. Mỗi một nước đi, từ vị trí đang đứng quân tượng chỉ có thể di chuyển đến được những ô trên cùng đường chéo với nó mà trên đường đi không phải qua các ô đã có quân


\includegraphics{http://vnoi.info/webcontent/VOI_files/image086.gif}

Cần phải đưa quân tượng từ ô xuất phát (p, q) về ô đích (s,t). Giả thiết là ở ô đích không có quân cờ. Nếu ngoài quân tượng không có quân nào khác trên bàn cờ thì chỉ có 2 trường hợp: hoặc là không thể tới được ô đích, hoặc là tới được sau không quá 2 nước đi (hình trái). Khi trên bàn cờ còn có các quân cờ khác, vấn đề sẽ không còn đơn giản như vậy.

Yêu cầu: Cho kích thước bàn cờ n, số quân cờ hiện có trên bàn cờ m và vị trí của chúng, ô xuất phát và ô đích của quân tượng. Hãy xác định số nước đi ít nhất cần thực hiện để đưa quân tượng về ô đích hoặc đưa ra số -1 nếu điều này không thể thực hiện được.
Hạn chế:
Trong tất cả các test: 1 ≤ n ≤ 200. Có 60\% số lượng test với n ≤ 20.