 

Cho dãy số nguyên $a_{1}$ , $a_{2}$ , …, $a_{n}$ và số nguyên dương k. Ta gọi k-phân đoạn của dãy số đã cho là cách chia dãy số đã cho ra thành k đoạn, mỗi đoạn là một dãy con gồm các phần tử liên tiếp của dãy. Chính xác hơn, một k-phân đoạn được xác định bởi dãy chỉ số

1  $\le$  $n_{1}$ $<$ $n_{2}$ $<$ $n_{3}$ $<$ ... $<$ $n_{k}$ = n

Đoạn thứ i là dãy con $a_{n}$_ i-1 $ +1 $ , $a_{n}$_ i-1 $ +2 $ , ..., $a_{n}$_ i $$ , i=1, 2, ..k. Ở đây quy ước $n_{0}$ =0

Yêu cầu: Hãy xác định số M nhỏ nhất để tồn tại k-phân đoạn sao cho tổng các phần tử trong mỗi đoạn đều không vượt quá M.