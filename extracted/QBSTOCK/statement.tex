Qua nhiều năm hoạt động, IOICamp đã tập hợp được nhiều tài năng tin học và qua đó dần dần trở thành một tập đoàn công nghệ lớn. Đến năm 2222, IOICamp đã chính thức trở thành một tập đoàn lớn có cổ phiếu trên thị trường chứng khoán. Tuy nhiên để tồn tại được trên thị trường chứng khoán thì việc nắm bắt được tình hình lên xuống của giá cổ phiếu là vô cùng cần thiết. Để làm được việc này, IOICamp rất cần đến các chuyên gia tin học của mình. Sau nhiều năm nghiên cứu các thành viên của IOICamp đã thành công trong việc xấp xỉ sự biến động của cổ phiếu thành một hàm đa thức với biến thời gian. Tuy nhiên công việc khó khăn trước mắt là phải tìm được thời điểm cổ phiếu có giá thấp nhất và cao nhất và IOICamp muốn nhờ các bạn làm giúp việc này.

Các bạn sẽ được cung cấp hàm đa thức mô tả biến động của giá cổ phiếu trong một khoảng thời gian cho trước. Hàm đa thức này có dạng:

f(t) = a.$t^{4}$ + b.$t^{3}$ + c.$t^{2}$ + d.t + e

Với a, b, c, d, e là các hệ số nguyên cho trước. Hãy tìm giá cổ phiếu cao nhất $f_{max}$ và giá cổ phiếu thấp nhất $f_{min}$ trong khoảng thời gian từ $t_{1}$ đến $t_{2}$ (nghĩa là $t_{1}$ ≤ t ≤ $t_{2}$ ).

\textbf{Đ}\textbf{ây là 1 bài khá đơn giản cho các bạn đã qua môn Phương Pháp Tính :D }