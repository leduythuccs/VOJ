Dòng đầu tiên ghi các số nguyên L1, L2, L3, C1, C2, C3 (1  $\le$  L1  $\le$  L2  $\le$  L3  $\le$  $10^{9}$   ;  1  $\le$  C1  $\le$  C2  $\le$  C3  $\le$  $10^{9}$   ) theo đúng thứ tự liệt kê ở trên.  

   Dòng thứ hai chứa số lượng nhà ga N ( 2  $\le$  N  $\le$  100000 )  

   Dòng thứ ba ghi hai số nguyên s, f là các chỉ số của hai nhà ga cần tìm cách đặt mua vé với chi phí nhỏ nhất để đi lại giữa chúng.  

   Dòng thứ i trong số N - 1 dòng tiếp theo ghi số nguyên là khoảng cách từ nhà ga A (ga 1) đến nhà ga thứ i + 1.