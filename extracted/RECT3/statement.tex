Cho bảng chữ nhật MxN (1 ≤ M, N ≤ 200) gồm các số 0 và 1. Ta gọi một       khối tam chữ nhật      là một hình gồm ba hình chữ nhật, mỗi hình chữ nhật gồm toàn số 1, xếp chồng lên nhau sao cho hình chữ nhật ở giữa phải rộng hơn về mỗi phía của hai hình chữ nhật trên và dưới ít nhất một ô.  

   Ví dụ, các hình sau đây là các khối tam chữ nhật:  
\begin{verbatim}
 1	
111
 1	
 
 11		
 11		
11111
   1		
   1		
\end{verbatim}

   Các hình sau đây không phải là khối tam chữ nhật:  
\begin{verbatim}
 111
	 				
11111
 111	
 111	
(Ba hình chữ nhật không liên thông)

 1111
 1111
11111
 111	
 111	
(Hình chữ nhật ở giữa không rộng hơn hình chữ nhật phía trên một ô về bên phải)
\end{verbatim}

   Trong bài toán này, bạn cần tìm khối tam chữ nhật có diện tích lớn nhất.