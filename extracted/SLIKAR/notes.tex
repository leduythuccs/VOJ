\begin{verbatim}
Input:
8
01010001
10100011
01010111
10101111
01010111
10100011
01010001
10100000

Output:
16
00000001
00000011
00000111
00001111
11110111
11110011
11110001
11110000
\end{verbatim}