Linear Probing is one of the most used Hashing techiniques. We define here a special hashing which is similar to linear probing.  

   The following operations are defined.  
\begin{itemize}
	\item     Hash: Hash is defined for a number N to be N\%k.   
	\item     Move forward : Move to the next number (the number connected by the forward link) . Initially, every number's forward link points to the itself. (currentIndex).   
	\item     Move backward : Move to the previous number (the number connected by the back link) . Initially, every number's back link points to the itself in the link (currentIndex).   
	\item     Insertion operation: Given a number N , find hash(N)=N\%k where k is the size of the list. If the list[hash(N)] is empty the element is inserted at position hash(N) in the list and forward link is made to point at (currentIndex+1)\%(size of list) and backward link is made to point at (currentIndex-1+sizeof list)\%(size of list). If it is filled , we do move\_forward/move\_backward as specified and then the same process is again repeated.   

    Note: Thus list is circular due to modulus property.   
	\item     Merge Operation: Let x be a index in the list which is not empty. Calculate xmin by doing a move\_backward from index x till the previous index is empty . Similarily calculate xmax by doing a move\_forward from index x till the next element is an empty space . Do the same for y to find out ymin and ymax. For a valid merge operation, the index x and y should not be empty and either xmax $<$ ymin or ymax $<$ xmin. Now, when merging x and y, if ymin $>$ xmax, the forward link of xmax is made to point at ymin and the backward link of ymin is made to point to xmax. Same approch is applied in the other case.   

    Note: for the merge operation take the min(b,c). The merge is only to be done from x(b)max to x(c)min if the merge was allowed.   
\end{itemize}

Constraints
\begin{verbatim}
Dataset 1:T$<$25, k ≤=10000,C $\le$ 25000 Score: 100
Time limit: 5s   Memory Limit: 128MB
Dataset 2:T$<$8, k ≤=400000,C $\le$ 800000 Score: 50
Time limit: 5s   Memory Limit: 128MB 
\end{verbatim}