Cho 2 xâu ký tự X = $x_{1$ , $x_{2}$ , .., $x_{m}$} và Y = $y_{1$ , $y_{2}$ , ..., $y_{n}$} . Cần xây dựng xâu T = $t_{1$ , $t_{2}$ , $t_{3}$ , ..,$t_{n+m}$}$_ . $ gồm tất cả các ký tự trong xâu X và tất cả các ký tự trong xâu Y , sao cho các ký tự trong X xuất hiện trong T theo thứ tự xuất hiện trong X và các ký tự trong Y xuất hiện trong T theo đúng thứ tự xuất hiện trong Y , đồng thời với tổng chi phí trộn là nhỏ nhất. Tổng chi phí trộn hai xâu X và Y để thu được xâu T được tính bởi công thức c( T ) = sum(c( $t_{k$ , $t_{k+1}$}$_$ )) với k = 1, 2, ..,  n+m-1; trong đó, các chi phí c( $t_{k$ , $t_{k+1}$} ) được tính như sau:
\begin{itemize}
	\item Nếu hai ký tự liên tiếp $t_{k$ , $t_{k+1}$} được lấy từ cùng một xâu X hoặc Y thì c( $t_{k$ , $t_{k+1}$} ) = 0
	\item Nếu hai ký tự liên tiếp $t_{k$ , $t_{k+}$ 1 } là $x_{i}$ $y_{j}$ thì chi phí phải trả là c( $x_{i$ , $y_{j}$} ). Nếu hai ký tự liên tiếp $t_{k$ , $t_{k+1}$}$_$ là $y_{j$ , $x_{i}$}$_$ thì chi phí phải trả là c( $y_{j$ , $x_{i}$} ) = c( $x_{i$ , $y_{j}$} )
\end{itemize}