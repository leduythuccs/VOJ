Dòng đầu tiên gồm 2 số N và d. (3 ≤   \textbf{    N   }   ≤ 1000 , 3 ≤   \textbf{    d   }   ≤ 20).  

\textbf{    N   }   -   \textbf{    d   }   +1 dòng tiếp theo chứa   \textbf{    N   }   -   \textbf{    d   }   +1 xâu con của s.  

\