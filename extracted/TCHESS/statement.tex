Minh Đức tuy rất giỏi lập trình nhưng chơi cờ lại hơi tệ. Sau một thời gian cố gắng mà không thành với loại cờ truyền thống, cậu quyết định sáng tạo ra một loại cờ của riêng mình: cờ tam giác. Đúng như tên gọi, bàn cờ có hình dạng một tam giác với độ dài cạnh là N, được chia làm N   $^    2   $   hình tam giác con. Hình vẽ dưới đây mô tả một bàn cờ với N = 5.  


\includegraphics{http://vn.spoj.com/content/tchess.gif}

   Một quân xe khi đứng trên một ô bàn cờ sẽ kiểm soát được tất cả những ô cùng hàng hoặc đường chéo với nó mà giữa chúng không có ô vật cản. Ở hình vẽ trên, với quân xe nằm ở ô đỏ và vật cản nằm ở ô đen, quân xe có thể kiểm soát được những ô màu xanh lá nhưng không kiểm soát được những ô màu vàng.   
\\   Yêu cầu: Cho biết vị trí của R quân xe và K ô vật cản, hãy đếm số ô trống bị kiểm soát bởi ít nhất một quân xe. Vị trí của một ô được thể hiện bằng cặp số (d,c) cho biết đó là ô thứ c tính từ trái sang trong các ô ở dòng thứ d. Ở hình vẽ trên, ô màu đen có tọa độ là (4,3).  

\