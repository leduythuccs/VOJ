Trong một bản nhạc thường có những đoạn nhạc mà tác giả sử dụng nó nhiều lần ( ít nhất 2 lần ). Những đoạn đó gọi là "đoạn cao trào". Do có thể sử dụng nhiều giọng khác nhau ( son, la, si...) nên nốt đầu tiên của các lần xuất hiện có thể khác nhau, nhưng chệnh lệnh độ cao giữa hai nốt liên tiếp thì chắc chắn giống.   
\\   VD: hai đoạn sau   
\\   1 2 5 4 10   
\\   và   
\\   4 5 8 7 13   
\\   được coi là một đoạn cao trào, vì chúng cùng sự chênh lệch độ cao : +1,+3,-1,+6   
\\   Cho một bản nhạc, yêu cầu tìm độ dài đoạn cao trào dài nhất.   
\\   + Đoạn cao trào phải có từ 5 nốt nhạc trở lên.   
\\   + Những lần xuất hiện của đoạn không được chồng lên nhau ( không có nốt nhạc chung ).   
\\

\