Đề bài
Số nguyên dương A được gọi là số lẻ hoàn toàn nếu mọi chữ số đều lẻ, chẳng hạn 9, 513, 77777. Số nguyên dương N được gọi là số đẹp nếu có thể biểu diễn N = A + B, trong đó A, B là hai số lẻ hoàn toàn. Ví dụ, 2 = 1 + 1 và 4752 = 1377 + 3375 là số đẹp, trong khi 3 và 220 thì không. Cho X, tìm số đẹp nhỏ nhất lớn hơn hoặc bằng X.  

   Đoạn 2
Giới hạn
\begin{itemize}
	\item     1  $\le$  X  $\le$  100,000,000   
\end{itemize}
Ví dụ
\begin{verbatim}
Dữ liệu
[CASE]
1

[CASE]
999

[CASE]
2000

[CASE]
4201234

[CASE]
10101010

[END]
Kết quả
2
1000
2000
4222222
10102222
\end{verbatim}