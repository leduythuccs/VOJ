Đề bài
Có N quân bài hai mặt, mỗi mặt có một số khác nhau. Mặt trước của các quân bài chứa N số phân biệt từ 1 đến N. Mặt sau của các quân bài cũng vậy.  

   Ta có thể bày N quân bài lên mặt bàn theo thứ tự bất kỳ và với mỗi quân bài có thể lật mặt trước hay mặt sau tùy ý. Đếm số cách bày bài khác nhau, biết rằng hai cách bày bài là khác nhau nếu có ít nhất một vị trí với hai lá bài tương ứng mang số khác nhau. Trả về phần dư của kết quả cho 1,000,000,007.
Giới hạn
\begin{itemize}
	\item     Số quân bài N nằm trong phạm vi từ 1 đến 50.   
\end{itemize}
Ví dụ
\begin{verbatim}
Dữ liệu
[CASE]
$<$$<$
1
2 
3
$>$$>$
$<$$<$
1 
3
2
$>$$>$

[END]
Kết quả
12
\end{verbatim}
Giải thích
Có 12 khả năng: (1,2,3), (1,3,2), (2,1,3), (2,3,1), (3,1,2), (3,2,1), (1,3,3), (3,1,3), (3,3,1), (1,2,2), (2,1,2), (2,2,1).