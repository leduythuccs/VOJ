Năm 2011, cả thế giới háo hức chờ đợi sự xuất hiện của ngôi sao chổi VO11. Người dân ở khu vực miền núi phía bắc nước ta cũng không phải ngoại lệ, tất cả mọi người đều đang đứng ở trên các sườn núi để chờ đợi khoảnh khắc sao chổi xuất hiện. Tuy nhiên, các ngọn núi có thể che khuất tầm nhìn của một số người. Nhiệm vụ của bạn là đếm xem có bao nhiêu người may mắn nhìn thấy sao chổi.

Hình dung trên mặt phẳng tọa độ, ngôi sao chổi xuất hiện ở vị trí (0,Ys) và di chuyển thẳng đến vị trí (0,Ye) trước khi biến mất (Ys $<$ Ye). Các sườn núi là một đường gấp khúc khi qua n điểm (x1,y1), (x2,y2), ..., (xn,yn), trong đó x1 = 0 $<$ x2 $<$ ... $<$ xn và y1 = 0 $<$ y2 $>$ y3 $<$ ... $>$ yn. Có k người đang đứng ở trên sườn núi với tọa độ X lần lượt là p1, p2, ..., pk (p1 $<$ p2 $<$ ... $<$ pk). Do những người này đang đứng trên sườn núi, bạn hoàn toàn có thể tính được tọa độ Y của từng người.

Bỏ qua chiều cao của mọi người, một người nhìn thấy sao chổi nếu ở một thời điểm nào đó, đoạn thẳng nối vị trí người này với vị trí sao chổi không cắt bất cứ sườn núi nào. Một đoạn thẳng nằm trên biên của một sườn núi \textbf{ không } được tính là cắt sườn núi đó.

\textbf{Input }
\begin{itemize}
	\item Dòng đầu ghi hai số N, K. (N  $\le$  50000, K  $\le$  50000, N lẻ)
	\item Dòng sau ghi hai số Ys, Ye.
	\item Trong N dòng sau, dòng thứ i ghi 2 số xi, yi.
	\item Dòng cuối ghi K số p1, p2, ... pk.
	\item Tất cả các tọa độ đều là số tự nhiên không vượt quá 1000000.
\end{itemize}