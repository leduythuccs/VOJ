HÌnh 1 mô tả một quân mã tấn công các ô trên bàn cờ vua.  

   Cho một bàn cờ vua có kích thước 3Xn, 3 hàng và n cột, trong đó 1 ≤ n ≤ 100, và một tập gồm Z ô. Các dòng được đánh số 1 đến 3 từ   trên xuống dưới, các cột được đánh số 1 đến n từ trái sang phải.  

   Các quân mã không được đặt trên các ô thuộc tập Z. Không có hai quân mã nào được tấn công lẫn nhau. Giả sử mỗi cột có nhiều nhất một ô   thuộc tập Z. Khi đó, tập Z có thể mô tả bởi dãy $k_{1}$   , $k_{2}$   ,... ,$k_{n}$   với $k_{i}$   thuộc \{0, 1, 2, 3\}. Nếu   $k_{i}$   =0, không có ô nào trên cột i thuộc tập Z, trong các trường hợp còn lại, $k_{i}$   là chỉ số dòng của ô trên cột này thuộc   tập Z.  

Yêu cầu
Cho biết số cột n của bàn cờ và dãy mô tả tập Z, hãy tìm số nhiều nhất quân mã M có thể đặt sao cho thỏa mãn các điều kiện đã nêu, và L,   số cách đặt M quân mã lên bàn cờ.