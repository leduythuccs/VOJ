Giới hạn
\begin{itemize}
	\item     Tất cả các test có    \textbf{     1    }    ≤    \textbf{     M    }    ,    \textbf{     N    }    ≤    \textbf{     1000    }
	\item     Trong 40\% test (tương ứng với 40\% số điểm),    \textbf{     1    }    ≤    \textbf{     M    }    ,    \textbf{     N    }    ≤    \textbf{     20    }
	\item      Trong quá trình thi, bài của bạn chỉ được chấm với 2 test ví dụ. Nếu được chấm đúng, kết quả sẽ được hiện là 100.    
\end{itemize}
Ví dụ
\begin{verbatim}
\textbf{Input 1:}
3 3
\\0 1 1
\\1 1 0
\\1 1 1
\end{verbatim}
\begin{verbatim}
\textbf{Output 1}
1 3
\\1
\\R 1 3
\end{verbatim}
\begin{verbatim}
\textbf{Input 2
\\}3 1
\\0
\\1
\\1\end{verbatim}
\begin{verbatim}
\textbf{Output 2
\\}0 0\end{verbatim}
Giải thích
Ở ví dụ 1, ta sẽ tiến hành đảo 2 hàng 1 và 3 cho nhau.  

    0 1 1      -$>$   1 1 1

   1 1 0  -$>$  1 1 0  

1 1 1   -$>$       0 1 1   

   Sau khi thực hiện thao tác trên ta có:  
\begin{itemize}
	\item     Hình chữ nhật có góc trái trên là (1, 1) và góc phải dưới là (1, 3) chỉ chứa toàn số 1.   
\end{itemize}

   =$>$   \textbf{    valid   }   (   \textbf{    1   }   ,   \textbf{    1   }   ,   \textbf{    1   }   ,   \textbf{    3   }   ) = true.  
\begin{itemize}
	\item     Hình chữ nhật này sẽ cho ra tổng    \textbf{     x0    }    +    \textbf{     y0    }    (= 1 + 3 = 4) là lớn nhất.   
\end{itemize}

   Nên 4 sẽ là giá trị của bảng sau khi đảo.  

   Trong ví dụ này ta sẽ không tìm được cách làm khác cho ra bảng có giá trị lớn hơn.  

   Ở ví dụ 2, vì không có cách biến đổi nào cho ra bảng với giá trị   \textbf{}\textbf{    $>$   }\textbf{    max   }   (3, 1) = 3.  

   Nên kết quả xuất ra là   \textbf{    0 0   }   .