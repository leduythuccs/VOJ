Ngày xửa ngày xửa, quần đảo của Pirate bị chia cắt bởi các thế lực cát cứ. Cuối cùng, có 3 thế lực mạnh nhất đã thâu tóm hết các nước lân bang để tạo thành thế Tam Đảo: đó là nước Rỉ, nước Quá và nước Đanh. Giữa ba nước này cứ diễn ra chọi dừa liên miên khiến dân chúng u đầu mẻ trán như cơm bữa, cuộc sống lầm than không sao kể xiết.  

   Pirate sinh ra trong thời loạn, xuất thân từ một gia đình trồng chuối, từ nhỏ đã nối tiếng với tài vừa xem đá bóng vừa... làm thơ. Thấy con trai mình có tài văn chương thiên bẩm, phụ thân Pirate mời cả "Bắc Cương, Nam Vũ" đến dạy chữ cho anh, những mong sau này con trai có thể trở thành một nhà thuyết khách, đi khắp Tam Đảo giảng hòa ba nước, tạo phúc cho bá tánh.  

   Sao bao nhiêu sóng gió, cuối cùng, bằng việc hứa sẽ truyền cho các chúa đảo về bí quyết trồng chuối to, Pirate đã đạt được một thỏa thuận hòa bình cho Tam Đảo. Đó là sự hình thành một "vùng ngừng chọi". Pirate kéo tấm bản đồ các hòn đảo trong quần đảo ra. Ở trên đó, mỗi đảo được kí hiệu bởi một điểm nguyên trên mặt phẳng tọa độ và bị duy nhất một trong ba nước chiếm giữ. "Vùng ngừng chọi" sẽ có hình dạng một tam giác với ba đỉnh là ba hòn đảo trên tấm bản đồ này mà trong đó không có hai hòn đảo nào bị chiếm giữ bởi cùng một nước. Ba nước cam đoan sẽ không chọi dừa trong phần diện tích che phủ bởi "vùng ngừng chọi".  

   Pirate chỉ muốn "vùng ngừng chọi" có diện tích lớn nhất để bá tánh bớt lầm than và có thể chăm lo trồng chuối. Hãy giúp anh ấy xác định diện tích lớn nhất có thể của "vùng ngừng chọi"!