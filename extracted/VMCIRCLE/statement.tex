Bọn trẻ con trong xóm của Dũng đang thách đố nhau một trò rất quái dị. Bọn con gái dùng phấn trắng vẽ   \textbf{    N   }   vòng tròn nhỏ (đánh số từ   \textbf{    1 đến N ngược chiều kim đồng hồ   }   ) xung quanh 1 vòng tròn lớn. Sau đó chúng đố bọn con trai dùng ít loại phấn màu nhất để vẽ   \textbf{    2N   }   đoạn nối: N đoạn nối vòng tròn to với N vòng tròn nhỏ và N đoạn nối vòng tròn nhỏ số i với vòng tròn nhỏ số   \textbf{    (i mod N)+1   }   , sao cho giữa 2 vòng tròn (i,j) bất kì luôn tồn tại một đường đi từ vòng tròn i đến vòng tròn j đi qua các đoạn nối mà không có 2 đoạn nối nào có màu giống nhau.  

   Với những trường hợp N nhỏ, bọn trẻ không gặp vấn đề gì. Nhưng với N lớn, bọn con trai không tìm ra cách giải, bọn con gái cũng không đủ sức mà vẽ hết N vòng tròn. Cuối cùng chúng nhờ Dũng tìm xem có thuật toán nào giải được trò chơi của chúng với mọi N hay không. Nếu có, chúng sẽ không cần phải chơi làm gì nữa cho mệt.  

   Dũng đang rất bận nên nhờ bạn tìm hộ thuật toán. Nếu có, Dũng cũng không cần phải nghĩ làm gì nữa cho mệt.