Bé năm nay 15 tuổi, học hết lớp 9, ở cái tuổi mà Bé đã bắt đầu có những tò mò về chuyện giải toán. Hàng ngày, Bé luôn mơ ước được ngắm nhìn những bài toán hóc búa, quyến rũ. Biết Bé đam mê giải toán, cô giáo vui lắm. Hôm nay cô cho Bé một bài toán:  

   Cho một bảng ô vuông kích thước   \textbf{    M   }   x   \textbf{    N   }   (M và N chẵn). Trên bảng có 2 ô cấm (   \textbf{    x   }   ,   \textbf{    y   }   ) và (   \textbf{    u   }   ,   \textbf{    v   }   ). Yêu cầu: đặt các quân domino lên bảng, sao cho mỗi ô trên bảng có đúng 1 viên domino đặt lên, và không có quân domino đặt lên ô cấm. Quân domino có dạng hình chữ nhật 1 x 2 hoặc 2 x 1.   
\\
\\   Dĩ nhiên, việc tìm 1 cách đặt với Bé là quá dễ, vì vậy, cô yêu cầu Bé phải tìm được 10 cách đặt khác nhau thì mới được điểm 10 (mỗi cách 1 điểm). Bạn có thể giúp bé được bao nhiêu điểm? Hai cách đặt được gọi là khác nhau nếu tồn tại hai quân domino (mỗi quân thuộc một cách) có đúng 1 ô chung. Ví dụ quân domino đặt ở vị trí (1, 1) - (1, 2) và quân domino đặt ở vị trí (1, 2) - (2, 2) là hai quân domino có đúng 1 ô chung (1, 2).