Trong mỗi file input, bạn được cho một mô tả của một mê cung, theo định dạng dưới đây:
\begin{itemize}
	\item Dòng đầu tiên là số nguyên dương \textbf{ N } - kích thước của mê cung.
	\item Dòng thứ hai chứa 4 số nguyên \textbf{ sx sy fx fy } , cách nhau bởi ít nhất một dấu cách.
	\item Dòng thứ ba là số nguyên dương \textbf{ M } - số lượng xác ướp trong mê cung.
	\item M dòng tiếp theo mô tả các xác ướp. Dòng thứ i gồm 3 số nguyên \textbf{ $x_{i}$}\textbf{ $y_{i}$}\textbf{ $z_{i}$} , trong đó ($x_{i}$ , $y_{i}$ ) là ô ban đầu của xác ướp, và $z_{i}$ là loại xác ướp ($z_{i}$ = 0 hoặc 1).
	\item Dòng tiếp theo là số nguyên \textbf{ WH } - số bức tường ngang.
	\item Tiếp theo là WH dòng, dòng thứ i là cặp số nguyên \textbf{ $u_{i}$ v }$_\textbf{ i }$ , cho biết có một bức tường ngăn cách giữa ô ($u_{i}$ , $v_{i}$ ) và ô ($u_{i}$ + 1, $v_{i}$ ). Ô ($u_{i}$ , $v_{i}$ ) và ô ($u_{i}$ + 1, $v_{i}$ ) luôn nằm trong mê cung.
	\item Dòng tiếp theo là số nguyên \textbf{ WV } - số bức tường dọc.
	\item Tiếp theo là WV dòng, dòng thứ i là cặp số nguyên \textbf{ $u_{i}$ $v_{i}$} , cho biết có một bức tường ngăn cách giữa ô ($u_{i}$ , $v_{i}$ ) và ô ($u_{i}$ , $v_{i}$ + 1). Ô ($u_{i}$ , $v_{i}$ ) và ô ($u_{i}$ , $v_{i}$ + 1) luôn nằm trong mê cung.
\end{itemize}