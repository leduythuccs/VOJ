Giới hạn
\begin{itemize}
	\item \textbf{1 ≤ N ≤ $10^{5}$ . }
	\item \textbf{1 ≤ $C_{i}$ ≤ $10^{9}$ . }
	\item \textbf{20\% } số test, \textbf{ N ≤ 100. }
	\item \textbf{15\% } số test, \textbf{ N ≤ $10^{5}$} , nhân viên thứ \textbf{ x } là cấp trên của nhân viên thứ \textbf{ x+1 } , với \textbf{ 1 ≤ x $<$ N. }
	\item \textbf{65\% } số test, \textbf{ N ≤ $10^{5}$ . }
	\item Trong quá trình thi, bài của bạn sẽ chỉ được chấm với test ví dụ.
\end{itemize}
Ví dụ
\paragraph{Input}
\begin{verbatim}
5
3
1 2
1 2
3 2
3 1\end{verbatim}

\paragraph{Output}
\begin{verbatim}
6\end{verbatim}
Giải thích
Sơ đồ biểu diễn cấp bậc của các nhân viên sẽ như sau:

  1


 /  $\backslash$


2    3


     /  $\backslash$


   4    5

Ta có 6 bộ ba như sau:
\begin{enumerate}
	\item \textbf{(1, 2, 3) } - Nhân viên thứ nhất có lương bằng 3 lớn hơn nhân viên thứ 2 và thứ 3 với tiền lương của mỗi người là 2. Nhân viên 1 đều là cấp trên của nhân viên 2 và 3.
	\item \textbf{(1, 2, 4) } - Tiền lương của nhân viên thứ 2 và thứ 4 đều là 2 và nhỏ hơn tiền lương của CEO và cả 2 đều là cấp dưới.
	\item \textbf{(1, 2, 5) }
	\item \textbf{(1, 3, 4) }
	\item \textbf{(1, 3, 5) }
	\item \textbf{(1, 4, 5) }
\end{enumerate}

Bộ \textbf{ (3, 4, 5) } không được tính vì lương của nhân viên thứ 3 chỉ bằng chứ không cao hơn tiền lương của nhân viên thứ 4.