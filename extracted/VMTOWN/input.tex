Dòng đầu tiên là 2 số nguyên dương N và M – Số lượng thành phố và số lượng con đường của đất nước X.  

   M dòng tiếp theo chứa 3 số nguyên dương u, v, t (u ≠ v, 1 ≤ u, v ≤ N) mang ý nghĩa có con đường kết nối 2 thành phố u, v, và nếu t = 0 thì con đường này là con đường trọng điểm, t = 1 thì đây là con đường bình thường. Hai thành phố không có quá 1 con đường kết nối chúng.  

   Dòng tiếp theo chứa số nguyên dương Q – số lượng các cải tổ hệ thống đường xá của KrK (xóa bỏ hoặc xây dựng)  

   Q dòng tiếp theo chưa 2 số nguyên dương u, v (u ≠ v, 1 ≤ u, v ≤ N). Nếu tồn tại con đường giữa u và v, nghĩa là phá hủy con đường kết nối u, v. Ngược lại nếu chưa tồn tại con đường kết nối u, v nghĩa là xây dựng thêm con đường nối 2 thành phố u và v.  

   Input luôn thỏa mãn:  
\begin{itemize}
	\item     Ban đầu có đúng N-1 con đường loại 0 (Con đường trọng điểm)   
	\item     Các con đường trọng điểm không bao giờ bị xóa đi trong các truy vẫn của KrK.   
\end{itemize}