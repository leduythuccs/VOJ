Đề bài
Cho một mê cung kích thước \textbf{ M*N } , được chia thành các hình vuông đơn vị cạnh 1*1. Các hàng của mê cung được đánh số từ 1 đến M từ trên xuống dưới, các cột của mê cung được đánh số từ 1 đến N từ trái sang phải. Ô ở hàng i, cột j được kí hiệu là ô (i, j). Mỗi ô của mê cung là tường hoặc ô trống.

Cho một xâu \textbf{ S } độ dài \textbf{ L } là chỉ dẫn đường đi, gồm các chữ số từ 1 đến 8, tương ứng với các hướng từ 1 đến 8 như bảng sau:
\begin{verbatim}

\texttt{8 1 2
7   3
6 5 4}\end{verbatim}

Vào ngày thứ i (1 ≤ i ≤ L), hoàng tử sẽ đứng im, hoặc là di chuyển sang một ô kề cạnh hoặc kề đỉnh, theo hướng là ký tự thứ i của xâu S (các ký tự của xâu S được đánh số từ 1 đến L).

Cho biết vị trí ban đầu của hoàng tử (vị trí này luôn luôn là ô trống). Tìm tất cả các ô (i, j), mà sau đúng L ngày, hoàng tử có thể ở ô đó. Chú ý rằng hoàng tử chưa bao giờ thoát khỏi mê cung, nên bạn có thể bỏ qua tất cả các nước đi khiến hoàng tử đi ra ngoài mê cung. Bạn cũng cần bỏ qua tất cả các nước di chuyển vào những ô tường, vì hoàng tử không có khả năng đi xuyên tường.
Giới hạn
\begin{itemize}
	\item 40\% test có M, N, L ≤ 100.
	\item 60\% test còn lại có M, N ≤ $10^{3}$ , L ≤ $10^{6}$
\end{itemize}
Chấm điểm
\begin{itemize}
	\item Trong thời gian thi, bài của bạn sẽ chỉ được chấm với duy nhất 1 test có trong đề bài.
\end{itemize}
Ví dụ
\begin{verbatim}
\textbf{Input:}
3 4
2 3
....
....
....
37\end{verbatim}
\begin{verbatim}
\textbf{Output:}
0000
0111
0000
\end{verbatim}