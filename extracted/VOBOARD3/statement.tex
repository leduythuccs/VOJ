Tuy mới 10 tuổi, nhưng Bé đã rất thích chơi với các số nhị phân. Hàng ngày, Bé thường lấy 1 tấm bảng kích thước M*N và chơi một trò chơi như sau:  
\begin{itemize}
	\item     Ban đầu, trên mỗi ô của bảng ghi một bit 0 hoặc 1.   
	\item     Mỗi nước đi, Bé chọn một vùng hình chữ nhật nằm hoàn toàn trong bảng, có các cạnh song song với các cạnh của bảng, và có kích thước R*C, và đảo bit tất cả các ô trong vùng đó.   
	\item     Mục tiêu của Bé là đưa tất cả các ô trên bảng về bit 0.   
	\item     Bé không thích sự lặp lại, vì thế Bé sẽ không bao giờ thực hiện 2 nước đi (R1, C1) và (R2, C2) mà R1 = R2 và C1 = C2 (Nói cách khác, Bé không bao giờ thực hiện 2 nước đi có cùng cả R và C).   
\end{itemize}

   Hôm nay Bé bị ốm, không thể chơi được. Bạn hãy giúp bé chơi trò chơi này nhé.  

   Các hàng của bảng được đánh số từ 1 đến M từ trên xuống dưới, các cột của bảng được đánh số từ 1 đến N từ trái sang phải. Ô ở hàng i, cột j của bảng được kí hiệu là ô (i, j).