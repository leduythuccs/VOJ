Ngoài việc chơi với các bảng số 01, Bé còn rất thích chơi với những dãy ngoặc. Hôm nay, mẹ cho Bé N dãy ngoặc. Bé muốn tạo được một dãy ngoặc đúng bằng việc nối một số dãy ngoặc lại với nhau. Bạn hãy giúp bé nhé.  

   Một dãy ngoặc đúng được định nghĩa theo kiểu đệ quy như sau:  
\begin{itemize}
	\item     Dãy rỗng - dãy không gồm ký tự nào - là dãy ngoặc đúng.   
	\item     Nếu X là một dãy ngoặc đúng, thì (X) cũng là một dãy ngoặc đúng. Ví dụ, vì X = ()() là một dãy ngoặc đúng nên (()()) cũng là một dãy ngoặc đúng.   
	\item     Nếu X và Y là hai dãy ngoặc đúng, thì XY là một dãy ngoặc đúng. Ví dụ, vì X = (()) và Y = () là các dãy ngoặc đúng, nên XY = (())() cũng là dãy ngoặc đúng.   
\end{itemize}

   Một số ví dụ về các dãy không phải là dãy ngoặc đúng: )(, (())), ((()...