 

Bờm có một cái sân hình chữ nhật có các kích thước là \textbf{ M } và \textbf{ N } mét. Khu vườn được chia thành \textbf{ M x N } ô vuông đơn vị \textbf{ 1 x 1 } mét vuông. Các hàng được đánh số liên tục từ \textbf{ 1 } đến \textbf{ M } , các cột được đánh số liên tục từ \textbf{ 1 } đến \textbf{ N } . Ô tại hàng \textbf{ i } cột \textbf{ j } sẽ có tọa độ \textbf{ [i, j] } .

Sắp đến giáng sinh, Bờm đã trang trí cho cái sân của mình một \emph{ hệ thống đèn } . \emph{ Hệ thống đèn } có thiết kế như sau:
\begin{itemize}
	\item Bờm làm một \emph{ hệ thống treo đèn } có kích thước \textbf{ M } , \textbf{ N } và cách mặt sân \textbf{ K + 1 } mét.
	\item Tại tâm của mỗi hình vuông đơn vị, \emph{ hệ thống treo } sẽ cho xuống đất một \emph{ dây đèn } .
	\item Trên mỗi \emph{ dây đèn } sẽ gắn đúng \textbf{ K } đèn, đèn thứ \textbf{ 1 } , \textbf{ 2 } , ..., \textbf{ i } , ..., \textbf{ K } sẽ lần lượt gắn tại các độ cao \textbf{ 1 } , \textbf{ 2 } , ..., \textbf{ i } , ..., \textbf{ K } mét.
	\item Đèn nằm trên ô vuông đơn vị có tọa độ \textbf{ [i, j] } , tại độ cao \textbf{ u } sẽ có tọa độ là \textbf{ [u, i, j]ta } trong hệ thống đèn.
\end{itemize}
\begin{itemize}
\end{itemize}

Bờm cũng đã thiết thế cho mỗi đèn một \emph{ công tắc } . Hệ thống có đến \textbf{ K x M x N }\emph{ công tắc } , mỗi lần muốn mở cả hệ thống là phải tốn rất nhiều thời gian. Nên Bờm đã lập trình ra một \emph{ hệ thống hỗ trợ mở đèn } . Các bước sử dụng như sau:
\begin{enumerate}
	\item Bờm sẽ chọn một số đèn để mở.
	\item Sau đó, Bờm mở \emph{ hệ thống hỗ trợ } . Khi \emph{ hệ trống hỗ trợ } được mở, tắc cả các \emph{ công tắc } sẽ bị liệt và không còn sử dụng được nữa (để tránh việc sinh ra các lỗi trong quá trình hệ thống đang chạy). Nhưng bù lại \emph{ hệ thống hộ trợ } sẽ mở thêm một số đèn theo nguyên tắc sau:
\end{enumerate}
\begin{itemize}
	\item Nếu 2 đèn tại tọa độ [ \textbf{ x } , y1, z1] và [ \textbf{ x } , y2, z2] đã mở thì đèn tại [ \textbf{ x } , y1, z2] và [ \textbf{ x } , y2, z1] sẽ tự động mở.
	\item Nếu 2 đèn tại tọa độ [x1, \textbf{ y } , z1] và [x2, \textbf{ y } , z2] đã mở thì đèn tại [x1, \textbf{ y } , z2] và [x2, \textbf{ y } , z1] sẽ tự động mở.
	\item Nếu 2 đèn tại tọa độ [x1, y1, \textbf{ z } ] \textbf{} và [x2, y2, \textbf{ z } ] đã mở thì đèn tại [x1, y2, \textbf{ z } ] \textbf{} và [x2, y1, \textbf{ z } ] sẽ tự động mở.
\end{itemize}
\begin{itemize}
\end{itemize}

Hiện tại Bờm đã mở một số đèn. Bạn hãy mở thêm ít nhất các đèn, để sau khi mở \emph{ hệ thống hỗ trợ } tắc cả đèn đều sáng.


\includegraphics{http://www.spoj.com/content/voj:VOLIGHTS.png}

Hình trên mô tả trạng thái các đèn tại cùng \textbf{\emph{ độ cao x }} . Khi 2 đèn \textbf{\emph{ vàng [x, y1, z1] }}\emph{ và }\textbf{\emph{ [x, y2, x2] }} đã mở, thì 2 đèn \textbf{\emph{ xanh [x, y1, z2] }}\emph{ và }\textbf{\emph{ [x, y2, z1] }} sẽ tự mở (sau khi mở \emph{ hệ thống hỗ trợ } ).

 