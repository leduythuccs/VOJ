\begin{verbatim}
\textbf{Input:}
3 4
2 1 2
2 3 3
1 1 1
3 3 3\end{verbatim}
\begin{verbatim}
\textbf{Output:}
54
2
3
0
3\end{verbatim}

\textbf{Giải thích : }

 

\emph{Cho khối lập phương 3x3x3. Con cáo ở ô (2,1,2) có thể nhảy tới (1,1,2) và sau đó là (1,1,1); con cáo ở (2,3,3) nhảy tới }

\emph{(1,3,3), (1,1,3) rồi (1,1,1). Con cáo ở (1,1,1) ở sẵn vị trí. Con cáo ở (3,3,3) nhảy tới (1,3,3), (1,1,3) rồi (1,1,1). Có thể }

\emph{có một số cách dịch chuyển khác với số lượt nhảy tương tự. }

\emph{Tổng số bước nhảy của 27 con cáo là 54. }

 

 

\emph{Từ 8 đến 1 có thể đi như sau: 8 -$>$ 9 -$>$ 1 }

 

\textbf{Giới hạn : }

1 $\le$ a,b,c $\le$ N $\le$ 10000

0 $\le$ K $\le$ 10000

Ngoài ra, trong 50\% test, N $\le$ 100.