Cho 2 dãy a[1..N] và b[1..M]. Gọi c[1..k] là 1 dãy con chung (không cần liên tiếp) bất kì của 2 dãy này. Đặt f(c) = abs(c[2] - c[1]) + abs(c[3] - c[2]) + .. + abs(c[k] - c[k - 1]). Nếu số phần tử của c $<$ 2 thì f(c) = 0.

Xác định dãy con chung có giá trị f lớn nhất và in ra giá trị đó.