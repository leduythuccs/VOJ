Một số nguyên X được gọi là số may mắn nếu ước nguyên tố của X chỉ có 2 và 5. Ví dụ các số 2, 10, 25... là các số may mắn, ngược lại các số 09, 69, 15, 24 không là số may mắn.  Cho một xâu S có độ dài N, chỉ gồm các số từ 0 tới 9. Hãy tính số chuỗi con gồm các kí tự liên tiếp của S, tạo thành một số không có các chữ số 0 vô nghĩa ở đầu và chia hết cho một số may mắn X cho trước.  

\