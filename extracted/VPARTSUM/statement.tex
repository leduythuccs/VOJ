Cho một dãy gồm N (1 ≤ N ≤ 100000) số nguyên dương a   $_    1   $   , a   $_    2   $   , ..., a   $_    n   $   . Tổng a   $_    i   $   + a   $_    i+1   $   + ... + a   $_    j   $   (1 ≤ i ≤ j ≤ N) được gọi là tổng bộ phận từ i đến j của dãy số.  

   Cho hai số nguyên dương P và K (1 $<$ P ≤ 10   $^    9   $   , 0 ≤ K $<$ P). Hãy tìm tổng bộ phận theo modulo P nhỏ nhất không bé hơn K.  

   Ví dụ, xét dãy số sau:  
\begin{verbatim}
12     13     15     11     16     26     11
\end{verbatim}

   Ở đây N=7, giả sử K=2 và P=17, ta có kết quả là 2 vì 11 + 16 + 26 = 53 và 53 mod 17 = 2. Nếu K=0 ta có kết quả bằng 0 vì 15 + 11 + 16 + 26 = 68 và 68 mod 17 = 0.  

\