



   Xét số nguyên dương   \textbf{x}   . Gọi   \textbf{S(x)}   là hàm tính tổng các chữ số của   \textbf{x}   (trong dạng biểu diễn cơ số 10). Ví dụ,   \textbf{S}   (21) = 2+1 = 3.  

   Cho số nguyên dương   \textbf{n}   . Ta có thể biểu diễn   \textbf{n}   dưới dạng tổng của   \textbf{k}   số nguyên   \textbf{a}$_    1   $   ,   \textbf{a}$_    2   $   , . . .,   \textbf{$a_{k$}}   .  

\textbf{Yêu cầu}   : Cho hai số nguyên dương   \textbf{n}   và   \textbf{m}   (   \textbf{n}   ,   \textbf{m}   ≤ $10^{12}$   ). Hãy xác định   \textbf{k}   nhỏ nhất, sao cho với nó tồn tại các số   \textbf{a}$_    1   $   ,   \textbf{a}$_    2   $   , . . .,   \textbf{$a_{k}$}   thỏa mãn:  
\begin{itemize}
	\item     a1+a2+ ...+ak= N   
	\item     S(a1)+ S(a2)+...+S(ak)= M   
\end{itemize}

\textbf{Dữ liệu}   : 2 dòng chứa hai số nguyên   \textbf{n}   và   \textbf{m}   .  

\textbf{Kết quả}   : kết quả đưa ra trên một dòng dưới dạng số nguyên. Nếu không tồn tại cách phân tích thì đưa ra số -1.  

   VD:  

   Input  

   100  

   1  

   Output  

   1  
