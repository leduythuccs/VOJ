



   Dãy ngoặc đúng được định nghĩa như sau:  
\begin{itemize}
	\item     Biểu thức rỗng là biểu thức ngoặc đúng có bậc bằng 0.   
	\item     Nếu A là biểu thức ngoặc đúng có bậc bằng k thì (A), [A], \{A\} cũng là biểu thức ngoặc đúng có bậc k+1.   
	\item     Nếu A và B tương ứng là hai biểu thức ngoặc đúng có bậc là kA, kB thì AB cũng là một biểu thức ngoặc đúng có bậc bằng max(kA, kB).   
\end{itemize}

   Ví dụ "()[()]" là một biểu thức ngoặc đúng có bậc bằng 2.  

   Với hai số n, k người ta tiến hành tạo ra tất cả các biểu thức ngoặc đúng có độ dài đúng bằng n và có bậc không quá k. Sắp xếp các biểu thức ngoặc này theo thứ tự từ điển, chú ý: '(' $<$ ')' $<$ '[' $<$ ']' $<$ '\{' $<$ '\}'.  

\textbf{      Yêu cầu     }   : Cho n, k và S là một biểu thức ngoặc đúng độ dài n và có bậc không quá k, hãy tìm thứ tự của S.  

\textbf{     Input    }   :  
\begin{itemize}
	\item     Dòng đầu chứa hai số nguyên n, k.   
	\item     Dòng thứ hai chứa xâu S.   
\end{itemize}

\textbf{      Output     }   :  
\begin{itemize}
	\item     Một dòng duy nhất chứa một số nguyên là thứ tự của biểu thức ngoặc đúng S.   
\end{itemize}

\textbf{      Example     }   :  

    INPUT      :   


   6 2   


   (())\{\}  

    OUTPUT      :   


   4  

   Giải thích:   


   1. "(()())"   


   2. "(())()"   


   3. "(())[]"   


   4. "(())\{\}"  

\textbf{      Ràng buộc     }   :  
\begin{itemize}
	\item     n chẵn   
	\item     2*k  $\le$  n   
\end{itemize}

\textbf{      Giới hạn     }   :  
\begin{itemize}
	\item     Trong 10\% test đầu tiên: n, k  $\le$  10.   
	\item     Trong 20\% test tiếp theo: n  $\le$  20, k  $\le$  5.   
	\item     Trong 30\% test tiếp theo: n  $\le$  100, k  $\le$  5.   
	\item     Trong 40\% test còn lại: n, k  $\le$  100.   
\end{itemize}

