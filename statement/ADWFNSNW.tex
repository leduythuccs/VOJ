







   Hôm nay RomanD3 trên đường đua về Đồ Sơn với hai nhóc lớp 10 bỗng nổi hứng phán một bài như sau:  

   Trên chặng đường đạp xe vất vả về Đồ Sơn, đội đua do chú lùn RomanD3 dẫn đầu gồm N nàng Bạch tuyết đi trên N chiếc xe đạp khác nhau, nhưng các nàng không ai chịu đi một mình cả :| Biết rằng nàng Bạch Tuyết thứ i muốn được chàng lùn nhà ta chở trong $T_{i}$   phút, sau đó nàng có thể tự đi được trong $D_{i}$   phút, sau đó mà không được chở tiếp là nàng sẽ khóc :((; tuy vậy tình yêu đích thực của chàng RomanD3 lại là nàng Bạch Tuyết trên chiếc xe bus hồng Thịnh Hưng, để gặp được nàng, chàng ta phải chở mỗi nàng đi xe đạp đúng một lần, sau đó cần ít nhất 1 phút để lên bus!  

   Hãy giúp RomanD3 tính toán xem có kịp hay không!  

\subsubsection{   Dữ liệu  }

   - Dòng đầu tiên chứa số N.   
\\   - Trong N dòng tiếp theo, dòng thứ i chứa hai số $T_{i}$   , $D_{i}$   .  

\subsubsection{   Kết quả  }

   - In ra -1 nếu chàng lùn không thể kịp, hoặc in ra thứ tự chở N nàng Bạch Tuyết trên cùng một dòng.  

\subsubsection{   Ví dụ  }
\begin{verbatim}
\textbf{Dữ liệu:}
2
2 11
11 30

\textbf{Kết quả:}
2 1
\end{verbatim}

\subsubsection{   Giới hạn  }

   - n ≤ $10^{5}$   .  

