







   Cho một bàn cờ gồm m ô được xếp thành một vòng tròn, đánh số từ 1 đến m. Trên bàn cờ có b quân cờ trắng và c quân cờ đen, có tối đa 1 quân cờ trong 1 ô. Hai người chơi trò chơi này như sau: Người cầm quân trắng bắt đầu trước, 2 người lần lượt thực hiện nước đi của mình. Mỗi nước đi, người chơi được quyền di chuyển quân cờ của mình tiến lên hoặc lùi đi một số ô trống. Ví dụ, trong hình vẽ dưới đây, người chơi cầm quân trắng có thể di chuyển quân từ ô 3 đến ô 4, hoặc quân từ ô 8 đến một trong các ô 7, 9, và 1.  


\includegraphics{http://vn.spoj.pl/content/grarys.jpg}

   Người chơi nào không thể thực hiện được nước đi của mình là người thua cuộc. Gỉả sử cả 2 người đề chơi tối ưu, hỏi ai là người chiến thắng? Có thể xảy ra trường hợp không có ai thắng (trò chơi không bao giờ kết thúc).  

\subsubsection{   Input  }

   Dòng đầu chứa số nguyên t là số lượng bộ test.  

   Các dòng tiếp theo mô tả lần lượt t bộ test, mỗi bộ gồm 3 dòng. Dòng đầu tiên chứa 3 số nguyên m, b và c (1 ≤ m ≤ $10^{9}$   , 1 ≤ b, c) thể hiện độ dài của bàn cờ, số lượng quân trắng, và số lượng quân đen. Dòng thứ hai chứa một dãy số tăng dần gồm b số nguyên (trong khoảng 1, . . . ,m) thể hiện vị trí của các quân cờ trắng. Dòng thứ 3 chứa một dãy số tăng dần gồm c số nguyên (trong khoảng 1, . . . ,m) thể hiện vị trí của các quân cờ đen. Tổng số quân cờ trên bàn cờ không vượt quá $10^{6}$   .  

\subsubsection{   Output  }

   Gồm đúng t dòng, ghi kết quả của t test. Kết quả là một trong số 3 loại kí tự: B, C, hoặc R, tuỳ thuộc vào việc người cầm quân trắng thắng (B), người cầm quân đen thắng (C) hay trò chơi không bao giờ kết thúc (R).  

\subsubsection{   Example  }

   Với dữ liệu:  
\begin{verbatim}
3
9 2 3
3 8
2 5 6
6 2 2
5 6
2 4
7 1 1
3
4
\end{verbatim}

   Kết quả đúng là:  
\begin{verbatim}
C
B
R
\end{verbatim}

