



   Cho n điểm đôi một phân biệt trên mặt phẳng (n ≥ 3). Có n(n-1)(n-2)/6 tam giác có các đỉnh là 3 đỉnh phân biệt trong số n điểm này (bao gồm cả những tam giác bị suy biến, nghĩa là khi cả 3 đỉnh thẳng hàng).  

   Ta cần tính tổng diện tích của các tam giác này. Phần mặt phẳng thuộc về nhiều tam giác sẽ được tính nhiều lần. Quy ước diện tích của các tam giác suy biến là 0.  

\subsubsection{   Dữ liệu  }

   Dòng đầu tiên chứa một số nguyên n (3 ≤ n ≤ 1000) cho biết số lượng điểm. Mỗi dòng trong số n dòng sau chứa 2 số nguyên $x_{i}$   và $y_{i}$   (0 ≤ $x_{i}$   , $y_{i}$   ≤ 10000) phân cách bởi một khoảng trắng xác định tọa độ của điểm thứ i (với i=1,2,...,n). Không có cặp (thứ tự) tọa độ nào xuất hiện nhiều hơn một lần.  

\subsubsection{   Kết quả  }

   In ra một số thực duy nhất bằng tổng diện tích của các tam giác có các đỉnh nằm trong n điểm đã cho. Kết quả phải được in ra với đúng một chữ số thập phân và không được chênh lệch với kết quả đúng nhiều hơn 0.1.  

\subsubsection{   Ví dụ  }
\begin{verbatim}
Dữ liệu
5
0 0
1 2
0 2
1 0
1 1

Kết quả
7.0
\end{verbatim}
\includegraphics{http://vn.spoj.pl/content/areatri.jpg}
