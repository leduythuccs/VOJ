

 

Trong một hội thi Ballgame, ban tổ chức chuẩn bị một bàn lớn. Trên mặt bàn có n bi xanh đánh số từ 1 đến n và n bi đỏ đánh số từ n + 1 đến 2n. Mỗi trận đấu, các vận động viên sẽ chơi luân phiên nhau. Đến lượt chơi của mình, Hùng cần tìm 3 bi mà vị trí của chúng là thằng hàng hanu và sao cho trong số đó có hai bi đỏ và 1 bi xanh (khi đó ăn được một bi đỏ), hoặc là có hai bi xanh và 1 bi đỏ (khi đó được ăn 1 bi xanh).

\subsubsection{Yêu cầu}

Cho biết tọa độ trên mặt phẳng tọa độ Đề-các của vị trí và màu của các bi hiện tại trên bàn, bạn hãy giúp Hùng chọn 3 bi để chơi.

\subsubsection{Input}
\begin{itemize}
	\item Dòng đầu ghi số nguên dương n.
	\item Dòng thứ i trong số n dòng tiếp theo ghi hai số nguyên là hoành độ và tung độ trên mặt phẳng tọa độ Đề-các của vị trí đặt bi xanh với chỉ sô i.
	\item Dòng thứ i trong số n dòng cuối cùng ghi hai số nguyên là hoàng độ và tung độ trên mặt phẳng tọa độ Đề-các của vị trí đặt bi đỏ với chỉ số n + i.
	\item Hoàng độ và tung độ không vượt quá 10^6, vị trí các bi là đôi một phân biệt.
\end{itemize}

\subsubsection{Giới hạn}
\begin{itemize}
	\item 30\% số test có n  $\le$  2;
	\item 30\% số test khác có n  $\le$  100.
	\item 40\% số test còn lại có n  $\le$  1000.
\end{itemize}

\subsubsection{Output}

Ghi ra 3 chỉ số của các viên bi mà Hùng cần chọn, nếu không thể chọn được 3 bi nào, ghi ra -1. Nếu có nhiều đáp án, ghi ra một đáp án bất kì.

\subsubsection{Example}
\begin{verbatim}
\textbf{Input:}
3
1 1
2 2
4 9
3 3
6 20
8 100

\textbf{Output:}
1 2 4
\end{verbatim}
