
\begin{verbatim}



\textbf{BỘI CHUNG NHỎ NHẤT}

 

Cho một số nguyên N, bạn hãy tính \textbf{bội chung nhỏ nhất} của tất cả các số nguyên từ \textbf{1} đến \textbf{N}.

 

Bội chung nhỏ nhất của một tập các số nguyên là \textbf{số nguyên dương nhỏ nhất} mà chia hết cho tất cả các số trong tập đó.

 

Ví dụ : BCNN(2, 5, 4) = 20, BCNN(3, 9) = 9, BCNN(6, 8, 12) = 24.

 

\textbf{Input}

Dòng đầu tiên chứa số nguyên \textbf{T} (≤10000) là số bộ test.

Tiếp theo là T dòng, mỗi dòng chứa một số nguyên \textbf{N} (2 ≤ N ≤ 10$^8$)

 

\textbf{Output}

Với mỗi test, in ra kết quả trên một dòng, theo định dạng như trong ví dụ. Vì kết quả có thể rất lớn nên chỉ cần in ra phần dư của kết quả cho 2$^32$

 

\textbf{Ví dụ :}
\begin{tabular}\hline 


Input & 

Output  
\hline


5

10

5

200

15

20 & 

Case 1: 2520

Case 2: 60

Case 3: 2300527488

Case 4: 360360

Case 5: 232792560  
\hline

\end{tabular}

 \end{verbatim}
