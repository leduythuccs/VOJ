

Năm 2112, nước ta đã trở thành một cường quốc kinh tế. Giá trị đồng tiền tăng lên 100 lần (200VNĐ tương ứng vói 20.000VNĐ vào năm 2012). Tuy mạnh về kinh tế nhưng giá điện lại vô cùng đắt đỏ. Công NVE là nhà cung cấp điện duy nhất trong thành phố Nam ở. NVE vừa tăng giá điện, bảng giá như sau :

 


\includegraphics{https://drive.google.com/uc?export=view&amp;id=1Qbqwyiv5OW2BoS5weLx4uNukZOrMRH-c}

 

Cách tính : 100kWh đầu tiên có giá 200VNĐ mỗi kWh, 9900  kWh tiếp theo (từ 101-10000) có giá 300 VNĐ mỗi kWh. Cứ như vậy tính tiếp. Ví dụ nếu sử dụng 10050 kWh thì Nam phải trả 200x100 + 300x9900 + 500 x 50 = 3.015.000 VNĐ.

Lợi dụng độc quyền cung cấp điện, NVE thường xuyên tìm cách lừa khách hàng (báo sai hóa đơn chả hạn) để moi thêm tiền. Tuy nhiên do Nam rất giỏi tính toán nên chưa bị lừa lần nào. Điều này làm NVE rất tức và quyết định bắt bí bằng được. Thay vì gửi hóa đơn báo lượng điện tiêu thụ và chi phí của Nam, NVE gửi hóa đơn cho Nam và hàng xóm là Việt báo 2 con số :

X : tổng số tiền phải trả nếu cộng lượng điện tiêu thụ của Nam và Việt lại.
\\Y : độ chênh giữa số tiền Nam phải trả và Việt phải trả.

Nếu không tính được hóa đơn của mình, Nam phải trả thêm 200.000 VNĐ “phí tính toán”. Không chịu đầu hàng, Nam quyết tâm tính ra hóa đơn của mình. Nam biết được một chi tiết quan trọng là mình không thể tiêu thụ điện nhiều hơn Việt (do Việt bật máy lạnh suốt ngày). Biết NVE sẽ còn làm khó mình dài dài, Nam viết hẳn một chương trình để tính.

\subsubsection{Input}

Gồm một dòng duy nhất ghi 2 số X, Y (0  $\le$  X , Y $\le$  $2x10^{10}$ )

\subsubsection{Output}

Ghi ra một số nguyên duy nhất là tiền điện Nam phải trả.

\subsubsection{Example}
\begin{verbatim}
\textbf{Input:}
110000  30000

\textbf{Output:}
35000
\emph{Giải thích : 
}Nam dùng 150kWh, Việt dùng 250kWh. Hai người dùng tổng cộng 400 kWh. 
Hóa đơn tổng cộng : 200 x 100 + 300 x 300 = 110000 VNĐ.

\textbf{Input:}
3551500  2761500

\textbf{Output:}
290000

\end{verbatim}

\emph{ }
