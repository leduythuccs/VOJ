

Cây nhị phân là cây mà mỗi nút có tối đa 2 con, thường được gọi là con trái và con phải. 3 phép duyệt cây nhị phân thông thường là tiền thứ tự (preorder), trung thứ tự (inorder) và hậu thứ tự (postorder):
\begin{itemize}
	\item Duyệt tiền thứ tự cây gốc A: thăm gốc A, thăm cây con trái của A, thăm cây con phải của A.
	\item Duyệt trung thứ tự cây gốc A: thăm cây con trái của A, thăm gốc A, thăm cây con phải của A.
	\item Duyệt hậu thứ tự cây gốc A: thăm cây con trái của A, thăm cây con phải của A, thăm gốc A.
\end{itemize}

Cho danh sách duyệt tiền thứ tự và trung thứ tự của 1 cây nhị phân N đỉnh, các đỉnh được đánh số từ 1 đến N.

Yêu cầu hãy xây dựng lại cây và in ra danh sách duyệt hậu thứ tự.

\subsubsection{Input \href{http://vi.wikipedia.org/wiki/Duy%E1%BB%87t_c%C3%A2y}{}}

Dòng đầu ghi số N là số đỉnh (N  $\le$  50 000, 40\% số test N  $\le$  1000).

Dòng thứ hai ghi N số là danh sách duyệt preorder.

Dòng thứ ba ghi N số là danh sách duyệt inorder.

\subsubsection{Output}

In ra N số là danh sách duyệt postorder của cây.

\subsubsection{Example}
\begin{verbatim}
\textbf{Input:}
6
1 5 3 4 6 2
5 1 4 6 3 2

\textbf{Output:}
5 6 4 2 3 1\end{verbatim}
\begin{verbatim}

\includegraphics{http://i37.photobucket.com/albums/e81/beo_chay_so/bintree.png}\end{verbatim}
