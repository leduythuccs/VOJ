



   Ở một công ty nọ có một quy định khá lạ lùng đó là Sếp thì không được thấp hơn nhân viên! Sếp trực tiếp của một nhân viên A được định nghĩa là nhân viên có chiều cao không thấp hơn A và có mức lương nhỏ nhất nhưng vẫn cao hơn A. Quan hệ lính được định nghĩa đệ quy như sau, nếu A là sếp của B thì B và các lính của B đều là lính của A.  

   Biết công ty này có N nhân viên và một số truy vấn đến các nhân viên. Với mỗi truy vấn, bạn hay tìm sếp trực tiếp và số lượng lính của nhân viên đó.  

\subsubsection{   Dữ liệu  }

   Dòng đầu tiên chứa số lượng test. Mỗi test có cấu trúc như sau:  
\begin{itemize}
	\item     Dòng 1 chứa hai số N và Q là số nhân viên và số truy vấn (1 ≤ N ≤ 30000, 1 ≤ Q ≤ 200)   
	\item     N dòng tiếp theo mỗi dòng chứa thông tin về một nhân viên gồm 3 số nguyên: số hiệu của nhân viên (luôn có 6 chữ số, chữ số đầu tiên khác 0), lương và chiều cao. Lương không vượt quá 10000000. Chiều cao trong phạm vi 1000000 đến 2500000.   
	\item     Q dòng tiếp theo mỗi dòng chứa số hiệu của một nhân viên cần truy vấn.   
\end{itemize}

\subsubsection{   Kết quả  }

   Với mỗi truy vấn in ra hai số nguyên là số hiệu của sếp trực tiếp và số lính. Nếu không có sếp thì in ra số 0.  

\subsubsection{   Ví dụ  }
\begin{verbatim}
Dữ liệu
2
3 3
123456 14323 1700000
123458 41412 1900000
123457 15221 1800000
123456
123458
123457
4 4
200002 12234 1832001
200003 15002 1745201
200004 18745 1883410
200001 24834 1921313
200004
200002
200003
200001

Kết quả
123457 0
0 2
123458 1
200001 2
200004 0
200004 0
0 3
\end{verbatim}