

Tèo được các admin VNOI đố  cho một trò chơi như sau :
\begin{itemize}
	\item Tèo có một bảng hình    chữ nhật N x M, mỗi ô có một màu nhất định và K lần    tô màu.
	\item Trong mỗi lần tô màu,    Tèo chọn một vùng có số hiệu màu như nhau và tô một    màu khác. (Một vùng là một dãy các ô chung cạnh có cùng    màu với nhau).
\end{itemize}
\begin{itemize}

Ví  dụ :
\end{itemize}

Ở hình bên trái nếu ta chọn  ô (3,3) màu xanh lá và tô thành màu vàng thì sẽ thu được  hình bên phải.\hypertarget{0.1_table01}{}
\begin{tabular}\hline 
\hypertarget{0.1_table02}{}
\begin{tabular}\hline 
 &   &   &   &    
\hline
  &   &   &    
\hline
  &   &   &    
\hline

\end{tabular}

  
\hypertarget{0.1_table03}{}
\begin{tabular}\hline 
 &   &   &   &    
\hline
  &   &   &    
\hline
  &   &   &    
\hline

\end{tabular}

  

\end{tabular}

  


Sau khi biến đổi, bảng màu sẽ  có Q vùng, mỗi vùng sẽ có c[i] ô (i=1->Q). Câu hỏi của  các admin VNOI cho Tèo là : Hãy chỉ ra cách tô màu sau K lần  sao cho c[1]\textasciicircum2 + .... + c[Q]\textasciicircum2 càng lớn càng tốt.

\subsubsection{Dữ liệu}
\begin{itemize}
	\item Dòng đầu tiên chứa    3 số n, m, k. (n, m, k $<$= 50)
	\item n dòng tiếp theo, mỗi    dòng chứa m số thể hiện màu được tô cho ô tương ứng    trên bảng (mỗi màu khác nhau được đánh một số hiệu    khác nhau và các ô có màu giống nhau thì được đánh cùng    một số). Số hiệu màu là số tự nhiên không vượt quá    3000.
\end{itemize}

\subsubsection{Kết quả}
\begin{itemize}
	\item Ghi ra trên k dòng, thể    hiện các bước tô màu theo thứ tự tăng dần của thời    gian, mỗi dòng có dạng: i, j, t tương ứng với việc tô    màu t cho ô (i, j).
\end{itemize}

\subsubsection{Ví dụ}

Dữ liệu:

3 4 2

1 1 2 3

1 2 2 3

2 2 3 3

Kết quả:

3 3 2

1 1 2

\subsubsection{Cách tính điểm}

Điểm cho mỗi output hợp lệ  sẽ tỷ lệ với tổng c[1]\textasciicircum2 + .... + c[Q]\textasciicircum2.

