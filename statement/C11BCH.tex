

Sirdat\_LS có căn cứ quân sự được thiết lập sâu trong rừng rậm. Có \textbf{n} trạm gác đặt máy phát siêu âm để bảo vệ căn cứ, trạm thứ \textbf{i} có tọa độ ( \textbf{$x_{i$}} , \textbf{$y_{i$}} ). Không có 3 trạm nào thẳng hàng. Nếu lấy các trạm này làm đỉnh ta có một đa giác lồi. Không có trạm nào nằm trong đa giác lồi. Vùng nằm hoàn toàn trong đa giác lồi được bảo vệ an toàn tuyệt đối. Nhưng Khaihankdk có thể tấn công và phá hủy một số trạm gác. Với những trạm còn lại, vùng được bảo vệ sẽ bị thu hẹp.

Bộ chỉ huy của Sirdat\_LS cần phải được đảm bảo an toàn ở mức cao nhất, sao cho để Bộ chỉ huy không còn được an toàn địch phải phá hủy một số nhiều nhất các trạm gác.

 


\includegraphics{http://d.f7.photo.zdn.vn/upload/original/2011/11/25/21/55/13222329541237269652_574_574.jpg}

\textbf{Yêu cầu:} Cho \textbf{n} và các  tọa độ nguyên \textbf{$x_{i$}} , \textbf{$y_{i$}} (3 ≤ \textbf{n} ≤ 5×$10^{4}$ , | \textbf{$x_{i$}} |, | \textbf{$y_{i$}} | ≤ $10^{6}$ , \textbf{i} = 1 ÷ \textbf{n} ). Hãy xác định số trạm gác tối đa cần phá để Bộ chỉ huy không còn được an toàn ứng với trường hợp vị trí đặt bộ chỉ huy được chọn tối ưu.

\subsubsection{Input}
\begin{itemize}
	\item Dòng đầu tiên chứa số nguyên \textbf{n} ,
	\item Dòng thứ \textbf{i} trong \textbf{n} dòng sau chứa 2 số nguyên \textbf{$x_{i$}} , \textbf{$y_{i$}} , các đỉnh được liệt kê theo chiều kim đồng hồ.
\end{itemize}

\subsubsection{Output}

Một số nguyên duy nhất – số trạm gác tối đa cần phá để Bộ chỉ huy không còn được an toàn.

\subsubsection{Example}
\begin{verbatim}
\textbf{Input:}
5
0 0
0 10
10 20
20 10
25 0
\textbf{Output:}
2
\end{verbatim}
