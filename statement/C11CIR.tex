



   Cho M (M $\le$  100) hình tròn trên 1 bảng kích thước N*N  ô vuông(N  $\le$  1000), Mỗi hình tròn có 1 bán kính R[i]-0.5 (R[i] nguyên, R[i]  $\le$  100) và tâm nằm trên  trọng tâm của 1 ô vuông ở dòng x[i], cột y[i]. Các hình tròn có thể đè lên nhau và có hình có một phần diện tích nằm ngoài bảng. Yêu cầu: Tìm hình chữ nhật có diện tích lớn nhất nằm trên bảng mà không đè lên các hình tròn. Lưu ý: tất cả các dữ diệu input và output đều là số nguyên dương, kể cả kích thước và diện tích hình chữ nhật tìm được.  

   Ví dụ: Hình tròn có tâm (4;4), bán kính là 4 và phần bị phủ là phần màu đỏ.  


\includegraphics{http://d.f7.photo.zdn.vn/upload/original/2011/11/18/21/32/13216267481064251649_574_574.jpg}


\includegraphics{[url=http://www.upanh.com/yenthanh132_h1_upanh/v/enof4e1a0vj.htm][img]http://nf5.upanh.com/b2.s7.d4/54f01f0697cf51d91b5490b55e48b83e_38032345.yenthanh132h1.png[/img][/url]}



\subsubsection{   Input  }
\begin{itemize}
	\item     Dòng đầu: số nguyên dương N và M.   
\end{itemize}
\begin{itemize}
	\item     M dòng sau: Mỗi dòng ghi 3 số, lần lượt là tọa độ hàng, cột của ô chứa tâm, bán kính hình tròn.   
\end{itemize}

\subsubsection{   Output  }
\begin{itemize}
	\item 

       Diện tích của hình chữ nhật tìm được, và phần diện tích bị các hình tròn phủ (không tính phần diện tính lan ra khỏi bảng).      



\end{itemize}

\subsubsection{   Ví dụ  }
\begin{verbatim}
\textbf{Input:}
8 1


4 4 4


\textbf{Output:}
8 45\end{verbatim}
