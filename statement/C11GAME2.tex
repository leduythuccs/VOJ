

Xét 1 bảng số gồm 2* n ô, mỗi ô chứa một số nguyên có giá trị nằm trong đoạn [-10, 10].

\textbf{Ví dụ }
\begin{verbatim}

\texttt{-3 -1 -2  0  5 -1  0
 0 -3  2  4  0  5 -2}\end{verbatim}

 

Ta gọi điểm của bảng là tổng tất cả các tích của số trên dòng 1 với số trên dòng 2 tương ứng( cùng cột). Với bảng trên điểm sẽ là -6.

Ta có loại phép biến đổi bảng như sau: Tráo 2 ô liên tiếp trên cùng một dòng cho nhau, điều kiện để thực hiện phép tráo là một ô phải khác 0 và ô còn lại = 0.

\textbf{Yêu cầu: } Cho bảng số, hãy biến đổi bảng để bảng có điểm lớn nhất.

\subsubsection{Input}
\begin{itemize}
	\item Dòng đầu là số n (n $\le$  200).
	\item Dòng thứ 2 là n số nguyên là n số được ghi trên dòng 1 của bảng số.
	\item Dòng thứ 3 là n số nguyên là n số được ghi trên dòng 2 của bảng số.
\end{itemize}

\subsubsection{Output}

Gồm 1 số duy nhất là  kết quả tìm được.

\subsubsection{Ví dụ:}
\begin{verbatim}
\textbf{\textbf{Input}}7
-3 -1 -2 0 5 -1 0
0 -3 2 4 0 5 -2
\textbf{Output
}36 \end{verbatim}
