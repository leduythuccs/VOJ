



   Dãy số A gồm n số nguyên khác nhau từng đôi: $a_{1}$   , $a_{2}$   , ..., $a_{n}$   được gọi là hoàn hảo nếu như nó thỏa mãn tính chất sau: "Không tồn tại 3 chỉ số p $<$ q $<$ r sao cho hoặc $a_{p}$   $<$ $a_{q}$   $<$ $a_{r}$   hoặc $a_{p}$   $<$ $a_{r}$   $<$ $a_{q}$   hoặc $a_{q}$   $<$ $a_{p}$   $<$ $a_{r}$   ".  

   Cho A là dãy hoàn hảo, một phép chèn một số nguyên b vào dãy A để tạo thành dãy B (b có thể được chèn vào trước $a_{1}$   , hoặc giữa $a_{i}$   , $a_{i+1}$   với 1 ≤ i $<$ n, hoặc sau $a_{n}$   ) được gọi là hợp lệ nếu như các điều kiện sau được thỏa mãn:  
\begin{itemize}
	\item     b $>$ $a_{i}$    với mọi 1 ≤ i ≤ n   
	\item     Dãy B là hoàn hảo   
\end{itemize}
\begin{itemize}
\end{itemize}

\subsubsection{   Yêu cầu  }
\begin{itemize}
	\item     Hãy tính số lượng phép chèn hợp lệ một số b vào dãy A.   
	\item     Giả sử B là một dãy thu được bởi một phép chèn hợp lệ b vào A. Hãy tính số lượng dãy hoàn hảo thu được bằng cách hoán vị các phần tử của B.   
\end{itemize}

\subsubsection{   Dữ liệu  }
\begin{itemize}
	\item     Dòng thứ nhất chứa hai số nguyên n và b tương ứng với số lượng phần tử của dãy A và số nguyên cần chèn. Biết rằng 3 ≤ n ≤ $10^{5}$    , |b| ≤ $10^{6}$    .   
	\item     Dòng thứ hai chứa n số nguyên $a_{1}$    , $a_{2}$    , ..., $a_{n}$    , mỗi số có trị tuyệt đối ≤ $10^{6}$    .   
	\item     Các số trên cùng một dòng được ghi cách nhau ít nhất một dấu cách.   
\end{itemize}

\subsubsection{   Kết quả  }
\begin{itemize}
	\item     Dòng thứ nhất ghi một số nguyên là số phép chèn hợp lệ số b vào dãy A.   
	\item     Dòng thứ hai ghi một số nguyên là phần dư trong phép chia cho $10^{9}$    của số lượng dãy hoàn hảo thu được bằng cách hoán vị các phần tử của dãy B. Nếu không thể chèn được b vào dãy A, ta ghi ra 0.   
\end{itemize}

\subsubsection{   Ví dụ  }
\begin{verbatim}
\textbf{Input:}
4 2012


55 25 9 20

\textbf{Output:}
2


8\end{verbatim}

\subsubsection{   Giới hạn  }

   Có 50\% số test có n ≤ 15.  
