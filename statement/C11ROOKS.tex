

Hai quân xe được đặt tại hai ô khác nhau trên một bàn cờ vua có NxN ô. Mỗi ô của bàn cờ có một giá trị từ 0 đến 1000.

Một ô trong bàn cờ được gọi là \textbf{ bị tấn công } nếu ô đó \textbf{ cùng hàng hoặc cùng cột } với ít nhất một quân xe. Hai ô vuông có chứa quân xe được xem như \textbf{ không bị tấn công } .

Bài toán đặt ra là tìm cách đặt hai quân xe lên bàn cờ, để tổng giá trị các ô bị tấn công là \textbf{ lớn nhất } .

\subsubsection{Dữ liệu}
\begin{itemize}
	\item Dòng 1: Số nguyên dương N (2 ≤ N ≤ 300).
	\item N dòng tiếp theo, mỗi dòng gồm N số. Đây là giá trị của các ô trên bàn cờ.
\end{itemize}

\subsubsection{Kết quả}
\begin{itemize}
	\item Tổng lớn nhất tìm được.
\end{itemize}

\subsubsection{Ví dụ}
\begin{verbatim}
\textbf{Input}
3
0 1 4
3 0 2
1 4 1

\textbf{Output}
15

\textbf{Input}
4
0 1 1 1
1 0 4 3
0 1 3 5
0 0 2 5

\textbf{Output}
23

\textbf{Input}
5
4 2 2 3 3
4 2 1 4 0
1 3 4 0 1
4 3 0 2 3
0 0 3 0 4

\textbf{Output}
40\end{verbatim}

 

\subsubsection{Giải thích}
\begin{itemize}
	\item Vị trí đặt hai quân xe trong ví dụ 1 là (1, 1) và (2, 2)
	\item Vị trí đặt hai quân xe trong ví dụ 2 là (1, 3) và (1, 4)
	\item Vị trí đặt hai quân xe trong ví dụ 3 là (2, 5) và (4, 3)
\end{itemize}

\subsubsection{Giới hạn}
\begin{itemize}
	\item Có 60\% số test, 2 ≤ N ≤ 100
\end{itemize}