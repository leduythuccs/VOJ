



   Cho dãy số F dài vô tận. Nguyên tắc xây dựng dãy F rất đơn giản.  
\begin{itemize}
	\item     F[1] = 1   
	\item     Với i $>$ 1, đầu tiên ta lấy F[i] = F[i - 1] * 2, sau đó sắp xếp các chữ số trong F[i] theo thứ tự tăng dần. (Các chữ số 0 ở đầu F[i], ta coi như là không có nghĩa và có thể xóa đi).   
\end{itemize}

   Như vậy, các số đầu tiên trong dãy số F là:  

   1, 2, 4, 8, 16, 23, 46, 29, 58, ...  

   Cho số n, bạn hãy tìm số F[n]. Biết rằng số lượng chữ số của F[n] luôn nhỏ hơn $10^{6.}$

\subsubsection{   Dữ liệu  }
\begin{itemize}
	\item     Một số nguyên dương duy nhất là n (1 ≤ n ≤ $10^{9}$    ).   
\end{itemize}

\subsubsection{   Kết quả  }
\begin{itemize}
	\item     Một số nguyên dương duy nhất là số cần tìm.   
\end{itemize}

\subsubsection{   Ví dụ  }
\begin{verbatim}
\textbf{Input:}
7

\textbf{Output:}
46\end{verbatim}
