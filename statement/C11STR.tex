

Cho 2 xâu ký tự S và T trong đó S chỉ bao gồm các ký tự 'a' -> 'z' còn T chỉ bao gồm các ký tự 'a' -> 'z' và 2 ký tự đặc biệt là '?' và '*'.


Một ký tự '*' trong xâu T có thể đại diện cho 1 số lượng bất kỳ (có thể bằng 0) các ký tự 'a' -> 'z'.


Một ký tự '?' trong xâu T có thể đại diện cho \textbf{ đúng 1 } ký tự trong số các ký tự 'a' -> 'z'.


Một xâu X gọi là thỏa mãn định dạng T nếu ta có thể thay các ký tự '*' và '?' trong xâu T để thu được xâu X.




\textbf{Ví dụ}


T = '*bc?' sẽ có các xâu X thỏa mãn như 'abca' , 'bca', 'bcd', 'aaaaabcz', ...


Trọng số của một xâu được tính bằng tổng trọng số của các ký tự có trong đó với quy ước 'a' = 1, 'b' = 2, 'c' = 3, ... 'z' = 26.

\textbf{Yêu cầu}

Hãy tìm xâu X là xâu con liên tiếp của xâu S mà X thỏa mãn định dạng T và có trọng số là nhỏ nhất.

\textbf{Input }
\begin{itemize}
	\item 

Gồm 1 dòng chứa 2 xâu T và S ( 1 ≤ |S| ≤ 10000, |T| ≤ 1000 )
\end{itemize}

\textbf{Output }
\begin{itemize}
	\item 

Ghi ra trọng số của xâu X tìm được, hoặc ghi ra -1 nếu không có xâu X nào thỏa mãn.
\end{itemize}

\textbf{Ví dụ}
\begin{verbatim}
\textbf{Input}
a?a alabala

\textbf{Output}
4

\textbf{Giải thích}aba là xâu có trọng số nhỏ nhất tìm được.


\textbf{Input}
a*c?a axcbaabcbax

\textbf{Output}
9

\textbf{Giải thích}abcba có trọng số nhỏ hơn axcba\end{verbatim}
