



   Xâu a được gọi là   \emph{    tiền tố   }   của xâu b nếu xâu a trùng với phần đầu của xâu b. Ví dụ       pre      là tiền tố của       prefix   

   Xâu a được gọi là   \emph{    hậu tố   }   của xâu b nếu xâu a trùng với phần cuối của xâu b. Ví dụ       fix      là hậu tố của       suffix   

\emph{    yenthanh132   }   vừa mới học về tiền tố và hậu tố nên hôm nay anh ta sẽ đố các bạn một bài toán đơn giản về tiền tố và hậu tố như sau:  
\begin{itemize}
	\item     Cho 2 xâu a,b gồm các kí tự latin thường ('a' đến 'z')   
	\item     Tìm 1 xâu c thỏa mãng:    
\begin{enumerate}
	\item       Xâu a là tiền tố của xâu c     
	\item       Xâu b là hậu tố của xâu c     
	\item       Độ xài xâu c là ngắn nhất.     
\end{enumerate}
\end{itemize}
\begin{enumerate}
\end{enumerate}

\subsubsection{   Input  }
\begin{itemize}
	\item     Dòng 1: Xâu a   
	\item     Dòng 2: Xâu b   
\end{itemize}

\subsubsection{   Output  }
\begin{itemize}
	\item     Một dòng duy nhất là xâu c.   
\end{itemize}

\textbf{     Giới hạn:    }
\begin{itemize}
	\item     40\% số test có độ dài 2 xâu a,b  $\le$  1000 kí tự   
	\item     Trong toàn bộ test, độ dài 2 xâu a,b  $\le$  10    $^     5    $    kí tự   
\end{itemize}

\subsubsection{\textbf{    Ví dụ:   }}
\begin{verbatim}
\textbf{Input 1:}
abca
\\cab
\\\textbf{
\\Output 1:
\\}abcab
\\\textbf{
\\Input 2:
\\}abc
\\abc
\\\textbf{
\\}\textbf{Output 2:
\\}abc
\\(2 xâu a,b không nhất thiết phải khác nhau).\end{verbatim}
