



   Cho dãy số A gồm N phần tử là hoán vị của N số nguyên từ 0 đến N - 1 và được đánh số lần lượt từ 1 đến N. Phép biến đổi Swap(x) sẽ đổi chỗ A[x] và A[x + 1] (1 ≤ x $<$ N). Một hoán vị B gọi là đẹp nếu thỏa mãn 2 điều kiện sau:  
\begin{enumerate}
	\item     Là hoán vị của N – 1 số gồm các số từ 1 đến N – 1.   
	\item     Sau khi thực hiện lần lượt các phép biến đổi Swap(B[1]), Swap(B[2]), ..., Swap(B[N - 1]) trên dãy số A đã cho thì được dãy số mới là dãy tăng dần.   
\end{enumerate}

   Yêu cầu: Hãy đếm số hoán vị đẹp.  

\subsubsection{   Dữ liệu  }
\begin{itemize}
	\item     Dòng 1: Số nguyên dương N.   
	\item     Dòng 2: Gồm N số nguyên, là giá trị ban đầu của dãy số A.   
\end{itemize}

\subsubsection{   Kết quả  }
\begin{itemize}
	\item     Phần dư khi chia số hoán vị đẹp cho $10^{9}$    + 7.   
\end{itemize}

\subsubsection{   Ví dụ  }
\begin{verbatim}
\textbf{Input:


}4


2 0 3 1





\textbf{Output:


}2\end{verbatim}

\subsubsection{   Giải thích:  }
\begin{itemize}
	\item     Dãy số A ban đầu là (2, 0, 3, 1).   
	\item     Các hoán vị được sinh ra là (1, 2, 3), (1, 3, 2), (2, 1, 3), (2, 3, 1), (3, 1, 2), (3, 2, 1).   
	\item     Có 2 hoán vị đẹp là (1, 3, 2) và (3, 1, 2).   
\end{itemize}

\subsubsection{\textbf{    Giới hạn:   }}

   1 ≤ N ≤ 100, 20\% số test N ≤ 10.  
