



   Cho bảng N x N (N ≤ 14). Mỗi ô là 1 số 0 hoặc 1. Một phép Touch(i, j) cho phép đổi ngược giá trị của ô (i, j) - hàng i, cột j - và các ô kề cạnh (nếu có) từ 0 thành 1 và từ 1 thành 0. Trạng thái đích là trạng thái mà cả bảng đều là 0 hoặc đều là 1.  

\textbf{    Yêu cầu:   }   Tìm số bước ít nhất để đến được trạng thái đích. Nếu ko có cách nào in ra -1.  

\subsubsection{\textbf{    Dữ liệu vào:   }}
\begin{itemize}
	\item     Dòng đầu tiên chứa số N (1 ≤ N ≤ 14).   
	\item     N dòng tiếp theo, mỗi dòng chứa N kí tự '0' hoặc '1' mô tả trạng thái của bảng.   
\end{itemize}

\subsubsection{   Dữ liệu ra:  }
\begin{itemize}
	\item     1 số duy nhất là số bước ít nhất để đến được trạng thái đích. Nếu ko có cách nào in ra -1.   
\end{itemize}

\textbf{     Lưu ý:    }

    Có 40\% số test có N ≤ 4   

\subsubsection{   Ví dụ:  }
\begin{verbatim}
\textbf{Input:


}3


110 


001


100


\textbf{


Output:


}4





\textbf{Giải thích: }4 phép touch cần dùng ở vị trí (1,1), (1,2), (1,3), (2,1).


\end{verbatim}
