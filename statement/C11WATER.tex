

Nền phẳng của một công trường xây dựng đã được chia thành lưới ô vuông đơn vị kích thước m x n ô. Trên mỗi ô (i,j) của lưới, người ta dựng một cột bê tông hình hộp có đáy là ô (i,j) và chiều lao là $h_{ij}$ đơn vị. Sau khi dựng xong, thì trời đổ mưa to và đủ lâu. Giả thiết rằng nước không thấm qua các cột bê tông cũng như không rò rỉ qua các đường ghép giữa chúng.
\\\textbf{Yêu cầu:} Xác định lượng nước đọng giữa các cột.

\subsubsection{Input}
\begin{itemize}
	\item Dòng đầu tiên chứa 2 số nguyên dương m,n (m, n  $\le$ 1000)
	\item m dòng tiếp theo, dòng thứ i chứa n số nguyên dương, số thứ j là $h_{ij}$ ( $\le$ $10^{6}$ )
\end{itemize}

Các số trên cùng một dòng cách nhau ít nhất 1 dấu cách.

\subsubsection{Output}
\begin{itemize}
	\item Khi ra số đơn vị khối nước đọng lại
\end{itemize}

\subsubsection{Example}
\begin{verbatim}
\textbf{Input:
}5 7
3 3 3 3 3 3 3
3 1 1 1 1 1 3
3 1 2 2 2 1 3
3 1 1 1 1 1 3
3 3 3 3 3 3 3
\textbf{Output:}
27
\end{verbatim}
