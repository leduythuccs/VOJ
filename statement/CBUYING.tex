

Những con bò rất thích ăn Sô-cô-la , nên Farmer John quyết định mua một ít cho chúng.

Cửa hàng có N loại sô-cô-la (được đánh số từ 1..N) với số lượng mỗi loại không hạn chế. Loại thứ i có giá P\_i tiền và có đúng C\_i con bò muốn ăn loại Sô-cô-la ấy. Farmer John có B tiền để mua Sô-cô-la cho lũ bò.

Hỏi số bò tối đa mà Farmer John có thể phục vụ là bao nhiêu ? Biết rằng mỗi con bò chỉ thích một loại sô-cô -la, và nó chỉ được ăn loại sô-cô-la ấy.

\subsubsection{Input}

Dòng đầu tiên là hai số nguyên N và B.

N dòng tiếp theo , dòng thứ i+1 là hai số nguyên dương P\_i và C\_i.

\subsubsection{Output}

Gồm một số duy nhất là kết quả.

\subsubsection{Example}

\textbf{Input:}
\\5 50
\\5 3
\\1 1
\\10 4
\\7 2
\\60 1
\\\textbf{Output:}
\\8
\\
\\
\\\textbf{​Giới hạn}
\\
\\1 $\le$ N $\le$ 10^5
\\1  $\le$  B  $\le$  10^18
\\1  $\le$  C\_i  $\le$  10^18
\\1  $\le$  P\_i  $\le$  10^18.
\\
\\\textbf{Giải thích}:
\\FJ sẽ mua như sau:
\\+Mua 3 gói sô-cô-la loại 1 mất 3*5= 15\$.
\\+Mua 1 gói sô-cô-la loại 2 mất 1*1= 1\$.
\\+Mua 2 gói sô-cô-la loại 3 mất 2*10= 20\$
\\+Mua 2 gói sô-cô-la loại 4 mất 2*7= 14\$.
\\Tổng cộng hết :15+1+20+14=50\$, và FJ đã phục vụ được 8 con bò.

 

 
