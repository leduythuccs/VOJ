

Mirko and Slavko are playing a new game. Again. Slavko starts each round by giving Mirko two numbers A and B, both smaller than 100. Mirko then has to slove the following task for Slavko: how to pair all given A numbers with all given B numbes so that the \textbf{maximal sum of such pairs is as small as possible}.

In other words, if during previous rounds Slavko gave numbers a1, a2, a3 .... an and b1, b2, b3 ... bn, determine n pairings (ai, bj) such that each number in A sequence is used in exactley one pairing, and each number in B sequenct is used in exactely one pairing and the maximum of all sums ai + bj is minimal.

\subsubsection{Input}

The first line of input contains a single integer N (1 ≤ \textbf{N} ≤ 100000), number of rounds. Next N lines contain two integers \textbf{A} and \textbf{B} (1 ≤ \textbf{A, B} ≤ 100), numbers given by Slavko in that round.

\subsubsection{Output}

Output consists of \textbf{N} lines, one for each round. Each line should contain the smallest maximal sum for that round.

\subsubsection{Example}
\begin{verbatim}
\textbf{Input1:}


2 8
3 1
\\1 4



extbf{Output1:}

10
10
9 \end{verbatim}
egin{verbatim}
extbf{Input2:
\\}3
\\1 1
2 2
3 3\textbf{ }\end{verbatim}
egin{verbatim}
\textbf{Output2:
\\}2
\\3
\\4\textbf{ }\end{verbatim}
