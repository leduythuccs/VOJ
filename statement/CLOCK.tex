

 

 


\includegraphics{http://vn.spoj.pl/content/clock1.jpg}

9 chữ số trong hình 1 là vị trí của 9 đồng hồ, trong đó mỗi đồng hồ có 1 trong 4 vị trí: Bắc (12 giờ) , Đông ( 3 giờ ) , Nam ( 6 giờ ) và Tây ( 9 giờ) . ( Các hướng được đánh số tương ứng từ 0 -> 3 ):



\includegraphics{http://vn.spoj.pl/content/clock2.jpg}


Có 9 cách khác nhau để quay các đồng hồ, mỗi cách được gọi là một dịch chuyển (Move). Mỗi dịch chuyển được chọn bằng một số từ 1 đến 9. Số đó sẽ quay các đồng hồ được đánh số 1 một góc 90 độ theo chiều kim đồng hồ. Các đồng hồ có đánh số 0 sẽ không bị tác động gì. 9 di chuyển được minh họa trong hình 2.


\includegraphics{http://vn.spoj.pl/content/clock3.jpg}

 

Hãy viết chương trình tính xem cần ít nhất bao nhiêu lần di chuyển để tất cả các đồng hồ đều chỉ 12h .

\subsubsection{Input}

Gồm 3 dòng , mỗi dòng gồm 3 chữ số cho biết hướng mà đồng hồ đang chỉ.

\subsubsection{Output}

Số lần dịch chuyển ít nhất.

\subsubsection{Ví dụ}
\begin{verbatim}
Input:
330
222
212

Output:
4
\end{verbatim}

Giải thích test ví dụ : Thực hiện các phép biến đổi 5, 8, 4 và 9 .
