



    "Qua đêm nay sóng gió sẽ về với mây ngàn"   

    "Dậy đi yêu thương ta hãy khép mi nỗi buồn"    \emph{
\\}

Qua đêm nay - Phương Linh - Mạnh Quân


\\

    Tuy vậy nhưng không phải lúc nào sóng gió cũng có thể dễ dàng về với mây ngàn được. Lần này cũng vậy, Quang Vũ đang bị thầy giáo dạy môn Data Structure đố một bài toán với giao hẹn nếu không giải được thì sẽ có một trận sóng gió khủng khiếp ( dĩ nhiên vì thế  sẽ không về được với mây ngàn ) . Bài toán như sau :   

    Cho N loại bi khác nhau, loại bi thứ i có a[i] viên. Giờ ta gọi tổng số các viên bi là S. Cho một số M ( S chia hết cho  M) , hãy sắp xếp S viên bi này vào S/M hộp sao cho mỗi hộp có M viên khác màu nhau. Dữ liệu đảm bảo có nghiệm.   

    Các bạn hãy giúp Quang Vũ nhé, trận sóng gió này có vẻ to quá, các bạn mà không giúp là cậu ấy về bàn thờ luôn chứ không phải mây ngàn đâu !    
\\

\subsubsection{    Dữ liệu    
\\}
\begin{itemize}
	\item      Dòng đầu tiên ghi 2 số N,M (M  $\le$  S , S $\le$  500000)    
\end{itemize}
\begin{itemize}
	\item      Dòng thứ 2 ghi N số nguyên a[i]  ( a[i] $>$ 0)    
\end{itemize}

\subsubsection{    Kết quả    
\\}

    S/M dòng mỗi dòng ghi M số là chỉ số của viên bi lựa chọn. Nếu có nhiều phương án chỉ cần in ra 1 đáp án.   

\subsubsection{    Ví dụ    
\\}
\begin{verbatim}
\textbf{Dữ liệu}
3 2
2 3 3

\textbf{Kết quả}
1 2
1 3
2 3
2 3

\\
\\Có ít nhất 50% số test có S  $\le$  1000\end{verbatim}

Tác giả: Phạm Quang Vũ
\\
