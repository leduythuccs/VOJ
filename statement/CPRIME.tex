



   Trong số học, định lý Số Nguyên Tố cho biết sự phân bố tiệm cận của các số nguyên tố. Gọi π(x) là số số nguyên tố không vượt quá x. Định lý Số Nguyên Tố khẳng định:  
\includegraphics{http://upload.wikimedia.org/math/c/9/3/c93061b930d29877a2364a62e5ecc1a5.png}

   Bạn hãy viết chương trình xác định xem định lý Số Nguyên Tố có thể dùng để tính xấp xỉ π(x) tốt đến đâu. Cụ thể hơn, với mỗi giá trị x, bạn cần tính sai số phần trăm |π(x) - x/lnx| / π(x) \%.  

\subsubsection{   Dữ liệu  }

   Dữ liệu bao gồm nhiều bộ test (không quá 1000). Mỗi bộ test chứa một giá trị x (2 ≤ x ≤ 10   $^    8   $   ) cho trên một dòng. Số 0 kết thúc dữ liệu.  

\subsubsection{   Kết quả  }

   Với mỗi giá trị x, in ra sai số phần trăm của phép xấp xỉ π(x), làm tròn đến một chữ số thập phân.  

\subsubsection{   Ví dụ  }
\begin{verbatim}
Dữ liệu:
10000000
2
3
5
1234567
0

Kết quả
6.6
188.5
36.5
3.6
7.7
\end{verbatim}