



   Bessie đã quay lại trường học nốt lớp 8 để lấy được tấm bằng. Cô giáo dạy toán của Bessie rất "khủng khiếp" và cô muốn các học sinh phải tính lũy thừa P (1  $\le$  P  $\le$  100,000) của 1 số nguyên N (1  $\le$  N  $\le$  2,000,000,000)  

   Ví dụ: 2 lũy thừa 3 = 2 * 2 * 2 = 8.  

   Tương tự, 123456 lũy thừa 88 = 123456 * 123456 * ... * 123456 (88  thừa số) =  1129987770413559019467963153621658978635389622595924947762339599136126 3387265547320084192414348663697499847610072677686227073640285420809119 1376617325522768826696494392126983220396307144829544079751988205731569 1498433718478969549886325738202371569900214092289842856905719188890170 0772424218248094640290736200969188059104939824466416330655204270246371 3699112106518584413775333247720509274637795508338904731884172716714194 40898407102819460020873199616  

   (Mỗi dòng 70 chữ số).  

   Hãy viết chương trình tính N lũy thừa P. Biết rằng đáp án có không quá 15,000 chữ số. Khi ghi kết quả thì ghi trên mỗi dòng 70 chữ số (ngoại trừ dòng cuối có thể ít hơn). Không ghi ra số 0 ở đầu (ví dụ, không ghi ra  008 mà phải ghi ra 8).  

\subsubsection{   Dữ liệu  }

   * Dòng 1: 2 số nguyên cách nhau bởi dấu cách: N và P  

\subsubsection{   Kết quả  }

   * Dòng 1..?: Một số nguyên là kết quả tính được.  Mỗi dòng ghi ra 70 chữ số (trừ dòng cuối có thể ít hơn).  

\subsubsection{   Ví dụ  }

   Dữ liệu  

   2 15  


\\



   Kết quả  

   32768  
