

Cho n bình chứa nước (1  $\le$  n  $\le$  4). Ban đầu, mỗi bình đều chứa đầy nước. Bình i có dung lượng là \emph{o}$_\emph{i}$\emph{, với 1  $\le$  o}$_\emph{i}$\emph{  $\le$  49.}

Bạn có thể thực hiện 1 trong 3 thao tác sau:
\begin{enumerate}
	\item Đổ tất cả nước ở trong bình A sang bình B. Thao tác này chỉ được thực hiện nếu bình B có đủ chỗ trống.
	\item Lấy 1 lượng nước ở bình A và đổ đầy hoàn toàn bình B.
	\item Đổ tất cả nước mà 1 bình đang chứa.
\end{enumerate}

\textbf{Yêu cầu: }
\begin{itemize}
	\item Cho 1 dãy $w_{i}$. Hỏi có tồn tại 1 dãy thao tác đổ nước để từ trạng thái ban đầu (mỗi bình chứa đầy nước), ta đến được trạng thái mà bình i chứa $w_{i}$ nước.
	\item Nếu tồn tại dãy thao tác đổ nước, tìm số lượng thao tác ít nhất.
\end{itemize}

\textbf{Input}

Dòng đầu: số lượng test T (T  $\le$  20).

Mỗi test gồm:
\begin{itemize}
	\item 1 dòng trống
	\item 1 dòng chứa số nguyên dương n (n  $\le$  4).
	\item 1 dòng chứa n số \emph{o}$_\emph{i}$\emph{, với 1  $\le$  o}$_\emph{i}$\emph{  $\le$  49.}
	\item 1 dòng chứa n số $w_{\emph{i}}$\emph{, với 1  $\le$  w}$_\emph{i}$\emph{  $\le$  $o_{i}$.}
\end{itemize}

\textbf{Output}

Nếu tồn tại 1 dãy thao tác, in ra số thao tác ít nhất, ngược lại in ra NO.

\textbf{Ví dụ}
\begin{verbatim}
Input:
2

3
3 5 5
0 0 4

2
20 25
10 16

Output:
6
NO
\end{verbatim}
