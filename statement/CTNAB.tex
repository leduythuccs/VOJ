



   Cho ba số nguyên dương N, A, B.  

   Cần tìm một hoán vị của N số nguyên dương đầu tiên thỏa mãn các điều kiện sau:  
\begin{enumerate}
	\item     Có dãy con tăng dài nhất gồm A số hạng;$<$không cần liên tục $>$   
	\item     Có dãy con giảm dài nhất gồm B số hạng;$<$không cần liên tục$>$   
	\item     Trong số các hoán vị thỏa mãn hai điều kiện trên, chọn hoán vị nhỏ nhất theo thứ tự từ điển;   
\end{enumerate}

   N, A, B, 1 ≤ N,A,B ≤ 100,000  

\subsubsection{   Input  }

   Dòng đầu tiên là số bộ test T  

   T dòng sau mỗi dòng gồm 3 số N,A,B cách nhau ít nhất 1 dấu cách.  

\subsubsection{   Output  }

   Với mỗi bộ test ,nếu không có hoán vị nào thỏa mãn ghi -1 , ngược lại  in ra hoán vị tìm được.  

\subsubsection{   Example  }
\begin{verbatim}
\textbf{Input:}
\\1
\\4 2 3
\\
\\\textbf{Output:}
\\
\\1 4 3 2
\\\end{verbatim}