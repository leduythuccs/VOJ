
\begin{verbatim}



\textbf{DIABLO}

 

Diablo có một đội quân rất lớn. Cậu ấy sắp xếp các binh lính thành một hàng, và mỗi người được nhận một số nguyên \textbf{id}.

 

Diablo sẽ thực hiện một loạt hiệu lệnh. Mỗi lệnh của cậu ấy sẽ hoặc là \textbf{thêm} một binh sỹ vào cuối hàng, hoặc là \textbf{gọi} một binh sỹ tại vị trí thứ \textbf{k} (tính từ trái sang) \textbf{ra khỏi} hàng.

 

Nhiệm vụ của bạn là mỗi khi Diablo gọi binh sỹ tại vị trí thứ \textbf{k} ra khỏi hàng thì bạn phải tính được \textbf{id} của binh sỹ đó.

 

\textbf{Input}

Dòng đầu chứa số nguyên \textbf{T} là số bộ test.

 

Mỗi bộ test bắt đầu bằng một dòng trống. Tiếp theo là một dòng chứa 2 số nguyên \textbf{N} (0 ≤ N ≤ 10$^5$) thể hiện số lượng binh lính ban đầu ở trong hàng, và số nguyên \textbf{Q} (1 ≤ Q ≤ 50000) thể hiện số lượng hiệu lệnh mà Diablo sẽ thực hiện.

 

Dòng tiếp theo chứa \textbf{N} số nguyên thể hiện \textbf{id} của N người tương ứng theo thứ tự từ trái sang phải trong hàng. Các id đều dương và nằm trong khoảng số nguyên 32 bit có dấu.

 

Mỗi dòng trong \textbf{Q} dòng tiếp theo sẽ chứa một hiệu lệnh, thuộc 1 trong 2 dạng sau :

\textbf{a p} (thêm một binh sỹ vào cuối hàng với \textbf{id} là \textbf{p})

\textbf{c k} (gọi binh sỹ thứ \textbf{k} ra khỏi hàng (tính từ bên trái), k nguyên dương 32 bit có dấu)

 

\textbf{Output}

Với mỗi bộ test, in ra chỉ số của bộ test (như trong ví dụ), sau đó là kết quả của các hiệu lệnh \textbf{‘c k’}, tức là in ra id của binh sỹ thứ \textbf{k}, hoặc in ra \textbf{‘none’} nếu không tồn tại.

 

\textbf{Ví dụ}
\begin{tabular}\hline 


Input & 

Output  
\hline


2

 

5 5

6 5 3 2 1

c 1

c 1

a 20

c 4

c 4

 

2 1

18811 1991

c 1 & 

Case 1:

6

5

20

none

Case 2:

18811  
\hline

\end{tabular}\end{verbatim}
