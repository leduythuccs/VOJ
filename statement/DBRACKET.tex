



   Người ta định nghĩa một dãy ngoặc đúng như sau:  
\begin{itemize}
	\item     Xâu rỗng là một dãy ngoặc đúng.   
	\item     Nếu A là dãy ngoặc đúng thì (A) cũng là một dãy ngoặc đúng   
	\item     Nếu A, B là những dãy ngoặc đúng thì AB cũng là dãy ngoặc đúng.   
\end{itemize}

   Những dãy ngoặc sau được xem là đúng:  
\begin{itemize}
	\item     ()(())   
	\item     ((()))   
\end{itemize}

   Những dãy ngoặc sau thì không:  
\begin{itemize}
	\item     )(   
	\item     (((()))   
	\item     )()()(   
\end{itemize}

   Một xâu S khác rỗng được gọi là xâu con của T nếu xâu S trùng với một dãy các kí tự liên tiếp của T. Ví dụ "bcd" là xâu con của xâu "abcde" nhưng xâu "dc" thì không.  

   Cho một xâu T chỉ gồm các kí tự '(' và ')'  (kí tự mở ngoặc và đóng ngoặc). Như vậy các xâu con của T có thể là một dãy ngoặc đúng hoặc không. Hãy đếm số lượng xâu con phân biệt của T mà là một dãy ngoặc đúng.  

\subsubsection{   Input  }
\begin{itemize}
	\item     Dòng đầu tiên chứa số n là số lượng bộ test (n $\le$ 20).   
	\item     n dòng tiếp theo, mỗi dòng là một bộ test chứa xâu T. Biết rằng độ dài của xâu T không vượt quá 100.000 kí tự.   
\end{itemize}

\subsubsection{   Output  }
\begin{itemize}
	\item     Với mỗi bộ test, xuất ra số lượng xâu con phân biệt của T mà là một dãy ngoặc đúng.   
\end{itemize}

\subsubsection{   Example  }
\begin{verbatim}
\textbf{Input:}
3


(()())()


(()()()()()


()()()(()())(()())


\textbf{Output:}
4


5


11





\textbf{Giải thích: }Với bộ test đầu, có 4 xâu con phân biệt là một dãy ngoặc đúng: "()" ; "()()"; "(()())"; "(()())()"


\end{verbatim}
