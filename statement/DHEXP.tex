

Một dãy gồm n số nguyên không âm a$_ 1 $ , a$_ 2 $ ,..., $a_{n$} được viết thành một hàng ngang, giữa hai số liên tiếp có một khoảng trắng, như vậy có tất cả ( n­ -1) khoảng trắng. Người ta muốn đặt k dấu cộng và ( n- 1- k ) dấu trừ vào ( n­ -1) khoảng trắng đó để nhận được một biểu thức có giá trị lớn nhất.

Ví dụ, với dãy gồm 5 số nguyên 28, 9, 5, 1, 69 và k = 2 thì cách đặt 28+9-5-1+69 là biểu thức có giá trị lớn nhất.

\textbf{Yêu cầu: } Cho dãy gồm n $_$ số nguyên không âm a$_ 1 $ , a$_ 2 $ ,..., $a_{n$} và số nguyên dương k , hãy tìm cách đặt k dấu cộng và ( n- 1- k ) dấu trừ vào ( n­ -1) khoảng trắng để nhận được một biểu thức có giá trị lớn nhất.

\subsubsection{Input}
\begin{itemize}
	\item Dòng đầu chứa hai số nguyên dương n, k ( k $<$ n );
	\item Dòng thứ hai chứa n số nguyên không âm a$_ 1 $ , a$_ 2 $ ,..., $a_{n$} ( $a_{n$} ≤ $10^{6}$ )
\end{itemize}

\subsubsection{Output}

Một số nguyên là giá trị của biểu thức đạt được.

\subsubsection{Example}
\begin{verbatim}
\textbf{Input:
}5 2
28 9 5 1 69
\textbf{Output:}
100
\end{verbatim}

\textbf{Ghi chú:}
\begin{itemize}
	\item Có 50\% số test ứng với 50\% số điểm có n≤ $10^{5}$ và k= 1;
	\item Có 50\% số test còn lại ứng với 50\% số điểm có n≤ $10^{5}$;
\end{itemize}
