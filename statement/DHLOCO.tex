

Carnaval Hạ Long 2015 với chủ đề “Hội tụ tinh hoa - Lan tỏa nụ cười”, điểm mới của lễ hội là sự song hành giữa biểu diễn nghệ thuật “Nơi tinh hoa hội tụ” và diễu hành đường phố “Nụ cười Hạ Long” với sự góp mặt của hơn 2000 diễn viên quần chúng. Có rất nhiều chương trình vui chơi được tổ chức, một trong những trò chơi thu hút được nhiều du khách tham gia đó là trò chơi nhảy lò cò, cụ thể: người chơi cần vượt qua một đoạn đường dài n mét, mỗi bước, người chơi có ba cách nhảy với độ dài bước nhảy tương ứng là 1 mét, 2 mét, 3 mét. Một cách đi chuyển đúng là dãy các bước nhảy có tổng đúng bằng n .

\textbf{Yêu cầu: } Cho n $_$ và M , gọi K là số cách đi chuyển đúng khác nhau để đi hết đoạn đường n mét, hãy tính phần dư của K chia M .

\subsubsection{Input}

Gồm một dòng chứa hai số nguyên dương n , M ( M ≤ 2015);

\subsubsection{Output}

Một số nguyên là phần dư của K chia M .

\subsubsection{Example}
\begin{verbatim}
\textbf{Input:}
5 100
\textbf{Output:}
13
\end{verbatim}

\textbf{\textbf{Ghi chú:}}
\begin{itemize}
	\item Có 20\% số test ứng với 20\% số điểm có n≤ 20;
	\item Có 40\% số test ứng với 40\% số điểm có n≤ $10^{6}$;
	\item Có 40\% số test còn lại ứng với 40\% số điểm có n≤ $10^{15}$.
\end{itemize}
