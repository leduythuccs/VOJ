

 

Công viên Disneyland khai trương N điểm vui chơi mới. Một hôm Bờm và Cuội đến chơi Công viên. Qua việc hỏi thăm người hướng dẫn Bờm và Cuội đã biết thời gian cụ thể để đi từ điểm vui chơi i đến địa điểm vui chơi j. Bờm và Cuội muốn đi thăm hết tất cả các địa điểm theo nguyên tắc sau:
\begin{itemize}
	\item Mỗi người đi từ điểm số 1, qua 1 số điểm và quay trở về điểm số 1.
	\item Các điểm đến trên đường đi của Bờm và Cuội là 1 dãy số tăng dần( trừ điểm số 1 khi quay về).
	\item Mỗi điểm vui chơi phải thuộc ít nhất 1 trong 2 đường đi của Bờm và Cuội.
	\item Thời gian đi là nhỏ nhất (tính cả thời gian quay về 1).
\end{itemize}

\subsubsection{Input}
\begin{itemize}
	\item Dòng đầu là số N (N$<$201)
	\item Tiếp theo là ma trận N dòng N cột: 1 số nguyên không âm($<$1001) trên hàng i cột j cho ta biết thời gian đi từ địa điểm i đến địa điểm j. (A[i][j] = A[j][i] , A[i][i] = 0 với mọi i, j).
\end{itemize}

\subsubsection{Output}

Gồm 1 số duy nhất là thời gian nhỏ nhất để Bờm và Cuội thăm quan Công Viên theo nguyên tắc trên.

\subsubsection{Example}
\begin{verbatim}
Input:
4
0 1 4 3
1 0 2 4
4 2 0 4
3 4 4 0


Output:
10

(Giải thích : cách đi tốt nhất là Bờm đi 1-$>$2-$>$3-$>$4-$>$1 và Cuội đứng yên)
\end{verbatim}