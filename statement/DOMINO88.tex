



\section{\textbf{    Bộ bài Domino với bản đồ số   }}




   Bộ bài domino gồm 28 quân đánh số từ 1 đến 28. Mỗi quân bài là một thanh hình chữ nhật được chia làm hai hình vuông bằng nhau. Trong đó người ta ghi các số từ 0 (để trống) đến 6 bằng cách trổ các dấu tròn trắng. Dưới đây liệt kê 28 quân bài domino:   



\includegraphics{http://i925.photobucket.com/albums/ad92/thanhbebi17/dat.jpg}


   Sắp xếp 28 quân bài domino ta có thể tạo ra một hìmh chữ nhật kích thước 7*8 ô vuông. Mỗi cách sắp xếp như vậy sẽ tạo ra một bản đồ số. Ngược lại, mỗi bản đồ số có thể tương ứng với một số cách xếp.   


   Bài toán đặt ra là cho trước một bảnng số, hãy liệt kê tất cả các cách xếp có thể tạo ra từ nó.  

\subsubsection{   Input  }

   ma trận 7*8 mô tả bản đồ số ban đầu.  

\subsubsection{   Output  }

   Kết quả ghi ra file DOMINO.OUT dòng đầu là số lượng p cách xếp tìm được. Tiếp theo là p nhóm dòng, mỗi nhóm gồm 7 dòng ghi các dòng của các bảng tương ứng với một bảng số tìm được.  

\subsubsection{   Example  }
\begin{verbatim}
Ví dụ bản đồ số:


\textbf{4 2 5 2 6 3 5 4 }


\textbf{5 0 4 3 1 4 1 1 }


\textbf{1 2 3 0 2 2 2 2 }


\textbf{1 4 0 1 3 5 6 5 }


\textbf{4 0 6 0 3 6 6 5 }


\textbf{4 0 1 6 4 0 3 0 }


\textbf{6 5 3 6 2 1 5 3} 


tương ứng với hai cách xếp mô tả bởi hai bảng số sau:


\textbf{16 16 24 18 18 20 12 11 }


\textbf{06 06 24 10 10 20 12 11 }


\textbf{08 15 15 03 03 17 14 14 }


\textbf{08 05 05 02 19 17 28 26 }


\textbf{23 01 13 02 19 07 28 26 }


\textbf{23 01 13 25 25 07 04 04 }


\textbf{27 27 22 22 09 09 21 21 }





\textbf{16 16 24 18 18 20 12 11 }


\textbf{06 06 24 10 10 20 12 11 }


\textbf{08 15 15 03 03 17 14 14 }


\textbf{08 05 05 02 19 17 28 26 }


\textbf{23 01 13 02 19 07 28 26 }


\textbf{23 01 13 25 25 07 21 04 }


\textbf{27 27 22 22 09 09 21 04}\end{verbatim}
