



   Nguyên vừa được tuyển dụng vào một công ty vẽ bản đồ. Phần việc của Nguyên là xác định các ‘đỉnh’ của một bản đồ. Rất tiếc đây không phải là một công việc dễ dàng.  

   Nguyên được cho một bản đồ dưới dạng bảng   \textbf{    h   }   *   \textbf{    w   }   số, số ở dòng i, cột j thể hiện độ cao cho điểm (i,j) trên bản đồ. Chúng ta gọi một điểm độ cao x bất kì là   \textbf{    d   }   -đỉnh khi và chỉ khi không thể đi tới một điểm cao hơn nó (đi ở đây là đi qua các ô kề cạnh) mà không đi qua một điểm có độ cao nhỏ hơn hoặc bằng x-d. Nguyên được yêu cầu đếm số lượng   \textbf{    d   }   -đỉnh của bản đồ đã cho.  

\subsubsection{   Input  }

   Một dòng duy nhất là 3 số nguyên   \textbf{    h   }   ,   \textbf{    w   }   ,   \textbf{    d   }   (1 ≤   \textbf{    h   }   ,   \textbf{    w   }   ≤ 500, 1≤   \textbf{    d   }   ≤ 10   $^    9   $   ).  

\textbf{    h   }   dòng sau mỗi dòng chứa   \textbf{    w   }   số, số thứ j ở dòng i thể hiện chiều cao của điểm (i,j). Các số này đều nằm trong khoảng 0 đến 10   $^    9   $   .  

\subsubsection{   Output  }

   In ra 1 số duy nhất là số điểm được coi là   \textbf{    d   }   -đỉnh.  

\subsubsection{   Example  }
\begin{verbatim}
\textbf{Input:}

6 10 2

0 0 0 0 0 0 0 0 0 0

0 1 2 1 1 1 1 0 1 0

0 2 1 2 1 3 1 0 0 0

0 1 2 1 3 3 1 1 0 0

0 2 1 2 1 1 1 0 2 0

0 0 0 0 0 0 0 0 0 0\textbf{Output:}

4\end{verbatim}