



   Cây khế nhà Khánh rất sai quả nên có một con chim to to đến ăn. Ăn xong, chim chở Khánh ra đảo để trả công bằng vàng. Đảo có N cục vàng. Anh ấy muốn chuyển hết cả N cục vàng của mình về nhà. Nhưng khổ nổi các cục vàng này lại có trọng lượng và kích thước khổng lồ. Khánh đem theo một cái túi ba trăm gang to đùng nhưng vẫn chưa chắc chứa hết đống vàng này. Khổ quá đi! Lấy cục nào, bỏ cục nào bây giờ! Các bạn giúp anh ấy tìm ra một cách chọn vàng để thu được giá trị lớn nhất mà vẫn không làm rách túi đi.  

\subsubsection{   Input  }
\begin{itemize}
	\item     Dòng 1: Chứa 2 số nguyên: số cục vàng N (1 ≤ N ≤ 40) và tải trọng tối đa của túi M (1 ≤ M ≤ $10^{9}$    ).   
	\item     N dòng sau: Mỗi dòng chứa 2 số nguyên: trọng lượng $W_{i}$    và giá trị $V_{i}$    của cục vàng thứ i (1 ≤ $W_{i}$    , $V_{i}$    ≤ $10^{8}$    ).   
\end{itemize}

\subsubsection{   Output  }
\begin{itemize}
	\item     Một số nguyên duy nhất là giá trị lớn nhất thu được.   
\end{itemize}

\subsubsection{   Example  }
\begin{verbatim}
\textbf{Input:}


3 4


1 4


2 5


3 6





\textbf{Output:


}10


\end{verbatim}
