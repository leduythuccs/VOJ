

 

Xét một mạng điện gồm N nút (đánh số từ 1 đến N) và hệ thống gồm M đường dây , mỗi đường dây nối trực tiếp một cặp nút nào đó của mạng . Với mục đính khảo sát hiệu thế giữa hai nút s, t nào đó của mạng ảnh hưởng đến điện áp của các nút trong mạng, người ta muốn xác định các nút gọi là các nút thế năng của mạng. Một nút của mạng được gọi là nút thế năng nếu như việc truyền tải điện năng từ nút s đến nút t trên mạng có thể thực hiện theo tuyến đường dây có đi qua nút này đồng thời mỗi nút của mạng xuất hiện trên tuyến đường dây này không quá một lần.




\textbf{Yêu cầu}: Xác định tất cả các nút thế năng của mạng điện .





Download test tại \href{http://vn.spoj.pl/content/ENET.rar{ đây } . Solution của bài này sẽ không được upload , các bạn phải tự giải. Lưu ý là đây là bộ test thử , còn bộ test dùng trong chương trình check có thể khác . }

\subsubsection{Input}

Dòng đầu tiên chứa bốn số N, M, s, t (N ≤ 1000, M ≤ 15000).


Dòng thứ i trong M dòng tiếp theo chứa hai số Di, Ci là các số hiệu hai nút tương ứng hai đầu mút của đường dây thứ i.

\subsubsection{Output}

Dòng đầu tiên ghi số K là số lượng nút thế năng tìm được.


Dòng thứ i trong K dòng cuối cùng ghi số hiệu của nút thế năng thứ i , các chỉ số được ghi theo thứ tự tăng dần .

\subsubsection{Example}


\includegraphics{http://www.spoj.pl/CSP/content/enet.gif}
\begin{verbatim}
Input:
3 2 1 3
3 1
1 2
Output:
2
1
3
\end{verbatim}
