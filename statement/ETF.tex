



   Trong số học, hàm Ơ-le   
\includegraphics{http://vn.spoj.pl/content/phi.jpg}   của một số nguyên dương n được định nghĩa là số lượng các số nguyên dương nhỏ hơn hoặc bằng n và nguyên tố cùng nhau với n.  

   Cho số nguyên dương n (1  $\le$  n  $\le$  10^6). Tính giá trị của hàm Ơ-le   
\includegraphics{http://vn.spoj.pl/content/phi.jpg}   .  

\subsubsection{   Input  }

   Dòng đầu chứa số nguyên T là số test (T  $\le$  20000)  

   T dòng tiếp theo, mỗi dòng chứa một số nguyên n.  

\subsubsection{   Output  }

   T dòng, mỗi dòng ghi kết quả của test tương ứng.  

\subsubsection{   Example  }
\begin{verbatim}
Input:
5
1
2
3
4
5

Output:
1
1
2
2
4

\end{verbatim}
