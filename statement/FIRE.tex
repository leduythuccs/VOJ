



   Để đối phó với tình hình biến động của giá xăng dầu, nước X quyết định xây dựng một kho dự trữ dầu với quy mô cực lớn. Kho chứa dầu sẽ bao gồm N bể chứa dầu hình trụ tròn mà ta sẽ biểu diễn trên bản đồ bằng N hình tròn, hình tròn thứ i có tọa độ là ($X_{i}$   , $Y_{i}$   ) và bán kính $R_{i}$   , các hình tròn không có điểm chung trong với nhau (nhưng có thể tiếp xúc).  

   Để đảm bảo an toàn phòng cháy chữa cháy, người ta cần xác định 2 bể chứa dầu gần nhau nhất để tăng cường cách ly khi xảy ra hỏa hoạn.  

   Biết rằng khoảng cách giữa 2 bể chứa dầu thứ i và thứ j chính bằng khoảng cách giữa 2 đường tròn tương ứng và bằng $D_{ij}$   – $R_{i}$   – $R_{j}$   , trong đó $D_{ij}$   là khoảng cách Euclide giữa 2 điểm ($X_{i}$   , $Y_{i}$   ) và ($X_{j}$   , $Y_{j}$   ).  

   Bạn hãy giúp những người quản lý tìm ra 2 bể chứa dầu này.  

\subsubsection{   Input  }

   Dòng thứ nhất ghi số nguyên dương N là số bể chứa dầu.  

   Dòng thứ i trong N dòng tiếp theo ghi 3 số nguyên $X_{i}$   , $Y_{i}$   , $R_{i}$   là tọa độ và bán kính bể chứa dầu thứ i.  

\subsubsection{   Output  }

   Gồm 1 dòng duy nhất là khoảng cách của 2 bể chứa dầu bé nhất tìm được.  

\subsubsection{   Example  }
\begin{verbatim}
\textbf{Input:}
3
0 0 1
4 0 2
5 5 3

\textbf{Output:}
0.0990

\textbf{Giới hạn:}
2 ≤ N ≤ 10000. 
|$X_{i}$|, |$Y_{i}$| ≤ $10^{6}$.
0 $<$ $R_{i}$ ≤ $10^{6}$. 
Kết quả ghi chính xác đến 4 chữ số sau dấu phẩy.
\end{verbatim}
