



   Cho đơn đồ thị vô hướng N đỉnh và M cạnh, trọng số các cạnh đều nguyên dương. Có 2 loại câu hỏi :   
\\   0 u v : Cho biết đường đi ngắn nhất từ u tới v có độ dài là bao nhiêu.   
\\   1 u v : Hãy chỉ ra 1 đường đi ngắn nhất từ u -> v   
\\   Bài cơ bản này nhằm kiểm tra kỹ năng xây dựng các module chương trình con dành cho truy vết 1 cách hợp lý, sử dụng nhuần nhuyễn chương trình con, lời gọi hàm .   
\\
\\\textit{    Download test và solution tại    \href{http://vn.spoj.pl/content/floyd.rar}{     đây    }}

\subsubsection{   Input  }

   Dòng 1 : 3 số nguyên N , M , K . ( 1 ≤ N ≤ 100 , 1 ≤ M ≤ N*(N-1)/2  , 1 ≤ K ≤ 1000 )   
\\   M dòng tiếp theo , dòng thứ i gồm 3 số nguyên dương u , v , c cho biết cạnh (u,v) có trọng số là c ( 1 ≤ c ≤ 10000 )   
\\   K dòng tiếp theo là K câu hỏi , dòng thứ j sẽ có định dạng như đã nêu ở trên .   
\\

\subsubsection{   Output  }

   Ứng với mỗi câu hỏi trong K câu hỏi thì ta phải trả lời trên mỗi dòng như sau .   
\\   Câu hỏi 0 u v : Ghi ra 1 số nguyên duy nhất là độ dài đường đi ngắn nhất từ u -> v.   
\\   Câu hỏi 1 u v : Ghi ra số đầu tiên là số X là số đỉnh trên đường đi ngắn nhất này , tiếp đó ghi ra X số là chỉ số các đỉnh theo thứ tự xuất hiện trên hành trình .   
\\

\subsubsection{   Example  }
\begin{verbatim}
Input:
3 3 2
1 2 3
2 3 1
1 3 5
0 1 2
1 1 3

Output:
3
3 1 2 3
\end{verbatim}
