

 

Sau khi tham dự IOI và OLPSV, Nguyên chuyển đến một ngôi nhà mới. Khu nhà mới của Nguyên có \textbf{ N } người bạn hàng xóm ( \textbf{ N } ≤ 200000). Vì dễ bị nhầm nên Nguyên đánh số các bạn ấy từ 1 đến \textbf{ N } . Giữa các ngôi nhà có đường đi tạo thành cây. Khoảng cách giữa hai căn nhà kề nhau là 1 đơn vị. Có \textbf{ K } cuộc hẹn ( \textbf{ K } ≤ \textbf{ N } /2) được Nguyên đưa ra để làm quen với các bạn mới. Để tính toán chi phí mời các bạn, Nguyên muốn biết xem khoảng cách xa nhất của 2 ngôi nhà trong một cuộc hẹn là bao nhiêu ? Bạn hãy giúp Nguyên giải quyết vấn đề này.

\subsubsection{Input}

- Dòng 1 gồm 2 số \textbf{ N } và \textbf{ K } .

- \textbf{ N } dòng tiếp theo, dòng thứ i gồm 2 số x y. Trong đó x là thứ tự của cuộc hẹn mà bạn thứ itham gia , y là nhà hàng xóm của bạn thứ i . Nếu y = 0 thì đó là gốc của khu dân cư (có thể hiểu là gốc của cây).

\subsubsection{Output}

Gồm \textbf{ K } dòng, dòng thứ i thể hiện đường đi xa nhất tìm được giữa 2 ngôi nhà của 2 người bạn trong cuộc hẹn thứ i .

\subsubsection{Example}
\begin{verbatim}
\textbf{Input:}
6 2
1 3
2 1
1 0
2 1
2 1
1 5
\textbf{Output:}
3
2
\textbf{Giải thích:}\end{verbatim}


\texttt{  -3-
\\   |
\\  -1-
\\/ | $\backslash$
\\2  4  5
\\      |
\\     -6-}
\\
\\​Trong cuộc hẹn thứ 1 gồm 3 bạn là bạn số 1, số 3 và số 6. Khoảng cách xa nhất giữa 2 ngôi nhà trong cuộc hẹn thứ 1 là 3 ( giữa nhà bạn số 3 và số 6). Tương tự, cuộc hẹn thứ 2 gồm 3 bạn số 2, số 4 và số 5, khoảng cách  xa nhất là 2.

 

 

 

 

 

 
