







   Công ti phần mềm FSoft mới nhận thêm N nhân viên. Lãnh đạo công ti muốn số nhân viên mới trên chia thành M nhóm khác nhau thỏa mãn rằng: số người trong các nhóm bằng nhau, mỗi người chỉ thuộc một nhóm. Qua phân tích bản CV và lúc phỏng vấn, công ti đưa ra bảng đánh giá A với $A_{ij}$   là sự hiệu quả khi người thứ i và j thuộc cùng một nhóm.   
\\   Ban lãnh đạo công ti nhờ bạn tìm cách phân chia N nhân viên thành M nhóm sao cho hiệu quả nhất. Sự hiệu quả của một cách phân chia bằng tổng sự hiệu quả của từng nhóm; sự hiệu quả của mỗi nhóm bằng tổng sự hiệu của của các cặp trong nhóm đó.  

\subsubsection{   Dữ liệu  }

   - Dòng đầu tiên chứa hai số N, M.   
\\   - N dòng tiếp theo, mỗi dòng ghi N số mô tả bảng A. ($A_{ii}$   =0; $A_{ji}$   =$A_{ij}$   )  

\subsubsection{   Kết quả  }

   - Dòng đầu tiên ghi sự hiệu quả của cách phân chia mà bạn tìm được.   
\\   - M dòng tiếp theo ghi danh sách nhân viên của mỗi nhóm trong cách chia của bạn.  

\subsubsection{   Ví dụ  }
\begin{verbatim}
\textbf{Dữ liệu:}
\\4 2
\\0 1 2 3
\\1 0 5 1
\\2 5 0 2
\\3 1 2 0
\\
\\\textbf{Kết quả:}
\\8
\\1 4
\\2 3
\\\end{verbatim}

\subsubsection{   Giới hạn  }

   - 1 ≤ N ≤ 100.   
\\   - 1 ≤ M ≤ 10. (N mod M = 0)  

\subsubsection{   Cách tính điểm  }

   - Có 10 tests; với mỗi test, gọi sự hiệu quả bạn tìm được là RES, đáp án là ANS; nếu RES khớp với danh sách các nhóm bạn đưa ra thì số điểm bạn đạt được là   \textbf{    (RES/ANS)    $^     4    $    x10   }   .   
\\   - Điểm của toàn bài bằng tổng điểm cho mỗi test.  

