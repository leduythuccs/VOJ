



   Nông dân John quyết định mang nước tới cho N (1  $\le$  N  $\le$  300) đồng cỏ của mình, để thuận tiện ta đánh số các đồng cỏ từ 1 đến N. Để tưới nước cho 1 đồng cỏ John có thể chọn 2 cách, 1 là đào ở đồng cỏ  đó 1 cái giếng hoặc lắp ống nối dẫn nước từ những đồng cỏ trước đó đã có nước tới.  

   Để đào một cái giếng ở đồng cỏ i cần 1 số tiền là W\_i (1  $\le$  W\_i  $\le$  100,000). Lắp ống dẫn nước nối 2 đồng cỏ i và j cần 1 số tiền là P\_ij (1  $\le$  P\_ij  $\le$  100,000; P\_ij = P\_ji; P\_ii=0).  

   Tính xem nông dân John phải chi ít nhất bao nhiêu tiền để tất cả các  đồng cỏ đều có nước.  

\subsubsection{   DỮ LIỆU  }
\begin{itemize}
	\item     Dòng 1: Một số nguyên duy nhất: N   
	\item     Các dòng 2..N + 1: Dòng i+1 chứa 1 số nguyên duy nhất: W\_i   
	\item     Các dòng N+2..2N+1: Dòng N+1+i chứa N số nguyên cách nhau bởi dấu cách; số thứ         j là P\_ij   
\end{itemize}

\subsubsection{   KẾT QUẢ  }
\begin{itemize}
	\item     Dòng 1: Một số nguyên duy nhất là chi phí tối thiểu         để cung cấp nước cho tất cả các đồng cỏ.   
\end{itemize}

\subsubsection{   VÍ DỤ  }
\begin{verbatim}
Dữ liệu
4
5
4
4
3
0 2 2 2
2 0 3 3
2 3 0 4
2 3 4 0


Kết quả
9
\end{verbatim}

\subsubsection{   GIẢI THÍCH  }

   Có 4 đồng cỏ. Mất 5 tiền để đào 1 cái giếng ở đồng cỏ 1, 4 tiền để đào ở đồng cỏ 2, 3 và 3 tiền để đào ở đồng cỏ 4. Các ống dẫn nước tốn 2, 3, và 4 tiền tùy thuộc vào nó nối đồng cỏ nào với nhau.  

   Nông dân John có thể đào 1 cái giếng ở đồng cỏ thứ 4 và lắp ống dẫn  nối đồng cỏ 1 với tất cả 3 đồng cỏ còn lại, chi phí tổng cộng là 3 + 2 + 2 + 2 = 9.  
