



   1 nhà thám hiểm nọ vừa phát hiện ra một bản đồ kho báu . Trên bản đồ miêu tả 1 hòn đảo  nằm ở nam Thái Bình Dương . Trên hòn đảo có N vị trí có kho báu là các mỏ vàng . Để được phép khai thác nhà thám hiểm quyết định dốc hết tiền của ra mua 1 mảnh đất và khai thác các mỏ vàng trên đó . Tuy nhiên Nhà thám hiểm cũng không giàu có gì lắm nên chỉ có thể mua được 1 miếng đất hình chữ nhật có kích thước tối đa là S * W và theo yêu cầu của Chúa Đảo thì miếng đất phải song song với 2 trục Ox và Oy để không làm mất mỹ quan của hòn đảo ( các mỏ vàng nằm trên đường biên của miếng đất cũng sẽ được quyền khai thác ) . Bạn hãy lập trình giúp Nhà thám hiểm tính xem ông ta có thể chiếm được nhiều nhất là bao nhiêu mỏ vàng .   
\\       Lưu ý : Bài này nếu không cẩn thận sẽ rất dễ bị ngộ nhận . Vì vậy nên phải đặc biệt chú ý. Là 1 bài khó vì thế nên sau một thời gian sẽ để cho các bạn có thể xem lời giải và download test .      
\\
\\\textit{    Download test và solution tại    \href{http://vn.spoj.pl/content/GOLD.rar}{     đây    }    .   }

\subsubsection{   Input  }

   Dòng 1 : 2 số nguyên dương S W ( 1 ≤ S , W ≤ 10000 ). S là độ dài cạnh song song với trục Ox . W là độ dài cạnh song song với trục Oy .   
\\   Dòng 2 : số nguyên dương N ( 1 ≤ N ≤ 15000 ) .   
\\   N dòng tiếp theo , dòng thứ i mô tả vị trí của mỏ vàng thứ i là 2 số nguyên xi và yi . ( -30000 ≤ xi , yi ≤ 30000 ) .  

\subsubsection{   Output  }

   Gồm 1 dòng duy nhất ghi ra số lượng nhiều nhất mỏ vàng mà Nhà thám hiểm có thể có được .  

\subsubsection{   Example  }
\begin{verbatim}
Input:
1 2
12
0 0
1 1
2 2
3 3
4 5
5 5
4 2
1 4
0 5
5 0
2 3
3 2

Output:
4
\end{verbatim}