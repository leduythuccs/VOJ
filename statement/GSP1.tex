

Cho n kí tự trong bảng kí tự.

Xét n hoán vị vòng quanh. Sắp xếp chúng theo thứ tự từ điển tăng dần.

Nếu 2 hoán vị có cùng thứ tự từ điển thì ưu tiên hoán vị có chỉ số của kí tự bắt đầu nhỏ hơn xếp trước.

\textbf{Input:}
\begin{itemize}
	\item Dòng đầu ghi số n (1 $\le$  n  $\le$  5 * 10 ^ 4)
	\item N dòng sau: dòng thứ i ghi a\_i (1  $\le$  a\_i  $\le$  2^31) là thứ tự của kí tự thứ i trong bảng chữ cái.
\end{itemize}

\textbf{Output:}

Ghi ra n dòng là n chỉ số tương ứng sau khi sắp xếp.

\textbf{Ví dụ}
\begin{verbatim}
Input
5
2
1
1
2
4
Kết qủa
2
3
1
4
5\end{verbatim}
