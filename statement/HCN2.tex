

 

Một mảnh đất được quy hoạch để làm khu vui chơi. Tuy nhiên mảnh đất lại không vuông vắn, trong khi khu vui chơi lại yêu cầu phải có dạng hình chữ nhật. Khuôn viên của mảnh đất được xác định bởi một đường gấp khúc khép kín không tự cắt với các hình chữ nhật có toạ độ nguyên trên hệ toạ độ vuông góc Đề-các Oxy.

Xác định trong khuôn viên mảnh đất hình chữ nhật có diện tích lớn nhất thoả mãn các điều kiện:
\begin{itemize}
	\item Hình chữ nhật phải nằm hoàn toàn trong khuôn viên mảnh đất đã cho (nghĩa là hình chữ nhật không được chứa điểm nằm ngoài khuôn viên mảnh đất)
	\item Bốn đỉnh của hình chữ nhật phải có toạ độ nguyên
	\item Các cạnh phải song song với trục toạ độ Ox hoặc Oy
\end{itemize}

\subsubsection{Input}
\begin{itemize}
	\item Dòng 1: N là số đỉnh của đường gấp khúc bao quanh mảnh đất. (N $\le$ 100)
	\item Dòng thứ i trong số N dòng tiếp theo chứa 2 số nguyên Xi, Yi (0 $\le$ Xi,Yi $\le$ 1000) được ghi cách nhau bởi 1 dấu cách. Các đỉnh của đường gấp khúc được đánh số từ 1 theo 1 chiều đi vòng quanh nó
\end{itemize}

\subsubsection{Output}

Gồm 1 dòng duy nhất chứa diện tích hình chữ nhật lớn nhất thoả mãn đề bài.

\subsubsection{Example}
\begin{verbatim}
Input:
7
0 5
2 7
3 5
4 9
6 5
5 0
0 0


Output:
25
\end{verbatim}

các bạn có thắc mắc về đề bài hoặc test xin liên hệ quynh6174 qua forum vnoi.info
