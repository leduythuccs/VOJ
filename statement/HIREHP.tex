



     Giáo sư X có một kỳ nghỉ kéo dài n ngày đánh số từ 1 tới n. Ông ta muốn thuê những chiếc mô-tô để đi ngắm cảnh bởi ông muốn thử cảm giác tốc độ giữa quang cảnh hoang dã của thiên nhiên. Dịch vụ du lịch có đúng n chiếc xe cho thuê. Ngày thứ i, người ta chỉ cho thuê chiếc xe thứ i, thời gian thuê từ đầu ngày thứ i tới hết ngày t\_i (t\_i≥i) với giá thuê là p\_i, tức là nếu vào ngày i giáo sư X trả p\_i đồng để thuê chiếc xe thứ i, ông ta phải trả lại nó không muộn hơn ngày t\_i và khi ông ta đã trả lại chiếc xe đang thuê mới được phép thuê một chiếc xe khác.       Yêu cầu: Bạn hãy giúp giáo sư X tính xem cần ít nhất bao nhiêu tiền để thuê xe sao cho ngày nào cũng có xe để đi    



   Giáo sư X có một kỳ nghỉ kéo dài n ngày đánh số từ 1 tới n. Ông ta muốn thuê những chiếc mô-tô để đi ngắm cảnh bởi ông muốn thử cảm giác tốc độ giữa quang cảnh hoang dã của thiên nhiên. Dịch vụ du lịch có đúng n chiếc xe cho thuê. Ngày thứ i, người ta chỉ cho thuê chiếc xe thứ i, thời gian thuê từ đầu ngày thứ i tới hết ngày t\_i (t\_i≥i) với giá thuê là p\_i, tức là nếu vào ngày i giáo sư X trả p\_i đồng để thuê chiếc xe thứ i, ông ta phải trả lại nó không muộn hơn ngày t\_i và khi ông ta đã trả lại chiếc xe đang thuê mới được phép thuê một chiếc xe khác.  

\textbf{Yêu cầu:}   Bạn hãy giúp giáo sư X tính xem cần ít nhất bao nhiêu tiền để thuê xe sao cho ngày nào cũng có xe để đi  





\subsubsection{   Input  }


\begin{itemize}
	\item     Dòng 1 chứa số nguyên dương n≤5.10^5   
	\item     n dòng tiếp theo, dòng thứ i chứa hai số nguyên dương t\_i,p\_i (i≤t\_i≤n;p\_i≤ 10^6) cách nhau ít nhất một dấu cách.   
\end{itemize}



\subsubsection{   Output  }

   Ghi ra một số nguyên duy nhất là số tiền thuê xe  

\subsubsection{   Example  }
\begin{verbatim}
\textbf{Input:}
4
3 10
3 20
4 1
4 40


\textbf{Output:}
11 \end{verbatim}
\begin{itemize}
	\item     Ít nhất 50\% số điểm ứng với các test có n≤10^3   
	\item     Ít nhất 75\% số điểm ứng với các test có n≤10^5   
\end{itemize}
