

Một mạng giao thông gồm N nút giao thông, và có M đường hai chiều nối một số cặp nút, thông tin về một đường gồm ba số nguyên dương u, v là tên hai nút đầu mút của đường, và l là độ dài đoạn đường đó. Biết rằng hai nút giao thông bất kì có không quá 1 đường hai chiều nhận chúng làm hai đầu mút.
\\Cho hai nút giao thông s và f, hãy tìm hai đường đi nối giữa s với f sao cho hai trên hai đường không có cạnh nào được đi qua hai lần và tổng độ dài 2 đường đi là nhỏ nhất.

\subsubsection{Input}
\begin{itemize}
	\item Dòng đầu ghi N, M (N ≤ 100)
	\item Dòng thứ 2 ghi hai số s, f.
	\item M dòng tiếp theo, mỗi dòng mô tả một đường gồm ba số nguyên dương u, v, l.
\end{itemize}

\subsubsection{Output}
\begin{itemize}
	\item Dòng đầu ghi T là tổng độ dài nhỏ nhất tìm được hoặc -1 nếu không tìm được.
	\item Nếu tìm được, hai dòng sau, mỗi dòng mô tả một đường đi gồm: số đầu là số nút trên đường đi này, tiếp theo là dãy các nút trên đường đi bắt đầu từ s, kết thúc tại f.
\end{itemize}

Chú ý: Phạm vi tính toán trong vòng Longint.

\subsubsection{Example}
\begin{verbatim}
\textbf{Input}
5 8
1 5
1 2 1
1 4 8
2 3 5
2 4 1
3 5 1
4 3 8
4 5 1
1 3 1

\textbf{Output}
5
3 1 3 5 
4 1 2 4 5
\end{verbatim}
