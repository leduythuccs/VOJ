

Từ cặp số (a, b) gồm 2 số nguyên dương, có thể sử dụng 1 trong 3 phép biến đổi sau để tạo ra cặp số mới • • • (a,b) → (a, a+b) (a,b) → (a+b, b) (a,b) → (b,a) Bắt đầu từ cặp số (1, 1) hãy dùng ít phép biến đổi nhất để tạo ra một cặp số có chứa số N.

Từ cặp số (a, b) gồm 2 số nguyên dương, có thể sử dụng 1 trong 3 phép biến đổi sau để tạo ra

cặp số mới

(a,b) → (a, a+b)

(a,b) → (a+b, b)

(a,b) → (b,a)

Bắt đầu từ cặp số (1, 1) hãy dùng ít phép biến đổi nhất để tạo ra một cặp số có chứa số N.

 

\subsubsection{Input}

Dòng đầu chứa số test T. Tiếp theo là T test, mỗi test chứa một số 1 ≤ N ≤ 10^6.

\subsubsection{Output}

Ứng với mỗi test, in ra trên một dòng số bước biến đổi ít nhất.

\subsubsection{Example}
\begin{verbatim}
\textbf{Input:}
4
1
3
5
7

\textbf{Output:}
0
2
3
4\end{verbatim}
