



     Cho D 0 là chuỗi hai kí tự "Fa". Với n ≥ 1, tạo ra D n từ D n-1 theo các quy tắc viết lại chuỗi:       •       •       "a" → "aRbFR"       "b" → "LFaLb"       Do đó, D 0 = "Fa", D 1 = "FaRbFR", D 2 = "FaRbFRRLFaLbFR", ...    

   Cho D(0) là chuỗi hai kí tự "Fa". Với n ≥ 1, tạo ra D(n) từ D(n-1) theo các quy tắc viết lại chuỗi:  

   "a" → "aRbFR"  

   "b" → "LFaLb"  

   Do đó, D(0) = "Fa", D(1) = "FaRbFR", D(2) = "FaRbFRRLFaLbFR", ...  

   Các chuỗi có thể được hiểu như chỉ dẫn của một chương trình đồ họa máy tính: "F" có nghĩa là "vẽ về phía trước đoạn một đơn vị"; "L" có nghĩa là "rẽ trái 90 độ"; "R" có nghĩa là "rẽ phải 90 độ"; "a" và "b" bị bỏ qua. Con trỏ máy tính ban đầu ở vị trí (0,0), hướng về phía (0,1). Khi đó bản vẽ kỳ lạ D(n) được gọi là Đường Rồng bậc n.  

   Ví dụ, D(10) được hiển thị ở hình bên. coi "F" là một bước, vị trí xanh tại (18,16) là vị trí đạt được sau 500 bước.  

   Vị trí của con trỏ X bước trong D(N) là ở đâu? Đưa ra câu trả lời của bạn ở dạng x, y không có khoảng trống ở giữa.  



\subsubsection{   Input  }

   Gồm nhiều bộ test, mỗi bộ test ghi trên một dòng gồm 2 số nguyên X ≤ 10^13 và N ≤ 100  

\subsubsection{   Output  }

   Với mỗi test, in ra trên một dòng vị trí x và y cách nhau 1 dấu cách  

\subsubsection{   Example  }
\begin{verbatim}
\textbf{Input:}
500 10

\textbf{Output:}
18 16
\end{verbatim}
