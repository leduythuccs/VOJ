



   Một ngày nọ, tại lớp học đội tuyển, Natodel và các học sinh khác đang làm bài tập thì bỗng dưng mất điện. Đây là trường hợp vô cùng bất ngờ đối với mọi người trong lớp đội tuyển tin. Để các học sinh của mình không có thời gian để "chém gió", Ktun đã nghĩ ra một bài tập để các học sinh của mình phải giải ra được, bài tập đó có nội dung như sau:  

   "Cho một số nguyên dương N, bài toán đặt ra là phải tìm tất cả các số X thoả mãn X! có số chữ số bằng N."  

   Ban đầu Ktun chỉ cho giới hạn rất nhỏ để thử các học sinh của mình, thế nên rất nhanh chóng Natodel và các bạn đã giải bài tập này bằng cách bấm máy tính cầm tay. Nhưng rồi oái oăm hơn, Ktun nhất quyết không cho học sinh chơi nên đã tăng giới hạn của N lên rất cao, điều đó làm cho cả lớp phải run sợ (do không có máy vi tính bên cạnh). Bạn đang sẵn máy tính, hãy giúp các Natodel giải bài tập này để họ có thời gian "chém gió" với nhau.   
\includegraphics{../../../gfx/jscripts/tiny_mce/plugins/emotions/img/smiley-tongue-out.gif}

\subsubsection{   Dữ liệu vào  }

   Một dòng duy nhất chứa số nguyên dương N (N ≤ 2.$10^{6}$   )  

\subsubsection{   Dữ liệu ra  }

   Nếu không có số X nào thỏa mãn thì in ra "NO".  

   Nếu có số X thỏa mãn thì in ra nhiều dòng:   


   - Dòng thứ nhất chứa số S là số lượng số X.   


   - S dòng tiếp theo chứa 1 số X ghi theo thứ tự tăng dần.  

\subsubsection{   Ví dụ  }
\begin{verbatim}
\textbf{Input:}


5


\textbf{Output:}


1


8\end{verbatim}
