



   Hệ thống ngôn ngữ ở đất nước X có   \textbf{    N   }   ký tự là   C$_    1   $   ,   C$_    2   $   , …   $C_{n$}   và mọi văn bản lưu thông ở đây đều phải viết bằng ngôn ngữ này. Để in 1 ký tự   $C_{i$}   nào đó sẽ tốn 1 lượng mực in là   $M_{i$}   đơn vị. Có một nhà toán học ngốc nghếch cứ thắc mắc là với một lượng mực in là   \textbf{    S   }   đơn vị thì có thể in được bao nhiêu xâu ký tự khác nhau mà lại sử dụng vừa hết đúng   \textbf{    S   }   đơn vị nhỉ ? Ông ta cứ thắc mắc câu hỏi đó đến mức lẩn thẩn hết cả người.  

   Bạn là một học sinh chuyên Tin/một sinh viên CNTT tài năng, hãy lập trình giúp ông ấy giải quyết bài toán này nhé.  

\subsubsection{   Input  }

   Dòng đầu tiên gồm 2 số nguyên dương   \textbf{    N   }   và   \textbf{    S   }   (   \textbf{    N   }   ≤ 26,   \textbf{    S   }   ≤ $10^{200}$   )  

\textbf{    N   }   dòng tiếp theo, dòng thứ i dòng gồm 1 số nguyên dương $M_{i}$   cho biết lượng mực cần thiết để in ký tự $C_{i}$   ( $M_{i}$   ≤ N ) .  

\subsubsection{   Output  }

   Ghi ra (số lượng xâu ký tự khác nhau có thể in được với lượng mực là   \textbf{    S   }   ) mod 777777777.  

\subsubsection{   Example  }
\begin{verbatim}
\textbf{Input:}


4 3


1


3


2


4


\textbf{Output:}


4\end{verbatim}
