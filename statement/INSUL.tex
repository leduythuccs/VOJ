



   Cho một dãy N viên gạch lần lượt có độ cách nhiệt là các số $a_{1}$   .. $a_{N}$   . Nếu xếp lần lượt các viên gạch theo trình tự đó thì độ cách nhiệt cả khối là $a_{1}$   + $a_{2}$   + ... + $a_{N}$   + max(0, $a_{2}$   - $a_{1}$   ) + max(0, $a_{3}$   - $a_{2}$   ) + ... + max(0, $a_{N}$   - a   $_    N - 1   $   ). Nhiệm vụ của bạn là tìm cách xếp sao cho độ cách nhiệt của cả khối là lớn nhất có thể.  

\subsubsection{   Dữ liệu  }
\begin{itemize}
	\item     Dòng đầu ghi số nguyên dương N (0 $<$ n ≤ 10^5).   
	\item     N dòng sau mỗi dòng ghi một số $a_{i}$    ( 1 ≤ i ≤ N và 1 ≤ $a_{i}$    ≤ 10000).   
\end{itemize}

\subsubsection{   Kết qủa  }

   Ghi trên một dòng kết quả cần tìm.  

\subsubsection{   Ví dụ  }
\begin{verbatim}
\textbf{Dữ liệu:} 
4
5
4
1
7
\textbf{Kết qủa} 
24 
\end{verbatim}
