



   Cho đồ thị cây có trọng số gồm N đỉnh , các đỉnh được đánh số từ 1 -> N . Gốc của cây là đỉnh 1 . Cha của đỉnh u là 1 đỉnh có số hiệu nhỏ hơn u . Mỗi đỉnh có một nhãn là 1 số thực A[i] . Trong đó nhãn của đỉnh 1 bằng 1 và nhãn của đỉnh lá bằng 0 . Biết rằng A[v] ≤ A[u] nếu v là con của u .   


   Giá trị của 1 cây = Tổng (  ( A[u] – A[v] ) * Trọng số cạnh (u,v)  , với u là cha của v )   


   Bây giờ người ta cho biết các cạnh của đồ thị và trọng số của các cạnh này nhưng không cho biết các A[i]. Hãy tính xem giá trị của cây thấp nhất là bao nhiêu.  

\subsubsection{   Input  }

   Dòng 1 là số nguyên T là số bộ test . (  1 ≤ T ≤ 50 ) . T nhóm dòng tiếp theo mô tả từng bộ test . Mỗi bộ test sẽ có cấu trúc như sau :   


   Dòng 1 : số nguyên dương N ( 1 ≤ N ≤ 1000 ) .   


   Từ dòng 2 -> dòng N : dòng thứ i gồm 2 số nguyên dương u và c ( 1 ≤ u $<$ i , 0 ≤ c ≤ 1000 ) cho biết cha của nút i là nút u và cạnh nối (u,i) có trọng số là c .  

\subsubsection{   Output  }

   Với mỗi test ghi ra giá trị thấp nhất có thể đạt được của cây trên 1 dòng với độ chính xác là 2 chữ số sau dấu chấm.  

\subsubsection{   Example  }
\begin{verbatim}
Input:
1
4
1 1
1 2
2 1

Output:
3.00
\end{verbatim}
