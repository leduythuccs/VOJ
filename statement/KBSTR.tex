

Có một băng giấy gồm N ô vuồng liên tiếp nhau. K rất thích chơi tô màu với băng giấy này. Mỗi ô vuông K sẽ tô một màu. Nhưng vì nhà quá nghèo, quá nghèo và quá nghèo, nên K chỉ đủ tiền mua 4 màu là 'a', 'b', 'c' và 'd'. Bần cùng sinh đạo tặc, K cũng chôm thêm một cuốn sách tô màu ở thư viện để bù đắp cho sự thiếu óc thẩm mĩ của mình. Sách có rất nhiều mẫu tô màu khác nhau, mỗi mẫu là một dãy N kí tự thể hiện từng màu trong từng ô của băng giấy. Các mẫu được xếp theo thứ tự từ điển ('a' $<$ 'b' $<$ 'c' $<$ 'd'). Sách còn dạy rằng có một số màu không nên tô ở sau một màu khác, như thế sẽ không đẹp. Một ngày đẹp trời, tâm trạng buồn bực, K xé nát trang sách in mẫu thứ T trong sách. Than ôi, hối hận quá, K quyết định khôi phục lại mẫu tô màu đó, nhưng làm sao đây?

\subsubsection{Input}
\begin{itemize}
	\item Dòng thứ nhất: Ba số nguyên là N (1 ≤ N ≤ 10$^3$), số ô vuông trên băng giấy, M (1 ≤ M ≤ 15), số yêu cầu của sách và T (1 ≤ T ≤ 10$^9$), thứ tự của mẫu tô màu cần tìm. 
	\item Dòng thứ 2 đến M + 1: Mỗi dòng là hai ký tự x$_i $và y$_i$ viết liền nhau thể hiện một yêu cầu của sách là ký tự$_$y$_i$ không thể đứng liền sau ký tự x$_i$ trong mẫu tô màu.
\end{itemize}

\subsubsection{Output}
\begin{itemize}
	\item N ký tự liên tiếp thể hiện mẫu tô màu cần tìm. Dữ liệu vào đảm bảo có kết quả.
\end{itemize}

\subsubsection{Example}
\begin{verbatim}
\textbf{Input:}
3 1 5
ab

\textbf{Output:}
acb

Giải thích: Các mẫu tô màu thỏa mãn là: aaa, aac, aad, aca, acb, acc,... Mẫu thứ 5 là acb.
\end{verbatim}
