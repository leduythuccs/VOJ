



   Bé tuy mới vào lớp 1 nhưng rất thông mình và ham thích làm toán. Dù cô giáo đưa cho Bé 2 số có số lượng chữ số rất lớn, Bé đều dễ dàng tính đúng tổng của 2 số đó. Hôm nay, cô giáo đưa cho Bé một bài tập khó hơn.  

   Ở bài tập này, cô giáo sẽ nghĩ ra một tổng có dạng   \textbf{    A+B=C   }   trong đó A, B, C là các số tự nhiên. Cả A, B và C đều được viết theo đúng nguyên tắc ở hệ thập phân. Nói cách khác, sẽ không có chữ số 0 ở đầu mỗi số nếu số đó là nguyên dương. Sau khi cô giáo viết biểu thức A+B=C ra giấy, cô sẽ xóa đi một số vị trí (có thể là chữ số hoặc dấu cộng hay dấu bằng). Nhiệm vụ của Bé là khôi phục lại biểu thức ban đầu. Trong trường hợp có nhiều kết quả, Bé cần tìm biểu thức có thứ tự từ điển nhỏ nhất (coi cả biểu thức như một xâu ký tự để so sánh). Thứ tự của các ký tự lần lượt là: +,0,...,9,=.  

   Bạn hãy giúp Bé giải bài toán hóc búa trên. Lưu ý, đây là bài tập may mắn nên ngoài kết quả của bài toán, các bạn cần phải in ra một con số may mắn ở dòng đầu tiên. Con số may mắn sẽ là 1, 2 hoặc 3. Bộ test sẽ đảm bảo rằng mỗi con số sẽ ứng với đúng một loại test (dễ/trung bình/khó) và mỗi loại test sẽ có số điểm như nhau. Hình thức chấm vẫn là so file, nghĩa là bạn chỉ được điểm cho một test nếu con số may mắn bạn in ra trùng với con số ở kết quả mẫu.  
\begin{itemize}
	\item     Với bộ test dễ, biểu thức của cô giáo sẽ bị xóa hoàn toàn và có độ dài không quá 30.   
\end{itemize}
\begin{itemize}
	\item     Với bộ test trung bình, biểu thức của cô giáo sẽ có ít nhất một ký tự chưa bị xóa và có độ dài không quá 8.   
\end{itemize}
\begin{itemize}
	\item     Với bộ test khó, biểu thức của cô giáo sẽ có ít nhất một ký tự chưa bị xóa và có độ dài không quá 30.   
\end{itemize}

   Ở ví dụ bên dưới, đó là một test trung bình và con số may mắn được chọn  là 1. Tuy nhiên, số 1 có thể không tương ứng với các test trung bình ở  bộ test chính thức.  

\subsubsection{   Input  }




   Gồm một dòng duy nhất ghi biểu thức của cô giáo. Các ký tự bị xóa được thể hiện bằng dấu gạch dưới (\_).  

\subsubsection{   Output  }

   Dòng đầu ghi con số may mắn bạn chọn.   


   Dòng thứ hai ghi biểu thức đã được khôi phục.  

\subsubsection{   Example  }
\begin{verbatim}
\textbf{Input:}


____8





\textbf{Output:}


1


0+8=8


\end{verbatim}
