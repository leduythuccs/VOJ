



   Một dãy số được gọi là   \textit{    K – không đơn độc   }   nếu mỗi phần tử của dãy đều thuộc một đoạn gồm ít nhất       K      phần tử liên tiếp có giá trị giống nhau. Ví dụ dãy 1 1 2 2 2 1 1 là   \textit{    2 – không đơn độc   }   , nhưng không phải là   \textit{    3 – không đơn độc   }   vì phần tử đầu tiên chỉ thuộc một đoạn gồm 2 số 1. Nếu một dãy số chưa phải là   \textit{    K – không đơn độc   }   , bạn có quyền thực hiện các thao tác biến đổi, mỗi thao tác sẽ cộng ( hoặc trừ ) một đơn vị vào một phần tử của dãy.  

\subsubsection{   Yêu cầu  }

   Hãy đếm số thao tác ít nhất cần thực hiện để biến một dãy số thành dãy   \textit{    K – không đơn độc   }   .  

\subsubsection{   Dữ liệu  }
\begin{itemize}
	\item     Dòng đầu ghi 2 số N, K. N là số lượng phần tử của dãy số.   
	\item     Dòng sau ghi N số tự nhiên thể hiện dãy số.   
\end{itemize}

\subsubsection{   Kết quả  }
\begin{itemize}
	\item     Gồm một dòng duy nhất ghi ra số thao tác cần thực hiện.   
\end{itemize}

\subsubsection{   Ví dụ  }
\begin{verbatim}
Dữ liệu
7 3
2 2 3 4 4 5 5

Kết quả
3
\end{verbatim}

\subsubsection{   Giới hạn  }
\begin{itemize}
	\item     1 ≤ N ≤ 10000   
	\item     1 ≤ K ≤ 100   
	\item     K ≤ N   
	\item     Phần tử của dãy có giá trị không quá 10    $^     9    $
\end{itemize}

\subsubsection{   Hạn chế  }
\begin{itemize}
	\item     Có 30\% số test thỏa mãn N ≤ 200.   
	\item     Có 50\% số test thỏa mãn N ≤ 2000.   
\end{itemize}