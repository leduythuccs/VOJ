



   Trong rạp chiếu phim có N ghế dành cho N vị khách.  

   Mỗi người lần lượt vào rạp xem phim và phải ngồi vào đúng chỗ của mình (người vào thứ i phải ngồi ghế thứ i).  

   Tuy nhiên, có một ông khách VIP vào đầu tiên và được giành lấy một ghế bất kì. Mỗi người tiếp theo khi vào rạp thì ý thức hơn nên sẽ ngồi vào đúng chỗ (nếu ghế của họ chưa có người ngồi). Nhưng nếu chỗ của họ đã bị lấy mất thì họ được quyền chọn 1 ghế khác bất kì.  

   Ta biết C[i,j] là độ hài lòng của vị khách thứ i khi ngồi vào ghế thứ j. Hãy tính giá trị lớn nhất có thể của tổng độ hài lòng của N vị khách.  

\subsubsection{   Input  }
\begin{itemize}
	\item     Dòng đầu tiên là số nguyên dương N (1  $\le$  N  $\le$  1000);   
	\item     Dòng thứ i (trong N dòng tiếp theo) gồm N số nguyên. Số thứ j cho là giá trị C[i,j] (|C[i,j]|  $\le$  1.000.000.000);   
\end{itemize}

\subsubsection{   Output  }
\begin{itemize}
	\item     Một số nguyên duy nhất là tổng lớn nhất có thể đạt được.   
\end{itemize}

\subsubsection{   Example  }
\begin{verbatim}
\textbf{Input:}
4
\\2 6 8 6 
\\5 0 6 7 
\\8 0 1 9 
\\2 7 2 4

\textbf{Output:}
24
\\\end{verbatim}
