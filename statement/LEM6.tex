



   Chỉ năm nay nữa thôi là sherry sẽ tốt nghiệp Đại Học rồi vì thế sherry muốn sinh nhật năm nay của mình sẽ thật ý nghĩa. Và Sherry mời tất cả bạn của mình đến dự sinh nhật ^^  

   Sherry tổ chức 1 trò chơi nhỏ cho tất cả các bạn cùng tham gia, sherry có 1 tờ giấy HCN kích thước 1 x N và M mảnh nhỏ hơn, mảnh giấy thứ i có kích thước 1 x A   $_    i   $   . bây giờ sherry đố các bạn của mình có bao nhiêu cách đặt các mảnh giấy nhỏ theo thứ tự từ 1 đến M  vào mảnh giấy 1 x N sao cho mỗi mảnh giấy cách nhau ít nhất 1 ô vuông ( Nếu i $<$ j thì mảnh giấy thứ i sẽ được đặt nằm trước mảnh giấy thứ j ). Sherry hứa sẽ tặng 1 món quà đặc biệt cho bạn nào trả lời nhanh nhất :D  

\subsubsection{   Input  }

   Dòng 1: N, M ( 1  $\le$  N  $\le$  1000, 1  $\le$  M  $\le$  N/2 )  

   Dòng 2: Gồm M số, số  thứ  i  là   A   $_    i   $

\subsubsection{   Output  }

   Gồm 1 dòng duy nhất là số cách tìm được  

\subsubsection{   Example  }
\begin{verbatim}
\textbf{Input:}
4 2
1 1
\textbf{Output:}
3
\end{verbatim}
