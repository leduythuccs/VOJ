

 

Trò chơi với băng số là trò chơi tham gia trúng thưởng được mô tả như sau: Có một băng hình chữ nhật được chia ra làm n ô vuông, đánh số từ trái qua phải bắt đầu từ 1. Trên ô vuông thứ i người ta ghi một số nguyên dương $a_{i$} , i = 1, 2, ..., n. Ở một lượt chơi, người tham gia trò chơi được quyền lựa chọn một số lượng tùy ý các ô trên băng số. Giả sử theo thứ tự từ trái qua phải, người chơi lựa chọn các ô $i_{1$} , $i_{2$} , ..., $i_{k$} . Khi đó điểm số mà người chơi đạt được sẽ là:
\begin{itemize}
	\item $a_{i}$_ 1 $$ - $a_{i}$_ 2 $$ + ... + (-1) $^ k-1 $ $a_{i}$_ k $$
\end{itemize}

Yêu cầu: Hãy tính số điểm lớn nhất có thể đạt được từ một lượt chơi.

\subsubsection{Dữ liệu:}
\begin{itemize}
	\item Dòng đầu tiên chứa số nguyên dương n ( n ≤ $10^{6}$ ) là số lượng ô của băng số;
	\item Dòng thứ hai chứa n số nguyên dương $a_{1}$ , $a_{2}$ , ..., $a_{n}$ ( $a_{i}$ ≤ $10^{4}$ , i = 1, 2, ..., n ) ghi trên băng số. Các số liên tiếp trên cùng dòng được ghi cách nhau bởi ít nhất một dấu cách.
\end{itemize}

\subsubsection{Kết quả:}
\begin{itemize}
	\item Một số nguyên duy nhất là số điểm lớn nhất có thể đạt được từ một lượt chơi.
\end{itemize}

\subsubsection{Ví dụ:}


\includegraphics{http://vn.spoj.pl/content/LINEGAME.jpg}
\begin{verbatim}
Dữ liệu Kết quả


7
4 9 2 4 1 3 717


\end{verbatim}

Ràng buộc: 60\% số tests ứng với 60\% số điểm của bài có 1 ≤ n ≤ 20.
