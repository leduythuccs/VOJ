

Dãy C = $c_{1}$ , $c_{2}$ , ..., $c_{k}$ là dãy con không liền kề của dãy A = $a_{1}$ , $a_{2}$ , ..., $a_{m}$ nếu C có thể nhận được bằng cách chọn một dãy các phần tử không liền kề của A, nghĩa là tìm dược dãy các chỉ số $i_{1}$ , $i_{2}$ , ..., $i_{k}$ sao cho:

1 ≤ $i_{1}$ , $i_{2}$ , ..., $i_{k}$ ≤ m;
\\$i_{1}$ $<$ $i_{2}$ - 1, $i_{2}$ $<$ $i_{3}$ - 1, ..., i $_ k - 1 $ $<$ $i_{k}$ - 1;
\\$c_{1}$ = $a_{i1}$ , $c_{2}$ = $a_{i2}$ , $c_{k}$ = $a_{ik}$ .

Ta gọi độ dài của dãy số là số phần tử của nó.

Cho hai dãy:
\\A = $a_{1}$ , $a_{2}$ , ..., $a_{m}$
\\và
\\B = $b_{1}$ , $b_{2}$ , ..., $b_{n}$

Dãy C được gọi là dãy con chung không liền kề của hai dãy A và B nếu như nó vừa là dãy con không liền kề của A, vừa là dãy con không liền kề của B.

\subsection{Yêu cầu}

Cho hai dãy số A và B. Hãy tìm độ dài của dãy con chung không liền kề dài nhất của hai dãy đã cho.

\subsection{Dữ liệu}

 
\begin{itemize}
	\item Dòng đầu tiên chứa hai số nguyên dương m và n (2 ≤ m, n ≤ $10^{3}$ ) được ghi cách nhau bởi dấu cách, lần lượt là số lượng phần tử của dãy A và dãy B.
	\item Dòng thứ i trong m dòng tiếp theo chứa số nguyên không âm $a_{i}$ ($a_{i}$ ≤ $10^{4}$ ), i = 1, 2, ..., m.
	\item Dòng thứ j trong n dòng tiếp theo chứa số nguyên không âm $b_{j}$ ($b_{j}$ ≤ $10^{4}$ ), j = 1, 2, ..., n.
\end{itemize}

 

\subsection{Kết quả}

Ghi ra trên một dòng duy nhất độ dài của dãy con chung không liền kề dài nhất của hai dãy A và B.

\subsection{Ví dụ}
\begin{verbatim}
Input:
4 5
4
9
2
4
1
9
7
3
4

Output:
2
\end{verbatim}
