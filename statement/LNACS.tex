

Dãy C = c $_ 1 $ , c $_ 2 $ , ..., c $_ k $ là dãy con không liền kề của dãy A = a $_ 1 $ , a $_ 2 $ , ..., a $_ m $ nếu C có thể nhận được bằng cách chọn một dãy các phần tử không liền kề của A, nghĩa là tìm dược dãy các chỉ số i $_ 1 $ , i $_ 2 $ , ..., i $_ k $ sao cho:

1 ≤ i $_ 1 $ , i $_ 2 $ , ..., i $_ k $ ≤ m;
\\i $_ 1 $ $<$ i $_ 2 $ - 1, i $_ 2 $ $<$ i $_ 3 $ - 1, ..., i $_ k - 1 $ $<$ i $_ k $ - 1;
\\c $_ 1 $ = a $_ i1 $ , c $_ 2 $ = a $_ i2 $ , c $_ k $ = a $_ ik $ .

Ta gọi độ dài của dãy số là số phần tử của nó.

Cho hai dãy:
\\A = a $_ 1 $ , a $_ 2 $ , ..., a $_ m $
\\và
\\B = b $_ 1 $ , b $_ 2 $ , ..., b $_ n $

Dãy C được gọi là dãy con chung không liền kề của hai dãy A và B nếu như nó vừa là dãy con không liền kề của A, vừa là dãy con không liền kề của B.

\subsection{Yêu cầu}

Cho hai dãy số A và B. Hãy tìm độ dài của dãy con chung không liền kề dài nhất của hai dãy đã cho.

\subsection{Dữ liệu}

 
\begin{itemize}
	\item Dòng đầu tiên chứa hai số nguyên dương m và n (2 ≤ m, n ≤ 10 $^ 3 $ ) được ghi cách nhau bởi dấu cách, lần lượt là số lượng phần tử của dãy A và dãy B.
	\item Dòng thứ i trong m dòng tiếp theo chứa số nguyên không âm a $_ i $ (a $_ i $ ≤ 10 $^ 4 $ ), i = 1, 2, ..., m.
	\item Dòng thứ j trong n dòng tiếp theo chứa số nguyên không âm b $_ j $ (b $_ j $ ≤ 10 $^ 4 $ ), j = 1, 2, ..., n.
\end{itemize}

 

\subsection{Kết quả}

Ghi ra trên một dòng duy nhất độ dài của dãy con chung không liền kề dài nhất của hai dãy A và B.

\subsection{Ví dụ}
\begin{verbatim}
Input:
4 5
4
9
2
4
1
9
7
3
4

Output:
2
\end{verbatim}