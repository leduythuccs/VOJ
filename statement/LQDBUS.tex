



   Trường THPT chuyên Lê Quý Đôn, TP Đà Nẵng có một hệ thống xe buýt riêng. Hằng ngày xe buýt sẽ chạy dọc theo bờ sông Hàn thơ mộng để đón các học sinh đến trường. Trên hành trình, xe buýt sẽ   tiếp nhận các học sinh đứng chờ ở các điển hẹn nếu như xe còn chỗ trống. Xe buýt có thể đỗ lại để chờ những học sinh chưa kịp đến điểm hẹn. Tuy nhiên vì học sinh trường Lê Quý Đôn là những học sinh rất   coi trọng việc giờ giấc nên họ có thể để xe buýt chờ chứ không bao giờ chịu chờ xe buýt. Điều đó có nghĩa là nếu họ đến điểm hẹn mà xe chưa đến thì họ sẽ tự đi bộ đến trường.  

   Cho biết thời điểm mà mỗi học sinh đến điểm hẹn của mình và thời điểm qua mỗi điểm hẹn của xe buýt. Giả thiết rằng xe buýt đến điểm hẹn đầu tiên tại thời điểm 0 và thời gian xếp khách lên xe được bằng   0.  

   Xe buýt cần phải chở một số lượng nhiều nhất các học sinh có thể được đến trường. Hãy xác định khoảng thời gian ngắn nhất để xe buýt thực hiện công việc.  

\subsubsection{   Dữ liệu vào  }
\begin{itemize}
	\item     Dòng đầu tiên chứa 2 số nguyên dương n, m theo thứ tự là số điểm hẹn và số chỗ ngồi của xe buýt.   
	\item     Dòng thứ i trong số n dòng tiếp theo chứa số nguyên ti là thời gian cần thiết để xe buýt di chuyển từ điểm hẹn thứ i đến điểm hẹn thứ i+1 (điểm hẹn thứ n+1 sẽ là trường Lê Quý Đôn) và số nguyên k là số   lượng học sinh đến điểm hẹn i, tiếp theo k số nguyên là các thời điểm đến điểm hẹn của k học sinh.   
\end{itemize}

\subsubsection{   Kết qủa  }

   Gồm một dòng duy nhất, là thời gian ngắn nhất tìm được.  

\subsubsection{   Giới hạn  }
\begin{itemize}
	\item     1 ≤ n ≤ 200000, 1 ≤ m ≤ 20000   
	\item     Tổng số học sinh không vượt quá 200000.   
	\item     Kết quả không vượt quá $2^{31}$    -1.   
\end{itemize}

\subsubsection{   Ví dụ  }
\begin{verbatim}
Dữ liệu:
3 5
1 2 0 1
1 1 2
1 4 0 2 3 4

Kết qủa
5
\end{verbatim}
