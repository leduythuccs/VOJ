












    He choose a number N ( 1 ≤ N ≤ 10^18 ),   

    then wrote all the numbers from 1 to N to form a continuous string of digits. Next he replaced substrings of indentical digits with a single digit. For example string fragment "14445556677666" would be changed to "145676". He named this shortened string S. Then he specified a problem for his fellow professors: given a length of string S determine the number N witch results in that kind of string S. The task has proven to be too much for Mathematicans. can you solve it?   



    During a meeting with professors in the Asian Confederation of Mathematics, a Russian professor came up with a problem:   



    He choose a number N ( 1 ≤ N ≤ 10^18 ), then wrote all the numbers from 1 to N to form a continuous string of digits. Next he replaced substrings of indentical digits with a single digit. For example string fragment "14445556677666" would be changed to "145676". Then he specified a problem for his fellow professors: given a length of string S determine the number N witch results in that kind of string S. The task has proven to be too much for Mathematicans. can you solve it?   



\textbf{     Your task:    }



    Write a program to help your country's mathematicians.   



\subsubsection{    Input   }



     A single number M, length of the string S  ( 1 ≤ M ≤ $10^{18}$     )    



\subsubsection{    Output   }



    A single number N, the number which Russian professor selected.   



\subsubsection{    Example   }

\textbf{    Input:   }
\begin{verbatim}
13\end{verbatim}

\textbf{    Output:   }
\begin{verbatim}
12\end{verbatim}

\textbf{    Explaination:   }
\begin{verbatim}
With N = 12, we get the string:
- 123456789101112
Because three consecutive number one in the same location adjacent to the end, we delete the first two numbers, then we have
- 1234567891012
The length of this string is 13\end{verbatim}
