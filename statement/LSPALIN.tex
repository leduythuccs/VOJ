



   Palindrome là xâu đọc từ trái qua phải giống như đọc từ phải qua trái, ví dụ xâu ‘abba’ hoặc ‘madam’.  

   Với xâu s bất kỳ người ta xác định phép chia đôi ký hiệu là half(s) và định nghĩa như sau:   
\\   • Nếu s không phải là palindrome thì half(s) không xác định,   
\\   • Nếu s có độ dài bằng 1 thì half(s) không xác định,   
\\   • Nếu s là palindrome độ dài n thì half(s) là xâu k ký tự đầu của s, trong đó k = (n+1) div 2.  

   Ví dụ, half(informatics) và half(i) là không xác định, half(abba) = ‘ab’, half(madam) =’mad’.  

   Bậc palindrome (ta sẽ gọi ngắn gọn là bậc) của xâu s là số lần tối đa có thể áp dụng phép chia đôi mà kết quả vẫn xác định. Ví dụ, các xâu ‘informatics’ và ‘i’ có bậc bằng 0 vì không thể áp dụng phép chia đôi một lần nào, các xâu ‘abba’, ‘madam’ có bậc bằng 1, còn xâu ‘totottotot’ có bặc bằng 3: ‘totottotot’ -$>$ ‘totot’ -$>$ ‘tot’ -$>$ ‘to’.       Yêu cầu:      Xét tất cả các xâu độ dài n chỉ chứa các chữ cái la tinh thường và có bậc palindrome bằng p. Hãy xác định xâu thứ k theo thứ tự từ điển (1 ≤ n ≤ 200, 0 ≤ p ≤ 8, 1 ≤ k ≤10^9). Dữ liệu đảm bảo tồn tại xâu cần tìm.  

\subsubsection{   Input  }

   Gồm một dòng duy nhất chứa 3 số n, p, k.  

\subsubsection{   Output  }

   In ra xâu tìm đựơc  

\subsubsection{   Example  }
\begin{verbatim}
Input:
4 1 1
Output:
abba



Input:
10 3 490
Output:
totottotot



Input:
5 0 6597777
Output:
olymp
\end{verbatim}
