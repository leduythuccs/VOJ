

\section{\textbf{Đề bài}}

Cho dãy A gồm n số nguyên dương. Sau khi loại bỏ từ dãy A các số hạng không là lũy thừa của 3 hoặc 5 ta nhân được dãy B gồm các số hạng còn lại theo thứ tự xuất hiện trong dãy A. Cần viết chương trình để tìm:

1. Số M là số số hạng của dãy B;

2. Số các dãy con gồm các số hạng liên tiếp của dãy B có độ dài là lũy thừa của 2 và có số số hạng là lũy thừa của 3 bằng số số hạng là lũy thừa của 5;

\section{\textbf{Input}}

Dòng thứ nhất ghi số N.

 N dòng tiếp theo  mỗi dòng ghi một số hạng của dãy A theo thứ tự trong dãy.

\section{\textbf{Output}}

Dòng thứ nhất ghi số M.

Dòng thứ 2 ghi số S là số dãy con cần tìm.

\section{\textbf{Hạn chế}}
\begin{itemize}
	\item 2$<$n$<$500000
\end{itemize}
\begin{itemize}
	\item Các số hạng của dãy A trong phạm vi 2...2*10$^9$
\end{itemize}
\begin{itemize}
	\item Dữ liệu vào đảm bảo M$<$40000
\end{itemize}


