

Cho một khối lập phương kích thước n chia làm n$^3 $ khối lập phương đơn vị. Mỗi khối lập phương đơn vị chứa 1 số nguyên.
\\Bạn hãy tìm một khối lập phương con của khối lập phương đã cho sao cho tổng các số trong khối lập phương con đó là lớn nhất.

\subsubsection{Input}
\begin{itemize}
	\item Dòng đầu: số lượng test.
	\item Tiếp theo là các test, mỗi test gồm: dòng đầu là n. Sau đó n nhóm dòng thể hiện lớp cắt của hình lập phương nhìn từ mặt trước từ gần ra xa, mỗi nhóm gồm n dòng, mỗi dòng gồm n số liệt kê các số trên lớp cắt từ trên xuống dưới, trái qua phải.
\end{itemize}

Chú ý: n  $\le$  30. Giá trị của khối lập phương đơn vị thuộc kiểu integer.

\subsubsection{Output}

Mỗi dòng chứa tổng của khối lập phương con lớn nhất của test tương ứng.

\subsubsection{Example}
\begin{verbatim}
\textbf{Input}
2
3
0 -1 3
-5 7 4
-8 9 1
-1 -3 -1
2 -1 5
0 -1 3
3 1 -1
1 3 2
1 -2 1
4
1 1 1 1
1 1 1 1
1 1 1 1
1 1 1 1
1 1 1 1
1 1 1 1
1 1 1 1
1 1 1 1
1 1 1 1
1 1 1 1
1 1 1 1
1 1 1 1
1 1 1 1
1 1 1 1
1 1 1 1
1 1 1 1

\textbf{Output}
27
64
\end{verbatim}
