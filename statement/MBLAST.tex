



   Cho hai xâu A, B.  Mở rộng của 1 xâu X là xâu thu được bằng cách chèn (0,1,2 ..) kí tự trống vào xâu.  Ví dụ :  X là ‘abcbcd’, thì 'abcb-cd', '-a-bcbcd-'  và 'abcb-cd-'  là các mở rộng của X. (Dấu cách kí hiệu bằng ‘-‘).  

   A1,B1 là mở rộng của A và B, và giả sử chúng cùng độ dài. Khoảng cách giữa  A1 và B1 là tổng khoảng cách giữa các kí tự cùng vị trí. Nếu hai kí tự  không là dấu cách thì khoảng cách giữa 2 kí tự này là trị tuyệt đối mã  ASCII của chúng. Còn ngược lại, khoảng cách là 1 số K cố định.  

   Cho hai xâu A, B. Tìm khoảng cách nhỏ nhất giữa hai xâu mở rộng của nó.  

\subsubsection{   Input  }

   Dòng 1 chứa A, dòng 2 chứa B, chỉ gồm chữ thường a-z và số kí tự  $\le$ 2000.  

   Dòng thứ 3 là số K, khoảng cách của 1 kí tự bất kỳ với kí tự trống, 1 ≤ K ≤ 100.  

\subsubsection{   Output  }

   Khoảng cách nhỏ nhất.  

\subsubsection{   Sample  }


\begin{verbatim}
blast.in 
 
cmc 
snmn 
2 
 
blast.out 
 
10 

blast.in 
 
koiv 
ua 
1 
 
blast.out 
 
5

blast.in 
 
mj 
jao 
4 
 
blast.out 
 
12 

\end{verbatim}

