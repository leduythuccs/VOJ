

Ghi 2n số 1, 2, 3, . . . , 2n − 1, 2n lên vòng tròn.

Kẻ n đường thẳng nối các cặp số sao cho số nào cũng được nối và các đường thẳng này không cắt nhau.

Đếm số cách nối.

\subsubsection{Input}

Mỗi dòng là một số nguyên dương n, kết thúc là số -1, 1 ≤ n ≤ 150. 

\subsubsection{Output}

Với mỗi n, in ra số cách nối trên 1 dòng.
\begin{verbatim}
\textbf{Sample Input}
2 -1
\textbf{Sample output}
2\end{verbatim}

 

Note : Làm quen với số Catalan và công thức liên quan đến số Catalan. Sau đó là bài \href{http://vnoi.info/problems/show/CATALAN/}{CATALAN}, \href{http://vnoi.info/problems/show/BRACKET/}{Dãy ngoặc bậc K}.