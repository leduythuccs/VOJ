

Sàn nhà là hình chữ nhật, chia thành ô. Trên đó có các ô sạch, bẩn và robot có thể dọn các ô bẩn thành ô sạch nếu nó ở ô đó. Robot có thể di chuyển qua các ô kề cạnh. Xác định số bước di chuyển ít nhất để có thể dọn sạch sàn nếu có thể.

\href{http://tinypic.com}{
\includegraphics{http://i41.tinypic.com/29f33f4.jpg}}

\subsubsection{Input}
\begin{verbatim}
Gồm nhiều test. Dòng đầu mỗi test là 2 số w,h là chiểu rộng và dài của sàn nhà. 

    w h
    c11 c12 c13 ... c1w
    c21 c22 c23 ... c2w
    ...
    ch1 ch2 ch3 ... chw

1 $\le$ w,h $\le$ 20.

Các ô của sàn có 4 giá trị sau:
    '.' : sạch
    '*' : bẩn
    'x' : vật cản.
    'o' : robot (1 con)

Có không quá 10 ô bẩn trên sàn. 

Kết thúc test là 2 số 0 0.

SAMPLE INPUT
7 5
.......
.o...*.
.......
.*...*.
.......
15 13
.......x.......
...o...x....*..
.......x.......
.......x.......
.......x.......
...............
xxxxx.....xxxxx
...............
.......x.......
.......x.......
.......x.......
..*....x....*..
.......x.......
10 10
..........
..o.......
..........
..........
..........
.....xxxxx
.....x....
.....x.*..
.....x....
.....x....
0 0

\end{verbatim}

\subsubsection{Output}
\begin{verbatim}
 
In ra số bước di chuyển ít nhất cần sử dụng. Nếu không thể làm sạch sàn, 
in ra -1. 

SAMPLE OUTPUT
8
49
-1
\end{verbatim}
