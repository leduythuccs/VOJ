
\begin{verbatim}
Sau khi giảng xong về  các hệ  đếm, thầy  Phương  yêu cầu  học sinh  đổi  một số  nguyên dương  ở  hệ  cơ  số  10 sang dạng biểu diễn  ở  hệ  cơ số  b. Thầy  Phương  viết trên bảng sốnguyên dương X và lần lượt gọi học sinh đọc kết quả biểu diễn ở các hệ cơ số 2,3,4... Bạn Long dần dần thấy chán và buồn ngủ, bỗng nhiên bạn Chương đọc lên một kết quả lạ làm Long  giật mình  ngỡ  ngàng, các chữ  số  biểu diễn  trong hệ  cơ số  mới  của Chương hoàn toàn  giống nhau.  Chuông reo hết giờ  và Long quyết định về  nhà phải tìm  cho  được dạng biểu diễn  của X  trong hệ  cơ số  nhỏ  nhất  sao cho  các chữ  số  biểu diễn hoàn toàn  giống nhau. Hãy giúp Long tìm ra đáp án.Input:  Dòng đầu chứa số nguyên dương T là số bộ test. (T ≤ 10).  N  dòng tiếp theo, mỗi dòng duy nhất một số  nguyên X  (X≤   1015)  ứng với mỗi bộtest.Output:  Ứng với mỗi số nguyên X, hãy in ra cơ số nhỏ nhất của dạng biểu diễn X bằng các chữ số giống nhau.Example:Input:1219Output:8
\begin{verbatim}
Sau khi giảng xong về  các hệ  đếm, thầy  Phương  yêu cầu  học sinh  đổi  một số  nguyên \end{verbatim}
\begin{verbatim}
dương  ở  hệ  cơ  số  10 sang dạng biểu diễn  ở  hệ  cơ số  b. Thầy  Phương  viết trên bảng số\end{verbatim}
\begin{verbatim}
nguyên dương X và lần lượt gọi học sinh đọc kết quả biểu diễn ở các hệ cơ số 2,3,4... Bạn \end{verbatim}
\begin{verbatim}
Long dần dần thấy chán và buồn ngủ, bỗng nhiên bạn Chương đọc lên một kết quả lạ làm \end{verbatim}
\begin{verbatim}
Long  giật mình  ngỡ  ngàng, các chữ  số  biểu diễn  trong hệ  cơ số  mới  của Chương hoàn \end{verbatim}
\begin{verbatim}
toàn  giống nhau.  Chuông reo hết giờ  và Long quyết định về  nhà phải tìm  cho  được dạng \end{verbatim}
\begin{verbatim}
biểu diễn  của X  trong hệ  cơ số  nhỏ  nhất  sao cho  các chữ  số  biểu diễn hoàn toàn  giống \end{verbatim}
\begin{verbatim}
nhau. Hãy giúp Long tìm ra đáp án.\end{verbatim}
\begin{verbatim}
Input:\end{verbatim}
\begin{verbatim}
  Dòng đầu chứa số nguyên dương T là số bộ test. (T ≤ 10).\end{verbatim}
\begin{verbatim}
  N  dòng tiếp theo, mỗi dòng duy nhất một số  nguyên X  (X≤   10^15)  ứng với mỗi bộ\end{verbatim}
\begin{verbatim}
test.\end{verbatim}
\begin{verbatim}
Output:\end{verbatim}
\begin{verbatim}
  Ứng với mỗi số nguyên X, hãy in ra cơ số nhỏ nhất của dạng biểu diễn X bằng các \end{verbatim}
\begin{verbatim}
chữ số giống nhau.\end{verbatim}
\begin{verbatim}
Example:\end{verbatim}
\begin{verbatim}
Input:\end{verbatim}
\begin{verbatim}
1\end{verbatim}
\begin{verbatim}
219\end{verbatim}
\begin{verbatim}
Output:\end{verbatim}
\begin{verbatim}
8\end{verbatim}\end{verbatim}