

"Dilworth" có một bộ sưu tập các con búp bê Nga. Búp bê với chiều rộng w1 và chiều cao h1 sẽ nằm trong được con lật đật chiều rộng w2 và chiều cao h2 nếu w1 $<$ w2 và h1 $<$ h2. Tính số lớp búp bê bao nhau ít nhất mà có thể tạo ra được từ các búp bê ban đầu.

\href{http://tinypic.com}{
\includegraphics{http://i39.tinypic.com/1199pb4.jpg}}

\subsubsection{Input}

Dòng đầu là số test, 1 ≤ t ≤ 20.

Mỗi test gồm:
\begin{itemize}
	\item số nguyên m, 1 ≤ m ≤ 20000, số lượng búp bê ban đầu.
	\item Dòng tiếp theo là 2m số nguyên w1, h1,w2, h2, ... ,wm, hm, là chiều rộng và chiều cao của con búp bê thứ i, 1 ≤ wi, hi ≤ 10000.
\end{itemize}

\subsubsection{Output}

Ghi số lớp búp bê bao nhau ít nhất có thể trên một dòng cho từng test.
\begin{verbatim}
SAMPLE INPUT
4
3
20 30 40 50 30 40
4
20 30 10 10 30 20 40 50
3
10 30 20 20 30 10
4
10 10 20 30 40 50 39 51

SAMPLE OUTPUT
1
2
3
2\end{verbatim}