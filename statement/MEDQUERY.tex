


%  		@page { margin: 0.79in } 		td p { margin-bottom: 0in; direction: ltr; color: #00000a; text-align: left; orphans: 2; widows: 2 } 		td p.western { font-family: "Liberation Serif", serif; font-size: 12pt; so-language: en-US } 		td p.cjk { font-family: "Droid Sans Fallback"; font-size: 12pt; so-language: zh-CN } 		td p.ctl { font-family: "FreeSans"; font-size: 12pt; so-language: hi-IN } 		p { margin-bottom: 0.1in; direction: ltr; color: #00000a; line-height: 120%; text-align: left; orphans: 2; widows: 2 } 		p.western { font-family: "Liberation Serif", serif; font-size: 12pt; so-language: en-US } 		p.cjk { font-family: "Droid Sans Fallback"; font-size: 12pt; so-language: zh-CN } 		p.ctl { font-family: "FreeSans"; font-size: 12pt; so-language: hi-IN } 	 


\textbf{Yêu cầu}:

Hãy quản lí một tập hợp các số nguyên không âm hỗ trợ các thao tác sau:

 1. Thêm một phân tử

 2. Xóa một phần tử

 3. Tìm trung vị của tập hợp




Định nghĩa trung vị:

Trung vị của dãy a[1..N] được sắp xếp không giảm là phần tử a[(N+1)/2].




\textbf{Input}

Dòng đầu tiên chứa Q là số truy vấn. (1 $<$= Q $<$= 10\textasciicircum6)

Q dòng tiếp theo, mỗi dòng gồm một kí tự '+' hoặc '-' và một số nguyên không âm nhỏ hơn 10\textasciicircum9. Dấu '+' biểu thị thao tác thêm phần tử x vào tập hợp (có thể tập hợp đã có phần tử x nhưng thao tác này vẫn được thực hiện). Dấu '-' biểu thị thao tác xóa phần tử x khỏi tập hợp (bộ test đảm bảo tồn tại phần tử x trong tập hợp).




\textbf{Output}

Q dòng, mỗi dòng là trung vị của tập hợp sau mỗi thao tác thêm/xóa.




\textbf{Sample}



\begin{tabular}

Input & 

Output & 

Explanation  


20

+ 6

+ 2

+ 5

+ 3

+ 6

- 3

+ 8

- 6

- 5

+ 5

+ 2

- 8

+ 5

+ 5

- 2

+ 1

- 5

+ 4

+ 8

+ 7 & 

6

2

5

3

5

5

6

5

6

5

5

2

5

5

5

5

5

4

5

5 & 

[6]

[2, 6]

[2, 5, 6]

[2, 3, 5, 6]

[2, 3, 5, 6, 6]

[2, 5, 6, 6]

[2, 5, 6, 6, 8]

[2, 5, 6, 8]

[2, 6, 8]

[2, 5, 6, 8]

[2, 2, 5, 6, 8]

[2, 2, 5, 6]

[2, 2, 5, 5, 6]

[2, 2, 5, 5, 5,    6]

[2, 5, 5, 5, 6]

[1, 2, 5, 5, 5,    6]

[1, 2, 5, 5, 6]

[1, 2, 4, 5, 5,    6]

[1, 2, 4, 5, 5,    6, 8]

[1, 2, 4, 5, 5,    6, 7, 8]
\end{tabular}
