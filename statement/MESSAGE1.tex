

 

An và Bình thường trao đổi thông tin qua mạng. Để tránh người khác đọc được, hai bạn đã thống nhất cách truyền thông tin qua hai bước :
\begin{itemize}
	\item Bước 1 : Giấu thông tin . Nội dung thông tin cần gửi sẽ được giấu vào một bảng kí tự hình chữ nhật bằng cách điền lần lượt các kí tự của xâu biểu diễn vào các ô của bảng từ trên xuống dưới theo mỗi hàng và từ trái qua phải theo mỗi cột. Bảng này được đặt gọn vào một bảng kí tự hình chữ nhật có kích thước MxN lớn hơn. Các ô trống của bảng MxN sẽ được điền kí tự ngẫu nhiên
\end{itemize}
\begin{itemize}
	\item Bước 2 : Giải thông tin . Bảng MxN được gửi qua mạng. Vị trí đặt hình chữ nhật chứa nội dung được gửi qua điện thoại bằng tin nhắn.
\end{itemize}

Trong một lần An chuyển bảng A qua cho Bình, tuy nhiên Bình không nhận được. An thực hiện lại và chuyển bảng B qua. 2 bảng A và B đều chứa hình chữ nhật nội dung thông tin, tuy nhiên vị trí đặt hình này có thể khác nhau. Em gái Bình biết được quy ước trao đổi thông tin. Tò mò, cô muốn biết An đã gửi thông tin gì cho Bình bằng cách tìm một bảng hình chữ nhất có diện tích lớn nhất xuất hiện trong cả 2 bảng A và B.

\subsubsection{Input}
\begin{itemize}
	\item Dòng đầu chứa T – số lượng testcase. T nhóm dòng, mỗi nhóm miêu tả 1 testcase.
	\item Dòng thứ nhất chứa 2 số nguyên dương M, N. (N  $\le$  100).
	\item Dòng thứ i trong M dòng tiếp theo chứa một xâu gồm N kí tự chỉ gồm các chữ cái la tinh thường mô tả bảng A.
	\item Dòng thứ j trong M dòng tiếp theo chứa một xâu gồm N kí tự chỉ gồm các chữ cái la tinh thường mô tả bảng B.
\end{itemize}

\subsubsection{Output}
\begin{itemize}
	\item Gồm T dòng, dòng thứ i ghi một số nguyen là diện tích hình chữ nhất lớn nhất tìm được tương ứng với testcase thứ i.
\end{itemize}

\subsubsection{Example}
\begin{verbatim}
\textbf{Input:}
1
4 5
tinaa
hocaa
aaaaa
ccccc
bbbbd
btind
bhocd
bbbbd\end{verbatim}
\begin{verbatim}
\textbf{Output:}
6\end{verbatim}
