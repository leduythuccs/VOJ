
\begin{verbatim}
G. HÌNH LẬP PHƯƠNGTrong đợt đi công tác Sài Gòn về, bố Phương mua cho Delta một bộ xếp hình bằng gỗ rất đẹp. Trong bộ  xếp hình có rất nhiều miếng gỗ  hình hộp chữ  nhật với nhiều kích thước khác nhau. Delta rất thích  chơi xếp hình và đặc biệt thích hình lập phương. Nhận món quà của bố  Phương, Delta soạn ra và loay hoay chọn ra từng bộ 3 miếng gỗ hình hộp chữnhật để  xếp lại thành một hình lập phương. Bạn hãy giúp Delta xác định xem với 3 hình hộp đã cho thì có thể xếp thành hình lập phương được không?Giả  sử  3 hình hộp chữ  nhật với các cạnh a1  b1c1, a2  b2c2, a3  b3c3 là số  thực dương. Input:   Dòng đầu chứa số nguyên dương N là số bộ test.  Mỗi bộ  test gồm 3 dòng, mỗi dòng ghi kích thước của các khối hình hộp a1, b1, c1, a2, b2, c2, a3, b3, c3.Output:   Mỗi dòng chứa một số nguyên duy nhất (0 hoặc 1): o  1: ghép đượco  0: không ghép đượcExampleInput23 3 1 3 3 1 3 3 1 3 3 1 3 3 1 3 2 2OutputTRUEFALSE
\begin{verbatim}
G. HÌNH LẬP PHƯƠNG\end{verbatim}
\begin{verbatim}
Trong đợt đi công tác Sài Gòn về, bố Phương mua cho Delta một bộ xếp hình bằng gỗ rất \end{verbatim}
\begin{verbatim}
đẹp. Trong bộ  xếp hình có rất nhiều miếng gỗ  hình hộp chữ  nhật với nhiều kích thước \end{verbatim}
\begin{verbatim}
khác nhau. Delta rất thích  chơi xếp hình và đặc biệt thích hình lập phương. Nhận món \end{verbatim}
\begin{verbatim}
quà của bố  Phương, Delta soạn ra và loay hoay chọn ra từng bộ 3 miếng gỗ hình hộp chữ\end{verbatim}
\begin{verbatim}
nhật để  xếp lại thành một hình lập phương. Bạn hãy giúp Delta xác định xem với 3 hình \end{verbatim}
\begin{verbatim}
hộp đã cho thì có thể xếp thành hình lập phương được không?\end{verbatim}
\begin{verbatim}
Giả  sử  3 hình hộp chữ  nhật với các cạnh a1$>$=b1$>$=c1, a2$>$= b2$>$=c2, a3$>$=  b3$>$=c3 là số  thực  dương. \end{verbatim}
\begin{verbatim}
Input: \end{verbatim}
\begin{verbatim}
  Dòng đầu chứa số nguyên dương N là số bộ test.\end{verbatim}
\begin{verbatim}
  Mỗi bộ  test gồm 3 dòng, mỗi dòng ghi kích thước của các khối hình hộp a1, b1, c1, \end{verbatim}
\begin{verbatim}
a2, b2, c2, a3, b3, c3.\end{verbatim}
\begin{verbatim}
Output: \end{verbatim}
\begin{verbatim}
  Mỗi dòng chứa một số nguyên duy nhất (0 hoặc 1): \end{verbatim}
\begin{verbatim}
o  TRUE: ghép được\end{verbatim}
\begin{verbatim}
o  FALSE: không ghép được\end{verbatim}
\begin{verbatim}
Example\end{verbatim}
\begin{verbatim}
Input\end{verbatim}
\begin{verbatim}
2\end{verbatim}
\begin{verbatim}
3 3 1 \end{verbatim}
\begin{verbatim}
3 3 1 \end{verbatim}
\begin{verbatim}
3 3 1 \end{verbatim}
\begin{verbatim}
3 3 1 \end{verbatim}
\begin{verbatim}
3 3 1 \end{verbatim}
\begin{verbatim}
3 2 2\end{verbatim}
\begin{verbatim}
Output\end{verbatim}
\begin{verbatim}
TRUE\end{verbatim}
\begin{verbatim}
FALSE\end{verbatim}\end{verbatim}