



   Hóa chất chỉ gồm các nguyên tố C, H, O có trọng lượng 12,1, 16 tương ứng.  

   Nó được biểu diễn dạng "nén", ví dụ COOHHH là CO2H3  hay CH (CO2H) (CO2H) (CO2H) là CH(CO2H)3. Nếu ở dạng nén thì số lần lặp $>$=2 và  $\le$ 9.  

   Tính khối lượng hóa chất.  



\subsubsection{   Input  }



   Gồm một dòng mô tả hóa chất không quá 100 kí tự chỉ gồm C, H, O, (, ), 2,..,9.  



\subsubsection{   Output  }



   Khối lượng của hóa chất, luôn  $\le$ 10000.  



\subsubsection{   Sample  }
\begin{verbatim}
MASS.IN

COOH

MASS.OUT

45
 
MASS.IN

CH(CO2H)3

MASS.OUT

148
 
MASS.IN

((CH)2(OH2H)(C(H))O)3

MASS.OUT

222
 

\end{verbatim}
