

Cho n hình chữ nhật đánh số từ 1 đến n, các hình chữ nhật này được đặt tiếp xúc với trục OX và nằm kề nhau từ trái qua phải theo thứ tự chỉ số Mỗi hình chữ nhật có thể tiếp xúc với trục Ox theo bất kỳ cạnh cạnh (xem hình). Cần tính độ dài lớn nhất của đường gấp phía trên (xem hình)

\href{http://tinypic.com}{
\includegraphics{http://i39.tinypic.com/dc5tls.jpg}}

 

\subsubsection{INPUT}

 

Dòng đầu tiên ghi số hình chữ nhật n, n dòng tiếp theo mỗi dòng ghi


hai số a\_i  và b\_i, độ dài cạnh của hình chữ nhật thứ i.


Ràng buộc : 0 $<$ n $<$ 1000; 0 $<$ a\_i $<$ b\_i $<$ 1000, với i = 1, 2, …, n. 

 

\subsubsection{OUTPUT}

Ghi ra độ dài lớn nhất tìm được





​Ví dụ:

Input

5 


2 5 


3 8 


1 10 


7 14 


2 5

Output 

68

Giải thích

Cách xếp mà thu được chiều dài lớn nhất là hình trên. Cạnh phía trên gồm các đoạn  DC, CG, GF, FJ,  JI,  IM, ML, LP,  và PO.

Độ dài của đoạn này là 5 + 6 + 3 + 7 + 10 + 13 + 7 + 12 + 5 = 68

Problem for kid - Please, think like kid.
