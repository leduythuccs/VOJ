









   Cho hình vuông NxN, một số ô có các hộp, mỗi ô có thể có nhiều hộp bao nhau. Tính số hộp cần chuyển ít nhất để các hộp này xếp thành 1 hình chữ nhật (mỗi ô chỉ có 1 hộp). Phép chuyển hộp là chuyển 1 hộp ở trên cùng ở một ô vuông sang và đặt ở trên cùng 1 ô vuông nào khác.  



\subsubsection{   Input  }



   DÒng đầu là 2 số N và M,  (1 ≤ N ≤ 100, 1 ≤ M ≤ N^2), kích thước hình vuông và số hộp. M dòng tiếp theo là tọa độ mỗi hộp (hàng, cột).  



\subsubsection{   Output  }



   Số hộp ít nhất cần chuyển  



\subsubsection{   Sample  }
\begin{verbatim}
input
\\4 3
\\2 2
\\4 4
\\1 1
\\output
\\2
\\
\\input
\\5 8
\\2 2
\\3 2
\\4 2
\\2 4
\\3 4
\\4 4
\\2 3
\\2 3
\\output
\\3
\\
\\Ở ví dụ 2, hộp chuyển từ (2, 3) tới (3, 3), từ (4, 2)
\\tới (2, 5) và từ (4, 4) tới (3, 5).
\\\end{verbatim}

