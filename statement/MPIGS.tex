



   Mirko làm việc ở nông trại có M chuồng lợn, nhưng anh ta không có khóa  để mở cửa chuồng lợn. Khách hàng lần lượt đến nông trại và họ có khóa một số chuồng lợn. Mỗi khách hàng đến, họ sẽ mở cửa mọi chuồng lợn mà họ có chìa khóa, và  họ có thể mua một số lượng lợn nào đó trong các chuồng này.  

   Dữ liệu về khách hàng được gửi đến Mirko và buổi sáng, và Mirko muốn bán  được nhiều lợn nhất. Sau khi bán một số lợn ở các chuồng này cho họ, Mirko có thể  chuyển lợn còn lại giữa các chuồng đã mở.  Mỗi chuồng chứa bao nhiêu lợn cũng được.  

   Xác định số lợn nhiều nhất có thể bán.  



\subsubsection{   Input  }

   Dòng đầu là hai số M và N, 1 ≤ M ≤ 1000, 1 ≤ N ≤ 100, số chuồng lợn và khách hàng.  

   Dòng tiếp theo ghi M số là số lợn trong mỗi chuồng (1,2..,M), số lợn thỏa mãn $>$=0 và  $\le$ 1000.  

   N dòng tiếp theo lưu thông tin về khách hàng và có dạng:  

   A K1 K2 ... KA B  

   Nghĩa là anh ta có A cái khóa K1,..,KA và muốn mua B lợn (các khóa sắp  tăng dần). A, B có thể bằng 0.  

\subsubsection{   Output  }

   Số lợn nhiều nhất có thể bán.  

\subsubsection{   Sample  }
\begin{verbatim}
pigs.in 
 
3 3 
3 1 10 
2 1 2 2 
2 1 3 3 
1 2 6 
 
pigs.out
 
7 

pigs.in 
 
6 6 
6 3 2 0 1 3 
2 1 2 0 
1 3 3 
1 1 1 
2 2 3 8 
2 4 5 2 
2 4 6 6 
 
pigs.out 
 
15 

pigs.in 
 
11 5 
1 2 2 1 0 2 4 1 1 1 2 
5 1 2 3 4 5 3 
4 1 2 6 7 5 
2 3 8 1 
3 3 6 11 5 
3 8 9 10 3 
 
pigs.out 
 
17  

\end{verbatim}
