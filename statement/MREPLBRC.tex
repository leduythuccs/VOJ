

Dãy ngoặc hợp lệ gồm:
\begin{itemize}
	\item Xâu rỗng.
	\item A hợp lệ thì (A), [A] và \{A\} cũng thế.
	\item A, B hợp lệ thì AB cũng thế.
\end{itemize}

Ví dụ : [(\{\})], []()\{\} và [\{\}]()[\{\}] là hợp lệ, [(\{\{([, [](\{)\} và [\{\}])([\{\}] không hợp lệ.

Cho một xâu chỉ gồm ( ) \{ \} [ ] và ?. Dấu ? có thể thay thế bằng ngoặc bất kỳ. Tính số cách thay thế mà thu được 1 dẫy ngoặc hợp lệ. Chỉ hiện 5 chữ số cuối cùng.

\subsubsection{Input}

Dòng đầu là N, độ dài xâu (2  $\le$  N  $\le$  200), Dòng thứ hai là xâu mô tả.

\subsubsection{Output}

5 chữ số cuối cùng của dẫy ngoặc hợp lệ thu được. ( $\le$  5 chữ số thì in ra hết cả kết quả).

\subsubsection{Sample}
\begin{verbatim}
input 

6 
()()() 
 
output 
 
1 

input 
 
10 
(?([?)]?}? 
 
output 
 
3

input 
 
16 
???[???????]???? 
 
output 
 
92202

Ví dụ thứ hai, 3 dãy ngoặc hợp lệ là  ({([()])}), ()([()]{}) và ([([])]{}).
\end{verbatim}
