

Có N thành phố và N-1 cặp đường nối chúng, có duy nhất 1 đường nối 2 thành phố khác nhau.

Đường đã bị xuống cấp và với mỗi đường ta biết 2 số A, B : A(s) thời gian để đi qua đường này và B(s) là thời gian ít nhất để đi qua đường này nếu nâng cấp hết cả đường.

Có 1 lượng tiền đầu tư để sửa đường, với mỗi đoạn đường, kết quả sẽ tỉ lệ với lượng tiền đầu tư. Đầu tư 1 euro cho 1 đoạn đường sẽ giảm thời gian trên đoạn đường đó đi 1s. Tuy nhiên nó không thể giảm quá thời gian tối thiểu B của đoạn đường này.

Cần phân bố lượng tiền trên cho các đoạn đường khác nhau để thời gian cần thiết đi từ thành phố 1 tới thành phố xa nhất (đi mất nhiều thời gian nhất sau khi thực hiện mọi sửa chữa) là nhỏ nhất có thể.

Xác định thời gian nhỏ nhất này.

\subsubsection{Input}

Dòng đầu gồm 2 số nguyên N và K, 2 ≤ N ≤ 100 000, 0 ≤ K ≤ 1 000 000, số thành phố và lượng tiền (euros)

Sau đó là N-1 dòng, mỗi dòng 4 số nguyên X, Y, A và B, 0 ≤ B ≤ A ≤ 10 000. Nghĩa là có đường đi từ nối X và Y với 2 thông tin A và B như trên.

\subsubsection{Output}

Thời gian nhỏ nhất cần tìm theo yêu cầu đề bài.

\subsubsection{Sample}
\begin{verbatim}
input 
 
3 200 
1 2 200 100 
2 3 450 250 
 
output 
 
450

input 
 
5 11 
1 2 10 5 
1 3 3 2 
1 4 9 6 
3 5 7 3 
 
output 
 
6

input 
 
11 12 
1 2 7 5 
1 3 20 15 
2 4 10 8 
2 5 5 3 
2 6 6 2 
4 7 3 0 
4 8 7 2 
5 9 8 4 
5 10 9 8 
5 11 6 5 
 
output 
 
17\end{verbatim}

 