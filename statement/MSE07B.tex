



   Ngân hàng  BIG-Bank mở một chi nhánh ở Bucharest và được trang bị một máy tính hiện đại với các công nghệ mới nhập, C2\#,VC3+ ... chỉ chuối mỗi cái là không ai biết lập trình.  

   Họ cần một phần mềm mô tả hoạt động của ngân hàng như sau: mỗi khách hàng có một mã số là số nguyên K, và khi đến ngân hàng giao dịch, họ sẽ nhận được 1 số P là thứ tự ưu tiên của họ. Các thao tác chính như sau  

   0  Kết thúc phục vụ  

   1  K P Thêm khách hàng K vào hàng đợi với độ ưu tiên P  

   2  Phục vụ người có độ ưu tiên cao nhất và xóa khỏi danh sách hàng đợi  

   3  Phục vụ người có độ ưu tiên thấp nhất và xóa khỏi danh sách hàng đợi.  

   Tất nhiên là họ cần bạn giúp rồi.  



\subsubsection{   Input  }



   Mỗi dòng của input là 1 yêu cầu, và chỉ yêu cầu cuối cùng mới có giá trị là 0. Giả thiết là khi có yêu cầu 1 thì không có khách hàng nào khác có độ ưu tiên là P.  

   K $\le$ 10^6, và P $\le$  10^7.Một khách hàng có thể yêu cầu phục vụ nhiều lần và với các độ ưu tiên khác nhau.  



\subsubsection{   Output  }



   Với mỗi yêu cầu 2 hoặc 3, in ra trên 1 dòng mã số của khách hàng được phục vụ tương ứng. Nếu có yêu cầu mà hàng đợi rỗng, in ra số 0.  



\subsubsection{   Sample  }
\begin{verbatim}
Input :
2 
1 20 14 
1 30 3 
2 
1 10 99 
3 
2 
2 
0 
Ouput: 
0 
20 
30 
10 
0 
\end{verbatim}
