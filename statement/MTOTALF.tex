

Để quản lý nước cho đàn bò, FJ đã vẽ bản đồ đường ống gồm N (1  $\le$  N  $\le$  700) ống trong trang trại mà nối bể nước với các chuồng. FJ thấy rằng các đường ống có kích thước khác nhau và được nối một cách rất kỳ lạ. Do đó, FJ muốn tính xem lượng nước truyền qua các ống. Hai ống nước được nối liên tiếp với nhau cho phép lượng nước không vượt quá thông lượng nhỏ nhất của hai ống. Một ống có thông lượng 5 nối với ống có thông lượng 3 sẽ tương đương với một ống có thông lượng 3:
\begin{verbatim}

\texttt{  +---5---+---3---+    ->    +---3---+}\end{verbatim}

Còn hai ống nối song song cho phép thông lượng nước là tổng thông lượng của từng ống.
\begin{verbatim}

\texttt{    +---5---+
 ---+       +---    ->    +---8---+
    +---3---+
}\end{verbatim}

Còn một ống mà không nối với chuồng bò nào hay ống nào khác có thể bỏ đi:
\begin{verbatim}

\texttt{    +---5---+
 ---+               ->    +---3---+
    +---3---+--
}\end{verbatim}

Sử dụng cách thức trên, cho bản đồ đường ống, xác định lượng nước có thể truyền từ nguồn (A) cho tới chuồng (Z).

Xét ví dụ sau:
\begin{verbatim}

\texttt{                 +-----------6-----------+
        A+---3---+B                      +Z
                 +---3---+---5---+---4---+
                         C       D
}\end{verbatim}

Ống BC và CD có thể gộp lại được:
\begin{verbatim}

\texttt{                 +-----------6-----------+
        A+---3---+B                      +Z
                 +-----3-----+-----4-----+
                             D
}\end{verbatim}

Sau đó gộp BD và DZ :
\begin{verbatim}

\texttt{                 +-----------6-----------+
        A+---3---+B                      +Z
                 +-----------3-----------+
}\end{verbatim}

Gộp hai nhánh nối B và Z:
\begin{verbatim}

\texttt{                 B
        A+---3---+---9---+Z
}\end{verbatim}

Gộp AB và BZ và thông lượng thu được là 3:
\begin{verbatim}

\texttt{        A+---3---+Z
}\end{verbatim}

FJ cần viết một chương trình đọc vào một tập các đường ống mà mỗi đường ống được mô tả thông qua 2 đầu nối và tính thông lượng từ 'A' tới 'Z'.

Mọi dữ liệu cần tính đều có thể suy luận theo các quy tắc bên trên.

Ống i nối hai nút a\_i và b\_i (a\_i,b\_i là các chữ cái hoa hoặc thường - 'A-Za-z') và có thông lượng F\_i (1  $\le$  F\_i  $\le$  1,000). Tên nút phân biệt chữ cái hoa và thường ('A' $<$$>$ 'a').

\subsubsection{INPUT}

 
\begin{itemize}
	\item Dòng 1: Số nguyên N
	\item Dòng 2..N + 1: Dòng i+1 mô tả ống i qua hai chữ cái và một số nguyên, cách nhau bởi 1 dấu trống: a\_i, b\_i, and F\_i
\end{itemize}

OUTPUT
\begin{itemize}
	\item Dòng 1: Lượng nước lớn nhất có thể truyền từ nguồn A tới chuồng Z.
\end{itemize}

\subsubsection{VÍ DỤ}

Input

5


A B 3


B C 3


C D 5


D Z 4


B Z 6





Output

3
