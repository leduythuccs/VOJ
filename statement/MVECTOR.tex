







   Vector được biểu diễn bởi 1 cặp (X,Y). Tổng các vecto là tổng các thành phần tương ứng. e.g. (1,2)+(3,4)+(5,6) = (1+3+5,2+4+6) = (9,12)  Khối lượng vecto (x,y) là x*x+y*y.  Cho N vecto, tìm một tập con mà tổng của chúng có khối lượng lớn nhất. Kết quả là số 64 bit.  

\subsubsection{   Input  }

   Dòng đầu là N, 1 ≤ N ≤ 30,000, số vector. N dòng tiếp theo là N vecto (X,Y) -30,000 ≤ X,Y ≤ 30,000. Không có vector nào là (0,0)  

\subsubsection{   Output  }

   Ghi khối lượng lớn nhất tìm được  

\subsubsection{   Sample  }
\begin{verbatim}
suma.in 
\\
\\5 
\\5 -8 
\\-4 2 
\\4 -2 
\\2 1 
\\-6 4 
\\
\\suma.out 
\\
\\202 
\\
\\suma.in 
\\
\\4 
\\1 4 
\\-1 -1 
\\1 -1 
\\-1 4 
\\
\\suma.out 
\\
\\64
\\
\\suma.in 
\\
\\9 
\\0 1 
\\6 8 
\\0 -1 
\\0 6 
\\-1 1 
\\-1 2 
\\5 -4 
\\1 0 
\\6 -5 
\\
\\suma.out 
\\
\\360 
\\\end{verbatim}

