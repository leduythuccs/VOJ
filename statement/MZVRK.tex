

Số "whirligig" của 1 số là số thu được bằng cách xóa tất cả các số nằm bên trái của số 1 ở bên phải phải nhất của 1 số trong biểu diễn nhị phân. Ví dụ, whirligig của 6 i.e. (110)2 là 2 i.e. (10)2, và whirligig của 40 i.e. (101000)2 là 8 i.e. (1000)2. Tính tổng tất cả các số whirligig của các số nằm trong khoảng [A,B].

\subsubsection{Input}

Gồm hai số nguyên A,B, 1 ≤ A ≤ B ≤ 10^15. 

\subsubsection{Output}

Ghi ra tổng tìm được, ko cần xài số lớn.

\subsubsection{Sample}
\begin{verbatim}
\textbf{Input} 
176 177
 
\textbf{Output }
17 

\textbf{Input} 
5 9 
 
\textbf{Output } 
13 

\textbf{Input } 
25 28 
 
\textbf{Output}
8 

\end{verbatim}
