




\includegraphics{http://www.spoj.pl/content/yellowflash12:NKCNT2_1.png}

\textbf{    Ví dụ với tam giác đều bậc 1, 2, 3 và 4.   }

   Từ hình trên ta có thể dễ dàng định nghĩa với tam giác đều bậc N (N  $\le$  3000).  

   Ta sẽ đánh dấu các "hàng" của tam giác như sau.  


\includegraphics{http://www.spoj.pl/content/yellowflash12:NKCNT2_2.png}

\textbf{    N=5 với ô đỏ (2,2,3) và ô vàng (1,4,2)   }

\textbf{    Qui ước:   }   để đọc vị trí một ô bất kì, ta đi ngược chiều kim đồng hồ từ đỉnh của tam giác và ghi nhận các "hàng" mà ô đó nằm trên (xem ví dụ ở trên).  

   Cho bảng tam giác bậc N. Hiện tại có 1 số ô đã có màu, các ô còn lại màu trắng.  

\textbf{    Yêu cầu:   }   Đếm số lượng tam giác màu trắng trên bảng.  

\subsubsection{   Input  }

   \_ Dòng đầu tiên chứa số N và K là số bậc của tam giác và số ô đã tô màu (K  $\le$  N^2).  

   \_ K dòng sau là bộ 3 các số a[i], b[i], c[i] thể hiện tọa độ của ô màu thứ i.  

\subsubsection{   Output  }

   \_ Một dòng duy nhất là kết quả bài toán.  

\subsubsection{   Example  }
\begin{verbatim}
\textbf{Input:}
5 2
\\2 2 3
\\1 4 2
\\
\\\textbf{Output:}
31
\end{verbatim}
