



   Thầy Hoàng xây dựng một dãy số vô hạn A từ dãy các số nguyên dương bằng cách lần lượt xét các số tự nhiên bắt đầu từ 1 và lần lượt chọn các số cho dãy A theo quy tắc : chọn một số chia hết cho 1 (hiển nhiên là số 1),  sau đó là hai số chia hết cho 2, tiếp theo là 3 số chia hết cho 3, 4 số chia hết cho 5, 5 số chia hết cho 5…. Như vậy các số đầu tiên của dãy A là: 1, 2, 4, 6, 9, 12, 16, 20, 24, 28, 30, 35, 40, 45, 50, 54, …..  

   Thầy Hoàng tìm ra quy luật xác định một cách nhanh chóng các phần tử của dãy. Bạn là người lập trình giỏi, hãy giúp các bạn Đội tuyển Toán viết chương trình kiểm tra quy luật mà Thầy Hoàng tìm ra có đúng hay không.  

   Yêu cầu: Cho số tự nhiên N. Hãy xác định số thứ N của dãy số.  

\subsubsection{   Dữ liệu  }

   Chứa duy nhất số N (1≤ N ≤100000).  

\subsubsection{   Kết quả  }

   Ghi ra số thứ N tìm được.  

\subsubsection{   Ví dụ  }
\begin{verbatim}
Dữ liệu
10
Kết quả
28

Dữ liệu
13
Kết quả
40
\end{verbatim}