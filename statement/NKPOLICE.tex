



   Để truy bắt tội phạm, cảnh sát xây dựng một hệ thống máy tính mới. Bản đồ khu vực bao gồm N thành phố và E đường nối 2 chiều. Các thành phố được đánh số từ 1 đến N.  

   Cảnh sát muốn bắt các tội phạm di chuyển từ thành phố này đến thành phố khác. Các điều tra viên, theo dõi bản đồ, phải xác định vị trí thiết lập trạm gác. Hệ thống máy tính mới phải trả lời được 2 loại   truy vấn sau:  
\begin{itemize}
	\item     1. Đối với hai thành phố A, B và một đường nối giữa hai thành phố $G_{1}$    , $G_{2}$    , hỏi tội phạm có thể di chuyển từ A đến B nếu đường nối này bị chặn (nghĩa là tên tội phạm không   thể sử dụng con đường này) không?   
	\item     2. Đối với 3 thành phố A, B, C, hỏi tội phạm có thể di chuyển từ A đến B nếu như toàn bộ thành phố C bị kiểm soát (nghĩa là tên tội phạm không thể đi vào thành phố này) không?   
\end{itemize}

\subsubsection{   Dữ liệu vào  }
\begin{itemize}
	\item     Dòng đầu tiên chứa 2 số nguyên N và E ( 2 ≤ N ≤ 100 000, 1 ≤ E ≤ 500 000), số thành phố và số đường nối.   
	\item     Mỗi dòng trong số E dòng tiếp theo chứa 2 số nguyên phân biệt thuộc phạm vi [1, N] - cho biết nhãn của hai thành phố nối với nhau bởi một con đường. Giữa hai thành phố có nhiều nhất một đường nối.   
	\item     Dòng tiếp theo chứa số nguyên Q (1 ≤ Q ≤ 300 000), số truy vấn được thử nghiệm trên hệ thống.   
	\item     Mỗi dòng trong Q dòng tiếp theo chứa 4 hoặc 5 số nguyên. Số đầu tiên cho biết loại truy vấn - 1 hoặc 2.    
\begin{itemize}
	\item       Nếu loại truy vấn là 1, tiếp theo trên cùng dòng là 4 số nguyên A, B, $G_{1}$      , $G_{2}$      với ý nghĩa như đã mô tả. A khác B; $G_{1}$      , $G_{2}$      mô tả một con đường có   sẵn.     
	\item       Nếu loại truy vấn là 2, tiếp theo trên cùng dòng là 3 số nguyên A, B, C với ý nghĩa như đã mô tả. A, B, C đôi một khác nhau.     
\end{itemize}
\end{itemize}

   Dữ liệu được cho sao cho ban đầu luôn có cách di chuyển giữa hai thành phố bất kỳ.  

\subsubsection{   Kết qủa  }

   Gồm Q dòng, mỗi dòng chứa câu trả lời cho một truy vấn. Nếu câu trả lời là khẳng định, in ra "yes". Nếu câu trả lời là phủ định, in ra "no".  

\subsubsection{   Ví dụ  }
\begin{verbatim}
Dữ liệu mẫu
13 15
1 2
2 3
3 5
2 4
4 6
2 6
1 4
1 7
7 8
7 9
7 10
8 11
8 12
9 12
12 13
5
1 5 13 1 2
1 6 2 1 4
1 13 6 7 8
2 13 6 7
2 13 6 8

Kết qủa
yes
yes
yes
no
yes
\end{verbatim}
