

Hai bạn học sinh trong lúc nhàn rỗi nghĩ ra trò chơi sau đây. Mỗi bạn chọn trước một dãy số gồm n số nguyên. Giả sử dãy số mà bạn thứ nhất chọn là:

\emph{$b_{1}$ , $b_{2}$ , ..., $b_{n}$}

còn dãy số mà bạn thứ hai chọn là

\emph{$c_{1}$ , $c_{2}$ , ..., $c_{n}$}

Mỗi lượt chơi mỗi bạn đưa ra một số hạng trong dãy số của mình. Nếu bạn thứ nhất đưa ra số hạng $b_{i}$ (1 ≤ i ≤ n), còn bạn thứ hai đưa ra số hạng $c_{j}$ (1 ≤ j ≤ n) thì giá của lượt chơi đó sẽ là |$b_{i}$ +$c_{j}$ |.

Ví dụ: Giả sử dãy số bạn thứ nhất chọn là 1, -2; còn dãy số mà bạn thứ hai chọn là 2, 3. Khi đó các khả năng có thể của một lượt chơi là (1, 2), (1, 3), (-2, 2), (-2, 3). Như vậy, giá nhỏ nhất của một lượt chơi trong số các lượt chơi có thể là 0 tương ứng với giá của lượt chơi (-2, 2).

\subsubsection{Yêu cầu}

Hãy xác định giá nhỏ nhất của một lượt chơi trong số các lượt chơi có thể.

\subsubsection{Dữ liệu}
\begin{itemize}
	\item Dòng đầu tiên chứa số nguyên dương n (n ≤ $10^{5}$ )
	\item Dòng thứ hai chứa dãy số nguyên $b_{1}$ , $b_{2}$ , ..., $b_{n}$ (|$b_{i}$ | ≤ $10^{9}$ , i=1, 2, ..., n)
	\item Dòng thứ hai chứa dãy số nguyên $c_{1}$ , $c_{2}$ , ..., $c_{n}$ (|$c_{i}$ | ≤ $10^{9}$ , i=1, 2, ..., n)
\end{itemize}

Hai số liên tiếp trên một dòng được ghi cách nhau bởi dấu cách.

\subsubsection{Kết quả}

Ghi ra giá nhỏ nhất tìm được.

\subsubsection{Ràng buộc}
\begin{itemize}
	\item 60\% số tests ứng với 60\% số điểm của bài có 1 ≤ n ≤ 1000.
\end{itemize}

\subsubsection{Ví dụ}
\begin{verbatim}
Dữ liệu:
2
1 -2
2 3

Kết qủa
0
\end{verbatim}
