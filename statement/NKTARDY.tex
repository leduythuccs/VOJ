



   Có n công việc đánh số từ 1 đến n và một máy để thực hiện chúng. Biết:  
\begin{itemize}
	\item     $p_{i}$    là thời gian cần thiết để hoàn thành công việc i.   
	\item     $d_{i}$    là thời hạn hoàn thành công việc i.   
\end{itemize}

   Máy bắt đầu hoạt động từ thời điểm 0. Mỗi công việc cần được thực hiện liên tục từ lúc bắt đầu cho tới khi kết thúc, không cho phép ngắt quãng. Giả sử $c_{i}$   là thời điểm hoàn thành công việc i. Khi đó, nếu $c_{i}$   $>$  $d_{i}$   ta nói công việc i bị hoàn thành trễ hạn, còn nếu  $c_{i}$   ≤ $d_{i}$   thì ta nói công việc i được hoàn thành đúng hạn.  

   Yêu cầu: Tìm trình tự thực hiện các công việc sao cho số công việc hoàn thành trễ hạn là ít nhất.  

\subsubsection{   Dữ liệu  }
\begin{itemize}
	\item     Dòng đầu tiên chứa số nguyên dương n (0 $<$ n ≤ 100000).   
	\item     Dòng thứ hai chứa n số nguyên dương $p_{1}$    , $p_{2}$    , ..., $p_{n}$    (0 $<$ $p_{i}$    ≤ 10000).   
	\item     Dòng thứ ba chứa n số nguyên dương $d_{1}$    , $d_{2}$    , ..., $d_{n}$    (0 $<$ $d_{i}$    ≤ 10000).   
\end{itemize}

\subsubsection{   Kết quả  }
\begin{itemize}
	\item     Dòng đầu tiên ghi số lượng công việc bị hoàn thành trễ hạn theo trình tự tìm được.   
	\item     Dòng tiếp theo ghi n số nguyên dương là chỉ số của các công việc theo trình tự thực hiện tìm được.   
\end{itemize}

\subsubsection{   Hạn chế  }
\begin{itemize}
	\item     Có 50\% số test có n ≤ 100.   
\end{itemize}

\subsubsection{   Ví dụ  }
\begin{verbatim}
\textbf{Dữ liệu}
6
2 4 1 2 3 1
3 5 6 6 7 8

\textbf{Kết quả}
2
1 3 4 6 2 5
\end{verbatim}
