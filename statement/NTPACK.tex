

LC chuẩn bị đi du lịch. Cậu ta đã chuẩn bị sẵn sàng một chiếc valy rất to. Tuy nhiên nếu để tất cả mọi thứ vào đó thì không được khoa học cho lắm. Vì thế LC đã ra siêu thị tìm cho mình 4 chiếc valy con để phân loại đồ dùng: một chiếc để quần áo+đồ dùng cá nhân, một chiếc để laptop, một chiếc để sách vở (sắp thi QG nên LC rất chăm học), một chiếc để đem quà từ nơi du lịch về. Ở siêu thị có rất nhiều loại valy với kích cỡ khác nhau, LC không biết chọn những chiếc valy nào để có thể xếp vừa vào chiếc valy lớn của mình.

Valy có dạng hình chữ nhật. Bạn hãy giúp LC tính xem với mỗi bộ 4 chiếc valy mà LC chọn thì diện tích nhỏ nhất mà chiếc valy lớn cần có là bao nhiêu, đồng thời cho biết kích thước của nó. Lưu ý: Các valy con không được chồng lên nhau.

\subsubsection{Input}
\begin{itemize}
	\item Gồm 4 dòng. Mỗi dòng ghi kích thước của 1 chiếc valy.
\end{itemize}

\subsubsection{Output}
\begin{itemize}
	\item Dòng đầu tiên ghi diện tích nhỏ nhất của chiếc valy có thể bao toàn bộ 4 chiếc valy trên.
	\item Các dòng tiếp theo, mỗi dòng thể hiện kích thước của một chiếc valy thỏa mãn yêu cầu.
\end{itemize}

\subsubsection{Example}
\begin{verbatim}
\textbf{Input:}
1 2
2 3
3 4
4 5

\textbf{Output:}
40
4 10
5 8\end{verbatim}

Kích thước các valy ≤ 50

 
