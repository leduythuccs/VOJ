



   Một hoán vị của các số 1,2,..,n được gọi là "hoán vị sóng -n"  nếu  (a[i]-a[i-1])*(a[i]-a[i+1])$>$0  (i=2,3,..,n-1). Ví dụ, (1,3,2) là một hoán vị sóng -3.  

   LC sắp xếp dãy hoán vị sóng -n theo thứ tự tăng dần, và muốn biết hoán vị thứ k trong dãy đó là gì.  

\subsubsection{   Input  }

   Gồm có nhiều dòng, mỗi dòng gồm 2 số n,k. (n $\le$ 50). Kết thúc là số 0 (không cần đưa ra đáp án cho trường hợp này).  

\subsubsection{   Output  }

   Tương ứng với mỗi dòng trong  input ghi ra hoán vị tương ứng.  

\subsubsection{   Example  }
\begin{verbatim}
\textbf{Input:}
\\3 2
\\5 10
\\4 8
\\0
\\
\\\textbf{Output:
\\}2 1 3
\\2 4 3 5 1
\\3 4 1 2
\\\end{verbatim}
