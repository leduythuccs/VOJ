



 You are to find all pairs of integers such that their sum is equal to the given integer number N and the second number results from the first one by striking out one of its digits. The first integer always has at least two digits and starts with a non-zero digit. The second integer always has one digit less than the first integer and may start with a zero digit.

 

\subsubsection{Input}



 The first line of the input file is the integer number t ( 1 ≤ t ≤ 20 ), the number of test cases. Then t lines follow, each test case in one line; the line consists of a single integer N (10 ≤ N ≤ 10\textasciicircum9).

 

\subsubsection{Output}



 For each test case: 



 On the first line write the total number of different pairs of integers that satisfy the problem statement. On the following lines write all those pairs. Write one pair on a line in ascending order of the first integer in the pair. Each pair must be written in the following format



X + Y = N



 

 Here X, Y, and N, must be replaced with the corresponding integer numbers. There should be exactly one space on both sides of '+' and '=' characters.

 

 

 

\subsubsection{Example}
\begin{verbatim}
\textbf{Input:}

2

302

11



\textbf{Output:}

5

251 + 51 = 302

275 + 27 = 302

276 + 26 = 302

281 + 21 = 302

301 + 01 = 302

1

10 + 1 = 11

\end{verbatim}