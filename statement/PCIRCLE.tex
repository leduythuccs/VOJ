



   Một vòng tròn chứa 2*n vòng tròn nhỏ (Xem hình vẽ). Các vòng tròn nhỏ được đánh số từ 1 đến 2*n theo chiều kim đồng hồ. Cần điền các số tự nhiên từ 1 đến 2*n mỗi số vào một vòng tròn nhỏ sao cho tổng của hai số trên hai vòng tròn nhỏ liên tiếp là số nguyên tố. Số điền ở vòng tròn nhỏ 1 luôn là số 1.  
\includegraphics{http://vn.spoj.pl/content/pcircle.gif}

\subsubsection{   Input  }

   Số nguyên dương n ( 1 $<$ n $<$ 10 ) .  

\subsubsection{   Output  }

   Dòng đầu tiên ghi ra số k là số cách tìm được.   
\\   K dòng tiếp theo mỗi dòng ghi ra 1 cách điền các số vào các vòng tròn nhỏ. Cách điền nào có thứ tự từ điển nhỏ hơn thì xếp trước. Nếu K $>$ 10000 thì chỉ cần ghi ra 10000 cách đầu tiên.  

\subsubsection{   Ví dụ  }
\begin{verbatim}
Input:
4

Output:
4
1 2 3 8 5 6 7 4
1 2 5 8 3 4 7 6
1 4 7 6 5 8 3 2
1 6 7 4 3 8 5 2

\end{verbatim}