

Nam rất thích hoán vị. Một hoán vị N là một cách sắp xếp N số nguyên dương từ 1 đến N, mỗi số chỉ xuất hiện một lần. Ví dụ 1 3 5 2 4 là một hoán vị 5.

Phép nhân 2 hoán vị N ($a_{1}$ , $a_{2}$ , $a_{3}$ , … , $a_{n}$ ) và ($b_{1}$ , $b_{2}$ , $b_{3}$ , … ,$b_{n}$ ) được định nghĩa như sau
\\($a_{1}$ , $a_{2}$ , $a_{3}$ , … , $a_{n}$ ) x ($b_{1}$ , $b_{2}$ , $b_{3}$ , … ,$b_{n}$ ) = ($a_{b1}$ ,$a_{b2}$ , $a_{b3}$ , …, $a_{bn}$ )
\\Ví dụ : (2 5 1 4 3) x (3 4 2 5 1) = (1 4 5 3 2)

Phép lũy thừa hoán vị được định nghĩa theo phép nhân hoán vị :
\\($a_{1}$ , $a_{2}$ , $a_{3}$ , … , $a_{n}$ ) $^ 2 $ = ($a_{1}$ , $a_{2}$ , $a_{3}$ , … , $a_{n}$ ) x ($a_{1}$ , $a_{2}$ , $a_{3}$ , … , $a_{n}$ )
\\($a_{1}$ , $a_{2}$ , $a_{3}$ , … , $a_{n}$ ) $^ k $ = ($a_{1}$ , $a_{2}$ , $a_{3}$ , … , $a_{n}$ ) x ($a_{1}$ , $a_{2}$ , $a_{3}$ , … , $a_{n}$ ) x … x ($a_{1}$ , $a_{2}$ , $a_{3}$ , … , $a_{n}$ )  (k phép nhân hoán vị)
\\
\\Nam nhận thấy có những số nguyên X mà ($a_{1}$ , $a_{2}$ , $a_{3}$ , … , $a_{n}$ ) $^ X $ = ($a_{1}$ , $a_{2}$ , $a_{3}$ , … , $a_{n}$ ). Khi đó ta gọi X là một chu trình của ($a_{1}$ , $a_{2}$ , $a_{3}$ , … , $a_{n}$ ).
\\Với một một hoán vị ban đầu ($a_{1}$ , $a_{2}$ , $a_{3}$ , … , $a_{n}$ ). Nam muốn tìm số nguyên dương K nhỏ nhất sao cho K+1 là một chu trình của ($a_{1}$ , $a_{2}$ , $a_{3}$ , … , $a_{n}$ ). Hãy giúp Nam.

\subsubsection{Input}

Dòng đầu ghi số nguyên dương N (1  $\le$  N  $\le$  $2x10^{6}$ )
\\Dòng thứ 2 gồm N số nguyên dương khác nhau đôi một thể hiện hoán vị ban đầu.

\subsubsection{Output}

Gồm một số nguyên M duy nhất là số dư của K cho $10^{9}$ +7.
\\Dữ liệu vào bảo đảm có kết quả.
\\Trong 30\% số test có N  $\le$  500,  K  $\le$  $5x10^{5}$
\\Trong 50\% số test, K  $\le$  $10^{18}$

\subsubsection{Example}
\begin{verbatim}
\textbf{Input:}
5
5 3 2 1 4\end{verbatim}
\begin{verbatim}
\textbf{Output:}
6\end{verbatim}
\begin{verbatim}
\textbf{Input:}
5
1 2 3 4 5\end{verbatim}
\begin{verbatim}
\textbf{Output:}
1\end{verbatim}
\begin{verbatim}
\textbf{Input:}
5
5 4 3 2 1\end{verbatim}
\begin{verbatim}
\textbf{Output:}
2\end{verbatim}
