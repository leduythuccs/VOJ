



   Khoa và Khiêm là đôi bạn rất thân, thân vì tên 2 bạn đều bắt đầu bằng chữ Kh, thân vì 2 người đều độc thân và đều cua gái không đc giỏi như nhau. Vào một ngày đẹp trời, Khoa vì không muốn mang danh không có tài cua gái nên đã bày ra một trò chơi với Khiêm để xem kẻ nào mới là người không có tài tiếp cận con gái. Khoa đặc biệt rất thích những con số nguyên dương A có dạng như sau:   
\\   \_ A có chữ số tận cùng là k.   
\\   \_ sau khi thực hiện phép nhân A * n ta sẽ được số nguyên dương mới mà tất cả các chữ số của A đều dịch sang phải 1 vị trí (chữ số tận cùng sẽ trở thành chữ số đầu tiên).   
\\   Ví dụ:      128205 * 4 = 512820.   
\\   230769 * 4 = 923076.  

   Trò chơi của Khoa như sau: Khoa viết lên tờ giấy 2 số nguyên dương n và k, nhiệm vụ của Khiêm là tìm số nguyên dương nhỏ nhất có dạng như trên. Nếu Khiêm tìm được, anh đã chứng tỏ được khả năng gái gú của mình, còn không thì Khoa mới là người có trình độ gái gú cao hơn.  

\subsubsection{   Input  }

   1 dòng duy nhất chứa 2 số nguyên dương n và k. ( 1  $\le$  n , k  $\le$  9 )  

\subsubsection{   Output  }

   1 dòng duy nhất chứa số nguyên dương A nhỏ nhất có chữ số tận cùng là k và sau khi thực hiện phép nhân A * n ta sẽ được số nguyên dương mới mà tất cả các chữ số của A đều dịch sang phải 1 vị trí (chữ số tận cùng sẽ trở thành chữ số đầu tiên). Nếu không tìm được, in ra 0.  

\subsubsection{   Example  }
\begin{verbatim}
\textbf{Input1:}
4 5

\textbf{Output1:}
128205\end{verbatim}
\begin{verbatim}
\textbf{Input2:
\\}2 1\end{verbatim}
\begin{verbatim}
\textbf{Output2:
\\}0\end{verbatim}
