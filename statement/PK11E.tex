







    Trò chơi Queen được chơi trên bàn cờ có R hàng và C cột. Các hàng dược đánh số từ 1 đến R và các cột được đánh số từ 1 đến C. Ô trên cùng ở hàng 1 cột 1. Đây là trò chơi đối kháng giữa hai người. Ban đầu có N quân hậu được đặt ở các ô vuông khác nhau trên bàn cờ. Khi tới lượt của mình, người chơi sẽ chọn 1 quân hậu và di chuyển nó hoặc là dọc về phía trên bàn cờ, hoặc ngang về bên trái bàn cờ, hoặc chéo về phía trái trên của bàn cờ, và quân hậu phải luôn luôn nằm trong bàn cờ. Khi một quân hậu đến được ô (1, 1) thì nó sẽ bị bỏ ra khỏi bàn cờ. Người nào thực hiện nước đi hợp lệ cuối cùng giành chiến thắng. Mỗi ô vuông đủ lớn để có thể đặt vào đó vô số quân hậu. Hai người chơi luận phiên nhau. Bạn được cho kích thước của bàn cờ và vị trí ban đầu của N quân hậu. Giả sử rằng cả 2 người đều chơi tối ưu, bạn hãy tính xem ai là người chiến thắng.   

\subsubsection{   Input  }

    Dòng đầu tiên chứa T là số lượng bộ test. Mỗi bộ test bắt đầu bằng 1 dòng chứa 3 số nguyên R(1≤R≤25), C(1≤C≤$10^{15}$    ) và N(1≤N≤1000). Mỗi dòng trong số N dòng tiếp theo lần lượt chứa vị trí của N quân hậu. Các vị trí được mô tả bằng 2 số nguyên, số đầu tiên là chỉ số hàng, số thứ hai là chỉ số cột.   

\subsubsection{   Output  }

    Với mỗi bộ test, in ra “YES” nếu người đi trước có chiến lược để chắc chắn thắng, hoặc “NO” trong trường hợp ngược lại. Xem ví dụ mẫu để biết thêm chi tiết.   

\subsubsection{   Example  }
\begin{verbatim}
\textbf{Input:}

3 

5 5 1 

2 3 

5 5 2 

4 4 

4 4 

5 5 3 

1 2 

2 1 

2 2 \textbf{Output:}

NO 

NO 

YES \end{verbatim}

