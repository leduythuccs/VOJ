



   Sau khi xây dựng xong khu du lịch, thầy Minh bắt tay vào khai thác bằng cách tổ chức các hành trình du lịch. Khu du lịch gồm N địa điểm đánh số từ 1 đến N. Hệ thống giao thông trong vùng gồm M tuyến đường 1 chiều khác nhau, tuyến đường thứ j (j = 1, 2, ..., M) cho phép đi từ địa điểm u\_j đến địa điểm v\_j với chi phí đi lại là số nguyên dương c\_j. Công ty vừa nhận được một hợp đồng yêu cầu xây dựng một hành trình du lịch xuất phát từ địa điểm du lịch bất kỳ và đi thăm một số địa điểm du lịch sau đó quay về địa điểm xuất phát mà chi phí trung bình là nhỏ nhất. Chi phí trung bình được tính bằng tổng chi phí của các tuyến đường mà hành trình đi qua chia cho số tuyến đường trên hành trình.  

\textbf{    Yêu cầu   }   : Cho thông tin về hệ thống giao thông, hãy xây dựng một hành trình du lịch với chi phí trung bình là nhỏ nhất.  

\subsubsection{   Input  }
\begin{itemize}
	\item     Dòng thứ nhất chứa 2 số nguyên dương N ≤ $10^{3}$    , M         ≤ $10^{4}$     .    
	\item      Dòng thứ j trong số M dòng tiếp theo chứa 3 số nguyên dương u\_j, v\_j, c\_j cho biết thông tin về tuyến đường thứ j. Giả thết là u\_j khác v\_j; c\_j  $\le$  $10^{9}$     với j = 1, 2, ..., M.    
\end{itemize}

\subsubsection{   Output  }

   Ghi ra giá trị tổng chi phí cho số địa điểm trên hành trình tìm được, làm tròn tới đúng 2 chữ số sau dấu phẩy. Ghi ra xâu NO TOUR nếu không tìm được hành trình du lịch thỏa mãn yêu cầu.  

\subsubsection{   Example  }
\begin{verbatim}
\textbf{Input:}
6 8
1 2 4 
2 4 1 
4 3 3 
3 1 4 
4 1 1 
3 5 5 
5 3 1 
5 6 7

\textbf{Output:}
2.00
\end{verbatim}


\begin{verbatim}
\textbf{Input:}
2 1
1 2 3
\textbf{Output:}
NO TOUR
\end{verbatim}
