



   Có N con bò (1  $\le$  N  $\le$  1,000), để thuận tiện ta đánh số từ 1->N, đang ăn cỏ trên N đồng cỏ, để thuận tiện ta cũng đánh số các đồng cỏ từ 1->N. Biết rằng con bò i đang ăn cỏ trên đồng cỏ i.  

   Một vài cặp đồng cỏ được nối với nhau bởi 1 trong N-1 con đường 2 chiều mà các con bò có thể đi qua. Con đường i nối 2 đồng cỏ A\_i và B\_i (1  $\le$  A\_i  $\le$  N; 1  $\le$  B\_i  $\le$  N) và có độ dài là L\_i (1  $\le$  L\_i  $\le$  10,000).  

   Các con đường được thiết kế sao cho với 2 đồng cỏ bất kỳ đều có duy nhất 1 đường đi giữa chúng. Như vậy các con đường này đã hình thành 1 cấu trúc cây.  

   Các chú bò rất có tinh thần tập thể và muốn được thăm thường xuyên. Vì vậy lũ bò muốn bạn giúp chúng tính toán độ dài đường đi giữa Q (1  $\le$  Q  $\le$  1,000) cặp đồng cỏ (mỗi cặp được mô tả là 2 số nguyên p1,p2 (1  $\le$  p1  $\le$  N; 1  $\le$  p2  $\le$  N).  

\subsubsection{   DỮ LIỆU  }
\begin{itemize}
	\item     Dòng 1: 2 số nguyên cách nhau bởi dấu cách: N và Q   
	\item     Dòng 2..N: Dòng i+1 chứa 3 số nguyên cách nhau bởi dấu cách: A\_i,         B\_i, và L\_i   
	\item     Dòng N+1..N+Q: Mỗi dòng chứa 2 số nguyên khác nhau cách nhau bởi dấu cách         mô tả 1 yêu cầu tính toán độ dài 2 đồng cỏ mà lũ bò muốn đi thăm qua lại p1 và p2.   
\end{itemize}

\subsubsection{   KẾT QUẢ  }
\begin{itemize}
	\item     Dòng 1..Q: Dòng i chứa độ dài đường đi giữa 2 đồng cỏ         ở yêu cầu thứ i.   
\end{itemize}

\subsubsection{   VÍ DỤ  }
\begin{verbatim}
Dữ liệu
4 2
2 1 2
4 3 2
1 4 3
1 2
3 2

Kết quả
2
7
\end{verbatim}

\subsubsection{   GIẢI THÍCH  }

   Yêu cầu 1: Con đường giữa đồng cỏ 1 và 2 có độ dài là 2. Yêu cầu 2: Đi qua con đường nối đồng cỏ 3 và 4, rồi tiếp tục đi qua con đường nối 4 và 1, và cuối cùng là con đướng nối 1 và 2, độ dài tổng cộng là 7.  
