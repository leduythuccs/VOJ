



   Vào năm 1945, Liên Xô đang đánh nhau với phát xít Đức hết sức ác liệt. Hàng triệu thanh niên Liên Xô phải lên đường nhập ngũ. Một cuộc duyệt binh diễn ra, các tân binh không biết đứng quay mặt về bên nào liền xếp tùy ý, vị tổng chỉ huy thấy thể liền ra lệnh: “Nếu hai tân binh liên tiếp và đối mặt với nhau thì ngay lập phải quay ngược lại(180 độ), động tác này diễn ra trong vòng 1s!”. Người tổng chỉ huy muốn biết sau bao lâu thì thì đội hình sẽ ngừng quay?  

\subsubsection{   Input  }

   Dòng đầu ghi số nguyên N là số tân binh.   


   Dòng thứ hai gồm đúng N kí tự ‘$<$’, ‘$>$’ thể hiện cách đứng của các tân binh. Nếu hai tân binh liên tiếp quay mặt vào nhau thì sẽ được biểu diễn bởi ‘$>$$<$’. ( 1 ≤ n ≤ 1000000 ).  

\subsubsection{   Output  }

   Gồm một số duy nhất ghi thời gian để đội hình ngừng quay.  

\subsubsection{   Example  }
\begin{verbatim}
Input:
4
$<$$>$$<$$>$

Output:
1
\end{verbatim}  Chú thích :  


  Tại thời điểm 0: $<$$>$$<$$>$  


