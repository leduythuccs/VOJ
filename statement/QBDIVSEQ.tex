



   Dãy số M phần tử B được gọi là dãy con của dãy số A gồm N phần tử nếu tồn tại một mã chuyển C gồm M phần tử thoả mãn B[i]=A[C[i]] với mọi I = 1…M và 1 ≤ C[1] $<$ C[2] $<$ ... $<$ C[m] ≤ N.  

   Một cách chia dãy A thành các dãy con "được chấp nhận" nếu các dãy con này là các dãy không giảm và mỗi phần tử của dãy A thuộc đúng một dãy con.  

   Yêu cầu: Bạn hãy chia dãy con ban đầu thành ít dãy con nhất mà vẫn "được chấp nhận".  

\subsubsection{   Input  }

   Dòng đầu tiên ghi số N là số phần tử của dãy A. ( N ≤ 10   $^    5   $   )  

   N dòng tiếp theo ghi N số tự nhiên là các phần tử của dãy A. (  A   $_    i   $   ≤ 10   $^    9   $   )  

\subsubsection{   Output  }

   Ghi một duy nhất là số lượng dãy con ít nhất thỏa mãn.  

\subsubsection{   Example  }
\begin{verbatim}
Input:
4
1
5
4
6

Output:
2

\end{verbatim}