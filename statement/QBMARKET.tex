

 

An được mời tham gia trò chơi “Siêu thị may mắn” do đài truyền hình ZTV tổ chức.

Siêu thị được đặt trong trường quay truyền hình có n mặt hàng được đánh số từ 1 đến n và mặt hàng thứ i được niêm yết giá là $c_{i}$ đồng, i = 1, 2, ..., n.

Theo thể lệ của trò chơi, An được ban tổ chức tặng một thẻ mua hàng có giá trị là s đồng và phải dùng hết số tiền trong thẻ này để mua hàng trong siêu thị với điều kiện mặt hàng thứ i chỉ được mua với số lượng nhiều nhất là $m_{i}$ , i = 1, 2, …, n.

An sẽ là người thắng cuộc nếu tìm được tổng số cách mua hàng thỏa mãn yêu cầu đặt ra và chỉ ra một cách mua hàng nếu có.

Yêu cầu: Hãy giúp An trở thành người thắng cuộc khi cho bạn biết trước các giá trị n, s, $c_{i}$ và $m_{i}$ (1 ≤ n ≤ 500; 1 ≤ s ≤ $10^{5}$ ; 1 ≤ $c_{i}$ ≤ $10^{4}$ ; 1 ≤ $m_{i}$ ≤ 100) với i = 1, 2, …, n.

\subsubsection{Input}

Dòng đầu tiên chứa hai số nguyên dương s và n.

Dòng thứ i trong n dòng tiếp theo chứa hai số nguyên dương $c_{i}$ và $m_{i}$ với i = 1, 2, …, n.

\subsubsection{Output}

Gồm 1 dòng duy nhất ghi số nguyên d là tổng số cách mua hàng tìm được.

\subsubsection{Example}
\begin{verbatim}
Input:
12 3
4 1
6 2
2 1

Output:
2
\end{verbatim}
