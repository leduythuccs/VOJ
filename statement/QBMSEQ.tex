

 

Cho dãy số nguyên dương $a_{1}$ , $a_{2}$ , ..., $a_{n}$ .

Dãy số: $a_{i}$ , $a_{i+1}$ , ..., $a_{j}$ thỏa mãn $a_{i}$ ≤ $a_{i+1}$ ≤ ... ≤ $a_{j}$ . Với 1 ≤ i ≤ j ≤ n được gọi là dãy con không giảm của dãy số đã cho và khi đó số j-i+1 được gọi là độ dài của dãy con này.

Yêu cầu: Trong số các dãy con không giảm của dãy số đã cho mà các phần tử của nó đều thuộc dãy số \{$u_{k}$ \} xác định bởi $u_{1}$ = 1, u $_ k = $u_{k}$ -1 $ + k (k ≥ 2), hãy tìm dãy con có độ dài lớn nhất.

\subsubsection{Input}

Dòng đầu tiên chứa một số nguyên dương n (n ≤ $10^{4}$ ).

Dòng thứ i trong n dòng tiếp theo chứa một số nguyên dương $a_{i}$ ($a_{i}$ ≤ $10^{8}$ ) là số hạng thứ i của dãy số đã cho, i = 1, 2, ..., n.

\subsubsection{Output}

Gồm 1 dòng duy nhất ghi số nguyên d là độ dài của dãy con không giảm tìm được (quy ước rằng nếu không có dãy con nào thỏa mãn điều kiện đặt ra thì d = 0).

\subsubsection{Example}
\begin{verbatim}
Input:
8
2
2007
6
6
15
16
3
21
Output:
3

\end{verbatim}
