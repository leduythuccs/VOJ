



   Trong các cuộc thi tin học, sự xuất hiện của những bài toán hình học làm đội tuyển CBQ khá lúng túng. Do đó thầy Thạch quyết định cho đội tuyển luyện tập các bài toán hình học. Bắt đầu từ điểm, thầy đưa ra bài toán:  

   Cho n điểm trong mặt phẳng Oxy, hãy đếm số bộ 3 điểm thằng hàng  

\subsubsection{   Input  }

   Dòng thứ nhất ghi số N là số điểm trên mặt phẳng.  

   N dòng tiếp theo, mỗi dòng ghi tọa độ của một điểm.  

\subsubsection{   Output  }

   Một số duy nhất là số bộ 3 điểm thẳng hàng.  

\subsubsection{   Example  }
\begin{verbatim}
Input:
6
0 0
0 1
0 2
1 1
2 0
2 2
Output:
3

Giới hạn:
1 ≤ N ≤ 2000.
Tọa độ các điểm có trị tuyệt đối không quá 10000.
\end{verbatim}