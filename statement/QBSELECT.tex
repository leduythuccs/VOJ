

Cho một bảng hình chữ nhật kích thước 4×n ô vuông. Các dòng được đánh số từ 1 đến 4, từ trên xuống dưới, các cột được đánh số từ 1 đến n từ trái qua phải.

Ô nằm trên giao của dòng i và cột j được gọi là ô (i,j). Trên mỗi ô (i,j) có ghi một số nguyên aij , i =1, 2, 3, 4; j =1, 2, ..., n. Một cách chọn ô là việc xác định một tập con khác rỗng S của tập tất cả các ô của bảng sao cho không có hai ô nào trong S có chung cạnh. Các ô trong tập S được gọi là ô được chọn, tổng các số trong các ô được chọn được gọi là trọng lượng của cách chọn. Tìm cách chọn sao cho trọng lượng là lớn nhất.

Ví dụ: Xét bảng với n=3 trong hình vẽ dưới đây:


\includegraphics{http://www.spoj.pl/content/cun:select.JPEG}

Cách chọn cần tìm là tập các ô S = \{(3,1), (1,2), (4,2), (3,3)\} với trọng lượng 32.

\subsubsection{Input}

Dòng đầu tiên chứa số nguyên dương n là số cột của bảng.

Cột thứ j trong số n cột tiếp theo chứa 4 số nguyên $a_{1j}$ , $a_{2j}$ , $a_{3j}$ , $a_{4j}$ , hai số liên tiếp cách nhau ít nhất một dấu cách, là 4 số trên cột j của bảng.

\subsubsection{Output}

Gồm 1 dòng duy nhất là trọng lượng của cách chọn tìm được.

\subsubsection{Example}
\begin{verbatim}
Input:
3
-1 9 3
-4 5 -6
7  8 9
9  7 2

Output:
32

\end{verbatim}

\subsubsection{Hạn chế}

Trong tất cả các test: n ≤ 10000, |$a_{ij}$ | ≤ 30000. Có 50\% số lượng test với n ≤ 1000.
