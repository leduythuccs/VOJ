

 

Như một phần của chiến dịch tiếp thị, một công ty lớn ở Gdynia muốn treo logo của mình ở một số nơi trong thành phố. Công ty dành hẳn một khoản ngân sách trong năm để đầu tư tiếp thị vì vậy logo tạo ra rất hoành tráng. Giám đốc tiếp thị quyết định dùng nguyên cả dãy các tòa nhà để treo các phần của logo.

Logo bao gồm \textbf{\emph{ n }} băng treo dọc với độ dài khác nhau. Các băng được đánh số từ 1 đến n từ trái sang phải. Logo được mô tả bởi hoán vị ( \textbf{\emph{ s $_ 1 $}} , \textbf{\emph{ s $_ 2 $}} , . . ., \textbf{\emph{ s $_ n $}} ) các số tự nhiên từ 1 đến \textbf{\emph{ n }} . Băng số \textbf{\emph{ s $_ 1 $}} là ngắn nhất, băng số \textbf{\emph{ s $_ 2 $}} dài hơn hoặc bằng băng thứ \textbf{\emph{ s $_ 1 $}} nhưng ngắn hơn các băng còn lại, . . ., băng \textbf{\emph{ s $_ n $}} là dài nhất. Độ dài thực sự của các băng không phải là vấn đề đáng lưu ý.

Đường phố chính của Gdynia có \textbf{\emph{ m }} tòa nhà. Điều có thể làm cho bạn ngạc nhiên là các tòa nhà có độ cao khác nhau. Vấn đề đặt ra là tìm tất cả các vị trí có thể để treo logo.

Công ty muốn tìm dãy \textbf{\emph{ n }} tòa nhà liên tiếp nhau để treo logo. Nhà \textbf{\emph{ s $_ 1 $}} phải là thấp nhất, nhà \textbf{\emph{ s $_ 2 $}} – thấp thứ nhì, . . . Ví dụ dãy 3 nhà với các độ cao (5, 10, 4) thì dãy sắp xếp sẽ là (3, 1, 2).


\includegraphics{http://vn.spoj.pl/content/MATCHING_VOJ.jpg}

\subsubsection{Input}
\begin{itemize}
	\item Dòng đầu tiên chứa 2 số nguyên \textbf{\emph{ n }} và \textbf{\emph{ m }} (1 ≤ \textbf{\emph{ n }} ≤ \textbf{\emph{ m }} ≤ 2.10 $^ 5 $ )
	\item Dòng thứ 2 chứa \textbf{\emph{ n }} số nguyên \textbf{\emph{ s $_ 1 $}} , \textbf{\emph{ s $_ 2 $}} , . . ., \textbf{\emph{ s $_ n $}}
	\item Dòng thứa 3 chứa \textbf{\emph{ m }} số nguyên \textbf{\emph{ h $_ i $}} , i = 1 ÷ \textbf{\emph{ m }} (1 ≤ \textbf{\emph{ h $_ i $}} ≤ 10 $^ 9 $ ) – độ cao của các tòa nhà
\end{itemize}

\subsubsection{Output}
\begin{itemize}
	\item Dòng đầu tiên chứa số nguyên \textbf{\emph{ k }} – số dãy tìm được
	\item Dòng thứ 2 chứa \textbf{\emph{ k }} số nguyên theo thứ tự tăng dần – số nhà đầu tiên của mỗi dãy
\end{itemize}

\subsubsection{Example}
\begin{verbatim}
\textbf{Input:}
5 10
2 1 5 3 4
5 6 3 8 12 7 1 10 11 9
\textbf{Output:}
2
2 6\end{verbatim}