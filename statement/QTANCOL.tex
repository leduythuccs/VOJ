

Ancol là một tay pha chế rượu có tiếng ở VOJ. Cách pha chế rượu của anh ta rất đặc biệt đó là:
\begin{itemize}
	\item Chỉ dùng N chai để pha chế rượu.
	\item Không có 2 chai nào có độ rượu bằng nhau.
	\item Chỉ dùng 2 loại rượu để pha chế cho mỗi ly rượu mà khách gọi.
	\item Chỉ lấy 2 loại rượu nằm ở gần nhau.
	\item Sau khi pha chế xong anh ta sẽ đổi vị trí 2 chai rượu đó.
\end{itemize}

Tuy nhiên do sức khỏe anh ta chỉ pha cho đúng K vị khách rồi nghỉ.

Vào một buổi sáng đẹp trời, anh ta nhận ra 1 điều là có rất nhiều lần sau khi nghỉ các chai rượu được sắp xếp với độ rượu tăng dần mặc dù ban đầu chúng được đề ở vị trí bất kì.

\textbf{Input:}

Gồm 2 số N,K.

\textbf{Output:}

Số trường hợp các chai ban đầu mà sau khi nghỉ các chai rượu được xếp tăng dần về độ rượu sau khi modul $10^{9}$+9.

\textbf{Examples:}

3 0       

1   

Giải thích: (1,2,3)

3 1 

2                  

Giải thích: (2,1,3) -> (1,2,3)

               (1,3,2) -> (1,2,3)

3 2                            

3

Giải thích: (3,1,2) -> (1,3,2) -> (1,2,3)

               (2,3,1) -> (2,1,3) -> (1,2,3)

               (1,2,3) -> (1,3,2) -> (1,2,3)

3 3    

3                

Giải thích: (2,1,3) -> (1,2,3) -> (2,1,3) -> (1,2,3)

               (1,3,2) -> (1,2,3) -> (1,3,2) -> (1,2,3)

               (3,2,1) -> (3,1,2) -> (1,3,2) -> (1,2,3)

\textbf{Lưu ý:} thứ tự đổi và cách đổi khác nhau với cùng 1 thứ tự chai ban đầu chỉ tính là 1.

(1,3,2) -> (1,2,3) -> (1,3,2) -> (1,2,3)

(1,3,2) -> (1,2,3) -> (2,1,3) -> (1,2,3)

2 cách cách đổi trên được xem là 1 vì đều xuất phát từ (1,3,2).

\textbf{Giới hạn:} 1≤N≤2000. 0≤K≤2000.

 

 
