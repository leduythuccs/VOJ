

Ngày 9/12/1995 là ngày mà cậu bé \textbf{\emph{TTN}} ra đời. Và đúng 1 tháng sau (ngày đầy tháng của \textbf{\emph{TTN}}), vào buổi tối hôm ấy, \textbf{\emph{TTN}} đã mơ thấy mình đang đứng trong 1 phòng học rất lớn trong một ngôi trường rất bé. Và trên bảng là 1 bài toán. Cậu không biết lớp học này là lớp mấy, cậu chỉ biết ngồi xung quanh mình là rất nhiều vị bác học nổi tiếng như \textbf{\emph{Anbe Anhxtanh}}, \textbf{\emph{Isaac Newton}}, \textbf{\emph{Pythagoras}},\textbf{\emph{Euclid}}, ……..Và tất cả đều đang đau đầu bởi bài toán ghi trên bảng. Nhưng chỉ mất 5 phút \textbf{\emph{TTN}} đã làm cho tất cả mọi người đều phải thán phục khả năng giải toán của mình bằng cách đưa ra lời giải.

Bạn hãy thử khả năng của mình với bài toán này xem:

Cho các số có dạng xA với:

- x là một toán tử + hoặc – hoặc *.

- A là một số nguyên từ 0 đến 9 (vì \textbf{\emph{TTN}} mới học đến đây nên chỉ giải được đến đây thôi ^^)

Hãy tìm giá trị trung bình của các biểu thức khi xếp các $x_{i$A$_i}$ liên tiếp.

Lưu ý: vì $x_{1}$ là + hoặc – thì vẫn có nghĩa nên khi $x_{1}$=* xem như không tồn tại $x_{1$A$_1}$ và xét tiếp qua $x_{2$A$_2}$ và cứ thế.

\textbf{Examples:}

$X_{1$A$_1}$ = -1

$X_{2$A$_2}$ = *2

$X_{3$A$_3}$ = +3

Các biểu thức được tạo ra:

-1*2+3 = 1

-1+3*2 = 5

*2-1+3 = 2 (vì *2 ở đầu biểu thức xem như không tồn tại và chỉ còn -1+3)

*2+3-1 = 2

+3-1*2 = 1

+3*2-1 = 5

Kết quả sẽ là (1+5+2+2+1+5)/6 = 2.667

Giả sử với trường hợp *1*2*3*4*5+6 thì *1 không tồn tại tiếp đến xét *2 cũng không tồn tại và cứ thể biểu thức chỉ còn +6.

\textbf{Input:}

Một số dòng với dòng n chứa $x_{n$A$_n}$.

\textbf{Ouput:}

Kết quả bài toán làm tròn 3 chữ số sau dấu phẩy, nếu không tồn tại 1 biểu thức nào thì đưa ra ‘No’.

\textbf{Giới hạn:} n≤50.

\emph{Thuật toán chuẩn cho bài này chỉ 1s với máy chấm Pyramid. Nhưng để hẳn timelimit 10s cho máy chấm Cube để các bạn có niềm tin hơn với các test n$<$10.}
