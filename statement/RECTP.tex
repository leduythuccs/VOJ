



   Cho N hình chữ nhật có các cạnh song song với các trục Ox và Oy .   


   Hãy tính chu vi của hình tạo bởi N hình chữ nhật này . Với định nghĩa chu vi của N HCN là tổng độ dài các đường biên giúp phân biệt đâu là miền nằm trong các HCN và miền nằm ngoài các HCN .   


\textit{    Chú ý : N hình chữ nhật này có thể tách rời nhau , không nhất thiết là đè lên nhau .   }

\subsubsection{   Input  }

   Dòng 1 : số nguyên N ( 1 ≤ N ≤ 10000 ) .   


   N dòng tiếp theo , mỗi dòng gồm 4 số nguyên x1 , y1 , x2 , y2 tương ứng là toạ độ góc trái dưới và góc phải trên của hình chữ nhật thứ i .( 0 ≤ x1 ≤ x2 ≤ 30000 , 0 ≤ y1≤ y2 ≤ 30000 ) .  

\subsubsection{   Output  }

   Gồm 1 dòng ghi ra chu vi của N hình chữ nhật .  

\subsubsection{   Example  }
\includegraphics{http://www.spoj.pl/CSP/content/rectp.gif}
\begin{verbatim}
Input:
2
10 10 20 20
15 15 25 30

Output:
70
\end{verbatim}
