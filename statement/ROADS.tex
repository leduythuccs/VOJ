



   Có N thành phố 1..N nối bởi các con đường một chiều. Mỗi con đường có hai giá trị: độ dài và chi phí phải trả để đi qua. Bob ở thành phố 1. Bạn hãy giúp Bob tìm đường đi ngắn nhất đến thành phố N, biết rằng Bob chỉ có số tiền có hạn là K mà thôi.  

\subsubsection{   Dữ liệu  }

   Dòng đầu tiên ghi t là số test. Với mỗi test, dòng đầu ghi K (0 ≤ K ≤ 10000). Dòng 2 ghi N, 2 ≤ N ≤ 100. Dòng 3 ghi R, 1 ≤ R ≤ 10000 là số đường nối. Mỗi dòng trong N dòng sau ghi 4 số nguyên S, D, L, T mô tả một con đường nối giữa S và D với độ dài L ( 1 ≤ L ≤ 100) và chi phí T (0 ≤ T ≤ 100). Lưu ý có thể có nhiều con đường nối giữa hai thành phố.  

\subsubsection{   Kết quả  }

   Với mỗi test, in ra độ dài đường đi ngắn nhất từ 1 đến N mà tổng chi phí không quá K. Nếu không tồn tại, in ra -1.  

\subsubsection{   Ví dụ  }
\begin{verbatim}
Dữ liệu
2
5
6
7
1 2 2 3
2 4 3 3
3 4 2 4
1 3 4 1
4 6 2 1
3 5 2 0
5 4 3 2
0
4
4
1 4 5 2
1 2 1 0
2 3 1 1
3 4 1 0

Kết quả
11
-1
\end{verbatim}