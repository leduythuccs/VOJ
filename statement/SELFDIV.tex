

Bạn có thấy số 324 rất đặc biệt? Nó chia hết cho 3, 2 và 4. Một số tự nhiên chia hết cho tất cả các chữ số của nó được gọi là \textbf{ số tự chia hết } .

Một số ví dụ khác về \textbf{ số tự chia hết } là: 5, 12, 784, 8736. Số 102 không phải là \textbf{ số tự chia hết } vì nó không chia hết cho 0.

\textbf{Yêu cầu: } Đếm số lượng \textbf{ số tự chia hết } có đúng \textbf{ N } chữ số. Vì kết quả có thể rất lớn, bạn chỉ cần in ra phần dư khi chia cho \textbf{ 10007 } . \textbf{}

\subsubsection{Input}
\begin{itemize}
	\item Một dòng duy nhất ghi số \textbf{ N } .
\end{itemize}

\subsubsection{Output}
\begin{itemize}
	\item Phần dư khi chia số lượng \textbf{ số tự chia hết } có \textbf{ N } chữ số cho \textbf{ 10007 } .
\end{itemize}

\subsubsection{Example}
\begin{verbatim}
\textbf{Input:}
3
\textbf{Output:}
56\end{verbatim}

\subsubsection{Giới hạn}
\begin{itemize}
	\item 3 ≤ \textbf{ N } ≤ 10 $^ 6 $ .
	\item Trong 80\% số test, \textbf{ N } không vượt quá 500. Giới hạn thời gian cho các test này là 10 giây.
	\item Giới hạn thời gian cho các test còn lại là 30 giây.
\end{itemize}