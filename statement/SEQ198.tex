

Con số 198 có gợi cho bạn điều gì không? Khi học lịch sử Việt Nam, Vinh biết rằng ngày 19-8-1945 là ngày Tổng khởi nghĩa, ngày nhân dân cả nước ta nhất tề đứng lên làm cuộc Cách mạng Tháng Tám vĩ đại. Hiện nay, 198 được đặt tên cho nhiều bệnh viện, công viên, đường phố trong cả nước. Con số này đã gợi ý cho Vinh khảo sát dãy số SEQ98 sau đây: Dãy số nguyên không âm a1, a2, ..., an được gọi là dãy SEQ198 nếu không tồn tại hai chỉ số i và j (1 ≤ i,j ≤ n) mà ai-aj hoặc là bằng 1 hoặc là bằng 8 hoặc là bằng 9.

\textbf{Ví dụ:}

Dãy số nguyên 1, 3, 5, 7 là dãy SEQ198.

Dãy số nguyên 7, 3, 5, 1, 9, 21 không phải là dãy SEQ198 bởi vì có hai phần tử 1 và 9 có hiệu 9 - 1 = 8. Tuy nhiên, sau khi xóa bớt phần tử 1, ta thu được dãy 7, 3, 5, 9, 21 là một dãy SEQ198.

Vinh quan tâm tới bài toán sau đây: Cho dãy số nguyên không âm b1, b2, ..., bm, hãy tìm cách loại bỏ một số ít nhất phần tử của dãy để được dãy còn lại là SEQ198.

\textbf{Yêu cầu:}

Hãy giúp Vinh giải quyết bài toán đặt ra

 

\textbf{Dữ liệu:}

Dòng đầu chứa số nguyên dương m;

Dòng thứ hai chứa m số nguyên không âm b1, b2, ..., bm (bi ≤ $10^{9}$).

 

\textbf{Kết quả:}

Ghi ra số nguyên k là số phần tử bị loại bỏ. Ghi số 0 nếu dãy đã cho là SEQ198.

 

\textbf{Ví dụ:}

Input

6

7 3 1 9 21

Output

1

 

\textbf{Ràng buộc:}

Có 50\% số test ứng với 50\% số điểm của bài có m ≤ 20.

Có 50\% số test còn lại ứng với 50\% số điểm của bài có m ≤ 2000.
