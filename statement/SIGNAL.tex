

Để chuẩn bị cho ASIAD 2010, thành phố Quảng Châu, Trung Quốc dự định lắp đặt 1 số trạm thu tín hiệu từ các địa điểm thi đấu trên khắp thành phố. Có N địa điểm thi đấu, có thể biểu thị bằng N điểm trên mặt phẳng. Trạm thu tín hiệu là 1 đường tròn đi qua 3 địa điểm thi đấu, và nó sẽ nhận thông tin từ \emph{ tất cả các địa điểm thi đấu nằm trong hình tròn đó. }

Xét ví dụ sau:


\includegraphics{https://drive.google.com/uc?export=view&amp;id=1T_v4HBpaB33VLwcH1sEK6AT-IoSEXxYm}

Trong hình, ta có 4 địa điểm thi đấu nằm tại A,B,C,D. Khi đó, trạm thu tín hiệu đi qua A,C,D (kí hiệu là (ACD)) và (ABD) chỉ chứa 3 điểm, trong khi (ABC) và (BCD) chứa cả 4 điểm. Như vậy, trung bình 1 trạm thu tín hiệu sẽ thu được (4 + 4 + 3 + 3)/4 = 3.5 địa điểm thi đấu.

 

Yêu cầu:
\begin{itemize}
	\item 

Cho biết tọa độ của N điểm trên mặt phẳng, xác định số địa điểm thi đấu trung bình mà 1 trạm thu tín hiệu có thể nhận được
\end{itemize}

Input:
\begin{itemize}
	\item 

Dòng đầu tiên ghi số nguyên dương N là số địa điểm thi đấu
	\item 

N dòng sau, mỗi dòng gồm 2 số nguyên x[i],y[i] là tọa độ của điểm thi đấu thứ i
\end{itemize}

Output:
\begin{itemize}
	\item 

Gồm 1 dòng duy nhất ghi 1 số thực: kết quả bài toán. Kết quả cần được viết chính xác đến 6 chữ số sau dấu phẩy thập phân
\end{itemize}

Giới hạn:
\begin{itemize}
	\item 

3  $\le$  N  $\le$  1500
	\item 

-10^6  $\le$  x[i],y[i]  $\le$  10^6
	\item 

Không có 3 điểm nào thẳng hàng
	\item 

Không có 4 điểm nào nằm trên cùng 1 đường tròn
\end{itemize}

Ví dụ:
\begin{verbatim}
Input

4

0 2

4 4

0 0

2 0

Output

3.500000

Input

10

-50 46

12 -45

-75 -9

63 -90

-55 58

34 -93

12 -56

-39 -92

-94 2

-97 56

Output

5.916667\end{verbatim}
