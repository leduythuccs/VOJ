







   Cho một đa giác lồi N đỉnh. Các đỉnh được đánh số từ 1 đến N. Hãy chọn ra K trong số N điểm tạo thành đa giác có diện tích lớn nhất có thể.  

\subsubsection{   Dữ liệu  }
\begin{itemize}
	\item     Dòng đầu tiên ghi số N và K (3 ≤ K ≤ N ≤ 200).   
	\item     Dòng thứ U trong N dòng tiếp theo ghi hai số nguyên $X_{u}$    và $Y_{u}$    là tọa độ của điểm thứ U. Các đỉnh được liệt kê theo chiều kim đồng hồ. Tọa độ các điểm có trị tuyệt đối nhỏ hơn hoặc bằng $10^{5}$    .   
\end{itemize}

\subsubsection{   Kết quả  }
\begin{itemize}
	\item     Dòng đầu tiên in ra diện tích của đa giác K đỉnh lớn nhất (làm tròn đến 2 chữ số thập phân).   
	\item     Dòng thứ hai là dãy tăng gồm K số là số hiệu của các điểm được chọn.   
\end{itemize}

\subsubsection{   Ví dụ  }
\begin{verbatim}
\textbf{Dữ liệu}
\\4 3
\\1 2
\\3 1
\\2 0
\\1 1
\\
\\\textbf{Kết quả}
\\1.50
\\1 2 3
\\\end{verbatim}

