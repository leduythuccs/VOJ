



   Cho một dãy số A   $_    1   $   .. A   $_    N   $   theo công thức sau A   $_    i   $   = (A   $_    i−1   $   + A   $_    i+1   $   ) / 2 − 1 với mọi 1 $<$ i $<$ N và A   $_    i   $   $>$= 0 với mọi 1  $\le$  i  $\le$  N. Biết N và A   $_    1   $   , tìm giá trị nhỏ nhất có thể của A   $_    N   $   .  

\subsubsection{   Dữ liệu  }

   Dòng duy nhất ghi số nguyên N và số thực A   $_    1   $   (3  $\le$  N  $\le$  1000, 10  $\le$  A   $_    1   $    $\le$  1000).  

\subsubsection{   Kết qủa  }

   Ghi giá trị nhỏ nhất của A   $_    N   $   có thể có, chính xác đến 2 chữ số sau dấu phẩy.  
\begin{verbatim}
\textbf{Dữ liệu:} 
692 532.81
\textbf{Kết qủa} 
446113.34 
\end{verbatim}
