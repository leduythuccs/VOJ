



   Mùng 1 Tết, Nuga quyết định khai bút đầu xuân bằng cách giải một bài tập tin rất khó mà bạn ấy đã thấy trong một giấc mơ.  

   Cho một dãy số nguyên dương. Tính xem có bao nhiêu dãy con gồm các phần tử liên tiếp của nó mà có số lượng phần tử khác nhau nằm trong khoảng [L, U].  

   Sau 2 tiếng ngồi cắn bút mà không giải xong, Nuga quyết định nhờ các bạn đội tuyển tin giúp đỡ. Nuga sẽ rất biết ơn nếu các bạn giải được bài tập này, và bạn ấy sẽ lì xì cho các bạn nhân dịp đầu năm mới.  

\subsubsection{   Dữ liệu  }

   Dòng đầu tiên chứa các số nguyên dương N, L, U. Sau đó là N dòng, mỗi dòng chứa một số nguyên dương X là một phần tử của dãy số.  

\subsubsection{   Kết quả  }

   Một dòng duy nhất chứa một số nguyên thể hiện số dãy con có số lượng phần tử khác nhau nằm trong khoảng từ L .. U.  

\subsubsection{   Giới hạn  }


\begin{verbatim}
1 ≤ L ≤ U ≤ N ≤ 2^20
1 ≤ X ≤ 2^31 - 1
\end{verbatim}



\subsubsection{   Ví dụ  }
\begin{verbatim}
Dữ liệu:
4 1 2
231
19
7
19


Kết quả:
8
\end{verbatim}