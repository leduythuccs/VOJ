
%  Always please tidy HTML code generated by MS Word and similar editors. Thank you, the admins. 


Let's 

  consider the 32 bit representation of all integers i from m up to n 

  inclusive (m ≤ i ≤ n; m × n ≥ 0, -2\textasciicircum31 ≤ m ≤ n ≤ 2\textasciicircum31-1). Note that a 

  negative number is represented in 32 bit Additional Code. That is the 32 bit 

  sequence, the binary sum of which and the 32 bit representation of the 

corresponding positive number is 2\textasciicircum32 (\texttt{1 

0000 0000 0000 0000 0000 0000 0000 0000} 

  in binary). 

For 

  example, the 32 bit representation of 6 is \texttt{

  0000 0000 0000 0000 0000 0000 0000 0110}



  and 

  the 32 bit representation of -6 is \texttt{1111 1111 

  1111 1111 1111 1111 1111 1010}



  because

\texttt{   

  0000 0000 0000 0000 0000 0000 0000 0110} 

  (6) 
\\

  + 
\\\texttt{   

  1111 1111 1111 1111 1111 1111 1111 1010} 

  (-6) 
\\ 

  -------------------------------------------------
\\\texttt{= 1 0000 0000 0000 0000 0000 0000 0000 0000} 

  (2\textasciicircum32) 

Let's 

  sort the 32 bit representations of these numbers in ireasing order of the 

  number of bit 1. If two 32 bit representations that have the same number of 

  bit 1, they are sorted in lexicographical order. 

For 

  example, with m = 0 and n = 5, the result of the sortg will be 

  
\begin{tabular}\hline 




    No. & 



    Decimal number & 



    Binary 32 bit representation \\ 
\hline




    1 & 



    0 & 

\texttt{

    0000 0000 0000 0000 0000 0000 0000 0000} \\ 
\hline




    2 & 



    1 & 

\texttt{

    0000 0000 0000 0000 0000 0000 0000 0001} \\ 
\hline




    3 & 



    2 & 

\texttt{

    0000 0000 0000 0000 0000 0000 0000 0010} \\ 
\hline




    4 & 



    4 & 

\texttt{

    0000 0000 0000 0000 0000 0000 0000 0100} \\ 
\hline




    5 & 



    3 & 

\texttt{

    0000 0000 0000 0000 0000 0000 0000 0011} \\ 
\hline




    6 & 



    5 & 

\texttt{

    0000 0000 0000 0000 0000 0000 0000 0101} \\ 
\hline

\end{tabular}



  with 

  m = -5 and n = -2, the result of the sorting will be 
\begin{tabular}\hline 




    No. & 



    Decimal number & 



    Binary 32 bit representation \\ 
\hline




    1 & 



    -4 & 

\texttt{

    1111 1111 1111 1111 1111 1111 1111 1100} \\ 
\hline




    2 & 



    -5 & 

\texttt{

    1111 1111 1111 1111 1111 1111 1111 1011} \\ 
\hline




    3 & 



    -3 & 

\texttt{

    1111 1111 1111 1111 1111 1111 1111 1101} \\ 
\hline




    4 & 



    -2 & 

\texttt{

    1111 1111 1111 1111 1111 1111 1111 1110} \\ 
\hline

\end{tabular}





  Given m, n 

  and k (1 ≤ k ≤ min\{n − m + 1, 2 147 473 547\}), 

  your task is to write a program to find a number corresponding to k-th 

  representation in the sorted sequence. 

\subsubsection{Input}



  The input 

  consists of several data sets. The first line of the input file contains the 

  number of data sets which is a positive integer and is not bigger than 1000. 

  The following lines describe the data sets. 



  For each 

  data set, the only line contains 3 integers m, n 

  and k separated by space. 

\subsubsection{Output}



  For each 

  data set, write in one line the k-th number of the sorted numbers. 

  

\subsubsection{Example}


\begin{verbatim}
\textbf{Sample input:}

2

0 5 3

-5 -2 2



\textbf{Sample output:}

2

-5 

\end{verbatim}
