



   Fibonacci sequence is defined as follow: $F_{1}$   = 1, $F_{2}$   = 2, $F_{i}$   = $F_{i-1}$   + $F_{i-2}$   (i $>$ 2).  

   Each natural number X can be expressed by the maximum numbers that are less than or equal to X in Fibonacci sequence: X = $a_{1}$   $xF_{1}$   + $a_{2}$   $xF_{2}$   + … Therefore, in Fibonacci system, X is known as: $a_{n}$   $a_{n-1}$   …$a_{1}$   . For example, 1 = $1_{F}$   , 2 = $10_{F}$   , etc. If we write all natural numbers successively in Fibonacci system, we will obtain a sequence like this: 1\_1\_0… This is called “Fibonacci bit sequence of natural numbers”.  

   Your task is counting the numbers of times that bit 1 appears in the first N bits of this sequence.  

\subsubsection{   Input  }

   Line 1: An integer N (1  $\le$  N  $\le$  $10^{15}$   )  

\subsubsection{   Output  }

   Line 1: An integer K is the result  

\subsubsection{   Example  }
\begin{verbatim}
\textbf{Input:}
\\2
\\
\\\textbf{Output:}
\\2
\\\end{verbatim}
