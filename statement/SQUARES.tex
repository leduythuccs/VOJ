



   Gọi R là một hình chữ nhật với các cạnh là số nguyên. Hình chữ nhật được chia thành các hình vuông đơn vị. Xét một đường chéo, ta biểu thị f(R) là số lượng hình vuông đơn vị có điểm chung trong với nó. Ví dụ, nếu 2 cạnh của R là 2 và 4 thì f(R) = 4. Viết chương trình squ để tính số lượng các hình chữ nhật R khác nhau mà f(R ) = N. Hai hình chữ nhật với 2 cạnh a×b và b×a không được coi là khác nhau.  

\subsubsection{   Input  }

   Trên một dòng duy nhất của standard input ghi số nguyên N (0 $<$ N $<$ 10^6).  

\subsubsection{   Output  }

   Một dòng duy nhất của standard output ghi một số nguyên – là số lượng hình chữ nhật tìm được.  

\subsubsection{   Example  }
\begin{verbatim}
Input:
4

Output:
4
\end{verbatim}
