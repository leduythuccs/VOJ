

Cho 2 xâu ký tự \emph{ X } = \emph{ $x_{1}$ , $x_{2}$ , .., $x_{m}$} và \emph{ Y } = \emph{ $y_{1}$ , $y_{2}$ , ..., $y_{n}$} . Cần xây dựng xâu \emph{ T } = \emph{ $t_{1}$ , $t_{2}$ , $t_{3}$ , ..,$t_{n+m}$}$_ . $ gồm tất cả các ký tự trong xâu X và tất cả các ký tự trong xâu \emph{ Y } , sao cho các ký tự trong \emph{ X } xuất hiện trong \emph{ T } theo thứ tự xuất hiện trong \emph{ X } và các ký tự trong \emph{ Y } xuất hiện trong T theo đúng thứ tự xuất hiện trong \emph{ Y } , đồng thời với tổng chi phí trộn là nhỏ nhất. Tổng chi phí trộn hai xâu \emph{ X } và \emph{ Y } để thu được xâu \emph{ T } được tính bởi công thức c( \emph{ T } ) = sum(c( \emph{ $t_{k}$ , $t_{k+1}$}$_$ )) với k = 1, 2, ..,  n+m-1; trong đó, các chi phí c( \emph{ $t_{k}$ , $t_{k+1}$} ) được tính như sau:
\begin{itemize}
	\item Nếu hai ký tự liên tiếp \emph{ $t_{k}$ , $t_{k+1}$} được lấy từ cùng một xâu X hoặc Y thì c( \emph{ $t_{k}$ , $t_{k+1}$} ) = 0
	\item Nếu hai ký tự liên tiếp \emph{ $t_{k}$ , $t_{k+}$ 1 } là $x_{i}$ $y_{j}$ thì chi phí phải trả là c( \emph{ $x_{i}$ , $y_{j}$} ). Nếu hai ký tự liên tiếp \emph{ $t_{k}$ , $t_{k+1}$}$_$ là \emph{ $y_{j}$ , $x_{i}$}$_$ thì chi phí phải trả là c( \emph{ $y_{j}$ , $x_{i}$} ) = c( \emph{ $x_{i}$ , $y_{j}$} )
\end{itemize}

\subsubsection{Input}

Dòng đầu tiên chứa Q là số lượng bộ dữ liệu. tiếp đến là Q nhóm dòng, mỗi nhóm cho thong tin về 1 bộ dữ liệu theo khuôn dạng sau:
\begin{itemize}
	\item Dòng thứ nhất chứa 2 số nguyên duong m, n (m, n  $\le$  1000);
	\item Dòng thứ \emph{ i } trong m dòng tiếp theo chứa n số nguyên dương, mỗi số không vượt quá 10^9: c( \emph{ $x_{i}$ , $y_{1}$} ), c( \emph{ $x_{i}$ , $y_{2}$} ), …, c( \emph{ $x_{i}$ , $y_{n}$} ), \emph{ i } = 1, 2,…, m.
\end{itemize}

Ràng buộc : Có 60\% số test ứng với 60\% số điểm của bài đó có m, n  $\le$  10

\subsubsection{Output}
\begin{itemize}
	\item Gồm Q dòng, mỗi dòng chứa một số nguyên là tổng chi phí theo cách xây dựng xâu \emph{ T } tìm được tương ứng với bộ dữ liệu vào.
\end{itemize}

\subsubsection{Example}
\begin{verbatim}
\textbf{Input:}
1
2 3
3 2 30
15 5 4
\textbf{Output:}
6\end{verbatim}
