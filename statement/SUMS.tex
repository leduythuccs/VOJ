



   Cho tập số nguyên A gồm n phần tử, A=\{a1, a2,..., an\}. Số k được gọi là phụ thuộc vào tập A, nếu k được tạo thành bằng cách cộng các phần tử của tập A(mỗi phần tử có thể cộng nhiều lần).  

   Ví dụ  cho A=\{2,5,7\}.  Các số như 2, 4(2+2), 12(5+7 hoặc 2+2+2+2+2) được gọi là phụ thuộc vào tập A. Số 0 cũng gọi là phụ thuộc vào tập A.  

\subsubsection{   Yêu cầu:  }

   Cho một dãy B, hãy kiểm tra xem bi có phải là số phụ thuộc vào tập A hay không .  

\subsubsection{   Dữ liệu:  }
\begin{itemize}
	\item     Dòng đầu tiên chứa số nguyên n (1 ≤ n ≤ 5000).   
	\item     N dòng tiếp theo chứa các phân tử của tập A, a1 $<$ a2 $<$ ... $<$ an  (1  ≤ ai  ≤ 50000 ).   
	\item     Dòng thứ N+2 chứa số nguyên m (1 ≤ m ≤ 10000 ).   
	\item     M dòng tiếp theo chứa dãy số nguyên b1, b2, ..., bm (0 ≤ bi ≤ 1000000000 ).   
\end{itemize}

\subsubsection{   Kết quả:  }

   Gồm m dòng, dòng thứ i ghi ra TAK nếu bi là số phụ thuôc vào tập A và NIE nếu không phải là số phụ thuộc.  

\subsubsection{   Ví dụ:  }
\begin{verbatim}
Dữ liệu :
3
2
5
7
6
0
1
4
12
3
2

Kết quả :
TAK
NIE
TAK
TAK
NIE
TAK

\\\end{verbatim}