



   Cho một cây N đỉnh, trong đó đỉnh i có giá trị là Vi. Cho một số nguyên S. Gốc của cây là đỉnh 1. Đếm số đường đi từ một đỉnh u đến một đỉnh v nào đó, với điều kiện u phải nằm trên đường đi từ v đến gốc, sao cho tổng giá trị của các nút trên đường đi bằng S.  

\subsubsection{   Dữ liệu  }
\begin{itemize}
	\item     Dòng đầu tiên chứa hai số N và S   
	\item     Dòng thứ i trong số N dòng tiếp theo chứa hai số Pi, Vi là đỉnh cha của đỉnh i và giá trị của đỉnh i. Ta quy ước P1 = 0.   
\end{itemize}

\subsubsection{   Giới hạn  }
\begin{itemize}
	\item     1  $\le$  N  $\le$  1000000   
	\item     Mọi tổng giá trị của các nút trên đường đi từ u đến v, trong đó u nằm trên đường đi từ v đến gốc, luôn nằm trong phạm vi số nguyên 32 bit có dấu.   
\end{itemize}

\subsubsection{   Kết quả  }

   In ra một số duy nhất là số đường đi tìm được.  

\subsubsection{   Ví dụ  }
\begin{verbatim}
\textbf{Dữ liệu}
5 3
0 1
1 2
2 1
1 -2
4 5

\textbf{Kết quả}
3
\end{verbatim}

\subsubsection{   Giải thích  }

   Có 3 đường đi là 1-2, 2-3, 4-5  
