

Cho một bảng kích thước M * N. George có tất cả K màu khác nhau. Cậu muốn tô màu bảng này sao cho:
\begin{itemize}
	\item 

Mỗi màu được sử dụng ít nhất một lần ở dòng trên cùng của bảng
	\item 

2 ô cùng màu phải thuộc cùng 1 thành phần liên thông
\end{itemize}

2 cách tô màu được gọi là khác nhau nếu dòng trên cùng của 2 bảng là khác nhau. Đếm số cách tô màu có thể, lấy modulo 10^9 + 7


\\ 

Input:
\begin{itemize}
	\item 

Dòng 1: 3 số nguyên M, N, K (0 $<$ M,N,K  $\le$  300)
\end{itemize}

Output:
\begin{itemize}
	\item 

1 dòng đưa ra số cách tô màu có thể.
\end{itemize}
\begin{verbatim}
Example:

\textbf{Input}
4 1 2

\textbf{Output}
6

\textbf{Input}
4 3 2

\textbf{Output}
12

\textbf{Input}
4 4 10

\textbf{Output}
0

\textbf{Input}
14 28 14

\textbf{Output}
178290591\end{verbatim}


\\ 

Trong test ví dụ số 1, 6 cách tô màu là "AAAB", "AABB", "ABBB", "BBBA", "BBAA" và "BAAA".

Trong test ví dụ số 2, có thêm 6 cách tô màu dòng đầu tiên. Bảng sau đây mô tả một số cách tô hợp lệ:

ABAA AABA ABBA BABB BBAB BAAB

ABBA AAAA ABAA BABB BAAB BBBB

AAAA AAAA AAAA BBBB BBBB BBBB


\\ 
