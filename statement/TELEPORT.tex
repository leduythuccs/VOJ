

 

Pháp sư vĩ đại Byter đã phù phép tạo nên 2 hòn đảo trên biển Baltic : đảoBornholm và đảo Gotland . Ở mỗi đảo thì ông cũng tạo nên một vài cổng dịch chuyển tức thời ( CDCTT ) . Mỗi CDCTT sẽ là 1 trong 2 loại hình sau :
\begin{enumerate}
	\item Cổng Đến : Người ta sẽ được di chuyển tới cổng này .
	\item Cổng Đi : Khi bước vào cổng này người ta sẽ được đưa tới 1 Cổng Đến duy nhất xác định nằm ở hòn đảo kia .
\end{enumerate}


\\Một lần Byter đã giao cho các học trò của mình bài toán như sau : Cho biết số lượng CDCTT ở mỗi hòn đảo . Các học trò phải xác định xem cổng nào sẽ là Cổng Đến , cổng nào sẽ là Cổng Đi sao cho thoả mãn yêu cầu sau : Giả sử cổng i được đặt là Cổng Đến thì có ít nhất 1 Cổng Đi sẽ đưa người được dịch chuyển tới cổng i này và ngược lại , cổng i được đặt là Cổng Đi thì cổng mà nó gửi người đến phải được đặt là Cổng Đến .

\subsubsection{Input}

Dòng 1 : 2 số nguyên N và M ( 1 ≤ N , M ≤ 50000 ) tương ứng là số CDCTT ở trên đảo Bornhom và Gotland .
\\Dòng thứ 2 gồm N số nguyên A[1] … A[N] mô tả các CDCTT ở đảo Bornhom : số A[i] cho biết nếu như Cổng thứ i trên đảo Bornhom được đặt là Cổng Đi thì nó sẽ gửi người đến Cổng A[i] trên đảo Gotland . ( 1 ≤ A[i] ≤ M ) .
\\Dòng thứ 3 gồm M số nguyên B[1] … B[M] mô tả các CDCTT ở đảo Gotland : số B[i] cho biết nếu như Cổng thứ i trên đảo Gotland được đặt là Cổng Đi thì nó sẽ gửi người đến Cổng B[i] trên đảo Bornhom . ( 1 ≤ B[i] ≤ N ) .

\subsubsection{Output}

Dòng 1 : N số nguyên C[1] … C[N] ghi cách nhau 1 dấu khoảng trắng , C[i] = 1 nếu cổng i trên đảo Bornhom là Cổng Đi và = 0 nếu cổng i là Cổng Đến .
\\Dòng 2 : M số nguyên D[1] … D[M] ghi cách nhau 1 khoảng trắng, D[i] = 1 nếu cổng i trên đảo Gotland là Cổng Đi và = 0 nếu cổng i là Cổng Đến .

\subsubsection{Example}
\begin{verbatim}
Input:
4 5
3 5 2 5
4 4 4 1 3


Output:
0 1 1 0
1 0 1 1 0

\end{verbatim}
