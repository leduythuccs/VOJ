

Cho một bảng N*M chỉ gồm các số 0 và 1. Có Q mệnh lệnh có dạng: (i, j) có nghĩa là thay đổi số trong ô (i, j) nếu đang từ 0 sẽ chuyển thành 1, còn nếu đang là 1 thì sẽ chuyển thành 0. Sau mỗi lần thay đổi, bạn phải tìm số các số 1 \textbf{liên tiếp và}\textbf{lớn nhất} trong N \textbf{hàng} của bảng.

\subsubsection{Input}

Dòng 1: 3 số nguyên n, m and q (1 ≤ n, m ≤ 500 và 1 ≤ q ≤ 5000)

Dòng thứ n tiếp theo, mỗi dòng gồm m số nguyên cách nhau bởi dấu cách. Mỗi số được gán là 0 hay 1 

Dòng q tiếp theo, mỗi dòng gồm 2 số nguyên i và j (1 ≤ i ≤ n và 1 ≤ j ≤ m), chỉ số hàng và chỉ số cột của ô phải thay đổi trạng thái.

\subsubsection{Output}

Gồm Q dòng, gồm 1 số chứa kết quả sau mỗi lần thay đổi

\subsubsection{Example}
\begin{verbatim}
\textbf{Input:}
\begin{verbatim}
5 4 5

0 1 1 0

1 0 0 1

0 1 1 0

1 0 0 1

0 0 0 0

1 1

1 4

1 1

4 2

4 3\end{verbatim}\textbf{
\begin{verbatim}
\textbf{Output:}
\begin{verbatim}
3





3

4\end{verbatim}

Giải thích:

Lần 1: Thay đổi ô (1,1) từ 0 thành 1

\textbf{1 1 1} 0
\\1 0 0 1
\\0 1 1 0
\\1 0 0 1
\\0 0 0 0
\\Số các số 1 liên tiếp nhiều nhất là 3 ( ở hàng đầu tiên )

 \end{verbatim}}\end{verbatim}
