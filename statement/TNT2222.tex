

 Đại ca Thắng có 1 cái bàn chia thành 100 hàng dọc và 100 hàng ngang (ma trận 100*100) hàng ngang được đánh số từ 1 đến 100 từ đáy lên đỉnh, hàng dọc được đánh số từ 1-100 từ trái sang phảiThắng chọn n hình chữ nhật có cạnh dọc theo viền của bàn (1 số hình chữ nhật có thể lấy nhiều lần). Sau đó, cho trong ô của bảng anh tính số hình chữ nhật mà chứa ô đó rồi đánh số vào đó. Bây giờ anh muốn tính tổng tất cả giá trị của các ô mà anh đã đánh số. Mà bàn to vl nên anh nhờ các em giải hộ phát :)))

Đại ca Thắng có 1 cái bàn chia thành 100 hàng dọc và 100 hàng ngang (ma trận 100*100) 

hàng ngang được đánh số từ 1 đến 100 từ đáy lên đỉnh, hàng dọc được đánh số từ 1-100 

từ trái sang phải

Thắng chọn n hình chữ nhật có cạnh dọc theo viền của bàn (1 số hình chữ nhật có thể 

lấy nhiều lần). Sau đó, tăng tất cả các ô trong hình chữ nhật đó lên 1 đơn vị. Bây giờ anh muốn tính tổng tất cả giá trị của các ô mà anh đã đánh số. 

Mà bàn to vl nên anh nhờ các em giải hộ phát :)))

 

\subsubsection{Input}

-Dòng đầu là số hình chữ nhật.

- Mỗi dòng tiếp theo chứa 4 số nguyên x1 y1 x2 y2 (1$<$=x1$<$=x2$<$=100,1$<$=y1$<$=y2$<$=100)

x1, y1 là số ô cột và hàng dưới của HCN, x2, y2 là số ô cột và hàng trên của HCN.

\subsubsection{Output}

1 dòng duy nhất in tổng giá trị các ô của bàn.

\subsubsection{Example}
\begin{verbatim}
\textbf{Input:}
\begin{verbatim}
2
1 1 2 3
2 2 3 3\end{verbatim}\textbf{Output:}
\begin{verbatim}
10\end{verbatim}
\begin{verbatim}
\textbf{Input:}
\begin{verbatim}

\begin{verbatim}
2
1 1 3 3
1 1 3 3\end{verbatim}\end{verbatim}\textbf{Output:}
\begin{verbatim}

\begin{verbatim}
18\end{verbatim}\end{verbatim}\textbf{Giải thích:}Test1:000000000Tăng các ô trong hình chữ nhật (1,1),(2,3) lên 1 đơn vị:\textbf{11}0\textbf{11}0\textbf{11}0Tăng các ô trong hình chữ nhật (2,2),(3,3) lên 1 đơn vị:1\textbf{21}1\textbf{21}110=$>$ tổng là 10Test 2:222222222=$>$ tổng là 18\textbf{
}\end{verbatim}\end{verbatim}
