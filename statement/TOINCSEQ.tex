

Cho một dãy số $a_{1}$ , $a_{2}$ , …, $a_{n}$ . Bạn được thực hiện các phép biến đổi trên dãy này. Ở mỗi phép biến đổi bạn có thể chọn một giá trị bất kỳ, sau đó tăng hoặc giảm \textbf{ tất cả } các phần tử mang giá trị đó 1 đơn vị.

Ví dụ, dãy số \emph{ 1 9 9 2 2 } sẽ trở thành dãy \emph{ 1 9 9 3 3 } nếu bạn tăng tất cả các phần tử mang giá trị 2 lên 1 đơn vị.

\textbf{Yêu cầu: } Hãy đếm số phép biến đổi ít nhất để dãy số đã cho là \textbf{ dãy không giảm } .

\subsubsection{Input}
\begin{itemize}
	\item Dòng đầu ghi số phần tử của dãy N.
	\item Dòng sau ghi N số tự nhiên thể hiện dãy số.
\end{itemize}

\subsubsection{Output}
\begin{itemize}
	\item Số phép biến đổi ít nhất cần thực hiện.
\end{itemize}

\subsubsection{Example}
\begin{verbatim}
\textbf{Input:}
5
1 9 9 2 2
\textbf{Output:}
7\end{verbatim}

\subsubsection{Giới hạn}
\begin{itemize}
	\item 1 ≤ N ≤ $10^{5}$ .
	\item Các phần tử của dãy số là số tự nhiên không vượt quá $10^{9}$ .
	\item Trong 50\% số test, N không vượt quá $10^{3}$.
\end{itemize}
