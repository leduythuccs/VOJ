

Công ty du lịch X có dự án tổ chức các hành trình du lịch trong vùng lãnh thổ gồm n điểm du lịch trọng điểm, được đánh số từ 1 tới n. Hệ thống giao thông trong vùng gồm m tuyến đường một chiều khác nhau, tuyến đường thứ j ( j = 1, 2, 3, …, m) cho phép đi từ địa diểm $u_{j}$ đến dịa diểm $v_{j}$ với chi phí đi lại là một số nguyên dương c($u_{j}$ , $v_{j}$ ). Vấn đề đặt ra cho công ty là xây dựng các hành trình du lịch cho mỗi điểm du lịch. Một hành trình du lịch cho địa điểm du lịch i phải được xây dựng sao cho xuất phát từ địa điểm i đi qua một số địa điểm khác rồi quay lại địa điểm xấu phát i với tổng chi phí (được tính như là tổng chi phí của các tuyến đường mà hành trình đi qua) nhỏ nhất.

\subsubsection{Input}

Dòng đầu tiên số T là số lượng bộ dữ liệu. tiếp đến là T nhóm dòng, mỗi dòng cho thông tin về một bộ dữ liệu theo khuôn dạng sau :
\begin{itemize}
	\item Dòng thứ nhất chứa 2 số nguyên dương n, m
	\item Dòng thứ j trong số m dòng tiếp theo chứa ba số nguyên duong $u_{j}$ , $v_{j}$ , c($u_{j}$ , $v_{j}$ ) cho biết thông tin về tuyến đường thứ j. Giả thiết là $u_{j}$ ≠ $v_{j}$ ; c($u_{j}$ , $v_{j}$ ) $<$ 10^6; j = 1, 2, …, m
\end{itemize}

\subsubsection{Output}

Gồm T nhóm dòng tương ứng với T bộ test vào, mỗi nhóm dòng gồm n dòng, dòng thứ i ghi chi phí của hành trình du lịch cho địa điểm i . Qui ước: Ghi số -1 trên dòng i nếu không tìm được hành trình du lịch cho địa điểm i thỏa mãn yêu cầu đặt ra

\subsubsection{Example}
\begin{verbatim}
\textbf{Input:}
1
6 8
1 2 4
2 4 2
4 3 3
3 1 4
4 1 5
3 5 5
5 3 1
5 6 7

\textbf{Output:}
11
11
6
11
6
-1\end{verbatim}

 

\textbf{Ràng buộc:}
\begin{itemize}
	\item Có 30\% số test tương ứng với 30\% số điểm của bài có n  $\le$  20.
	\item Có 30\% số test tương ứng với 30\% số điểm của bài có 20 $<$ n  $\le$  100, m  $\le$  $10^{4}$.
	\item Có 40\% số test tương ứng với 30\% số điểm của bài có 100 $<$ n  $\le$  $10^{3}$, m  $\le$  $10^{5}$
\end{itemize}
\begin{itemize}
\end{itemize}
