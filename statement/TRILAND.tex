



   Đợt khủng hoảng kinh tế toàn cầu đã tác động mạnh vào nền kinh tế Việt Nam, đáng kể nhất trong ngành bất động sản. Từ sau khủng hoảng, giá nhà đất các thành phố lớn sụt thê thảm. Ngày càng ít người mua nhà. Các khu đất, căn hộ chung cư cao cấp, … ế ẩm đầy rẫy.  

   Chuyện nhà đất giờ ra sao chắc ai cũng biết. Tuy nhiên chỉ vài năm trước thì cơn sốt nhà đất ở Sài Gòn (tức Tp Hồ Chí Minh) vẫn còn rất mạnh. Các tập đoàn nước ngoài đầu tư không tiếc tay vào bất động sản. Trong đó phải kể đến tập đoàn RICH của Hàn Quốc. Ban giám đốc RICH rất thích những khu đất “vàng” : khu đất có hình tam giác cân nhưng không đều (vì họ tin như vậy sẽ đem đến thuận lợi và may mắn trong kinh doanh). Để mua một khu đất, RICH sẽ chọn 3 địa điểm “đẹp” (theo phong thủy Hàn Quốc J) và mua toàn bộ diện tích đất của tam giác tạo thành bởi 3 điểm này.  

   Do ở Sài Gòn có rất nhiều địa điểm đẹp (hòn ngọc Viễn Đông ngày xưa mà lại) nên ban giác đốc RICH muốn biết có bao nhiêu cách chọn một khu đất “vàng”.  

\subsubsection{   Input  }

   Dòng đầu ghi số nguyên dương N (3  $\le$  N  $\le$  1000) là số địa điểm đẹp.  

   N dòng sau, mỗi dòng ghi cặp số nguyên dương $X_{i}$   , $Y_{i}$   là tọa độ của một địa điểm ( 1  $\le$  $X_{i}$   , $Y_{i}$    $\le$  $10^{6}$   ). Không có ba địa điểm nào thẳng hàng.  

\subsubsection{   Output  }

   In ra một số nguyên duy nhất là số khu đất “vàng” có thể chọn.  

\subsubsection{   Example  }
\begin{verbatim}
\textbf{Input:}

5

1 2


2 12 2
\\
\\1 1

\\1000 1000000

\textbf{Output:}
4
\end{verbatim}
