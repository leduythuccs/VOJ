



   Đề bài tương tự bài TRIPOD nhưng có giới hạn lớn hơn.   
\\   Trong một chuyến đi dã ngoại, BB muốn cùng các bạn bắc bếp nấu cơm trên khu đất cắm trại. Trên mặt đất có n hòn đá, không có 2 hòn đá nào có cùng vị trí. Mọi người sẽ chọn ra 3 trong số các hòn đá này để bắc nồi lên. 3 hòn đá phải được chọn sao cho bán kính hình tròn chứa chúng là nhỏ nhất vì nếu có một cái nồi nhỏ bắc được lên 3 hòn đá này thì cái nồi có bán kính lớn hơn cũng có thể bắc được lên 3 hòn đá này. Bạn hãy giúp BB tìm ra 3 hòn đá cần chọn.  

\subsubsection{   Input  }

   Dòng đầu tiên ghi số n là số hòn đá.  

   Trong n dòng sau, dòng thứ u ghi một cặp số nguyên $x_{u}$   , $y_{u}$   là tọa độ của hòn đá thứ u.  

\subsubsection{   Output  }

   Ghi ra bán kính hình tròn chứa 3 hòn đá tìm được ( Chính xác đến 5 chữ số sau dấu phẩy ).  

\subsubsection{   Example  }
\begin{verbatim}
Input:
3
0 0
4 0
0 4

Output:
2.82842

Giới hạn:
3 ≤ N ≤ 100000. 
|$x_{u}$|, |$y_{u}$| ≤ 2100000000

\end{verbatim}
