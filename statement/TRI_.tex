



   You are given K points with positive integer coordinates. You are also given M triangles, each   
\\   of them having one vertex in the origin and the other 2 vertices with non-negative integer   
\\   coordinates.   
\\   You are asked to determine for each triangle whether it has at least one of the K given points   
\\   inside. (None of the K points are on any edge of any triangle.)  


\\   Input   
\\   The first line will contain K and M. The following K lines will   
\\   contain 2 positive integers x y separated by one space that represent the coordinates of   
\\   each point. The next M lines have 4 non-negative integers separated by one space, (x1,y1)   
\\   and (x2, y2), that represent the other 2 vertices of each triangle, except the origin.  


\\   Output   
\\   The output should contain exactly M lines. The k-th line should contain the   
\\   character Y if the k-th triangle (in the order of the input) contains at least one point   
\\   inside it, or N otherwise.  


\\   Constraints   
\\   · 1 ≤ K,M ≤ 100 000   
\\   · 1 ≤ each coordinate of the K points ≤ 10^9   
\\   · 0 ≤ each coordinate of the triangle vertices ≤ 10^9   
\\   · Triangles are not degenerate (they all have nonzero area).  

   SAMPLE 1  

   Input  

   4 3   
\\   1 2   
\\   1 3   
\\   5 1   
\\   5 3   
\\   1 4 3 3   
\\   2 2 4 1   
\\   4 4 6 3  

   Output  

   Y   
\\   N   
\\   Y  

   SAMPLE 2  

   Input  

   4 2   
\\   1 2   
\\   1 3   
\\   5 1   
\\   4 3   
\\   0 2 1 0   
\\   0 3 5 0  

   Output  

   N   
\\   Y  
