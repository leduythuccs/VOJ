



\subsubsection{   Đề bài  }

   Cho hai biến X và Y, ban đầu có giá trị 1. Mỗi bước ta có thể thực hiện một trong hai phép gán X:=X+Y (ký hiệu X) hoặc Y:=X+Y (ký hiệu Y). Cho trước một số r, tìm cách thực hiện ít phép gán nhất sao cho biến X mang giá trị r (biến Y có thể mang giá trị bất kỳ).  

   Nếu có nhiều cách thực hiện, trả về cách mang thứ tự từ điển nhỏ nhất.  

\subsubsection{   Dữ liệu  }
\begin{itemize}
	\item     Mỗi test bắt đầu bằng thẻ "[CASE]", các test cách nhau bởi một dòng trắng. Thẻ "[END]" báo hiệu kết thúc file input.   
	\item     Mỗi test chứa một số nguyên r duy nhất   
\end{itemize}

\subsubsection{   Kết quả  }
\begin{itemize}
	\item     Mỗi test chứa một dòng duy nhất là dãy bao gồm các ký tự X hoặc Y mô tả dãy phép gán.   
\end{itemize}

\subsubsection{   Giới hạn  }
\begin{itemize}
	\item     1  $\le$  R  $\le$  1000000   
\end{itemize}

\subsubsection{   Ví dụ  }
\begin{verbatim}
Dữ liệu
[CASE]
10

[CASE]
3

[CASE]
20

[CASE]
34

[END]

Kết quả
XXYYX
XX
XYYYYXX
XYXYXYX
\end{verbatim}
