



   Một công ty du lịch tổ chức cho 1 đoàn du khách nước ngoài đi du lịch M thành phố ở Việt Nam . Tuy nhiên mỗi du khách lại có 2 yêu cầu . Mỗi yêu cầu có dạng “Không muốn đi thành phố A” hoặc “Muốn đi thành phố A” ( A là chỉ số thành phố mà người đó yêu cầu ) . ( Có thể có trường hợp 2 yêu cầu của khách là “Muốn đi thành phố A” và  Vì các du khách này là người nước ngoài nên rất khó tính , họ muốn ít nhất  1 trong 2 yêu cầu của họ phải được đáp ứng . Bên công ty du lịch đã đau đầu tìm cách chọn ra các thành phố để đưa đoàn du khách đi mà vẫn chưa tìm được cách nào cả . Bạn được yêu cầu giúp công ty du lịch này chọn ra 1 số thành phố để đưa đoàn du khách này đi mà lại vừa thoả mãn được các du khách này .  

\subsubsection{   Input  }

   Dòng 1 : 2 số nguyên N và M ( 1 ≤ N ≤ 20000 , 1 ≤ M ≤ 8000 ) tương ứng là số khách du lịch và số thành phố .   
\\   M dòng tiếp theo gồm 2 số nguyên u , v , -M ≤ u ,v ≤ M ( u $<$$>$ 0 , v $<$$>$ 0 ) mô tả yêu cầu của khách thứ i ( số dương nếu yêu cầu du khách i muốn đi thành phố đó và số âm nếu không muốn đi thành phố đó ).   
\\

\subsubsection{   Output  }

   Dòng 1 : Ghi YES nếu có phương án thoả mãn yêu cầu các du khách và ghi NO trong trường hợp ngược lại .   
\\   Nếu YES thì ghi tiếp theo như sau :   
\\   Dòng 2 : số nguyên dương K là số thành phố được chọn .   
\\   Dòng 3 : Gồm K số nguyên là chỉ số của các thành phố được chọn .  

\subsubsection{   Example  }
\begin{verbatim}
Input:
2 3
-1 -2
1 2

Output:
YES
2
2 3
\end{verbatim}