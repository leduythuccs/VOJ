



   Nông dân John đang ở trên một con thuyền nhỏ và đang tìm kiếm  kho báu ở 1 trong số N (1  $\le$  N  $\le$  100) hòn đảo (đánh số từ 1..N)  ở vùng biển Ca-ri-bò.  

   Bản đồ kho báu cho John biết John cần phải thực hiện 1 hành  trình đi qua đảo A\_1, A\_2, … A\_M (2  $\le$  M  $\le$  10,000), bắt đầu từ  đảo 1 và kết thúc ở đảo N trước khi kho báu biến mất. Anh ta có thể  đến thăm các đảo khác và thăm bao nhiêu lần tùy thích, miễn là hành  trình của ông ta phải chứa dãy A\_1,..A\_M là 1 dãy con (không nhất  thiết phải liên tiếp nhau).  

   John muốn tránh đụng độ cướp biển và biết được mức-độ-bị-cướp  (0  $\le$  mức-độ-bị-cướp  $\le$  100,000) khi đi lại giữa 2 hòn đảo với  nhau. Độ nguy hiểm của hành trình của John sẽ là tổng các mức-độ-bị-cướp  trên các tuyến đường mà John đi qua.  

   Hãy giúp John tìm được 1 hành trình ít nguy hiểm nhất để  có thể lấy được kho báu.  

\subsubsection{   Dữ liệu  }
\begin{itemize}
	\item     Dòng 1: 2 số nguyên cách nhau bởi dấu cách: N và M   
	\item     Dòng 2..M+1: Dòng i+1 mô tả chứa 1 số nguyên là đảo thứ i         mà John cần phải tới: A\_i   
	\item     Dòng M+2..N+M+1: Dòng i+M+1 chứa N số nguyên cách nhau bởi dấu cách         tương ứng là mức-độ-bị-cướp trên tuyến đường đi giữa đảo i và          đảo 1, 2,…N; đảm bảo số nguyên thứ i luôn là số 0.   
\end{itemize}

\subsubsection{   Kết quả  }
\begin{itemize}
	\item     Dòng 1: Độ nguy hiểm nhỏ nhất của hành trình của John.   
\end{itemize}

\subsubsection{   Ví dụ  }
\begin{verbatim}
Dữ liệu
3 4
1
2
1
3
0 5 1
5 0 2
1 2 0

Giải thích:
Có 3 hòn đảo và bản đồ kho báu yêu cầu John phải thực hiện 1 hành 
trình tới các đảo như sau: từ đảo 1 tới đảo 2, quay lại đảo 1 và cuối 
cùng là tới đảo 3. Mức-độ-bị-cướp trên các tuyến đường đã được 
cho: (1, 2); (2, 3); (3, 1) có độ lớn tương ứng là 5, 2 và 1.

Kết quả
7

Giải thích:
Hành trình có độ nguy hiểm nhỏ nhất là 7. John sẽ đi như sau: 
1, 3, 2, 3, 1, and 3. Yêu cầu của bản đồ là phải chứa dãy
(1, 2, 1, và 3) và hành trình này thỏa mãn yêu cầu. Chúng ta sẽ tránh đi 
trên đường nối giữa 2 đảo 1 và 2 vì nó có mức-độ-bị-cướp lớn.
\end{verbatim}
