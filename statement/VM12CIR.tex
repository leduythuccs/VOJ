

 

Kết thúc phần 3 của bộ phim bom tấn Madagascar, sư tử Marty cùng các bạn của mình đã gia nhập nhóm xiếc thú Zaragoza. Có lẽ phần 4 của bộ phim sẽ lại xoay quanh những nhân vật trong gánh xiếc cùng những màn trình diễn hấp dẫn.Thế nhưng, các nhà đạo diễn luôn biết cách đem đến cho khán giả những điều bất ngờ. Phần tiếp theo sẽ bắt đầu bằng một thay đổi trong suy nghĩ của các con vật, chúng nhận ra chỗ thích hợp nhất với mình là rừng và quyết định quay trở về.


\includegraphics{http://3.bp.blogspot.com/-6I1FZg_iXak/TWOe9N-ZJgI/AAAAAAAAAAY/LykqLX1_SyI/s400/Madagascar%2B4%2BMovie.jpg}

Chắc chắn các con vật sẽ lại gặp những người bạn mới, có những chuyến phiêu lưu mới. Tuy vậy trong kích thước giới hạn của một bài tập, chúng ta chỉ cần quan tâm đến chi tiết sau đây: Gần đến ngày sinh nhật Marty, Alex Melman và Gloria quyết định mang đến một món quà đặc biệt cho chú ngựa vằn. Đêm trước ngày sinh nhật Marty, 3 người bạn sẽ sơn các thân cây trong rừng thành màu hồng. Như vậy khi tỉnh dậy, Marty có thể nghĩ mình vẫn đang mơ và làm một vài điều kì quặc. Do chỉ có một đêm để chuẩn bị, Alex nghĩ ra rằng mình chỉ cần sơn những phần thân cây mà Marty có thể nhìn thấy, rõ ràng sẽ giảm đi đáng kể khối lượng công việc. Bây giờ chỉ còn một vấn đề nhỏ dành cho bạn: giúp các con vật tính lượng sơn cần dùng.

Xét trên trục tọa độ Oxy với gốc O là chỗ Marty ngủ. Các thân cây có thể coi là các hình trụ vuông góc mới mặt đất. Marty sẽ nhìn thấy một điểm A trên thân cây nếu như đoạn thẳng nối A với O không bị bị một thân cây khác che khuất (nói cách khác đoạn AO không có điểm chung với các thân cây khác). Biết rằng ở mỗi cây, các con vật sẽ sơn ở những điểm độ cao không quá 1 đơn vị, các cây đều có độ cao lớn hơn hơn 1 đơn vị. Bạn hãy tính diện tích cần sơn, biết rằng hai cây khác nhau không có điểm chung, và vị trí Marty ngủ không nằm trong hoặc tiếp xúc với cây nào.

\subsubsection{Input}
\begin{itemize}
	\item Dòng 1 ghi số tự nhiên N (1  $\le$  N  $\le$  10,000) là số cây trong rừng.
	\item N dòng tiếp theo, mỗi dòng ghi 3 số nguyên x, y, z (|x|, |y|  $\le$  2000, 1  $\le$  z  $\le$  10) thể hiện các đáy của hình trụ mô tả cây có tọa đô(x, y), bán kính z.
	\item Không có hai đáy nào có điểm chung hoặc chứa điểm (0, 0).
\end{itemize}

\subsubsection{Output}

Ghi ra một số thực thể hiện kết quả của bài toán.

Kết quả của bạn được tính là đúng nếu sai số so với đáp án của ban tổ chức không quá 0.01

\subsubsection{Example}
\begin{verbatim}
\textbf{Input:}
3
0 6 3
4 0 2
-4 0 2
\textbf{Output:}
14.66\end{verbatim}
