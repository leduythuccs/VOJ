

Tom và Jerry rất thích uống cafe. Mỗi ngày, Tom uống cafe buổi sáng còn Jerry uống buổi chiều. Để cafe được ngon nhất, trước mỗi lần uống cafe mới được pha. Trong nhà bếp có N bao cafe, mỗi bao thuộc một loại khác nhau. Bao thứ i có $S_{i}$ gam cà phê với liều lượng $A_{i}$ . Mỗi lần pha Tom sẽ chọn một loại cafe i, lấy ra K gam bột cafe (K là số nguyên dương và không được vượt quá liều lượng cho phép $A_{i}$ nếu không sẽ gây tác dụng phụ không lường trước được). Để bảo đảm chất lượng không được trộn cafe khác loại với nhau.

Nhà bếp sắp hết cafe. Ai không tìm thấy cafe để pha (tức là đến lượt mình uống mà nhà bếp không còn cafe) sẽ phải đi mua. Vì rất lười nên Tom có thể nhịn chịu uống ít cafe một chút (nhưng vẫn uống!), chứ nhất định không chịu đi mua. Dĩ nhiên Jerry cũng không muốn phải đi mua cafe nên sẽ tìm cách bắt Tom phải mua.

Hãy cho biết xem Tom có thể đùn đẩy việc đi mua cafe cho Jerry được không. Vì Tom uống vào buổi sáng nên có thể xem là Tom uống trước Jerry.

\subsubsection{Input}

Dòng đầu tiên là T - số testcase. Trong mỗi testcase :
\begin{itemize}
	\item Dòng thứ nhất N là số bao cafe
	\item Dòng thứ 2 gồm N số nguyên dương $S_{i}$ là trữ lượng cafe còn trong bao i
	\item Dòng thứ 3 gồm N số nguyên dương $A_{i}$ là lừu lượng cafe của bao i
\end{itemize}

\subsubsection{Giới hạn}
\begin{itemize}
	\item N  $\le$  $10^{5}$
	\item 1  $\le$  $A_{i}$ , $S_{i}$  $\le$  $10^{9}$
	\item Subtask 1 (30\%) : N=1; $S_{1}$  $\le$  $10^{5}$ ; $A_{1}$  $\le$  200
	\item Subtask 2 (20\%) : N=1; $S_{1}$ , $A_{1}$  $\le$  $10^{9}$
	\item Subtask 3 (50\%) : N  $\le$  $10^{5}$ ; 1  $\le$  $A_{i}$ , $S_{i}$  $\le$  $10^{9}$
\end{itemize}

\subsubsection{Output}

Với mỗi testcase in ra Tom nếu Tom phải đi mua cafe, ngược lại in ra Jerry.

\subsubsection{Chấm bài}

Bài của bạn sẽ được chấm trên thang điểm 100. Điểm mà bạn nhận được sẽ tương ứng với \% test mà bạn giải đúng.

Trong quá trình thi, bài của bạn sẽ chỉ được chấm với 2 test ví dụ.

Khi vòng thi kết thúc, bài của bạn sẽ được chấm với bộ test đầy đủ.

\subsubsection{Example}
\begin{verbatim}
\textbf{Input 1:}
1

1
28
6
\textbf{Output 1:}
Tom

\textbf{Input 2:}
2

1
20
6

2
8 18
3 5
\textbf{Output 2:}
Jerry
Tom\end{verbatim}
