



   RR đang trọ trên đảo Pirate. Do sống trên đảo quá buồn chán nên RR có thói quen nhìn lên bầu trời, từ đó lại sinh sở thích về thiên văn. Thế là RR bèn đầu tư cho mình một đài thiên văn hoành tráng trên cây cọ to nhất của hòn đảo. Nhờ nghiên cứu thiên văn nên RR bớt buồn đời...  

   Nhật thực là một hiện tượng thiên văn kinh điển mà tất cả những người yêu thích bộ môn này đều từng nghiên cứu qua. Nhưng vì quá quen thuộc nên RR lại muốn tìm một lối đi riêng hòng kiếm một giải Nobel thiên văn học cho riêng mình. Theo suy luận của RR, nếu mặt trăng có thể che mặt trời, thì sao Kim cũng có thể. Và RR bắt đầu tiến hành nghiên cứu hiện tượng nhật thực sao Kim. RR thuê chân sai vặt giongto35 đi thu thập dữ liệu. Do tính tình vụng về bẩm sinh nên cứ mỗi lần đo giongto35 lại thu được những bộ dữ liệu khác nhau. RR thường mất hàng giờ cho mỗi phép toán cộng trừ nhân chia nên khi gặp đống dữ liệu trên, RR bắt đầu suy sụp và lại buồn đời.  

   RR cần sự giúp đỡ từ các bạn. Sử dụng các dữ liệu này, các bạn hãy giúp RR xác định thời điểm đầu tiên xảy ra nhật thực để anh ấy vui vẻ trở lại nhé.  

   Nhật thực sao Kim xảy ra khi sao Kim, trái đất và mặt trời thẳng hàng, đồng thời sao Kim nằm giữa trái đất và mặt trời.  

   Như thực tế, sao Kim luôn ở gần mặt trời hơn trái đất.  

   Như thực tế, sao Kim và trái đất luôn quay ngược chiều kim đồng hồ.  

\textbf{    Không như thực tế, sao Kim và trái đất có cùng quĩ đạo là đường tròn.   }

   Ta quy ước mặt trời sẽ nằm tại tọa độ (0, 0).  

\subsubsection{   Input  }
\begin{itemize}
	\item     Dòng 1: x1, y1, v1 là tọa độ và vận tốc dài của trái đất (đơn vị tọa độ).   
	\item      Dòng 2: x2, y2, v2 là tọa độ và vận tốc dài của sao Kim (đơn vị tọa độ).    
\end{itemize}

\subsubsection{   Output  }
\begin{itemize}
	\item     Thời điểm sớm nhất xảy ra nhật thực tính từ khi vị trí của ba vật thể được xác định. Đáp án của bạn sẽ được chấm đúng nếu chênh lệch giữa kết quả của bạn và đáp án không quá $10^{-6}$    .   
	\item     Nếu không bao giờ xảy ra nhật thực, in ra -1.   
\end{itemize}

\subsubsection{   Giới hạn  }
\begin{itemize}
	\item     0 ≤ |x1|, |x2|, |y1|, |y2| ≤ $10^{6}$
	\item     0 $<$ v1, v2 ≤ $10^{6}$
	\item     Các số trong đề bài đều là số nguyên   
	\item      Dữ liệu đảm bảo khoảng cách giữa sao Kim đến mặt trời bé hơn khoảng cách giữa trái đất đến mặt trời    
\end{itemize}

\subsubsection{   Chấm bài  }

   Bài của bạn sẽ được chấm trên thang điểm 100. Điểm mà bạn nhận được sẽ tương ứng với \% test mà bạn giải đúng.  

   Trong quá trình thi, bài của bạn sẽ chỉ được chấm với 2 test ví dụ có trong đề bài.  

   Khi vòng thi kết thúc, bài của bạn sẽ được chấm với bộ test đầy đủ.  
\begin{itemize}
\end{itemize}

\subsubsection{   Example  }

\textbf{    Input:   }
\begin{verbatim}
4 0 4
0 2 1\end{verbatim}

\textbf{    Output:   }
\begin{verbatim}
3.141593

\end{verbatim}



\textbf{    Input:   }
\begin{verbatim}
0 10 2
0 -5 1\end{verbatim}

\textbf{    Output:   }
\begin{verbatim}
-1\end{verbatim}
