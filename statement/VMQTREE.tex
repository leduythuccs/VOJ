



   RR và TA ngoài việc làm admin ở VNOI còn có nghề tay trái là người mẫu chụp ảnh trang bìa. Nghề này giúp đôi bạn không những nuôi sống được bản thân mà còn dành dụm được một số tiền kha khá. Vào một ngày đẹp trời, hai người quyết định đi tìm cho mình một tổ ấm. Các bạn đừng suy nghĩ lệch lạc, họ chỉ muốn mua hai ngồi nhà ở gần nhau để tiện bề đàm đạo với nhau về nhân tình thế thái, V-Pop, K-Pop và các thứ tương tự...  

   RR và TA tìm mãi mới ra một khu phố rất đặc biệt mà hai người rất hài lòng. Ngoài N ngồi nhà đẹp rạng ngời được đánh số từ 1 đến N thì hệ thống đường xá của khu phố này cũng rất đáng chú ý. Tại đây, một số cặp ngôi nhà được nối với nhau bởi một con phố có độ dài một đơn vị. Điều này thì chẳng có gì lạ, nhưng đặc biệt là số lượng các con phố dù rất ít nhưng lại được thiết kế rất khoa học. Cụ thể là chỉ có N - 1 con phố nhưng cũng đủ để hai ngôi nhà bất kì đều đến được nhau. Mua nhà ở đây thì không sợ bị lạc mà chỉ đường cho bạn bè đến chơi cũng tiện, hạnh phúc là đây rồi! Thế là RR và TA hăm hở đi rút tiền mua nhà.  

   flashmt (vừa trúng vé số) cũng định mua nhà ở khu phố này. Nhưng buồn một nỗi, thầy phong thủy trứ danh technolt (cũng là nghề tay trái thôi) lại phán rằng flashmt và "đôi bạn hoàn hảo" RR - TA không hợp tuổi nên anh không nên ở gần hai người này, nếu không sự nghiệp cũng tiêu tan mà tình duyên cũng ngang trái. Nhưng vì là bạn thân của hai người, flashmt vẫn muốn mua nhà ở đây để tiện bề hú hí với nhau. Thầy phong thủy cảnh báo flashmt rằng độ an toàn của flashmt chính bằng khoảng cách từ nhà anh đến ngôi nhà gần nhất trong số hai ngôi nhà của RR và TA. Muốn được bình yên thì dĩ nhiên độ an toàn phải càng lớn càng tốt!  

   Biết được vị trí hai ngôi nhà của RR và TA, bạn hãy tính xem độ an toàn lớn nhất mà flashmt có thể đạt được là bao nhiêu!  

\subsubsection{   Input  }

   Dòng đầu ghi số nguyên dương T - số bộ test (T ≤ 5).  

   Tiếp theo là T test, mỗi test gồm:  
\begin{itemize}
	\item     Dòng đầu tiên chứa số nguyên dương N.   
	\item     Tiếp theo là N-1 dòng, mỗi dòng gồm 2 số nguyên dương u, v cho biết có cạnh nối giữa đỉnh ngôi nhà u và ngôi nhà v.   
	\item     Dòng tiếp theo chứa số nguyên dương Q.   
	\item     Q dòng tiếp theo, mỗi dòng gồm 2 số nguyên dương u, v mô tả vị trí hai ngôi nhà của RR và TA.   
\end{itemize}

   Các dấu cách và dòng trống thừa có thể xuất hiện ở bất kỳ vị trí nào trong file input.  

\subsubsection{   Giới hạn:  }
\begin{itemize}
	\item     1 ≤ N, Q ≤ 50,000   
	\item     Trong 30\% số test, N và Q ≤ 100   
\end{itemize}

\subsubsection{   Output  }

   Ouput gồm Q dòng. Mỗi dòng in ra một số nguyên dương duy nhất là đáp án của truy vấn tương ứng.  

\subsubsection{   Example  }
\begin{verbatim}
\textbf{Input:}
1
7
1 2
1 3
3 4
3 5
3 6
5 7
7
3 7
5 7
4 6
1 2
1 1
3 5
7 2

\textbf{Output:}
2
3
3
3
3
2
3
\end{verbatim}