

 

Kì thi VM đang dần đi đến những vòng thi cuối cùng. Trong khi các thí sinh hăng say với những bài tập hóc búa, mang đậm tính chất Marathon thì các admin VNOI cũng phải đối mặt với những vấn đề rất nan giải, chẳng hạn như thống kê số lượng thí sinh, số lượng thí sinh giải được từng bài tập...

Trong không khí căng thẳng của cuộc đua, ban tổ chức đã quyết định tặng một món quà đặc biệt cho các bạn. Và món quà đó chính là bài tập này!

Bạn sẽ phải giúp các admin làm một nhiệm vụ sau: Cho dữ liệu mô tả các thí sinh tham gia mỗi một trong \textbf{ K } vòng của cuộc thi VNOI Marathone 20xx, hãy tính xem có bao nhiêu thí sinh tham gia ít nhất một vòng thi.

\subsubsection{Input}
\begin{itemize}
	\item Dòng 1: Ghi số nguyên dương K (1 ≤ K ≤ 5).
	\item K nhóm dòng sau: Mỗi nhóm thể hiện dữ liệu của một vòng thi: Dòng đầu ghi số nguyên N (1 ≤ N ≤ 200) là số lượng thí sinh tham gia vòng thi đó. N dòng sau, mỗi dòng ghi một nick của thí sinh, dưới dạng một xâu khác rỗng, độ dài không quá 20 kí tự (đảm bảo xâu không chứa khoảng trống). Trong mỗi nhóm dòng mô tả một vòng thi, không có tên thí sinh nào được lặp lại hai lần.
\end{itemize}

\subsubsection{Output}
\begin{itemize}
	\item Ghi ra một số nguyên duy nhất là số thí sinh tham gia ít nhất một vòng thi.
\end{itemize}

\subsubsection{Example}
\begin{verbatim}
\textbf{Input:}
3
4
flashmt
ll931110
technolt
tuananhnb93
3
mr_invincible
conankudo
ll931110
3
khanhptnk
hphong
technolt\end{verbatim}
\begin{verbatim}
\textbf{Output:}
8\end{verbatim}