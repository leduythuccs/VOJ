



   Định nghĩa:  

   + "Chữ số" là các kí tự từ "0" đến "9".  

   + "Chữ cái" là các kí tự "a" đến "z" và từ "A" đến "Z" (kí tự Latin in hoa và in thường).  

   + "Số" là dãy liên tiếp các chữ số, có thể chứa tối đa một dấu trừ ("-") ở đầu nếu là số âm và tối đa một dấu thập phân (".") nếu là số thực.  

   + "Từ" là dãy liên tiếp các chữ cái.  



\subsubsection{   Input  }

   Input gồm nhiều dòng:  
\begin{itemize}
	\item     Mỗi dòng là một dãy các từ hoặc các số. Các số và từ được ngăn cách với nhau bởi một hoặc nhiều dấu cách.   
	\item     Input sẽ kết thúc bằng một dòng chứa một dấu chấm hỏi ("?").   
\end{itemize}

\subsubsection{   Output  }

   Output gồm nhiều dòng, mỗi dòng tương ứng với một dòng ở Input (trừ dòng chứa dấu chấm hỏi):  
\begin{itemize}
	\item     Nếu dòng tương ứng ở Input gồm toàn các số, in ra tổng của các số đó (làm tròn đến 6 chữ số thập phân).   
	\item     Nếu dòng tương ứng ở Input gồm toàn các từ, in ra xâu tổng của các từ đó (nối các từ lại với nhau theo thứ tự trong Input).   
	\item     Nếu không phải một trong hai trường trên, ghi ra xâu "Error!" (không bao gồm cặp dấu "").   
\end{itemize}

\subsubsection{   Giới hạn  }
\begin{itemize}
	\item     Input không có quá 20 dòng.   
	\item     Mỗi dòng không có quá 100 kí tự.   
	\item     Mỗi số không có quá 9 chữ số.   
	\item     Mỗi từ không có quá 9 kí tự.   
\end{itemize}

\subsubsection{   Example  }
\begin{verbatim}
\textbf{Input:}





12 23 34
abd   dbs


12 a 4 


?





\textbf{Output:}

69.000000


abddbs


Error! 





\end{verbatim}

Các bạn nhập xuất bằng Standard Input/Output cho cả Pascal và C++. Lưu ý rằng ở cuối Ouput không được có kí tự lạ(ngoại trừ kí tự kết thúc dòng). Ví dụ: Ouput là "69" nhưng các bạn in ra "69 " thì vẫn tính là sai kết quả.

