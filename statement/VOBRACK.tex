



   Ngoài việc chơi với các bảng số 01, Bé còn rất thích chơi với những dãy ngoặc. Hôm nay, mẹ cho Bé N dãy ngoặc. Bé muốn tạo được một dãy ngoặc đúng bằng việc nối một số dãy ngoặc lại với nhau. Bạn hãy giúp bé nhé.  

   Một dãy ngoặc đúng được định nghĩa theo kiểu đệ quy như sau:  
\begin{itemize}
	\item     Dãy rỗng - dãy không gồm ký tự nào - là dãy ngoặc đúng.   
	\item     Nếu X là một dãy ngoặc đúng, thì (X) cũng là một dãy ngoặc đúng. Ví dụ, vì X = ()() là một dãy ngoặc đúng nên (()()) cũng là một dãy ngoặc đúng.   
	\item     Nếu X và Y là hai dãy ngoặc đúng, thì XY là một dãy ngoặc đúng. Ví dụ, vì X = (()) và Y = () là các dãy ngoặc đúng, nên XY = (())() cũng là dãy ngoặc đúng.   
\end{itemize}

   Một số ví dụ về các dãy không phải là dãy ngoặc đúng: )(, (())), ((()...  

\subsubsection{   Input  }
\begin{itemize}
	\item     Dòng 1: Số nguyên dương N duy nhất   
	\item     N dòng tiếp, mỗi dòng ghi 1 dãy ngoặc   
\end{itemize}

\subsubsection{   Output  }

   Độ dài của dãy ngoặc lớn nhất tìm được  

\subsubsection{   Chấm điểm  }
\begin{itemize}
	\item     Trong quá trình thi, bài của bạn sẽ chỉ được chấm với một test duy nhất là test đề bài.   
\end{itemize}

\subsubsection{   Giới hạn  }
\begin{itemize}
	\item     Trong tất cả các test, N ≤ 1000, tổng độ dài các dãy ngoặc trong input không quá 10,000   
	\item     Trong 30\% test, N ≤ 8, tổng độ dài các dãy ngoặc không quá 100.   
\end{itemize}

\subsubsection{   Example  }
\begin{verbatim}
\textbf{Input:}
3
(
)
)(

\textbf{Output:}
4
\end{verbatim}

\subsubsection{   Giải thích  }

   Bé có thể nối dãy ngoặc thứ 1 với thứ 2 tạo thành dãy ngoặc () là dãy ngoặc đúng. Bé cũng có thể nối dãy ngoặc thứ 1, thứ 3 rồi thứ 2, tạo thành dãy ngoặc ()() là dãy ngoặc đúng.  