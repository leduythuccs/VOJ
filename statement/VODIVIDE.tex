

Vinh và Sơn có N đồng tiền cổ, N là số chẵn. Ta tạm đánh số các đồng tiền cổ này theo thứ tự từ 1-$>$N. Đồng tiền thứ i Vinh định giá nó là ai còn Sơn định giá nó là bi.

Vinh và Sơn tiền hành phân chia số tiền này, lần lượt thực hiện N/2 lượt . Ở lượt chơi thứ t, Sơn lấy ra 2 đồng tiền, Vinh sẽ lấy đồng tiền nào có giá trị lớn hơn trong 2 đồng tiền theo như Vinh định giá, và Sơn sẽ lấy đồng tiền còn lại.

Sơn biết tất cả các số ai, bi (i=1..N) và mỗi bước Sơn được quyền chọn cách lấy 2 đồng tiền tùy theo ý thích của mình.

Hãy giúp Sơn tiền hành việc phân chia các đồng tiền sao cho tổng giá trị các đồng tiền Sơn nhận được là lớn nhất có thể.

\subsubsection{Input}
\begin{itemize}
	\item Dòng đầu tiên chứa số nguyên dương chẵn N là số đồng tiền.
	\item Dòng thứ 2 chứa N số nguyên dương a1, a2, ..., aN mỗi số ko vượt quá 400000
	\item Dòng thứ 3 chứa N số nguyên dương b1, b2, ..., bN mỗi số ko vượt quá 400000.
\end{itemize}

\subsubsection{Output}

Dòng đầu tiên ghi ra tổng số tiền lớn nhất mà Sơn nhận được.

N/2 dòng tiếp theo mỗi dòng ghi cặp 2 chỉ số các đồng tiền mà Sơn lấy ra theo đúng thứ tự mỗi lượt thực hiện chia phần. Nếu có nhiều phương án in ra 1 phương án bất kì.

\subsubsection{Example}
\begin{verbatim}
\textbf{Input:
}6
6 10  11 18 5 14
1 7 6 12 15 1
\textbf{Output:
}28
5 1
2 6
3 4
\end{verbatim}

\textbf{Giới hạn:}
\begin{itemize}
	\item 30\% số test với N  $\le$  5000 và ai = bi với i = 1..N;
	\item 30\% số test với N  $\le$  20
	\item 40\% số test còn lại với N  $\le$  5000 
\end{itemize}
