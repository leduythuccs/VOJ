

Từ một số tự nhiên X, ta có thể thực hiện 2 thao tác sau:
\begin{itemize}
	\item Tăng X lên 1 đơn vị
	\item Hoán vị các chữ số của X ( chữ số đầu tiên phải khác 0 )
\end{itemize}

Hỏi từ X = 1, chúng ta cần ít nhất bao nhiêu thao tác để X = N

\subsubsection{Input}

Dòng 1: Chứa số nguyên T là số test

T dòng tiếp theo, mỗi dòng chứa một số tự nhiên N

\subsubsection{Output}

Gồm T dòng, dòng thứ i in ra một số tự nhiên duy nhất là số thao tác ít nhất của test thứ i

\subsubsection{Example}
\begin{verbatim}
\textbf{Input:}
1
21
\textbf{Output:}
12\end{verbatim}

\subsubsection{Giải thích:}

1 -> 2 -> 3 -> 4 -> 5 -> 6 -> 7 -> 8 -> 9 -> 10 -> 11 -> 12 -> 21

\subsubsection{Giới hạn:}
\begin{itemize}
	\item T  $\le$  100
	\item 30\% số test có N  $\le$  10^5
	\item 70\% số test còn lại có N  $\le$  10^18
\end{itemize}
