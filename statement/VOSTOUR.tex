



   Quang Đạt được cho một danh sách F chuyến bay một chiều giữa C thành phố (đánh số từ 0 đến C-1). Anh ta muốn thăm T thành phố đầu tiên (từ 0 đến T-1). Nhiệm vụ của bạn là tìm số lần di chuyển ít nhất giữa các thành phố để Quang Đạt hoàn thành mục tiêu.  

   Quang Đạt sống ở thành phố 0. Do đó chuyến đi của anh ta phải bắt đầu và kết thúc ở thành phố 0. Một thành phố có thể được thăm nhiều lần.  

   Dữ liệu đảm bảo luôn bạn luôn tìm được đường bắt đầu và kết thúc tại thành phố 0 và đi qua tất cả các thành phố từ 0 đến T-1.  

\subsubsection{   Input  }

   Dòng đầu chứa 3 số nguyên C, T và F.  

   F dòng tiếp chứa 2 số nguyên a, b thể hiện đường đi một chiều từ a đến b.  

\subsubsection{   Output  }

   In ra số nguyên duy nhất là số lần di chuyển ít nhất để Quang Đạt hoàn thành mục tiêu.  

\subsubsection{   Example  }
\begin{verbatim}
\textbf{Input:}

7 4 12

0 5

0 4

1 0

1 2

2 6

3 0

3 6

4 3

4 5

6 1

6 26 5\end{verbatim}
\begin{verbatim}
\textbf{Output:}\end{verbatim}
\begin{verbatim}
7\end{verbatim}

\textbf{      Giải thích     }     :    


\includegraphics{https://dl.dropboxusercontent.com/u/44735005/C11%20Contest/25.jpg}

    Trong ví dụ, có 7 thành phố, Fred chỉ cần thăm các thành phố 0, 1, 2, 3. Nếu anh ta thăm theo thứ tự:   

    0 → 4 → 3 → 6 → 2 → 6 → 1 → 0   

    Anh ta chỉ cần di chuyển 7 lần, và đây chính là lộ trình ngắn nhất có thể để thăm hết các thành phố từ 0 đến 3.   
\begin{verbatim}


\textbf{Giới hạn dữ liệu}:

3 $\le$ T $\le$ 8

T $\le$ C $\le$ 5000

C $\le$ F $\le$ 100000

Trong 60% test: C $\le$ 80

Trong 30% test: T $\le$ 5, C $\le$ 10\end{verbatim}
