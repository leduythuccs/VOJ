

 

Cho một đồ thị cây có N đỉnh N-1 cạnh, mỗi đỉnh của cây có một giá trị là A $_ i $ , Người ta tiến hành thao tác sau:

Bước 1: Chọn một cây con có gốc là đỉnh 1, gọi là cây con T, sao cho khoảng cách từ các đỉnh của cây con T tới đỉnh 1 có thể tạo thành một dãy tăng nghiêm ngặt.

Bước 2: Tiến hành tăng hoặc giảm giá trị tất cả các đỉnh của cây con T đã chọn ở bước 1 đi 1 đơn vị.

Nhiệm vụ của bạn tính số thao tác nhỏ nhất để tất cả các đỉnh của cây đều có giá trị bằng 0.

\subsubsection{Input}

Dòng 1: Một số nguyên T, số test đề bài (1≤T≤10).

T bộ test tiếp theo có dạng:

Dòng 1: Số nguyên N, số đỉnh của cây (1≤N≤10 $^ 5 $ ).

N-1 dòng tiếp theo, mỗi dòng gồm hai số nguyên u và v cho biết cạnh nối giữa hai đỉnh u và v (1≤u,v≤N).

Dòng tiếp theo: Gồm  N số nguyên A $_ 1 $ ... A $_ N $ (1≤|A $_ i $ |≤10 $^ 5 $ ).

\subsubsection{Output}

Gồm T dòng, mỗi dòng một số nguyên là số thao tác ít nhất ứng với bộ test đó

\subsubsection{Example}
\begin{verbatim}
\textbf{Input:
}1
3
1 2
1 3
-1 -1 -1
\textbf{Output:}
3
\end{verbatim}

\textbf{Giải thích test:}

Thao tác 1: Chọn cây con T gồm hai đỉnh 1 và 2, tăng giá trị tất cả các đỉnh lên 1, mảng A=[0,0,-1].

Thao tác 2: Chọn cây con T gồm hai đỉnh 1 và 3, tăng giá trị tất cả các đỉnh lên 1, mảng A=[1,0,0].

Thao tác 3, chọn cây con T gồm đỉnh 1, giảm giá trị các đỉnh đi 1, mảng A=[0,0,0].

Bạn không thể chọn cây con T gồm 3 đỉnh 1, 2, 3 bởi vì khi đó khoảng cách từ các đỉnh tới 1 sẽ lần lượt là 0,1,1

-$>$không tạo thành một dãy tăng.

Thông tin về tree cho các bạn chưa biết: http://en.wikipedia.org/wiki/Tree\_(data\_structure)