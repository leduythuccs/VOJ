



   Cho một dãy gồm N (1 ≤ N ≤ 100000) số nguyên dương $a_{1}$   , $a_{2}$   , ..., $a_{n}$   . Tổng $a_{i}$   + $a_{i+1}$   + ... + $a_{j}$   (1 ≤ i ≤ j ≤ N) được gọi là tổng bộ phận từ i đến j của dãy số.  

   Cho hai số nguyên dương P và K (1 $<$ P ≤ $10^{9}$   , 0 ≤ K $<$ P). Hãy tìm tổng bộ phận theo modulo P nhỏ nhất không bé hơn K.  

   Ví dụ, xét dãy số sau:  
\begin{verbatim}
12     13     15     11     16     26     11
\end{verbatim}

   Ở đây N=7, giả sử K=2 và P=17, ta có kết quả là 2 vì 11 + 16 + 26 = 53 và 53 mod 17 = 2. Nếu K=0 ta có kết quả bằng 0 vì 15 + 11 + 16 + 26 = 68 và 68 mod 17 = 0.  

\subsubsection{   Dữ liệu  }
\begin{itemize}
	\item     Dòng 1: N, K, P.   
	\item     Dòng 2..n+1: $a_{1}$    , $a_{2}$    ,... , $a_{N}$    , mỗi số trên một dòng.   
\end{itemize}

\subsubsection{   Kết quả  }

   In ra tổng bộ phận theo modulo P nhỏ nhất không bé hơn K.  

\subsubsection{   Ví dụ  }
\begin{verbatim}
Dữ liệu
7 2 17
12
13
15
11
16
26
11

Kết quả
2
\end{verbatim}
