



   Chú chó sói Vjekoslav đang chạy trốn khỏi một đám thợ săn khát máu. Những người thợ săn rất thông minh và họ đang nấp sau những cái cây. Vjekoslav biết điều đó, nhưng không biết chính xác cây nào. Con sói muốn về nơi ở của nó một cách an toàn nhất , tức là càng xa cây càng tốt !  

   Khu rừng có thể được mô tả bằng một hình chữ nhật kích thước N*M. Những ô trống được đánh dấu bằng ký hiệu '.' , những ô có cây là '+' , vị trí ban đầu của Vjekoslav là 'V' và nhà của nó là 'J'. Vjekoslav có thể chạy từ ô nó đang đứng đến 4 ô  chung cạnh xung quanh nó đứng.  

   Nếu Vjekoslav đang ở ô (R,C) và có một cái cây ở ô (A,B) thì khoảng cách được tính theo công thức :|R-A| + |C-B|. Hãy giúp Vjekoslav tìm đường đi an toàn nhất để về nhà . Đường đi an toàn nhất được hiểu là đường đi mà   \textbf{    khoảng cách  bé nhất từ một ô nào đó trên đường đi đó đến tất cả các cây   }   là lớn nhất.  

\subsubsection{   Input  }

   Dòng đầu tiên là hai số N,M (0$<$N,M  $\le$ 500) là kích thước của khu rừng.  

   N dòng sao mỗi dòng gồmN ký tự thuộc tập \{'+','.','V','J'\} mô tả khu rừng.  

   Input luôn đảm bảo chứa một ký tự 'V', 1 ký tự 'J' và ít nhất một ký tự '+'.  

\subsubsection{   Output  }

   Gồm một dòng duy nhất là kết quả.  

\subsubsection{   Example  }
\begin{verbatim}
\textbf{Input:}
\\
\\4 4
\\+...
\\....
\\....
\\V..J
\\
\\\textbf{Output:}
\\
\\3
\\
\\
\\\textbf{Input}
\\
\\4 5
\\.....
\\.+++.
\\.+.+.
\\V+.J+
\\
\\\textbf{Output}
\\
\\0.
\\\end{verbatim}
